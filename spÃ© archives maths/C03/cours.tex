\documentclass{article}
\usepackage{amsmath,amssymb,mathtools}
\usepackage{xcolor}
\usepackage{minted}
\usepackage{enumitem}
\usepackage{multicol}
\usepackage{changepage}
\usepackage{stmaryrd}
\usepackage{graphicx}
\graphicspath{ {./images/} }
\usepackage[framemethod=tikz]{mdframed}
\usepackage{tikz,pgfplots}
\pgfplotsset{compat=1.18}

% physique
\definecolor{oranges}{RGB}{255, 242, 230}
\definecolor{rouges}{RGB}{255, 230, 230}
\definecolor{rose}{RGB}{255, 204, 204}

% maths - info
\definecolor{rouge_fonce}{RGB}{204, 0, 0}
\definecolor{rouge}{RGB}{255, 0, 0}
\definecolor{bleu_fonce}{RGB}{0, 0, 255}
\definecolor{vert_fonce}{RGB}{0, 69, 33}
\definecolor{vert}{RGB}{0,255,0}

\definecolor{orange_foncee}{RGB}{255, 153, 0}
\definecolor{myrtille}{RGB}{225, 225, 255}
\definecolor{mayonnaise}{RGB}{255, 253, 233}
\definecolor{magenta}{RGB}{224, 209, 240}
\definecolor{pomme}{RGB}{204, 255, 204}
\definecolor{mauve}{RGB}{255, 230, 255}


% Cours

\newmdenv[
  nobreak=true,
  topline=true,
  bottomline=true,
  rightline=true,
  leftline=true,
  linewidth=0.5pt,
  linecolor=black,
  backgroundcolor=mayonnaise,
  innerleftmargin=10pt,
  innerrightmargin=10pt,
  innertopmargin=5pt,
  innerbottommargin=5pt,
  skipabove=\topsep,
  skipbelow=\topsep,
]{boite_definition}

\newmdenv[
  nobreak=false,
  topline=true,
  bottomline=true,
  rightline=true,
  leftline=true,
  linewidth=0.5pt,
  linecolor=white,
  backgroundcolor=white,
  innerleftmargin=10pt,
  innerrightmargin=10pt,
  innertopmargin=5pt,
  innerbottommargin=5pt,
  skipabove=\topsep,
  skipbelow=\topsep,
]{boite_exemple}

\newmdenv[
  nobreak=true,
  topline=true,
  bottomline=true,
  rightline=true,
  leftline=true,
  linewidth=0.5pt,
  linecolor=black,
  backgroundcolor=magenta,
  innerleftmargin=10pt,
  innerrightmargin=10pt,
  innertopmargin=5pt,
  innerbottommargin=5pt,
  skipabove=\topsep,
  skipbelow=\topsep,
]{boite_proposition}

\newmdenv[
  nobreak=true,
  topline=true,
  bottomline=true,
  rightline=true,
  leftline=true,
  linewidth=0.5pt,
  linecolor=black,
  backgroundcolor=white,
  innerleftmargin=10pt,
  innerrightmargin=10pt,
  innertopmargin=5pt,
  innerbottommargin=5pt,
  skipabove=\topsep,
  skipbelow=\topsep,
]{boite_demonstration}

\newmdenv[
  nobreak=true,
  topline=true,
  bottomline=true,
  rightline=true,
  leftline=true,
  linewidth=0.5pt,
  linecolor=white,
  backgroundcolor=white,
  innerleftmargin=10pt,
  innerrightmargin=10pt,
  innertopmargin=5pt,
  innerbottommargin=5pt,
  skipabove=\topsep,
  skipbelow=\topsep,
]{boite_remarque}


\newenvironment{definition}[2]
{
    \vspace{15pt}
    \begin{boite_definition}
    \textbf{\textcolor{rouge}{Définition #1}}
    \if\relax\detokenize{#2}\relax
    \else
        \textit{ - #2}
    \fi \\ \\
}
{
    \end{boite_definition}
    
}

\newenvironment{exemple}[2]
{
    \vspace{15pt}
    \begin{boite_exemple}
    \textbf{\textcolor{bleu_fonce}{Exemple #1}}
    \if\relax\detokenize{#2}\relax
    \else
        \textit{ - #2}
    \fi \\ \\ 
}
{   
    \end{boite_exemple}
    
}

\newenvironment{proposition}[2]
{
    \vspace{15pt}
    \begin{boite_proposition}
    \textbf{\textcolor{rouge}{Proposition #1}}
    \if\relax\detokenize{#2}\relax
    \else
        \textit{ - #2}
    \fi \\ \\
}
{
    \end{boite_proposition}
    
}

\newenvironment{theoreme}[2]
{
    \vspace{15pt}
    \begin{boite_proposition}
    \textbf{\textcolor{rouge}{Théorème #1}} 
    \if\relax\detokenize{#2}\relax
    \else
        \textit{ - #2}
    \fi \\ \\
}
{
    \end{boite_proposition}
    
}

\newenvironment{demonstration}
{
    \vspace{15pt}
    \begin{boite_demonstration}
    \textbf{\textcolor{rouge}{Démonstration}}\\ \\
}
{
    \end{boite_demonstration}
    
}

\newenvironment{remarque}[2]
{
    \vspace{15pt}
    \begin{boite_remarque}
    \textbf{\textcolor{bleu_fonce}{Remarque #1}}
    \if\relax\detokenize{#2}\relax
    \else
        \textit{ - #2}
    \fi \\ \\   
}
{  
    \end{boite_remarque}
    
}



%Corrections
\newmdenv[
  nobreak=true,
  topline=true,
  bottomline=true,
  rightline=true,
  leftline=true,
  linewidth=0.5pt,
  linecolor=black,
  backgroundcolor=mayonnaise,
  innerleftmargin=10pt,
  innerrightmargin=10pt,
  innertopmargin=5pt,
  innerbottommargin=5pt,
  skipabove=\topsep,
  skipbelow=\topsep,
]{boite_question}


\newenvironment{question}[2]
{
    \vspace{15pt}
    \begin{boite_question}
    \textbf{\textcolor{rouge}{Question #1}}
    \if\relax\detokenize{#2}\relax
    \else
        \textit{ - #2}
    \fi \\ \\
}
{
    \end{boite_question}
    
}

\newenvironment{enumeratebf}{
    \begin{enumerate}[label=\textbf{\arabic*.}]
}
{
    \end{enumerate}
}

\begin{document}
\begin{adjustwidth}{-3cm}{-3cm}
\begin{document}
\begin{adjustwidth}{-3cm}{-3cm}
% commandes
\newcommand{\notion}[1]{\textcolor{vert_fonce}{\textit{#1}}}
\newcommand{\mb}[1]{\mathbb{#1}}
\newcommand{\mc}[1]{\mathcal{#1}}
\newcommand{\mr}[1]{\mathrm{#1}}
\newcommand{\code}[1]{\texttt{#1}}
\newcommand{\ccode}[1]{\texttt{|#1|}}
\newcommand{\ov}[1]{\overline{#1}}
\newcommand{\abs}[1]{|#1|}
\newcommand{\rev}[1]{\texttt{reverse(#1)}}
\newcommand{\crev}[1]{\texttt{|reverse(#1)|}}

\newcommand{\ie}{\textit{i.e.} }

\newcommand{\N}{\mathbb{N}}
\newcommand{\R}{\mathbb{R}}
\newcommand{\C}{\mathbb{C}}
\newcommand{\K}{\mathbb{K}}
\newcommand{\Z}{\mathbb{Z}}

\newcommand{\A}{\mathcal{A}}
\newcommand{\bigO}{\mathcal{O}}
\renewcommand{\L}{\mathcal{L}}

\newcommand{\rg}[0]{\mathrm{rg}}
\newcommand{\re}[0]{\mathrm{Re}}
\newcommand{\im}[0]{\mathrm{Im}}
\newcommand{\cl}[0]{\mathrm{cl}}
\newcommand{\grad}[0]{\vec{\mathrm{grad}}}
\renewcommand{\div}[0]{\mathrm{div}\,}
\newcommand{\rot}[0]{\vec{\mathrm{rot}}\,}
\newcommand{\vnabla}[0]{\vec{\nabla}}
\renewcommand{\vec}[1]{\overrightarrow{#1}}
\newcommand{\mat}[1]{\mathrm{Mat}_{#1}}
\newcommand{\matrice}[1]{\mathcal{M}_{#1}}
\newcommand{\sgEngendre}[1]{\left\langle #1 \right\rangle}
\newcommand{\gpquotient}[1]{\mathbb{Z} / #1\mathbb{Z}}
\newcommand{\norme}[1]{||#1||}
\renewcommand{\d}[1]{\,\mathrm{d}#1}
\newcommand{\adh}[1]{\overline{#1}}
\newcommand{\intint}[2]{\llbracket #1 ,\, #2 \rrbracket}
\newcommand{\seg}[2]{[#1\, ; \, #2]}
\newcommand{\scal}[2]{( #1 | #2 )}
\newcommand{\distance}[2]{\mathrm{d}(#1,\,#2)}
\newcommand{\inte}[2]{\int_{#1}^{#2}}
\newcommand{\somme}[2]{\sum_{#1}^{#2}}
\newcommand{\deriveref}[4]{\biggl( \frac{\text{d}^{#1}#2}{\text{d}#3^{#1}} \biggr)_{#4}}






\begin{definition}{3.1}{caractérisation de norme sur un $\K$-espace vectoriel}
    $\K = \R$ ou $\K = \mb{C}$.\\
    Soit $E$ un $\K$-espace vectoriel. une \notion{norme sur $E$} est une application $\norme{.} : E \to \R$ vérifiant pour tout $(x,y) \in E^2$~:  
    \begin{enumeratebf}
        \item \notion{positivité} :
            $$\norme{x} \geq 0$$
        \item \notion{Axiome de séparation} : 
            $$\norme{x} = 0 \implies x = 0$$
        \item \notion{Absolue homogénéité} : 
            $$\forall \lambda \in \K,\, \norme{\lambda x} = \abs{\lambda} \cdot \norme{x}$$
        \item \notion{Inégalité triangulaire} :
            $$\norme{x} + \norme{y} \geq \norme{x + y}$$
    \end{enumeratebf}
    
\end{definition}

\begin{exemple}{3.3 (1)}{normes de $\K^n$}
    Les applications suivantes sont des normes sur $\K^n$~:
    \begin{enumeratebf}
        \item $\displaystyle \norme{.}_1 : (x_1,\, \dots,\, x_n) \mapsto \somme{i=1}{n}\abs{x_i}$
        \item la \notion{norme euclidienne associée au produit scalaire canonique sur $\K^n$}~:
        $$\norme{.}_2 : (x_1,\, \dots,\, x_n) \mapsto \sqrt{\somme{i=1}{n} \abs{x_i}^2}$$
        \item la \notion{norme infinie} $\displaystyle \norme{.}_\infty : (x_1,\, \dots,\, x_n) \mapsto \max_{i \in \intint{1}{n}}(\abs{x_i})$
    \end{enumeratebf}
\end{exemple}

\begin{exemple}{3.3 (2)}{normes de $\mc{C}^0(\seg{a}{b}, \R)$}
    Les applications suivantes sont des normes sur $\mc{C}^0(\seg{a}{b}, \R)$~:
    \begin{enumeratebf}
        \item la \notion{norme de la convergence en moyenne} $\displaystyle \norme{.}_1 : f \mapsto \inte{a}{b}\abs{f(t)}\d{t}$
        \item la \notion{norme euclidienne associée au produit scalaire canonique sur $\mc{C}^0(\seg{a}{b}, \R)$}~: 
        $$\norme{.}_2 : f \mapsto \sqrt{\inte{a}{b}f(t)^2\d{t}}$$
        \item la \notion{norme infinie} $\displaystyle \norme{.}_\infty : f \mapsto \sup_{x \in \seg{a}{b}}(\abs{f(x)})$
    \end{enumeratebf}
\end{exemple}

\begin{theoreme}{3.7}{norme euclidienne associée à un produit scalaire}
    Soit $(E, \scal{.}{.})$ un espace préhilbertien réel. L'application $\norme{.}:x \mapsto \sqrt{\scal{x}{x}}$ est une norme, appelée \notion{norme euclidienne associée à $\scal{.}{.}$}.
\end{theoreme}

\begin{definition}{3.12}{espace métrique}
    Soit $E$ un ensemble. Une application $d : E \times E \to \R$ est appelée \notion{distance} si elle vérifie ces propriétés~:
    \begin{enumeratebf}
        \item $\forall (x,y) \in E^2,\, \distance{x}{y} \geq 0$
        \item $\forall (x,y) \in E^2,\, \distance{x}{y} = 0 \implies x = y$
        \item $\forall (x,y) \in E^2,\, \distance{x}{y} = d(y,\,x)$
        \item $\forall (x,y,z) \in E^3,\, \distance{x}{z} \leq \distance{x}{y} + \distance{y}{z}$
    \end{enumeratebf}
    Une telle application munit $E$ d'une structure d'\notion{espace métrique}. 
\end{definition}

\begin{definition}{3.14}{distance d'un point à une partie non vide}
    Soit $(E, \text{d})$ un espace métrique. Étant donnée une partie $A$ de $E$ et $x$ un élément de $E$, on appelle \notion{distance de $x$ à $A$} la borne inférieure des distances de $x$ à tous les éléments de $A$~:
    $$\distance{x}{A} = \inf_{a \in A} \distance{x}{a}$$
\end{definition}

\begin{definition}{3.15}{sphère d'un espace vectoriel normé}
    Soit $E$ un espace vectoriel normé. On appelle \notion{sphère de centre $a \in E$ et de rayon $r \geq 0$} de $E$ l’ensemble $\mc{S}(a,\, r) = \{ x \in E,\, \norme{x - a} = r\}$.
\end{definition}

\begin{definition}{3.16}{partie bornée d'un espace vectoriel normé}
    Soit $E$ un espace vectoriel normé. On dit qu'une partie $A$ de $E$ est \notion{bornée} lorsqu'il existe une boule fermée la contenant~: 
    $$\exists a \in E,\, \exists r \geq 0,\, A \subset \overline{B}(a,\, r)$$
    Soit : 
    $$\exists r \geq 0,\, \forall x \in A,\, \norme{x} \leq r$$
\end{definition}

\begin{definition}{3.19}{application bornée}
    Soit $E$ un espace vectoriel normé et $X$ un ensemble fini. Une application $\varphi : X \to E$ est dite bornée lorsque l'ensemble $\im(\varphi)    $ est borné.
\end{definition}

\begin{definition}{3.20}{applications bornées sur un espace vecotriel normé}
    Soit $E$ un $\K$-espace vectoriel normé et $X$ un ensemble fini. L'ensemble $\mc{B}(X,\,E)$ est un $\K$-espace vectoriel et l'application : 
    $$f \mapsto \sup_{x \in X}\norme{f(x)}$$
    est une norme sur $\mc{B}(X,\,E)$, appelée \notion{norme infinie}.
\end{definition}

\begin{definition}{3.21}{application lipschitzienne}
    Soit $E$ et $F$ deux espaces vectoirles normés, $A \subset E$ et $k \geq 0$. Une application $f : A \to F$ est dite $k$-lipschitzienne sur $A$ lorsque~:
    $$\forall (x,y) \in A^2,\, \norme{f(x) -  f(y)}_F \leq k\norme{x-y}_E$$
    L'ensemble des applications $k$-lipschitziennes de $A$ dans $F$ est noté $\text{Lip}_k(A,\, F)$.
\end{definition}

\begin{definition}{3.30}{normes équivalentes} 
    Soit $N$ et $N'$ deux normes sur un espace vectoriel $E$. On dit que \notion    {$N$ et $N'$ sont équivalentes} lorsque :
    $$\exists(\alpha,\, \beta) \in (\R_+^*)^2,\, \alpha N \leq N' \leq \beta N$$
    Il s'agit d'une relation d'équivalence.
\end{definition}

\begin{proposition}{3.31}{normes équivalentes et convergence de suites}
    Deux normes sur un espace vectoriel sont équivalentes si et seulement si toute suite qui converge pour l'une vers $x \in E$ converge pour l'autre, également vers $x$.
\end{proposition}

\begin{proposition}{3.32}{normes équivalentes et boules incluses}
    Deux normes sur un espace vectoriel sont équivalentes si et seulement si toute boule pour l'une contient une boule pour l'autre.
\end{proposition}

\end{adjustwidth}
\end{document}