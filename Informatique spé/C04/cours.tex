\documentclass{article}
\usepackage{amsmath,amssymb,mathtools}
\usepackage{esint} % intégrale avec un round
\usepackage{xcolor}
\usepackage{listings}
% \usepackage{minted}
\usepackage{enumitem}
\usepackage{fourier-orns}
\usepackage{multicol}
\usepackage{changepage}
\usepackage{stmaryrd}
\usepackage{graphicx}
\graphicspath{ {./images/} }
\usepackage[framemethod=tikz]{mdframed}
\usepackage{tikz,pgfplots}
\pgfplotsset{compat=1.18}
\usetikzlibrary{arrows}
\usepackage{forest}
\usepackage{hyperref}

% physique
% \renewcommand*{\overrightarrow}[1]{\vbox{\halign{##\cr 
%   \tiny\rightarrowfill\cr\noalign{\nointerlineskip\vskip1pt} 
%   $#1\mskip2mu$\cr}}}

  \newenvironment{enumeratebf}{
    \begin{enumerate}[label=\textbf{\arabic*.}]
}
{
    \end{enumerate}
}
  
\definecolor{oranges}{RGB}{255, 242, 230}
\definecolor{rouges}{RGB}{255, 230, 230}
\definecolor{rose}{RGB}{255, 204, 204}

% maths - info
\definecolor{rouge_fonce}{RGB}{204, 0, 0}
\definecolor{rouge}{RGB}{255, 0, 0}
\definecolor{bleufonce}{RGB}{0, 0, 255}
\definecolor{vert_fonce}{RGB}{0, 69, 33}
\definecolor{vert}{RGB}{0,255,0}

\definecolor{orange_foncee}{RGB}{255, 153, 0}
\definecolor{myrtille}{RGB}{225, 225, 255}
\definecolor{mayonnaise}{RGB}{255, 253, 233}
\definecolor{magenta}{RGB}{224, 209, 240}
\definecolor{pomme}{RGB}{204, 255, 204}
\definecolor{mauve}{RGB}{255, 230, 255}


% Cours

\newmdenv[
  nobreak=true,
  topline=true,
  bottomline=true,
  rightline=true,
  leftline=true,
  linewidth=0.5pt,
  linecolor=black,
  backgroundcolor=mayonnaise,
  innerleftmargin=10pt,
  innerrightmargin=2.5em,
  innertopmargin=5pt,
  innerbottommargin=5pt,
  skipabove=\topsep,
  skipbelow=\topsep,
]{boite_definition}


\newenvironment{definition}[2]
{
    \vspace{15pt}
    \begin{boite_definition}
    \textbf{\textcolor{rouge}{Définition #1}}
    \if\relax\detokenize{#2}\relax
    \else
        \textit{ - #2}
    \fi \\ \\
}
{
    \end{boite_definition}
    
}

\newmdenv[
  nobreak=true,
  topline=true,
  bottomline=true,
  rightline=true,
  leftline=true,
  linewidth=0.5pt,
  linecolor=white,
  backgroundcolor=white,
  innerleftmargin=10pt,
  innerrightmargin=2.5em,
  innertopmargin=5pt,
  innerbottommargin=5pt,
  skipabove=\topsep,
  skipbelow=\topsep,
]{boite_exemple}


\newenvironment{exemple}[2]
{
    \vspace{15pt}
    \begin{boite_exemple}
    \textbf{\textcolor{bleufonce}{Exemple #1}}
    \if\relax\detokenize{#2}\relax
    \else
        \textit{ - #2}
    \fi \\ \\ 
}
{   
    \end{boite_exemple}
    
}


\newmdenv[
  nobreak=true,
  topline=true,
  bottomline=true,
  rightline=true,
  leftline=true,
  linewidth=0.5pt,
  linecolor=black,
  backgroundcolor=magenta,
  innerleftmargin=10pt,
  innerrightmargin=2.5em,
  innertopmargin=5pt,
  innerbottommargin=5pt,
  skipabove=\topsep,
  skipbelow=\topsep,
]{boite_proposition}

\newenvironment{proposition}[2]
{
    \vspace{15pt}
    \begin{boite_proposition}
    \textbf{\textcolor{rouge}{Proposition #1}}
    \if\relax\detokenize{#2}\relax
    \else
        \textit{ - #2}
    \fi \\ \\
}
{
    \end{boite_proposition}
}


\newmdenv[
  nobreak=true,
  topline=true,
  bottomline=true,
  rightline=true,
  leftline=true,
  linewidth=0.5pt,
  linecolor=black,
  backgroundcolor=magenta,
  innerleftmargin=10pt,
  innerrightmargin=2.5em,
  innertopmargin=5pt,
  innerbottommargin=5pt,
  skipabove=\topsep,
  skipbelow=\topsep,
]{boite_theoreme}


\newenvironment{theoreme}[2]
{
    \vspace{15pt}
    \begin{boite_theoreme}
    \textbf{\textcolor{rouge}{Théorème #1}} 
    \if\relax\detokenize{#2}\relax
    \else
        \textit{ - #2}
    \fi \\ \\
}
{
    \end{boite_theoreme}
    
}


\newmdenv[
  nobreak=true,
  topline=true,
  bottomline=true,
  rightline=true,
  leftline=true,
  linewidth=0.5pt,
  linecolor=black,
  backgroundcolor=white,
  innerleftmargin=10pt,
  innerrightmargin=2.5em,
  innertopmargin=5pt,
  innerbottommargin=5pt,
  skipabove=\topsep,
  skipbelow=\topsep,
]{boite_demonstration}


\newenvironment{demonstration}
{
    \vspace{15pt}
    \begin{boite_demonstration}
    \textbf{\textcolor{rouge}{Démonstration}}\\ \\
}
{
    \end{boite_demonstration}
    
}


\newmdenv[
  nobreak=true,
  topline=true,
  bottomline=true,
  rightline=true,
  leftline=true,
  linewidth=0.5pt,
  linecolor=white,
  backgroundcolor=white,
  innerleftmargin=10pt,
  innerrightmargin=2.5em,
  innertopmargin=5pt,
  innerbottommargin=5pt,
  skipabove=\topsep,
  skipbelow=\topsep,
]{boite_remarque}


\newenvironment{remarque}[2]
{
    \vspace{15pt}
    \begin{boite_remarque}
    \textbf{\textcolor{bleufonce}{Remarque #1}}
    \if\relax\detokenize{#2}\relax
    \else
        \textit{ - #2}
    \fi \\ \\   
}
{  
    \end{boite_remarque}
    
}

\newmdenv[
  nobreak=true,
  topline=true,
  bottomline=true,
  rightline=true,
  leftline=true,
  linewidth=0.5pt,
  linecolor=bleufonce,
  backgroundcolor=white,
  innerleftmargin=10pt,
  innerrightmargin=2.5em,
  innertopmargin=5pt,
  innerbottommargin=5pt,
  skipabove=\topsep,
  skipbelow=\topsep,
]{boite_OCaml}


\definecolor{keywordcolor}{RGB}{133, 153, 0}  % les mots-clés
\definecolor{commentcolor}{RGB}{147, 161, 161} % les commentaires
\definecolor{stringcolor}{RGB}{42, 161, 152}  % les chaînes de caractères
\newenvironment{OCaml}[2]
{   
    \vspace{15pt}
    \begin{boite_OCaml}
    \textbf{\textcolor{bleufonce}{Implémentation #1}}
    \if\relax\detokenize{#2}\relax
    \else
        \textit{ - #2}
    \fi \\ \\   

    \lstset{
    language=[Objective]Caml,
    basicstyle=\ttfamily,          % Police par défaut pour le code
    keywordstyle=\color{keywordcolor}, % Mots-clés en bleu doux
    commentstyle=\color{commentcolor}, % Commentaires en vert pâle
    stringstyle=\color{stringcolor},   % Chaînes en orange léger
    backgroundcolor=\color{white},   % Fond très clair
    numbers=left,                  % Numérotation à gauche
    numberstyle=\ttfamily,             % Taille des numéros de ligne
    stepnumber=1,                  % Numérotation de chaque ligne
    frame=single,                  % Cadre autour du code
    breaklines=true,               % Retour à la ligne automatique
    tabsize=2,                        % Taille des tabulations          
    }

    \begin{lstlisting}
}
{      
    \end{lstlisting}
    \end{boite_OCaml}
}

\newmdenv[
  nobreak=true,
  topline=true,
  bottomline=true,
  rightline=true,
  leftline=true,
  linewidth=0.5pt,
  linecolor=black,
  backgroundcolor=mayonnaise,
  innerleftmargin=10pt,
  innerrightmargin=2.5em,
  innertopmargin=5pt,
  innerbottommargin=5pt,
  skipabove=\topsep,
  skipbelow=\topsep,
]{boite_question}


\newenvironment{question}[2]
{
    \vspace{15pt}
    \begin{boite_question}
    \textbf{\textcolor{rouge}{Question #1}}
    \if\relax\detokenize{#2}\relax
    \else
        \textit{ - #2}
    \fi \\ \\
}
{
    \end{boite_question}
    
}

\newmdenv[
  nobreak=true,
  topline=true,
  bottomline=true,
  rightline=true,
  leftline=true,
  linewidth=0.5pt,
  linecolor=black,
  backgroundcolor=white,
  innerleftmargin=10pt,
  innerrightmargin=2.5em,
  innertopmargin=5pt,
  innerbottommargin=5pt,
  skipabove=\topsep,
  skipbelow=\topsep,
]{boite_corollaire}



\newenvironment{corollaire}[2]
{
    \vspace{15pt}
    \begin{boite_corollaire}
    \textbf{\textcolor{rouge}{Corollaire #1}}
    \if\relax\detokenize{#2}\relax
    \else
        \textit{ - #2}
    \fi \\ \\   
}
{
    \end{boite_corollaire}
    
}

\begin{document}
\begin{adjustwidth}{-3cm}{-3cm}
% commandes
\newcommand{\notion}[1]{\textcolor{vert_fonce}{\textit{#1}}}
\newcommand{\mb}[1]{\mathbb{#1}}
\newcommand{\mc}[1]{\mathcal{#1}}
\newcommand{\code}[1]{\texttt{#1}}
\newcommand{\ccode}[1]{\texttt{|#1|}}
\newcommand{\ov}[1]{\overline{#1}}
\newcommand{\abs}[1]{|#1|}
\newcommand{\rev}[1]{\texttt{reverse(#1)}}
\newcommand{\crev}[1]{\texttt{|reverse(#1)|}}

\newcommand{\ie}{\textit{i.e.} }

\newcommand{\N}{\mathbb{N}}
\newcommand{\R}{\mathbb{R}}
\newcommand{\C}{\mathbb{C}}
\newcommand{\K}{\mathbb{K}}

\newcommand{\A}{\mathcal{A}}
\newcommand{\bigO}{\mathcal{O}}
\renewcommand{\L}{\mathcal{L}}

\newcommand{\rg}[0]{\text{rg}}
\newcommand{\re}[0]{\text{Re}}
\newcommand{\im}[0]{\text{Im}}
\newcommand{\cl}[0]{\text{cl}}
\newcommand{\mat}[1]{\text{Mat}_{#1}}
\newcommand{\matrice}[1]{\mathcal{M}_{#1}}
\newcommand{\sgEngendre}[1]{\left\langle #1 \right\rangle}
\newcommand{\norme}[1]{||#1||}
\renewcommand{\d}[1]{\,\text{d}#1}
\newcommand{\intint}[2]{\llbracket #1 ,\, #2 \rrbracket}
\newcommand{\seg}[2]{[#1\, ; \, #2]}
\newcommand{\scal}[2]{\left\langle #1 ,\, #2 \right\rangle}
\newcommand{\inte}[2]{\int_{#1}^{#2}}
\newcommand{\somme}[2]{\sum_{#1}^{#2}}






\begin{definition}{4.1}{jeu sans mémoire}
    On appelle \notion{jeu sans mémoire} un jeu dans lequel à tout instant de la partie, il est possible de déterminer si un joueur a gagné ou si un coup est valide, \notion{indépendament des précédents coups joués}.
\end{definition}

\begin{definition}{4.2}{jeu à information complète}
    On appelle \notion{jeu à information complète} un jeu dans lequel il n'y \notion{aucune information cachée} que les joueurs ne puissent \notion{savoir ou prévoir}.
\end{definition}

\begin{definition}{4.3}{graphe associé à un jeu à deux joueurs}
    Un jeu à deux joueurs $J_1$ et $J_2$, sans mémoire, à information complète peut être \notion{représenté par un graphe orienté biparti}~:
    $$G = (S,A) \quad \text{ où } S = S_1 \sqcup S_2,\, A \subset (S_1 \times S_2) \cup (S_2 \times S_1) $$
    Les sommets de $S_1$ sont appelés les \notion{états contrôlés par $J_1$} et ceux de $S_2$ les \notion{états conrôlés par $J_2$}.
\end{definition}

\begin{definition}{4.4}{coup possible pour un joueur}
    Soit un jeu à deux joueurs $J_1$ et $J_2$, sans mémoire, à information complète, de graphe associé $G=(S_1 \sqcup S_2,A)$. Pour $a \neq b$ dans $\{1;2\}$, on appelle l'arc \notion{$(s_a,s_b) \in A \cap (S_a \times S_b)$ un coup possible pour $J_a$ depuis l'état $s_a$ vers un état $s_b$ contrôlé par $J_b$}.
\end{definition}

\begin{definition}{4.5}{jeu d'accessiblité}
    Un jeu à deux joueurs $J_1$ et $J_2$, sans mémoire, à information complète de graphe associé $G=(S_1 \sqcup S_2,A)$ est dit \notion{d'accessibilité} si toute partie du jeu prend fin dès lors qu'un joueur atteint un état dit \notion{final} : il en existe trois types~:
    \begin{enumeratebf}
        \item les états gagnants pour $J_1$, dont l'ensemble est appelé \notion{condition de gain $F_1$}
        \item les états gagnants pour $J_2$, dont l'ensemble est appelé \notion{condition de gain $F_2$}
        \item les états de match nul, dont l'ensemble est appelé $F_0$
    \end{enumeratebf}
    Nécessairement, ces trois ensembles sont deux à deux disjoints.
\end{definition}

\begin{definition}{4.6}{partie partielle}
    Soit $G = (S,A)$ le graphe associé à un jeu d'accessiblité. On appelle \notion{partie partielle du jeu} tout chemin de $G$ partant de l'état inital de $G$ à un état quelconque de $G$. \notion{L'ensemble des parties partielles du jeu est noté $S^w$}.
\end{definition}

\begin{definition}{4.7}{stratégie}
    Soit $G = (S,A)$ le graphe associé à un jeu d'accessiblité. Une stratégie est une \notion{application $\varphi : S^w \to S$ qui à une partie partielle $(s_0, \dots, s_p)$ associe un sommet la prolongeant} : $\Big(s_p,\,  \varphi\big((s_0, \dots, s_p)\big)\Big) \in A$. On dit alors qu'\notion{un joueur suit une stratégie}. \\
    Pour un jeu sans mémoire, \underline{l'image d'une stratégie ne dépend que du dernier sommet de la partie partielle}~: $\varphi\big((s_0, \dots, s_p)\big) = \varphi(s_p)$ moralement.
\end{definition}

\begin{definition}{4.8}{stratégie gagnante}
    Soit $G = (S,A)$ le graphe associé à un jeu d'accessiblité. Une \notion{stratégie $\varphi$ est gagnante depuis un état $s$} lorsque depuis $s$, le joueur qui la suit gagne peu importe le choix de l'adversaire.
\end{definition}

\begin{definition}{4.9}{suite convergeant vers l'attracteur}
    Soit $G = (S,A)$ le graphe associé à un jeu d'accessiblité. Pour $a \neq b$ dans $\{1;2\}$, on définit \notion{$(\attracteur{a}{j})_{j \in \N}$ l'ensemble des positions permettant au joueur $J_a$ de gagner en au plus $j$ coups}. On a bien sûr $\attracteur{a}{0} = F_a$, puis pour gagner en au plus $j \in \N^*$ coups~:
    \begin{itemize}
        \item ou bien $J_a$ peut gagner en au plus $j-1$ coups,
        \item ou bien $J_a$ dispose d'un coup le permettant ensuite de gagner en au plus $j-1$ coups,
        \item ou bien enfin, $J_b$ se trouvant sur une position non finale, ne dispose que de coups permettant ensuite à $J_a$ de gagner en au plus $j-1$ coups.
    \end{itemize}
    En conclusion :
    \begin{align*}
        \attracteur{a}{j} = \,&\attracteur{a}{j-1} \,\cup \\
        &\Big\{s \in S_a,\, \exists t \in \attracteur{a}{j-1},\,  (s,t) \in A\Big\} \, \cup\\
        &\Big\{s \in S_b,\,s \text{ non final et } \forall t \in S,\, (s,t) \in A \implies t \in \attracteur{a}{j-1}\Big\}
    \end{align*} 
\end{definition}

\begin{definition}{4.10}{attracteur d'un joueur}
    Soit $G = (S,A)$ le graphe associé à un jeu d'accessiblité. Pour $a \neq b$ dans $\{1;2\}$, la suite $(\attracteur{a}{j})_{j \in \N}$ est croissante pour l'inclusion ($\attracteur{a}{0} \subset \attracteur{a}{1} \subset \dots$) et majorée par $S$. C'est d'après le théorème de la limite monotone une suite convergente, on note \notion{$\attracteur{a}{}$ sa limite, appelée attracteur du joueur $J_a$}~:
    $$\attracteur{a}{} = \lim_{j \to + \infty} \attracteur{a}{j}$$
    Il s'agit de l'ensemble des positions gagnantes pour le joueur $J_a$.
\end{definition}

\begin{theoreme}{4.11}{de Zermelo}
    Dans un jeu à \textbf{deux joueurs} \textbf{fini} (à parties finies), à \textbf{information complète} et \textbf{sans match nul}, pour tout état du jeu, il existe une stratégie gagnante pour l'un des joueurs partant de cet état. 
\end{theoreme}

\begin{definition}{4.12}{heuristique pour MinMax}
    Dans le cadre de l'algorithme MinMax appliqué à un jeu d'accessibilité de joueurs $J_1$ et $J_2$ de graphe associé $G = (S,A)$, \notion{une fonction heuristique pour faire gagner $J_1$} doit vérifier~:

    $$\fonction{h}{S}{\Z}{s}{\begin{cases*}
    M & si $s \in F_1$ \\
    -M & si $s \in F_2$ \\
    h(s) \in \intint{-M+1}{M-1} & sinon
    \end{cases*}}$$
    où $M$ est un entier naturel assez grand pour modéliser l'infini.
\end{definition}

\begin{implementation}{MinMax classique - obtention du score}
    \begin{itemize}
        \item \textbf{Entrée}~: \begin{itemize}
            \item $\mc{A}$ l'arbre associé au jeu
            \item $n$ un noeud de $\mc{A}$ (état du jeu)
            \item $p$ la profondeur d'exploration
            \item $h:S\to \Z$ une heuristique
        \end{itemize}
        \item \textbf{Sortie}~: score du sommet $n$ 
    \end{itemize}
    \begin{lstLNat}
    MinMax($\mc{A}$, $n$, $p$, $h$) :
        si $n$ est final ou $p=0$ :
            renvoyer $h(n)$
        sinon :
            si $n \in S_1$:
                res = $-\infty$ // ou tout minorant de $h$
                pour tout fils $f$ de $n$:
                    v = MinMax($\mc{A}$, $f$, $p-1$, $h$)
                    res = max(v,res)
                renvoyer res
            si $n \in S_2$:
                res = $+\infty$ // ou tout majorant de $h$
                pour tout fils $f$ de $n$:
                    v = MinMax($\mc{A}$, $f$, $p-1$, $h$)
                    res = min(v,res)
                renvoyer res
    \end{lstLNat}
\end{implementation}

\begin{implementation}{MinMax - meilleur coup possible}
    \begin{itemize}
        \item \textbf{Entrée}~: \begin{itemize}
            \item $\mc{A}$ l'arbre associé au jeu
            \item $n$ un noeud de $\mc{A}$ (état \textbf{non final} du jeu)
            \item $p$ la profondeur d'exploration
            \item $h:S\to \Z$ une heuristique
        \end{itemize}
        \item \textbf{Sortie}~: $f$ le fils de $n$ correspondant au meilleur coup (au score le plus souhaitable)
    \end{itemize}
    \begin{lstLNat}
    score = MinMax($\mc{A}$, $n$, $p$, $h$)
    pour tout $f$ fils de $n$ :
        v = MinMax($\mc{A}$, $f$, $p-1$, $h$)
        si v == score :
        renvoyer $f$
    \end{lstLNat}
\end{implementation}

\begin{implementation}{MinMax avec élagage Alpha-Bêta - obtention du score}
    L'appel initial se fait avec $\alpha = - \infty$ et $\beta = + \infty$.
    \begin{itemize}
        \item \textbf{Entrée}~: \begin{itemize}
            \item $\mc{A}$ l'arbre associé au jeu
            \item $n$ un noeud de $\mc{A}$ (état du jeu)
            \item $p$ la profondeur d'exploration
            \item $h:S\to \Z$ une heuristique
            \item $\alpha$ et $\beta$ bornant le score de $n$
        \end{itemize}
        \item \textbf{Sortie}~: score de $n$
    \end{itemize}
    \begin{lstLNat}
    AlphaBeta($\mc{A}$, $n$, $p$, $h$, $\alpha$, $\beta$) :
        si $n$ est final ou $p=0$ :
            renvoyer $h(n)$
        sinon :
            si $n \in S_1$:
                a = $\alpha$
                pour tout fils $f$ de $n$:
                    v = AlphaBeta($\mc{A}$, $f$, $p-1$, $h$, a, $\beta$)
                    a = max(v,a)
                    si a >= $\beta$ // $\intint{\alpha}{\beta}$ est vide ou un singleton : on élague
                        renvoyer a
                renvoyer a
            si $n \in S_2$:
                b = $\beta$
                pour tout fils $f$ de $n$:
                    v = AlphaBeta($\mc{A}$, $f$, $p-1$, $h$, $\alpha$, b)
                    b = min(v,b)
                    si b <= $\alpha$ // $\intint{\alpha}{\beta}$ est vide ou un singleton : on élague
                        renvoyer b
                renvoyer b
        \end{lstLNat}
\end{implementation}

\input{../../stock/pied.tex}