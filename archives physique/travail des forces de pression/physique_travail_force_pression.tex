\documentclass{article}
\usepackage{amsmath,amssymb}
\usepackage{xcolor}
\usepackage{enumitem}
\usepackage{multicol}
\usepackage{changepage}

% physique
\definecolor{oranges}{RGB}{255, 242, 230}
\definecolor{rouges}{RGB}{255, 230, 230}
\definecolor{rose}{RGB}{255, 204, 204}

% maths
\definecolor{saumon}{RGB}{224, 209, 240}


\renewcommand{\labelitemi}{--}
\begin{document}
\begin{adjustwidth}{-3cm}{-3cm}
    \pagecolor{rose}
    Le travail des forces de pression reçu algébriquement par le système lors d’une transformation d’un système soumis à une pression extérieure $P_{ext}$ l’amenant d’un état d’équilibre initial pour lequel le volume du système est $V_I$ à un état d’équilibre final pour lequel le volume du système est $V_F$ s’écrit :
    $$ \mathcal{W}_P = \int_{V_I}^{V_F} -P_{ext} \,dV  $$
    Dans le cas d'une transformation quasi-statique, la pression $P$ du système est définie à chaque instant et est égale, au niveau des parois mobiles, à la pression extérieure $P_{ext}$ à chaque instant. Le travail des forces de pression s'écrit alors :
    $$ \mathcal{W}_P = \int_{V_I}^{V_F} -P \,dV  $$ 

\end{adjustwidth}
\end{document}