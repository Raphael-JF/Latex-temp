\documentclass{article}
\usepackage{amsmath,amssymb,mathtools}
\usepackage{xcolor}
\usepackage{minted}
\usepackage{enumitem}
\usepackage{multicol}
\usepackage{changepage}
\usepackage{stmaryrd}
\usepackage{graphicx}
\graphicspath{ {./images/} }
\usepackage[framemethod=tikz]{mdframed}
\usepackage{tikz,pgfplots}
\pgfplotsset{compat=1.18}

% physique
\definecolor{oranges}{RGB}{255, 242, 230}
\definecolor{rouges}{RGB}{255, 230, 230}
\definecolor{rose}{RGB}{255, 204, 204}

% maths - info
\definecolor{rouge_fonce}{RGB}{204, 0, 0}
\definecolor{rouge}{RGB}{255, 0, 0}
\definecolor{bleu_fonce}{RGB}{0, 0, 255}
\definecolor{vert_fonce}{RGB}{0, 69, 33}
\definecolor{vert}{RGB}{0,255,0}

\definecolor{orange_foncee}{RGB}{255, 153, 0}
\definecolor{myrtille}{RGB}{225, 225, 255}
\definecolor{mayonnaise}{RGB}{255, 253, 233}
\definecolor{magenta}{RGB}{224, 209, 240}
\definecolor{pomme}{RGB}{204, 255, 204}
\definecolor{mauve}{RGB}{255, 230, 255}


% Cours

\newmdenv[
  nobreak=true,
  topline=true,
  bottomline=true,
  rightline=true,
  leftline=true,
  linewidth=0.5pt,
  linecolor=black,
  backgroundcolor=mayonnaise,
  innerleftmargin=10pt,
  innerrightmargin=10pt,
  innertopmargin=5pt,
  innerbottommargin=5pt,
  skipabove=\topsep,
  skipbelow=\topsep,
]{boite_definition}

\newmdenv[
  nobreak=false,
  topline=true,
  bottomline=true,
  rightline=true,
  leftline=true,
  linewidth=0.5pt,
  linecolor=white,
  backgroundcolor=white,
  innerleftmargin=10pt,
  innerrightmargin=10pt,
  innertopmargin=5pt,
  innerbottommargin=5pt,
  skipabove=\topsep,
  skipbelow=\topsep,
]{boite_exemple}

\newmdenv[
  nobreak=true,
  topline=true,
  bottomline=true,
  rightline=true,
  leftline=true,
  linewidth=0.5pt,
  linecolor=black,
  backgroundcolor=magenta,
  innerleftmargin=10pt,
  innerrightmargin=10pt,
  innertopmargin=5pt,
  innerbottommargin=5pt,
  skipabove=\topsep,
  skipbelow=\topsep,
]{boite_proposition}

\newmdenv[
  nobreak=true,
  topline=true,
  bottomline=true,
  rightline=true,
  leftline=true,
  linewidth=0.5pt,
  linecolor=black,
  backgroundcolor=white,
  innerleftmargin=10pt,
  innerrightmargin=10pt,
  innertopmargin=5pt,
  innerbottommargin=5pt,
  skipabove=\topsep,
  skipbelow=\topsep,
]{boite_demonstration}

\newmdenv[
  nobreak=true,
  topline=true,
  bottomline=true,
  rightline=true,
  leftline=true,
  linewidth=0.5pt,
  linecolor=white,
  backgroundcolor=white,
  innerleftmargin=10pt,
  innerrightmargin=10pt,
  innertopmargin=5pt,
  innerbottommargin=5pt,
  skipabove=\topsep,
  skipbelow=\topsep,
]{boite_remarque}


\newenvironment{definition}[2]
{
    \vspace{15pt}
    \begin{boite_definition}
    \textbf{\textcolor{rouge}{Définition #1}}
    \if\relax\detokenize{#2}\relax
    \else
        \textit{ - #2}
    \fi \\ \\
}
{
    \end{boite_definition}
    
}

\newenvironment{exemple}[2]
{
    \vspace{15pt}
    \begin{boite_exemple}
    \textbf{\textcolor{bleu_fonce}{Exemple #1}}
    \if\relax\detokenize{#2}\relax
    \else
        \textit{ - #2}
    \fi \\ \\ 
}
{   
    \end{boite_exemple}
    
}

\newenvironment{proposition}[2]
{
    \vspace{15pt}
    \begin{boite_proposition}
    \textbf{\textcolor{rouge}{Proposition #1}}
    \if\relax\detokenize{#2}\relax
    \else
        \textit{ - #2}
    \fi \\ \\
}
{
    \end{boite_proposition}
    
}

\newenvironment{theoreme}[2]
{
    \vspace{15pt}
    \begin{boite_proposition}
    \textbf{\textcolor{rouge}{Théorème #1}} 
    \if\relax\detokenize{#2}\relax
    \else
        \textit{ - #2}
    \fi \\ \\
}
{
    \end{boite_proposition}
    
}

\newenvironment{demonstration}
{
    \vspace{15pt}
    \begin{boite_demonstration}
    \textbf{\textcolor{rouge}{Démonstration}}\\ \\
}
{
    \end{boite_demonstration}
    
}

\newenvironment{remarque}[2]
{
    \vspace{15pt}
    \begin{boite_remarque}
    \textbf{\textcolor{bleu_fonce}{Remarque #1}}
    \if\relax\detokenize{#2}\relax
    \else
        \textit{ - #2}
    \fi \\ \\   
}
{  
    \end{boite_remarque}
    
}



%Corrections
\newmdenv[
  nobreak=true,
  topline=true,
  bottomline=true,
  rightline=true,
  leftline=true,
  linewidth=0.5pt,
  linecolor=black,
  backgroundcolor=mayonnaise,
  innerleftmargin=10pt,
  innerrightmargin=10pt,
  innertopmargin=5pt,
  innerbottommargin=5pt,
  skipabove=\topsep,
  skipbelow=\topsep,
]{boite_question}


\newenvironment{question}[2]
{
    \vspace{15pt}
    \begin{boite_question}
    \textbf{\textcolor{rouge}{Question #1}}
    \if\relax\detokenize{#2}\relax
    \else
        \textit{ - #2}
    \fi \\ \\
}
{
    \end{boite_question}
    
}

\newenvironment{enumeratebf}{
    \begin{enumerate}[label=\textbf{\arabic*.}]
}
{
    \end{enumerate}
}

\begin{document}
\begin{adjustwidth}{-3cm}{-3cm}
\begin{document}
\begin{adjustwidth}{-3cm}{-3cm}
% commandes
\newcommand{\notion}[1]{\textcolor{vert_fonce}{\textit{#1}}}
\newcommand{\mb}[1]{\mathbb{#1}}
\newcommand{\mc}[1]{\mathcal{#1}}
\newcommand{\mr}[1]{\mathrm{#1}}
\newcommand{\code}[1]{\texttt{#1}}
\newcommand{\ccode}[1]{\texttt{|#1|}}
\newcommand{\ov}[1]{\overline{#1}}
\newcommand{\abs}[1]{|#1|}
\newcommand{\rev}[1]{\texttt{reverse(#1)}}
\newcommand{\crev}[1]{\texttt{|reverse(#1)|}}

\newcommand{\ie}{\textit{i.e.} }

\newcommand{\N}{\mathbb{N}}
\newcommand{\R}{\mathbb{R}}
\newcommand{\C}{\mathbb{C}}
\newcommand{\K}{\mathbb{K}}
\newcommand{\Z}{\mathbb{Z}}

\newcommand{\A}{\mathcal{A}}
\newcommand{\bigO}{\mathcal{O}}
\renewcommand{\L}{\mathcal{L}}

\newcommand{\rg}[0]{\mathrm{rg}}
\newcommand{\re}[0]{\mathrm{Re}}
\newcommand{\im}[0]{\mathrm{Im}}
\newcommand{\cl}[0]{\mathrm{cl}}
\newcommand{\grad}[0]{\vec{\mathrm{grad}}}
\renewcommand{\div}[0]{\mathrm{div}\,}
\newcommand{\rot}[0]{\vec{\mathrm{rot}}\,}
\newcommand{\vnabla}[0]{\vec{\nabla}}
\renewcommand{\vec}[1]{\overrightarrow{#1}}
\newcommand{\mat}[1]{\mathrm{Mat}_{#1}}
\newcommand{\matrice}[1]{\mathcal{M}_{#1}}
\newcommand{\sgEngendre}[1]{\left\langle #1 \right\rangle}
\newcommand{\gpquotient}[1]{\mathbb{Z} / #1\mathbb{Z}}
\newcommand{\norme}[1]{||#1||}
\renewcommand{\d}[1]{\,\mathrm{d}#1}
\newcommand{\adh}[1]{\overline{#1}}
\newcommand{\intint}[2]{\llbracket #1 ,\, #2 \rrbracket}
\newcommand{\seg}[2]{[#1\, ; \, #2]}
\newcommand{\scal}[2]{( #1 | #2 )}
\newcommand{\distance}[2]{\mathrm{d}(#1,\,#2)}
\newcommand{\inte}[2]{\int_{#1}^{#2}}
\newcommand{\somme}[2]{\sum_{#1}^{#2}}
\newcommand{\deriveref}[4]{\biggl( \frac{\text{d}^{#1}#2}{\text{d}#3^{#1}} \biggr)_{#4}}






\begin{theoreme}{14.50 (0)}{de la limite monotone}
    Soit $(u_n)_{n \in \N}$ une suite réelle.
    \begin{enumeratebf}
        \item si $(u_n)_{n \in \N}$ est croissante et majorée, alors :
        $$\begin{cases*}
            \lim\limits_{n \to +\infty} u_n = \sup\limits_{n\in \N}(u_n) \\
            \forall n \in \N,\, u_n \leq \sup\limits_{n\in \N}(u_n)
        \end{cases*}$$
        \item si $(u_n)_{n \in \N}$ est strictement croissante et majorée, alors :
        $$\begin{cases*}
            \lim\limits_{n \to +\infty} u_n = \sup\limits_{n\in \N}(u_n) \\
            \forall n \in \N,\, u_n < \sup\limits_{n\in \N}(u_n)
        \end{cases*}$$
        \item si $(u_n)_{n \in \N}$ est décroissante et minorée, alors :
        $$\begin{cases*}
            \lim\limits_{n \to +\infty} u_n = \inf\limits_{n\in \N}(u_n) \\
            \forall n \in \N,\, u_n \geq \inf\limits_{n\in \N}(u_n)
        \end{cases*}$$
        \item si $(u_n)_{n \in \N}$ est décroissante et minorée, alors :
        $$\begin{cases*}
            \lim\limits_{n \to +\infty} u_n = \inf\limits_{n\in \N}(u_n) \\
            \forall n \in \N,\, u_n > \inf\limits_{n\in \N}(u_n)
        \end{cases*}$$
        
    \end{enumeratebf}
\end{theoreme}

\begin{proposition}{14.50 (1)}{caractérisation de la convergence d'une suite monotone}
    Une suite monotone est convergente si et seulement si elle est bornée.
\end{proposition}

\begin{proposition}{14.50 (2)}{limites de suites croissante non majorée, décroissante non minorée}
    Une suite croissante et non majorée diverge vers $+ \infty$.\\
    De même, Une suite décroissante et non minorée diverge vers $- \infty$.
\end{proposition}

\begin{theoreme}{14.65}{monotonie d'une suite récurrente définie par $u_{n+1} = f(u_n)$}
    Soit $D \subset \mathbb{R}$, $u_0 \in D$, $f:D \rightarrow D$ une fonction et $(u_n) \in D^{\mathbb{N}}$ l'unique suite définie par la relation $u_{n+1} = f(u_n)$.
        \begin{enumeratebf}
            \item Le signe de $x \mapsto f(x) - x$ renseigne sur la monotonie de $(u_n)$ :\begin{equation*}\begin{cases}
                \forall x \in D, f(x) \geq x \implies \forall n \in \mathbb{N}, u_{n+1} \geq u_n \\
                \forall x \in D, f(x) \leq x \implies \forall n \in \mathbb{N}, u_{n+1} \leq u_n 
                \end{cases}\end{equation*}
    
            \item  Si $f$ est croissante, alors $(u_n)$ est :
            \begin{itemize}
                \item croissante si $u_1 \geq u_0$
                \item décroissante si $u_1 \leq u_0$
            \end{itemize}
    
            \item si $f$ est décroissante, alors $(u_{2n})$ et $(u_{2n+1})$ sont monotones et de sens contraire :
            \begin{itemize}
                \item si $u_2 \geq u_0$ alors $(u_{2n})$ est croissante et $(u_{2n+1})$ est décroissante
                \item si $u_2 \leq u_0$ alors $(u_{2n})$ est décroissante et $(u_{2n+1})$ est croissante
            \end{itemize}
        \end{enumeratebf}
    \end{theoreme}

\begin{theoreme}{14.66}{du point fixe}
    Soit $D \subset \mathbb{R}$, $u_0 \in D$, $f:D \rightarrow D$ une fonction et $(u_n) \in D^{\mathbb{N}}$ l'unique suite définie par la relation $u_{n+1} = f(u_n)$. \\
    Si $\lim\limits_{n \in \N} u_n = \ell \in D$ et si $f$ est continue en $\ell$, alors $f(\ell) = \ell$.
\end{theoreme}

Soit u € K" une suite telle que uo = a, u1 = B et pour tout entier naturel n, Un+2 + aUn+1 + bun = 0, avec
b + 0.
1. Si x2 + ax + b = 0 admet une solution double r E K, alors 30, p) € K2, Vn € N. Un = Ar.n + unr".
2. Si x2 + ax + b = 0 admet deux solutions distinctes r1 E K et T2 E K, alors
3(X, p) € K', Vn E N, Un = ArR ur?.
3. De plus, si K = R et (a,b) E R2 (autrement dit si u est une suite récurrente linéaire double réelle), et S?
12 + ax + b = 0 admet deux solutions complexes conjuguées ri et T2 (c'est-à-dire lorsque a2 - 46 < 0), alors
r1 = peio avec p e R- et 0 E R, et : 3(a, b) € R2, Vn E N, un = p' (A cos(n6) + B sin(nt)).
\begin{theoreme}{14.bonus}{suites récurrentes linéaires  du deuxième ordre}
    Soit $u \in \K^\N$ une suite telle que pour tout $n \in \N,\,  u_{n+2} + au_{n+1} + bu_{n} = 0$ avec $(a,b) \in \K \times \K^*$. On note $P = X^2 + aX + b$ le polynôme caractéristique associé à $(u_n)_{n \in \N}$
    \begin{enumeratebf}
        \item Si $P$ admet une racine double $r \in \K$, alors~: 
        $$\exists (\alpha, \beta) \in \K^2,\, \forall n \in \N ,\,u_n = \alpha r^n + \beta n r^n$$
        \item Si $P$ admet deux solutions distinctes $r_1$ et $r_2$ dans $\K$, alors~: 
        $$\exists (\alpha, \beta) \in \K^2,\, \forall n \in \N ,\,u_n = \alpha r_1^n + \beta r_2^n$$
        \item Si $\K = \R$ et $P$ admet deux solutions complexes conjuguées $r_1 = \rho^{i \theta}$ et $r_2 = \rho^{-i \theta}$ dans $\K$, alors~: 
        $$\exists (\alpha, \beta) \in \K^2,\, \forall n \in \N ,\,u_n = \rho(\alpha \cos(n \theta) + \beta \sin(n \theta))$$
    \end{enumeratebf}
    

\end{theoreme}

\end{adjustwidth}
\end{document}