\documentclass{article}
\usepackage{amsmath,amssymb,mathtools}
\usepackage{xcolor}
\usepackage{minted}
\usepackage{enumitem}
\usepackage{multicol}
\usepackage{changepage}
\usepackage{stmaryrd}
\usepackage{graphicx}
\graphicspath{ {./images/} }
\usepackage[framemethod=tikz]{mdframed}
\usepackage{tikz,pgfplots}
\pgfplotsset{compat=1.18}

% physique
\definecolor{oranges}{RGB}{255, 242, 230}
\definecolor{rouges}{RGB}{255, 230, 230}
\definecolor{rose}{RGB}{255, 204, 204}

% maths - info
\definecolor{rouge_fonce}{RGB}{204, 0, 0}
\definecolor{rouge}{RGB}{255, 0, 0}
\definecolor{bleu_fonce}{RGB}{0, 0, 255}
\definecolor{vert_fonce}{RGB}{0, 69, 33}
\definecolor{vert}{RGB}{0,255,0}

\definecolor{orange_foncee}{RGB}{255, 153, 0}
\definecolor{myrtille}{RGB}{225, 225, 255}
\definecolor{mayonnaise}{RGB}{255, 253, 233}
\definecolor{magenta}{RGB}{224, 209, 240}
\definecolor{pomme}{RGB}{204, 255, 204}
\definecolor{mauve}{RGB}{255, 230, 255}


% Cours

\newmdenv[
  nobreak=true,
  topline=true,
  bottomline=true,
  rightline=true,
  leftline=true,
  linewidth=0.5pt,
  linecolor=black,
  backgroundcolor=mayonnaise,
  innerleftmargin=10pt,
  innerrightmargin=10pt,
  innertopmargin=5pt,
  innerbottommargin=5pt,
  skipabove=\topsep,
  skipbelow=\topsep,
]{boite_definition}

\newmdenv[
  nobreak=false,
  topline=true,
  bottomline=true,
  rightline=true,
  leftline=true,
  linewidth=0.5pt,
  linecolor=white,
  backgroundcolor=white,
  innerleftmargin=10pt,
  innerrightmargin=10pt,
  innertopmargin=5pt,
  innerbottommargin=5pt,
  skipabove=\topsep,
  skipbelow=\topsep,
]{boite_exemple}

\newmdenv[
  nobreak=true,
  topline=true,
  bottomline=true,
  rightline=true,
  leftline=true,
  linewidth=0.5pt,
  linecolor=black,
  backgroundcolor=magenta,
  innerleftmargin=10pt,
  innerrightmargin=10pt,
  innertopmargin=5pt,
  innerbottommargin=5pt,
  skipabove=\topsep,
  skipbelow=\topsep,
]{boite_proposition}

\newmdenv[
  nobreak=true,
  topline=true,
  bottomline=true,
  rightline=true,
  leftline=true,
  linewidth=0.5pt,
  linecolor=black,
  backgroundcolor=white,
  innerleftmargin=10pt,
  innerrightmargin=10pt,
  innertopmargin=5pt,
  innerbottommargin=5pt,
  skipabove=\topsep,
  skipbelow=\topsep,
]{boite_demonstration}

\newmdenv[
  nobreak=true,
  topline=true,
  bottomline=true,
  rightline=true,
  leftline=true,
  linewidth=0.5pt,
  linecolor=white,
  backgroundcolor=white,
  innerleftmargin=10pt,
  innerrightmargin=10pt,
  innertopmargin=5pt,
  innerbottommargin=5pt,
  skipabove=\topsep,
  skipbelow=\topsep,
]{boite_remarque}


\newenvironment{definition}[2]
{
    \vspace{15pt}
    \begin{boite_definition}
    \textbf{\textcolor{rouge}{Définition #1}}
    \if\relax\detokenize{#2}\relax
    \else
        \textit{ - #2}
    \fi \\ \\
}
{
    \end{boite_definition}
    
}

\newenvironment{exemple}[2]
{
    \vspace{15pt}
    \begin{boite_exemple}
    \textbf{\textcolor{bleu_fonce}{Exemple #1}}
    \if\relax\detokenize{#2}\relax
    \else
        \textit{ - #2}
    \fi \\ \\ 
}
{   
    \end{boite_exemple}
    
}

\newenvironment{proposition}[2]
{
    \vspace{15pt}
    \begin{boite_proposition}
    \textbf{\textcolor{rouge}{Proposition #1}}
    \if\relax\detokenize{#2}\relax
    \else
        \textit{ - #2}
    \fi \\ \\
}
{
    \end{boite_proposition}
    
}

\newenvironment{theoreme}[2]
{
    \vspace{15pt}
    \begin{boite_proposition}
    \textbf{\textcolor{rouge}{Théorème #1}} 
    \if\relax\detokenize{#2}\relax
    \else
        \textit{ - #2}
    \fi \\ \\
}
{
    \end{boite_proposition}
    
}

\newenvironment{demonstration}
{
    \vspace{15pt}
    \begin{boite_demonstration}
    \textbf{\textcolor{rouge}{Démonstration}}\\ \\
}
{
    \end{boite_demonstration}
    
}

\newenvironment{remarque}[2]
{
    \vspace{15pt}
    \begin{boite_remarque}
    \textbf{\textcolor{bleu_fonce}{Remarque #1}}
    \if\relax\detokenize{#2}\relax
    \else
        \textit{ - #2}
    \fi \\ \\   
}
{  
    \end{boite_remarque}
    
}



%Corrections
\newmdenv[
  nobreak=true,
  topline=true,
  bottomline=true,
  rightline=true,
  leftline=true,
  linewidth=0.5pt,
  linecolor=black,
  backgroundcolor=mayonnaise,
  innerleftmargin=10pt,
  innerrightmargin=10pt,
  innertopmargin=5pt,
  innerbottommargin=5pt,
  skipabove=\topsep,
  skipbelow=\topsep,
]{boite_question}


\newenvironment{question}[2]
{
    \vspace{15pt}
    \begin{boite_question}
    \textbf{\textcolor{rouge}{Question #1}}
    \if\relax\detokenize{#2}\relax
    \else
        \textit{ - #2}
    \fi \\ \\
}
{
    \end{boite_question}
    
}

\newenvironment{enumeratebf}{
    \begin{enumerate}[label=\textbf{\arabic*.}]
}
{
    \end{enumerate}
}

\begin{document}
\begin{adjustwidth}{-3cm}{-3cm}
\begin{document}
\begin{adjustwidth}{-3cm}{-3cm}
% commandes
\newcommand{\notion}[1]{\textcolor{vert_fonce}{\textit{#1}}}
\newcommand{\mb}[1]{\mathbb{#1}}
\newcommand{\mc}[1]{\mathcal{#1}}
\newcommand{\mr}[1]{\mathrm{#1}}
\newcommand{\code}[1]{\texttt{#1}}
\newcommand{\ccode}[1]{\texttt{|#1|}}
\newcommand{\ov}[1]{\overline{#1}}
\newcommand{\abs}[1]{|#1|}
\newcommand{\rev}[1]{\texttt{reverse(#1)}}
\newcommand{\crev}[1]{\texttt{|reverse(#1)|}}

\newcommand{\ie}{\textit{i.e.} }

\newcommand{\N}{\mathbb{N}}
\newcommand{\R}{\mathbb{R}}
\newcommand{\C}{\mathbb{C}}
\newcommand{\K}{\mathbb{K}}
\newcommand{\Z}{\mathbb{Z}}

\newcommand{\A}{\mathcal{A}}
\newcommand{\bigO}{\mathcal{O}}
\renewcommand{\L}{\mathcal{L}}

\newcommand{\rg}[0]{\mathrm{rg}}
\newcommand{\re}[0]{\mathrm{Re}}
\newcommand{\im}[0]{\mathrm{Im}}
\newcommand{\cl}[0]{\mathrm{cl}}
\newcommand{\grad}[0]{\vec{\mathrm{grad}}}
\renewcommand{\div}[0]{\mathrm{div}\,}
\newcommand{\rot}[0]{\vec{\mathrm{rot}}\,}
\newcommand{\vnabla}[0]{\vec{\nabla}}
\renewcommand{\vec}[1]{\overrightarrow{#1}}
\newcommand{\mat}[1]{\mathrm{Mat}_{#1}}
\newcommand{\matrice}[1]{\mathcal{M}_{#1}}
\newcommand{\sgEngendre}[1]{\left\langle #1 \right\rangle}
\newcommand{\gpquotient}[1]{\mathbb{Z} / #1\mathbb{Z}}
\newcommand{\norme}[1]{||#1||}
\renewcommand{\d}[1]{\,\mathrm{d}#1}
\newcommand{\adh}[1]{\overline{#1}}
\newcommand{\intint}[2]{\llbracket #1 ,\, #2 \rrbracket}
\newcommand{\seg}[2]{[#1\, ; \, #2]}
\newcommand{\scal}[2]{( #1 | #2 )}
\newcommand{\distance}[2]{\mathrm{d}(#1,\,#2)}
\newcommand{\inte}[2]{\int_{#1}^{#2}}
\newcommand{\somme}[2]{\sum_{#1}^{#2}}
\newcommand{\deriveref}[4]{\biggl( \frac{\text{d}^{#1}#2}{\text{d}#3^{#1}} \biggr)_{#4}}






\begin{definition}{2.1}{automate fini déterministe}
    Un automate fini déterministe ${\mc{A}}$ est défini par un quintuplet $(\Sigma, Q, q_0, F, \delta)$, où~:
    \begin{enumeratebf}
        \item $\Sigma$ est un alphabet fini ;
        \item $Q$ est un \notion{ensemble fini d'états} ;
        \item $q_0 \in Q$ est l'\notion{état initial} ;
        \item $\mc{F} \subset Q$ est un \notion{ensemble d'états finaux} ;
        \item $\delta$ est une application d'une partie de $Q \times \Sigma$ dans $Q$ est la \notion{fonction de transition}.
    \end{enumeratebf}
    Il est commun de représenter par un tableau à double entrées, dit \notion{table de transition}, les valeurs prises par $\delta$.
\end{definition}

\begin{definition}{2.2}{chemin, étiquette d'un chemin}
    Un \notion{chemin} dans un automate est une suite finie d'états $(q_1,\, \dots,q_n)$ telle qu'il existe $a_1,\, \dots,\, a_{n-1}$ dans $\Sigma$, tel que~:
    $$\forall i \in \intint{1}{n},\, \delta(q_i,\, a_i) = q_{i+1}$$ 
    Assertion que l'on notera~:
    $$q_1 \xrightarrow{a_1} q_2 \xrightarrow{a_2} \dots \xrightarrow{a_{n-1}} q_n$$
    L'\notion{étiquette du chemin} est alors le mot $a_1 \dots a_{n-1}$.
\end{definition}

\begin{definition}{2.3}{chemin acceptant}
    Un chemin $(q_1,\, \dots,q_n)$ d'un automate $\mc{A}$ est \notion{acceptant} lorsque $q_1$ est l'état initial (ou un état initial si AFND) de $\mc{A}$ et $q_n$ est un état final de $\mc{A}$.\\
    On dit alors que l'étiquette de $(q_1,\, \dots,q_n)$, qui est un mot, \notion{est reconnue par $\mc{A}$}.
\end{definition}

\begin{definition}{2.4}{langage reconnu}
    L'ensemble des mots reconnus par $\mc{A}$ un automate fini est noté $\mc{L}(\mc{A})$ et est appelé \notion{langage reconnu par $\mc{A}$}.
\end{definition}

\begin{definition}{2.5}{état accessible}
    Un état $q$ d'un automate fini $\mc{A}$ est dit accessible lorsqu'il existe un chemin de $\mc{A}$ de l'état initial (ou d'un état initial) de $\mc{A}$ menant à $q$.
\end{definition}

\begin{definition}{2.6}{état co-accessible}
    Un état $q$ d'un automate fini $\mc{A}$ est dit co-accessible lorsqu'il existe un chemin de $\mc{A}$ $q$ vers un état final de $\mc{A}$.
\end{definition}

\begin{definition}{2.7}{état utile}
    Un état d'un automate fini est dit utile s'il est accessible et co-accessible.
\end{definition}

\begin{definition}{2.8}{automate fini émondé}
    La présence d'états non utiles (non accessibles ou non co-accessibles) n'altère pas le langage reconnu par un automate fini $\mc{A}$. \\
    On dit alors qu'un automate $\mc{A}'$ est \notion{émondé} s'il ne contient que des états utiles.
\end{definition}

\begin{definition}{2.9}{automate des parties d'un AFND}
    Soit $\mc{A}_{\text{ND}} = (\Sigma, Q, I, F, \delta)$ un automate fini non déterministe. On appelle \notion{automate des parties de $\mc{A}_{\text{ND}}$},\, noté $\mc{A}_{\text{D}} = (\Sigma, Q_{\text{D}}, q_{0,\text{D}}, F_{\text{D}}, \delta_{\text{D}})$ tel que~:
    \begin{enumeratebf}
        \item $Q_{\text{D}} = \mc{P}(Q)$ ;
        \item $q_{0,\text{D}} = I$ ;
        \item $F_{\text{D}} = \{P \in \Q_{\text{D}},\, P\cap F \neq \varnothing\}$, l'ensemble des états de $\mc{A}_{\text{D}}$ contenant au moins un état final de $\mc{A}_{\text{ND}}$.
        \item $ \displaystyle \delta_{\text{D}} : \begin{array}{rcl}
            Q_{\text{D}} \times \Sigma &  \to & Q_{\text{D}} \\
            (P,a) & \mapsto & \{q \in Q,\, \exists p \in P,\, q \in \delta(p,a)\} = \bigcup_{p \in P} \{q\in Q,\, \delta(p,a) = q\}
            \end{array}$
    \end{enumeratebf}
    Pour $P \in Q_{\text{D}}$ et $a \in \Sigma$, $\delta_{\text{D}}(P,a)$ est l'ensemble des états de $Q$ accessibles en lisant $a$ depuis un élément de $P$.\\
    $\mc{A}_{\text{D}}$ est alors un automate fini déterministe.

\end{definition}


\end{adjustwidth}
\end{document}