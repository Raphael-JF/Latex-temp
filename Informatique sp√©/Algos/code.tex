\documentclass{article}
\usepackage{amsmath,amssymb,mathtools}
\usepackage{xcolor}
\usepackage{minted}
\usepackage{enumitem}
\usepackage{multicol}
\usepackage{changepage}
\usepackage{stmaryrd}
\usepackage{graphicx}
\graphicspath{ {./images/} }
\usepackage[framemethod=tikz]{mdframed}
\usepackage{tikz,pgfplots}
\pgfplotsset{compat=1.18}

% physique
\definecolor{oranges}{RGB}{255, 242, 230}
\definecolor{rouges}{RGB}{255, 230, 230}
\definecolor{rose}{RGB}{255, 204, 204}

% maths - info
\definecolor{rouge_fonce}{RGB}{204, 0, 0}
\definecolor{rouge}{RGB}{255, 0, 0}
\definecolor{bleu_fonce}{RGB}{0, 0, 255}
\definecolor{vert_fonce}{RGB}{0, 69, 33}
\definecolor{vert}{RGB}{0,255,0}

\definecolor{orange_foncee}{RGB}{255, 153, 0}
\definecolor{myrtille}{RGB}{225, 225, 255}
\definecolor{mayonnaise}{RGB}{255, 253, 233}
\definecolor{magenta}{RGB}{224, 209, 240}
\definecolor{pomme}{RGB}{204, 255, 204}
\definecolor{mauve}{RGB}{255, 230, 255}


% Cours

\newmdenv[
  nobreak=true,
  topline=true,
  bottomline=true,
  rightline=true,
  leftline=true,
  linewidth=0.5pt,
  linecolor=black,
  backgroundcolor=mayonnaise,
  innerleftmargin=10pt,
  innerrightmargin=10pt,
  innertopmargin=5pt,
  innerbottommargin=5pt,
  skipabove=\topsep,
  skipbelow=\topsep,
]{boite_definition}

\newmdenv[
  nobreak=false,
  topline=true,
  bottomline=true,
  rightline=true,
  leftline=true,
  linewidth=0.5pt,
  linecolor=white,
  backgroundcolor=white,
  innerleftmargin=10pt,
  innerrightmargin=10pt,
  innertopmargin=5pt,
  innerbottommargin=5pt,
  skipabove=\topsep,
  skipbelow=\topsep,
]{boite_exemple}

\newmdenv[
  nobreak=true,
  topline=true,
  bottomline=true,
  rightline=true,
  leftline=true,
  linewidth=0.5pt,
  linecolor=black,
  backgroundcolor=magenta,
  innerleftmargin=10pt,
  innerrightmargin=10pt,
  innertopmargin=5pt,
  innerbottommargin=5pt,
  skipabove=\topsep,
  skipbelow=\topsep,
]{boite_proposition}

\newmdenv[
  nobreak=true,
  topline=true,
  bottomline=true,
  rightline=true,
  leftline=true,
  linewidth=0.5pt,
  linecolor=black,
  backgroundcolor=white,
  innerleftmargin=10pt,
  innerrightmargin=10pt,
  innertopmargin=5pt,
  innerbottommargin=5pt,
  skipabove=\topsep,
  skipbelow=\topsep,
]{boite_demonstration}

\newmdenv[
  nobreak=true,
  topline=true,
  bottomline=true,
  rightline=true,
  leftline=true,
  linewidth=0.5pt,
  linecolor=white,
  backgroundcolor=white,
  innerleftmargin=10pt,
  innerrightmargin=10pt,
  innertopmargin=5pt,
  innerbottommargin=5pt,
  skipabove=\topsep,
  skipbelow=\topsep,
]{boite_remarque}


\newenvironment{definition}[2]
{
    \vspace{15pt}
    \begin{boite_definition}
    \textbf{\textcolor{rouge}{Définition #1}}
    \if\relax\detokenize{#2}\relax
    \else
        \textit{ - #2}
    \fi \\ \\
}
{
    \end{boite_definition}
    
}

\newenvironment{exemple}[2]
{
    \vspace{15pt}
    \begin{boite_exemple}
    \textbf{\textcolor{bleu_fonce}{Exemple #1}}
    \if\relax\detokenize{#2}\relax
    \else
        \textit{ - #2}
    \fi \\ \\ 
}
{   
    \end{boite_exemple}
    
}

\newenvironment{proposition}[2]
{
    \vspace{15pt}
    \begin{boite_proposition}
    \textbf{\textcolor{rouge}{Proposition #1}}
    \if\relax\detokenize{#2}\relax
    \else
        \textit{ - #2}
    \fi \\ \\
}
{
    \end{boite_proposition}
    
}

\newenvironment{theoreme}[2]
{
    \vspace{15pt}
    \begin{boite_proposition}
    \textbf{\textcolor{rouge}{Théorème #1}} 
    \if\relax\detokenize{#2}\relax
    \else
        \textit{ - #2}
    \fi \\ \\
}
{
    \end{boite_proposition}
    
}

\newenvironment{demonstration}
{
    \vspace{15pt}
    \begin{boite_demonstration}
    \textbf{\textcolor{rouge}{Démonstration}}\\ \\
}
{
    \end{boite_demonstration}
    
}

\newenvironment{remarque}[2]
{
    \vspace{15pt}
    \begin{boite_remarque}
    \textbf{\textcolor{bleu_fonce}{Remarque #1}}
    \if\relax\detokenize{#2}\relax
    \else
        \textit{ - #2}
    \fi \\ \\   
}
{  
    \end{boite_remarque}
    
}



%Corrections
\newmdenv[
  nobreak=true,
  topline=true,
  bottomline=true,
  rightline=true,
  leftline=true,
  linewidth=0.5pt,
  linecolor=black,
  backgroundcolor=mayonnaise,
  innerleftmargin=10pt,
  innerrightmargin=10pt,
  innertopmargin=5pt,
  innerbottommargin=5pt,
  skipabove=\topsep,
  skipbelow=\topsep,
]{boite_question}


\newenvironment{question}[2]
{
    \vspace{15pt}
    \begin{boite_question}
    \textbf{\textcolor{rouge}{Question #1}}
    \if\relax\detokenize{#2}\relax
    \else
        \textit{ - #2}
    \fi \\ \\
}
{
    \end{boite_question}
    
}

\newenvironment{enumeratebf}{
    \begin{enumerate}[label=\textbf{\arabic*.}]
}
{
    \end{enumerate}
}

\begin{document}
\begin{adjustwidth}{-3cm}{-3cm}
\begin{document}
\begin{adjustwidth}{-3cm}{-3cm}
% commandes
\newcommand{\notion}[1]{\textcolor{vert_fonce}{\textit{#1}}}
\newcommand{\mb}[1]{\mathbb{#1}}
\newcommand{\mc}[1]{\mathcal{#1}}
\newcommand{\mr}[1]{\mathrm{#1}}
\newcommand{\code}[1]{\texttt{#1}}
\newcommand{\ccode}[1]{\texttt{|#1|}}
\newcommand{\ov}[1]{\overline{#1}}
\newcommand{\abs}[1]{|#1|}
\newcommand{\rev}[1]{\texttt{reverse(#1)}}
\newcommand{\crev}[1]{\texttt{|reverse(#1)|}}

\newcommand{\ie}{\textit{i.e.} }

\newcommand{\N}{\mathbb{N}}
\newcommand{\R}{\mathbb{R}}
\newcommand{\C}{\mathbb{C}}
\newcommand{\K}{\mathbb{K}}
\newcommand{\Z}{\mathbb{Z}}

\newcommand{\A}{\mathcal{A}}
\newcommand{\bigO}{\mathcal{O}}
\renewcommand{\L}{\mathcal{L}}

\newcommand{\rg}[0]{\mathrm{rg}}
\newcommand{\re}[0]{\mathrm{Re}}
\newcommand{\im}[0]{\mathrm{Im}}
\newcommand{\cl}[0]{\mathrm{cl}}
\newcommand{\grad}[0]{\vec{\mathrm{grad}}}
\renewcommand{\div}[0]{\mathrm{div}\,}
\newcommand{\rot}[0]{\vec{\mathrm{rot}}\,}
\newcommand{\vnabla}[0]{\vec{\nabla}}
\renewcommand{\vec}[1]{\overrightarrow{#1}}
\newcommand{\mat}[1]{\mathrm{Mat}_{#1}}
\newcommand{\matrice}[1]{\mathcal{M}_{#1}}
\newcommand{\sgEngendre}[1]{\left\langle #1 \right\rangle}
\newcommand{\gpquotient}[1]{\mathbb{Z} / #1\mathbb{Z}}
\newcommand{\norme}[1]{||#1||}
\renewcommand{\d}[1]{\,\mathrm{d}#1}
\newcommand{\adh}[1]{\overline{#1}}
\newcommand{\intint}[2]{\llbracket #1 ,\, #2 \rrbracket}
\newcommand{\seg}[2]{[#1\, ; \, #2]}
\newcommand{\scal}[2]{( #1 | #2 )}
\newcommand{\distance}[2]{\mathrm{d}(#1,\,#2)}
\newcommand{\inte}[2]{\int_{#1}^{#2}}
\newcommand{\somme}[2]{\sum_{#1}^{#2}}
\newcommand{\deriveref}[4]{\biggl( \frac{\text{d}^{#1}#2}{\text{d}#3^{#1}} \biggr)_{#4}}






\begin{implementation}{tri par fusion}
    \begin{lstOCaml}
    let rec casser l =
        match l with
        | [] -> [], []
        | [e1] -> [e1], []
        | e1::e2::q -> 
            let l1, l2 = casser q in
            e1::l1, e2::l2

    let rec fusion l1 l2 = 
        match l1, l2 with
        | [], _ -> l2
        | _, [] -> l1
        | e1::q1, e2::q2 ->
            if e2 > e1 then
                e1::(fusion q1 l2)
            else
                e2::(fusion l1 q2)

    let rec tri_fusion l =
        match l with
        | [] -> []
        | [e1] -> [e1]
        | _ -> 
            let l1, l2 = casser l in
            fusion (tri_fusion l1) (tri_fusion l2)
    \end{lstOCaml}
\end{implementation}

\begin{implementation}{parcours en largeur d'un graphe (1/3)}
    \begin{lstOCaml}
    type file = {e:int list; s:int list}
    type graphe = int list array 
        
    let file_vide = {e=[]; s=[]}             
    
    let rec ajoute f liste = match liste with
        | [] -> f
        | elt::q -> ajoute {e=(elt::f.e); s=f.s} q 
    \end{lstOCaml}
\end{implementation}

\begin{implementation}{parcours en largeur d'un graphe (2/3)}
    \begin{lstOCaml}
    let pop_opt f = 
        let rec retourne sub_f =
            match sub_f.e with
            | [] -> sub_f
            | elt::q -> retourne {e=q; s=elt::sub_f.s}   
        in let new_f = 
            if f.s = [] then
                retourne f 
            else f 
        in match new_f.s with 
        | [] -> file_vide, None
        | elt::q -> {e=new_f.e; s=q}, Some elt
    \end{lstOCaml}
\end{implementation}

\begin{implementation}{parcours en largeur d'un graphe (3/3)}
    \begin{lstOCaml}
    let parcours_largeur g s =
        let n = Array.length g in
        let non_vus = Array.make n true in
        let rec parcours f =
            match (pop_opt f) with
            | _, None -> ()
            | new_f, Some v when non_vus.(v) ->
                non_vus.(v) <- false;
                print_int v;
                parcours (ajoute new_f g.(v))
            | new_f, Some v ->
                parcours new_f
        in parcours {e=[]; s=[s]} 
    \end{lstOCaml}
\end{implementation}

\begin{implementation}{liste chainée en C (1/3)}
    \begin{lstC}
    typedef int elemtype;

    struct Maillon{
        elemtype val;
        struct Maillon* suivant;
    };
    typedef struct Maillon maillon;

    \end{lstC}
\end{implementation}

\begin{implementation}{liste chainée en C (2/3)}
    \begin{lstC}
    maillon* ajoute(elemtype x, maillon* c){
        maillon* res = malloc(sizeof(maillon));
        assert(res != NULL);
        res->val = x;
        res->suivant = c;
        return res;
    };
    \end{lstC}
\end{implementation}

\begin{implementation}{liste chainée en C (3/3)}
    \begin{lstC}
    int main(){
        maillon* a = ajoute(1,NULL);
        a = ajoute(2,a);
        a = ajoute(3,a);
        return 0;
    };
    \end{lstC}
\end{implementation}

\begin{implementation}{file d'entiers}
    \begin{lstC}
    struct Maillon{
        int val;
        struct Maillon* suivant;
    };
    typedef struct Maillon maillon;

    struct File{
        maillon* e; //maillon d'entrée
        maillon* s; //maillon de sortie
    };
    typedef struct File file;

    file* file_vide(){
        file* res = malloc(sizeof(file));
        assert(res != NULL);
        res->e = NULL;
        res->s = NULL;
        return res;
    }
    \end{lstC}
\end{implementation}

\begin{implementation}{file de priorité : \texttt{vide} et \texttt{ajoute} (1/2) }
    \begin{lstOCaml}
    type tas_binaire_min = {
        mutable nb_elts: int; 
        mutable data: (char*int) array
    }

    let tbmin_vide () = {nb_elts = 0; data = [||]}

    let prio couple = 
        let _,b = couple in b

    let redim tbmin new_taille cur_taille = 
        assert (new_taille >= tbmin.nb_elts);
        let new_data = Array.make new_taille ('\000',0) in
        for i=0 to tbmin.nb_elts - 1 do
            new_data.(i) <- tbmin.data.(i)
        done;
        tbmin.data <- new_data
    \end{lstOCaml}
\end{implementation}

\begin{implementation}{file de priorité : \texttt{vide} et \texttt{ajoute} (2/2)}
    \begin{lstOCaml}
    let tbmin_ajoute tbmin x p  =
        (*redimensionnement*)
        let n = Array.length tbmin.data in
        if tbmin.nb_elts >= n then 
            redim tbmin (2*n+1) n;

        (*ajout et percolations vers le haut*)
        tbmin.data.(tbmin.nb_elts) <- (x,p);
        tbmin.nb_elts <- tbmin.nb_elts + 1;
        let rec percole_haut i =
            let daron = if (i-1)/2 < 0 then 0 else (i-1)/2 in
            if prio tbmin.data.(daron) > prio tbmin.data.(i) then begin
                let temp = tbmin.data.(i) in
                tbmin.data.(i) <- tbmin.data.(daron);
                tbmin.data.(daron) <- temp;
                percole_haut daron;
            end;
        in percole_haut (tbmin.nb_elts - 1)
    \end{lstOCaml}
\end{implementation}

\begin{implementation}{file de priorité : \texttt{pop} (1/3)}
    \begin{lstOCaml}
    type tas_binaire_min = {
        mutable nb_elts:int; 
        mutable data: (char*int) array
    }

    let prio couple = 
        let _,b = couple in b

    let redim tbmin new_taille cur_taille = 
        assert (new_taille >= tbmin.nb_elts);
        let new_data = Array.make new_taille ('\000',0) in
        for i=0 to tbmin.nb_elts - 1 do
            new_data.(i) <- tbmin.data.(i)
        done;
        tbmin.data <- new_data
    \end{lstOCaml}
\end{implementation}

\begin{implementation}{file de priorité : \texttt{pop} (2/3)}
    \begin{lstOCaml}
    let tbmin_pop tbmin =
        (*formalité*)
        assert (tbmin.nb_elts > 0); 
    
        (*remplacement de la première case par la dernière*)
        let res =  tbmin.data.(0) in 
        tbmin.data.(0) <- tbmin.data.(tbmin.nb_elts-1);
        tbmin.nb_elts <- tbmin.nb_elts - 1;
    
        (*redimensionnement*)
        let n = Array.length tbmin.data in
        if tbmin.nb_elts <= n/2 then 
            redim tbmin (n/2) n;
    \end{lstOCaml}
\end{implementation}

\begin{implementation}{file de priorité : \texttt{pop} (3/3)}
    \begin{lstOCaml}
        (*percolations du nouveau premier élément*)
        let rec percole_bas i =
            let max = tbmin.nb_elts - 1 in
            let fils_g = if 2*i+1 <= max then 2*i+1 else max in
            let fils_d = if 2*i+2 <= max then 2*i+2 else max in
            if (prio tbmin.data.(fils_d) < prio tbmin.data.(i)|| 
                prio tbmin.data.(fils_g) < prio tbmin.data.(i)) then begin
                (*on va percoler le fils de plus basse priorité*)
                if prio tbmin.data.(fils_d) < prio tbmin.data.(fils_g) then
                    let temp = tbmin.data.(fils_d) in
                    tbmin.data.(fils_d) <- tbmin.data.(i);
                    tbmin.data.(i) <- temp;
                    percole_bas fils_d
                else 
                    let temp = tbmin.data.(fils_g) in
                    tbmin.data.(fils_g) <- tbmin.data.(i);
                    tbmin.data.(i) <- temp;
                    percole_bas fils_g
                end;
        in if tbmin.nb_elts > 0 then percole_bas 0;
        res
    \end{lstOCaml}
\end{implementation}

\end{adjustwidth}
\end{document}
