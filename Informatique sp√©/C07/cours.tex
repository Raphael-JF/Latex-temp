\documentclass{article}
\usepackage{amsmath,amssymb,mathtools}
\usepackage{xcolor}
\usepackage{minted}
\usepackage{enumitem}
\usepackage{multicol}
\usepackage{changepage}
\usepackage{stmaryrd}
\usepackage{graphicx}
\graphicspath{ {./images/} }
\usepackage[framemethod=tikz]{mdframed}
\usepackage{tikz,pgfplots}
\pgfplotsset{compat=1.18}

% physique
\definecolor{oranges}{RGB}{255, 242, 230}
\definecolor{rouges}{RGB}{255, 230, 230}
\definecolor{rose}{RGB}{255, 204, 204}

% maths - info
\definecolor{rouge_fonce}{RGB}{204, 0, 0}
\definecolor{rouge}{RGB}{255, 0, 0}
\definecolor{bleu_fonce}{RGB}{0, 0, 255}
\definecolor{vert_fonce}{RGB}{0, 69, 33}
\definecolor{vert}{RGB}{0,255,0}

\definecolor{orange_foncee}{RGB}{255, 153, 0}
\definecolor{myrtille}{RGB}{225, 225, 255}
\definecolor{mayonnaise}{RGB}{255, 253, 233}
\definecolor{magenta}{RGB}{224, 209, 240}
\definecolor{pomme}{RGB}{204, 255, 204}
\definecolor{mauve}{RGB}{255, 230, 255}


% Cours

\newmdenv[
  nobreak=true,
  topline=true,
  bottomline=true,
  rightline=true,
  leftline=true,
  linewidth=0.5pt,
  linecolor=black,
  backgroundcolor=mayonnaise,
  innerleftmargin=10pt,
  innerrightmargin=10pt,
  innertopmargin=5pt,
  innerbottommargin=5pt,
  skipabove=\topsep,
  skipbelow=\topsep,
]{boite_definition}

\newmdenv[
  nobreak=false,
  topline=true,
  bottomline=true,
  rightline=true,
  leftline=true,
  linewidth=0.5pt,
  linecolor=white,
  backgroundcolor=white,
  innerleftmargin=10pt,
  innerrightmargin=10pt,
  innertopmargin=5pt,
  innerbottommargin=5pt,
  skipabove=\topsep,
  skipbelow=\topsep,
]{boite_exemple}

\newmdenv[
  nobreak=true,
  topline=true,
  bottomline=true,
  rightline=true,
  leftline=true,
  linewidth=0.5pt,
  linecolor=black,
  backgroundcolor=magenta,
  innerleftmargin=10pt,
  innerrightmargin=10pt,
  innertopmargin=5pt,
  innerbottommargin=5pt,
  skipabove=\topsep,
  skipbelow=\topsep,
]{boite_proposition}

\newmdenv[
  nobreak=true,
  topline=true,
  bottomline=true,
  rightline=true,
  leftline=true,
  linewidth=0.5pt,
  linecolor=black,
  backgroundcolor=white,
  innerleftmargin=10pt,
  innerrightmargin=10pt,
  innertopmargin=5pt,
  innerbottommargin=5pt,
  skipabove=\topsep,
  skipbelow=\topsep,
]{boite_demonstration}

\newmdenv[
  nobreak=true,
  topline=true,
  bottomline=true,
  rightline=true,
  leftline=true,
  linewidth=0.5pt,
  linecolor=white,
  backgroundcolor=white,
  innerleftmargin=10pt,
  innerrightmargin=10pt,
  innertopmargin=5pt,
  innerbottommargin=5pt,
  skipabove=\topsep,
  skipbelow=\topsep,
]{boite_remarque}


\newenvironment{definition}[2]
{
    \vspace{15pt}
    \begin{boite_definition}
    \textbf{\textcolor{rouge}{Définition #1}}
    \if\relax\detokenize{#2}\relax
    \else
        \textit{ - #2}
    \fi \\ \\
}
{
    \end{boite_definition}
    
}

\newenvironment{exemple}[2]
{
    \vspace{15pt}
    \begin{boite_exemple}
    \textbf{\textcolor{bleu_fonce}{Exemple #1}}
    \if\relax\detokenize{#2}\relax
    \else
        \textit{ - #2}
    \fi \\ \\ 
}
{   
    \end{boite_exemple}
    
}

\newenvironment{proposition}[2]
{
    \vspace{15pt}
    \begin{boite_proposition}
    \textbf{\textcolor{rouge}{Proposition #1}}
    \if\relax\detokenize{#2}\relax
    \else
        \textit{ - #2}
    \fi \\ \\
}
{
    \end{boite_proposition}
    
}

\newenvironment{theoreme}[2]
{
    \vspace{15pt}
    \begin{boite_proposition}
    \textbf{\textcolor{rouge}{Théorème #1}} 
    \if\relax\detokenize{#2}\relax
    \else
        \textit{ - #2}
    \fi \\ \\
}
{
    \end{boite_proposition}
    
}

\newenvironment{demonstration}
{
    \vspace{15pt}
    \begin{boite_demonstration}
    \textbf{\textcolor{rouge}{Démonstration}}\\ \\
}
{
    \end{boite_demonstration}
    
}

\newenvironment{remarque}[2]
{
    \vspace{15pt}
    \begin{boite_remarque}
    \textbf{\textcolor{bleu_fonce}{Remarque #1}}
    \if\relax\detokenize{#2}\relax
    \else
        \textit{ - #2}
    \fi \\ \\   
}
{  
    \end{boite_remarque}
    
}



%Corrections
\newmdenv[
  nobreak=true,
  topline=true,
  bottomline=true,
  rightline=true,
  leftline=true,
  linewidth=0.5pt,
  linecolor=black,
  backgroundcolor=mayonnaise,
  innerleftmargin=10pt,
  innerrightmargin=10pt,
  innertopmargin=5pt,
  innerbottommargin=5pt,
  skipabove=\topsep,
  skipbelow=\topsep,
]{boite_question}


\newenvironment{question}[2]
{
    \vspace{15pt}
    \begin{boite_question}
    \textbf{\textcolor{rouge}{Question #1}}
    \if\relax\detokenize{#2}\relax
    \else
        \textit{ - #2}
    \fi \\ \\
}
{
    \end{boite_question}
    
}

\newenvironment{enumeratebf}{
    \begin{enumerate}[label=\textbf{\arabic*.}]
}
{
    \end{enumerate}
}

\begin{document}
\begin{adjustwidth}{-3cm}{-3cm}
\begin{document}
\begin{adjustwidth}{-3cm}{-3cm}
% commandes
\newcommand{\notion}[1]{\textcolor{vert_fonce}{\textit{#1}}}
\newcommand{\mb}[1]{\mathbb{#1}}
\newcommand{\mc}[1]{\mathcal{#1}}
\newcommand{\mr}[1]{\mathrm{#1}}
\newcommand{\code}[1]{\texttt{#1}}
\newcommand{\ccode}[1]{\texttt{|#1|}}
\newcommand{\ov}[1]{\overline{#1}}
\newcommand{\abs}[1]{|#1|}
\newcommand{\rev}[1]{\texttt{reverse(#1)}}
\newcommand{\crev}[1]{\texttt{|reverse(#1)|}}

\newcommand{\ie}{\textit{i.e.} }

\newcommand{\N}{\mathbb{N}}
\newcommand{\R}{\mathbb{R}}
\newcommand{\C}{\mathbb{C}}
\newcommand{\K}{\mathbb{K}}
\newcommand{\Z}{\mathbb{Z}}

\newcommand{\A}{\mathcal{A}}
\newcommand{\bigO}{\mathcal{O}}
\renewcommand{\L}{\mathcal{L}}

\newcommand{\rg}[0]{\mathrm{rg}}
\newcommand{\re}[0]{\mathrm{Re}}
\newcommand{\im}[0]{\mathrm{Im}}
\newcommand{\cl}[0]{\mathrm{cl}}
\newcommand{\grad}[0]{\vec{\mathrm{grad}}}
\renewcommand{\div}[0]{\mathrm{div}\,}
\newcommand{\rot}[0]{\vec{\mathrm{rot}}\,}
\newcommand{\vnabla}[0]{\vec{\nabla}}
\renewcommand{\vec}[1]{\overrightarrow{#1}}
\newcommand{\mat}[1]{\mathrm{Mat}_{#1}}
\newcommand{\matrice}[1]{\mathcal{M}_{#1}}
\newcommand{\sgEngendre}[1]{\left\langle #1 \right\rangle}
\newcommand{\gpquotient}[1]{\mathbb{Z} / #1\mathbb{Z}}
\newcommand{\norme}[1]{||#1||}
\renewcommand{\d}[1]{\,\mathrm{d}#1}
\newcommand{\adh}[1]{\overline{#1}}
\newcommand{\intint}[2]{\llbracket #1 ,\, #2 \rrbracket}
\newcommand{\seg}[2]{[#1\, ; \, #2]}
\newcommand{\scal}[2]{( #1 | #2 )}
\newcommand{\distance}[2]{\mathrm{d}(#1,\,#2)}
\newcommand{\inte}[2]{\int_{#1}^{#2}}
\newcommand{\somme}[2]{\sum_{#1}^{#2}}
\newcommand{\deriveref}[4]{\biggl( \frac{\text{d}^{#1}#2}{\text{d}#3^{#1}} \biggr)_{#4}}






\begin{definition}{7.1}{système de preuve}
    Un système de preuve est un cadre formel permettant de dériver des énoncés logiques à partir de règles bien définies et d'hypothèses. On le caractérise par~:
    \begin{enumeratebf}
        \item \notion{un ensemble d'axiomes} : propositions admises comme vraies.
        \item \notion{un ensemble de règles d'inférence}.
    \end{enumeratebf}
    On représente une preuve par un arbre dont \notion{les feuilles sont des instances d'axiomes} et \notion{les noeuds internes des instances de règles d'inférence}.
\end{definition}

\begin{definition}{7.2}{instance d'un axiome, d'une règle d'inférence}
    Une \notion{instance d'un axiome ou d'une règle d'inférence} est obtenue à partir de l'axiome ou de la règle, en choisissant pour chaque variable une formule logique et en remplaçant chaque occurrence de la variable par la formule.
\end{definition}

\newcommand{\these}[0]{\vdash}

\begin{definition}{7.3}{séquent}
    Un séquent (également jugement), est une \notion{affirmation qui exprime que, sous certaines hypothèses, une conclusion peut être déduite}. Il est généralement écrit sous la forme~:
    $$\Gamma \these C$$
    où : $\begin{cases*}
        \Gamma & est un ensemble d'hypothèses (formules de la logique propositionnelle supposées vraies)\\
        C & est la conclusion qui peut être déduite à partir des hypothèses de $\Gamma$
    \end{cases*}$\\\\
    Intuitivement, cela signifie : \textit{"Si les hypothèses de $\Gamma$ sont vérifiées, alors $C$ peut être démontrée"}
\end{definition}

\begin{definition}{7.4}{règle d'inférence}
    Dans un système de preuve, une \notion{règle d'inférence} est constituée d'une \notion{famille de prémisses $P_1,\, \dots,\, P_k$}, et d'une \notion{conclusion $C$}. On représente une règle d'inférence par~:
    $$\frac{P_1 \quad \dots \quad P_k}{C}$$
\end{definition}

\begin{definition}{7.5}{règle d'inférence de l'axiome}
    En déduction naturelle, Pour un ensemble $\Gamma$ de formules de la logique propositionnelle, pour toute formule $A \in \Gamma$, le séquent $T \these A$ est prouvable~:
    $$\frac{(\mr{axiome})}{\Gamma \these A}$$
    en constitue une démonstration.
\end{definition}

\begin{definition}{7.6}{séquent prouvable}
    Un séquent est dit \notion{prouvable} lorsqu'il existe un \notion{un arbre de preuve} de celui-ci. Plus précisément, le séquent $\Gamma \these A$ est prouvable si l'une des conditions suivantes est satisfaite~:
    \begin{itemize}
        \item \textbf{cas de base} : $A \in \Gamma$. $\displaystyle \frac{(\text{axiome})}{\Gamma \these A}$ est alors une démonstration
        \item \textbf{cas inductif} : il existe $\Gamma_1 \these C_1,\, \dots,\,  \Gamma_k \these C_k$ des séquents prouvables, ainsi qu'une règle d'inférence $(\gamma)$ dont~:
        $$\frac{\Gamma_1 \these C_1 \quad \dots \quad \Gamma_k \these C_k}{\Gamma \these A}(\gamma)$$
        est une instance. Le cas échéant $\Gamma \these A$ est prouvable via la règle $(\gamma)$.
    \end{itemize}
\end{definition}

\begin{definition}{7.7}{inductive d'un séquent prouvable}
    L'ensemble des séquents prouvables en déduction naturelle est défini inductivement à partir des règles de déduction naturelle~:
    \begin{itemize}
        \item Pour $\Gamma$ un ensemble de formules de la logique propositionnelle, et $A \in \Gamma$~:
        $$\Gamma \these A \text{ prouvable par axiome}$$
        \item Pour $\Gamma$ un ensemble de formules de la logique propositionnelle, et $A \in \Gamma$~:

    \end{itemize}
\end{definition}

\begin{definition}{7.8}{règle d'introduction du "et"}
    En déduction naturelle, Pour un ensemble $\Gamma$ de formules de la logique propositionnelle, pour toutes formules $A$, $B$ et $C$~:
    $$\frac{\Gamma \these A \quad \Gamma \these B}{\Gamma \these A \wedge B}(\mr{i}\wedge)$$
    telle est la \notion{règle d'introduction du "et"}.
\end{definition}

\begin{definition}{7.9}{règle d'élimination du "et"}
    En déduction naturelle, Pour un ensemble $\Gamma$ de formules de la logique propositionnelle, pour toutes formules $A$ et $B$~:
    $$\frac{\Gamma \these A \wedge B}{\Gamma \these A}(\mr{e}\wedge\mr{g}) \qquad \frac{\Gamma \these A \wedge B}{\Gamma \these B}(\mr{e}\wedge\mr{d})$$
    telle est la \notion{règle d'élimination du "et"}.
\end{definition}

\newcommand{\ou}[0]{\vee}

\begin{definition}{7.10}{règle d'introduction du "ou"}
    En déduction naturelle, Pour un ensemble $\Gamma$ de formules de la logique propositionnelle, pour toutes formules $A$ et $B$~:
    $$\frac{\Gamma \these B}{\Gamma \these A \ou B}(\mr{i}\ou\mr{d}) \qquad \frac{\Gamma \these A}{\Gamma \these A \ou B}(\mr{i}\ou\mr{g})$$
    telle est la \notion{règle d'introduction du "ou"}.
\end{definition}

\begin{definition}{7.11}{règle d'élimination du "ou"}
    En déduction naturelle, Pour un ensemble $\Gamma$ de formules de la logique propositionnelle, pour toutes formules $A$, $B$ et $C$~:
    $$\frac{\Gamma \these A \ou B \quad \Gamma,A \these C \quad \Gamma,B \these C}{\Gamma \these C}(\mr{e}\ou)$$
    telle est la \notion{règle d'élimination du "ou"}.
\end{definition}

\begin{definition}{7.12}{règle d'introduction du "non"}
    En déduction naturelle, Pour un ensemble $\Gamma$ de formules de la logique propositionnelle, pour toute formule $A$~:
    $$\frac{\Gamma,A \these \bot}{\Gamma \these \lnot A}(\mr{i}\lnot)$$
    telle est la \notion{règle d'introduction du "non"}.
\end{definition}

\begin{definition}{7.13}{règle d'élimination du "non"}
    En déduction naturelle, Pour un ensemble $\Gamma$ de formules de la logique propositionnelle, pour toute formule $A$~:
    $$\frac{\Gamma \these A \quad \Gamma \these \lnot A}{\Gamma \these \bot}(\mr{e}\lnot)$$
    telle est la \notion{règle d'élimination du "non"}.
\end{definition}

\begin{definition}{7.14}{règle d'introduction du "implique"}
    En déduction naturelle, Pour un ensemble $\Gamma$ de formules de la logique propositionnelle, pour toutes formules $A$ et $B$~:
    $$\frac{\Gamma,A \these B}{\Gamma \these A \to B}(\mr{i}\to)$$
    telle est la \notion{règle d'introduction du "implique"}.
\end{definition}

\begin{definition}{7.15}{règle d'élimination du "implique"}
    En déduction naturelle, Pour un ensemble $\Gamma$ de formules de la logique propositionnelle, pour toutes formules $A$ et $B$~:
    $$\frac{\Gamma \these A \to B \quad \Gamma \these A}{\Gamma \these B}(\mr{e}\to)$$
    telle est la \notion{règle d'élimination du "implique"}.
\end{definition}

\begin{theoreme}{7.16}{propriété d'affaiblissement}
    En déduction naturelle, pour des ensembles $\Gamma$ et $\Delta$ de formules de la logique propositionnelle, pour toute formule $A$ de la logique propositionnelle, si $\Gamma \these A$ est prouvable, alors il en est de même pour $\Gamma,\Delta \these A$.
\end{theoreme}


\end{adjustwidth}
\end{document}