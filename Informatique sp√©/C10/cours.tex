\documentclass{article}
\usepackage{amsmath,amssymb,mathtools}
\usepackage{xcolor}
\usepackage{minted}
\usepackage{enumitem}
\usepackage{multicol}
\usepackage{changepage}
\usepackage{stmaryrd}
\usepackage{graphicx}
\graphicspath{ {./images/} }
\usepackage[framemethod=tikz]{mdframed}
\usepackage{tikz,pgfplots}
\pgfplotsset{compat=1.18}

% physique
\definecolor{oranges}{RGB}{255, 242, 230}
\definecolor{rouges}{RGB}{255, 230, 230}
\definecolor{rose}{RGB}{255, 204, 204}

% maths - info
\definecolor{rouge_fonce}{RGB}{204, 0, 0}
\definecolor{rouge}{RGB}{255, 0, 0}
\definecolor{bleu_fonce}{RGB}{0, 0, 255}
\definecolor{vert_fonce}{RGB}{0, 69, 33}
\definecolor{vert}{RGB}{0,255,0}

\definecolor{orange_foncee}{RGB}{255, 153, 0}
\definecolor{myrtille}{RGB}{225, 225, 255}
\definecolor{mayonnaise}{RGB}{255, 253, 233}
\definecolor{magenta}{RGB}{224, 209, 240}
\definecolor{pomme}{RGB}{204, 255, 204}
\definecolor{mauve}{RGB}{255, 230, 255}


% Cours

\newmdenv[
  nobreak=true,
  topline=true,
  bottomline=true,
  rightline=true,
  leftline=true,
  linewidth=0.5pt,
  linecolor=black,
  backgroundcolor=mayonnaise,
  innerleftmargin=10pt,
  innerrightmargin=10pt,
  innertopmargin=5pt,
  innerbottommargin=5pt,
  skipabove=\topsep,
  skipbelow=\topsep,
]{boite_definition}

\newmdenv[
  nobreak=false,
  topline=true,
  bottomline=true,
  rightline=true,
  leftline=true,
  linewidth=0.5pt,
  linecolor=white,
  backgroundcolor=white,
  innerleftmargin=10pt,
  innerrightmargin=10pt,
  innertopmargin=5pt,
  innerbottommargin=5pt,
  skipabove=\topsep,
  skipbelow=\topsep,
]{boite_exemple}

\newmdenv[
  nobreak=true,
  topline=true,
  bottomline=true,
  rightline=true,
  leftline=true,
  linewidth=0.5pt,
  linecolor=black,
  backgroundcolor=magenta,
  innerleftmargin=10pt,
  innerrightmargin=10pt,
  innertopmargin=5pt,
  innerbottommargin=5pt,
  skipabove=\topsep,
  skipbelow=\topsep,
]{boite_proposition}

\newmdenv[
  nobreak=true,
  topline=true,
  bottomline=true,
  rightline=true,
  leftline=true,
  linewidth=0.5pt,
  linecolor=black,
  backgroundcolor=white,
  innerleftmargin=10pt,
  innerrightmargin=10pt,
  innertopmargin=5pt,
  innerbottommargin=5pt,
  skipabove=\topsep,
  skipbelow=\topsep,
]{boite_demonstration}

\newmdenv[
  nobreak=true,
  topline=true,
  bottomline=true,
  rightline=true,
  leftline=true,
  linewidth=0.5pt,
  linecolor=white,
  backgroundcolor=white,
  innerleftmargin=10pt,
  innerrightmargin=10pt,
  innertopmargin=5pt,
  innerbottommargin=5pt,
  skipabove=\topsep,
  skipbelow=\topsep,
]{boite_remarque}


\newenvironment{definition}[2]
{
    \vspace{15pt}
    \begin{boite_definition}
    \textbf{\textcolor{rouge}{Définition #1}}
    \if\relax\detokenize{#2}\relax
    \else
        \textit{ - #2}
    \fi \\ \\
}
{
    \end{boite_definition}
    
}

\newenvironment{exemple}[2]
{
    \vspace{15pt}
    \begin{boite_exemple}
    \textbf{\textcolor{bleu_fonce}{Exemple #1}}
    \if\relax\detokenize{#2}\relax
    \else
        \textit{ - #2}
    \fi \\ \\ 
}
{   
    \end{boite_exemple}
    
}

\newenvironment{proposition}[2]
{
    \vspace{15pt}
    \begin{boite_proposition}
    \textbf{\textcolor{rouge}{Proposition #1}}
    \if\relax\detokenize{#2}\relax
    \else
        \textit{ - #2}
    \fi \\ \\
}
{
    \end{boite_proposition}
    
}

\newenvironment{theoreme}[2]
{
    \vspace{15pt}
    \begin{boite_proposition}
    \textbf{\textcolor{rouge}{Théorème #1}} 
    \if\relax\detokenize{#2}\relax
    \else
        \textit{ - #2}
    \fi \\ \\
}
{
    \end{boite_proposition}
    
}

\newenvironment{demonstration}
{
    \vspace{15pt}
    \begin{boite_demonstration}
    \textbf{\textcolor{rouge}{Démonstration}}\\ \\
}
{
    \end{boite_demonstration}
    
}

\newenvironment{remarque}[2]
{
    \vspace{15pt}
    \begin{boite_remarque}
    \textbf{\textcolor{bleu_fonce}{Remarque #1}}
    \if\relax\detokenize{#2}\relax
    \else
        \textit{ - #2}
    \fi \\ \\   
}
{  
    \end{boite_remarque}
    
}



%Corrections
\newmdenv[
  nobreak=true,
  topline=true,
  bottomline=true,
  rightline=true,
  leftline=true,
  linewidth=0.5pt,
  linecolor=black,
  backgroundcolor=mayonnaise,
  innerleftmargin=10pt,
  innerrightmargin=10pt,
  innertopmargin=5pt,
  innerbottommargin=5pt,
  skipabove=\topsep,
  skipbelow=\topsep,
]{boite_question}


\newenvironment{question}[2]
{
    \vspace{15pt}
    \begin{boite_question}
    \textbf{\textcolor{rouge}{Question #1}}
    \if\relax\detokenize{#2}\relax
    \else
        \textit{ - #2}
    \fi \\ \\
}
{
    \end{boite_question}
    
}

\newenvironment{enumeratebf}{
    \begin{enumerate}[label=\textbf{\arabic*.}]
}
{
    \end{enumerate}
}

\begin{document}
\begin{adjustwidth}{-3cm}{-3cm}
\begin{document}
\begin{adjustwidth}{-3cm}{-3cm}
% commandes
\newcommand{\notion}[1]{\textcolor{vert_fonce}{\textit{#1}}}
\newcommand{\mb}[1]{\mathbb{#1}}
\newcommand{\mc}[1]{\mathcal{#1}}
\newcommand{\mr}[1]{\mathrm{#1}}
\newcommand{\code}[1]{\texttt{#1}}
\newcommand{\ccode}[1]{\texttt{|#1|}}
\newcommand{\ov}[1]{\overline{#1}}
\newcommand{\abs}[1]{|#1|}
\newcommand{\rev}[1]{\texttt{reverse(#1)}}
\newcommand{\crev}[1]{\texttt{|reverse(#1)|}}

\newcommand{\ie}{\textit{i.e.} }

\newcommand{\N}{\mathbb{N}}
\newcommand{\R}{\mathbb{R}}
\newcommand{\C}{\mathbb{C}}
\newcommand{\K}{\mathbb{K}}
\newcommand{\Z}{\mathbb{Z}}

\newcommand{\A}{\mathcal{A}}
\newcommand{\bigO}{\mathcal{O}}
\renewcommand{\L}{\mathcal{L}}

\newcommand{\rg}[0]{\mathrm{rg}}
\newcommand{\re}[0]{\mathrm{Re}}
\newcommand{\im}[0]{\mathrm{Im}}
\newcommand{\cl}[0]{\mathrm{cl}}
\newcommand{\grad}[0]{\vec{\mathrm{grad}}}
\renewcommand{\div}[0]{\mathrm{div}\,}
\newcommand{\rot}[0]{\vec{\mathrm{rot}}\,}
\newcommand{\vnabla}[0]{\vec{\nabla}}
\renewcommand{\vec}[1]{\overrightarrow{#1}}
\newcommand{\mat}[1]{\mathrm{Mat}_{#1}}
\newcommand{\matrice}[1]{\mathcal{M}_{#1}}
\newcommand{\sgEngendre}[1]{\left\langle #1 \right\rangle}
\newcommand{\gpquotient}[1]{\mathbb{Z} / #1\mathbb{Z}}
\newcommand{\norme}[1]{||#1||}
\renewcommand{\d}[1]{\,\mathrm{d}#1}
\newcommand{\adh}[1]{\overline{#1}}
\newcommand{\intint}[2]{\llbracket #1 ,\, #2 \rrbracket}
\newcommand{\seg}[2]{[#1\, ; \, #2]}
\newcommand{\scal}[2]{( #1 | #2 )}
\newcommand{\distance}[2]{\mathrm{d}(#1,\,#2)}
\newcommand{\inte}[2]{\int_{#1}^{#2}}
\newcommand{\somme}[2]{\sum_{#1}^{#2}}
\newcommand{\deriveref}[4]{\biggl( \frac{\text{d}^{#1}#2}{\text{d}#3^{#1}} \biggr)_{#4}}





\newcounter{chapitre}
\setcounter{chapitre}{10}

\title{Chapitre 10 : Décidabilité}
\maketitle

Au programme : concepts à comprendre, démonstration à connaître !

\section{Problème de décision et décidabilité}

\begin{definition}{}{problème de décision}
    Un \notion{problème de décision} est un problème dont la réponse attendue est binaire : Vrai ou Faux. Plus précisément, un problème $\mc{P}$ est la donnée de~:
    \begin{itemize}
        \item $I$ un ensemble d'instances
        \item $S$ un ensemble de solutions : l'union des solutions pour chaque instance.
        \item $f : I \to S$ ou pour $i$ une instance, $f(i)$ est la réponse attendue pour l'instance $i$.
    \end{itemize}
    Pour un problème de décision, $S = \{\mr{Vrai}, \mr{Faux}\}$. On appelle la fonction $f$ \notion{fonction de prédicat} du problème $\mc{P}$ de décision.\\\\
    $\mc{P}$ peut aussi être défini à l'aide d'un sous-ensemble $P$ de $I \times S$ tel que ~:
    $$(i,s) \in P \Leftrightarrow s \text{ solution de $\mc{P}$ pour l'instance $i$}$$
\end{definition}



Pour $\mc{P}$ un problème de décision, défini par $f : I \to \{\mr{Vrai},\, \mr{Faux}\}$ sa fonction de prédicat, on définit~:
$$I_{\mc{P}}^+ = \{i \in I,\, f(i) = \mr{Vrai}\}$$
l'ensemble des \notion{instances passives} du problème $\mc{P}$ de décision.


\begin{definition}{}{décidabilité d'un problème de décision}
    Un problème de décision $\mc{P}$ est dit \notion{décidable} lorsqu'il existe $\mc{A}$ un algorithme qui pour toute instance du problème $\mc{P}$ renvoie la solution attendue.\\
    Autrement dit, pour $f : I \to S$ la fonction de prédicat de $\mc{P}$, il existe un algorithme $\mc{A}$ tel que~:
    $$\forall i \in I, f(i) = \mc{A}(i)$$
    $f(i)$ est ici la solution attendue tandis que $\mc{A}(i)$ est la solution attendue pour $i$.\\\\
    Le cas échéant, $\mc{A}$ termine pour toute instance.
    \notion{$\mc{A}$ résout $\mc{P}$} ou que \notion{$\mc{P}$ est décidé par $\mc{A}$}.
\end{definition}

\begin{definition}{}{indécidabilité d'un problème}
    Un problème de décision $\mc{P}$ est dit \notion{indécidable} lorsqu'il n'existe pas d'algorithme resolvant $\mc{P}$.
\end{definition}

\begin{remarque}{}{sur l'indécidabilité}
    Un problème indécidable est un problème intrinsèquement infaisable~: inutile d'essayer de le résoudre car c'est impossible.
\end{remarque}

\begin{exemple}{}{de problème décidable}
    $f : I \to S = \{\mr{Vrai},\, \mr{Faux}\}$\\
    $$f(\code{l}) = \mr{Vrai} \Leftrightarrow \text{\code{l} a exactement 5 éléments}$$
    $f$ est la fonction de prédicat d'un problème de décision~:
    \begin{itemize}
        \item \textbf{Instance}~: \code{l} une liste d'entiers
        \item \textbf{Question}~: Est-ce que \code{l} contient 5 éléments.
    \end{itemize}
    Ce problème est décidable car on peut écrire en OCaml~:
    \begin{lstOCaml}
    let longueur 5 l =
        match l with
        | e1::e2::e3::e4::e5::[] -> true
        | _ -> false
    \end{lstOCaml}
\end{exemple}

\subsection{semi-décidabilité (HP)}

\begin{definition}{}{semi-décidabilité d'un problème}
    Un problème de décision $\mc{P}$ défini par $f:I \to \{\mr{Vrai},\, \mr{Faux}\}$ est dit \notion{semi-décidable} lorsqu'il existe un algorithme $\mc{A}$ tel que pour $i \in I$~:
    \begin{itemize}
        \item si $f(i) = \mr{Vrai}$ alors $\mc{A}(i) = f(i)$ et $\mc{A}$ termine sur $i$.
        \item si $f(i) = \mr{Faux}$, alors $\mc{A}(i) = f(i)$ et $\mc{A}$ termine ou bien $\mc{A}$ ne termine pas sur $i$. Ecrire systeme.
    \end{itemize}
\end{definition}

\begin{exemple}{}{important}
    Il existe des problèmes non semi-décidables. On considère par exemple le suivant~:
    \begin{itemize}
        \item \textbf{Instance}~: une fonction \code{f : string -> bool} et \code{f} son code source.
        \item \textbf{Question}~: est ce que l'appel \code{f code\_f} ne renvoie pas true ? \ie est ce que l'appel renvoie false ou bien ne termine pas ? N'a-t-on que ces deux possibilités ?
    \end{itemize}
    Montrons que ce problème n'est pas semi-décidable.\\\\
    Supposons par l'absurde qu'il existe un algorithme $\mc{A}$ implémenté par une fonction \code{diag : string -> bool} qui semi-décide le problème. Par définition~:
    \begin{enumerate}
        \item \code{diag code\_f} renvoie \code{true} lorsque \code{f code\_f} ne renvoie pas \code{true}
        \item \code{diag code\_f} renvoie \code{false} ou ne termine pas lorsque \code{f code\_f} revoie true.
    \end{enumerate}
    On applique la fonction \code{diag} à son propre code, noté \code{code\_diag}.
    \begin{itemize}
        \item Si \code{diag code\_diag} renvoie \code{false}~: par définition de \code{diag} (2.), on a \code{diag code\_diag} renvoie \code{true}. Il y a alors contradiction.
        \item Si \code{diag code\_diag} renvoie \code{true}. Par définition de \code{diag} (1.), on a \code{diag code\_diag} ne renvoie pas \code{true}. C'est également absurde.
        \item Si \code{diag code\_diag} ne termine pas ou échoue, par définition de \code{diag} (2.), \code{diag code\_diag} renvoie \code{true}. On a une contradiction.
    \end{itemize}
    Aucune possibilité n'est viable. D'où l'absurdité de l'hypothèse.
\end{exemple}

\subsection{Problème de l'arrêt (au programme)}

\begin{definition}{}{problème de l'arrêt}
    Le \notion{problème de l'arrêt} est le problème qui consiste à décider si un programme ou un algorithme termine sur une entrée.
    \begin{itemize}
        \item \textbf{Instance}~: un programme \code{p} donné par son code \code{code\_p} et une entrée \code{x}
        \item \textbf{Question}~: est ce que l'appel \code{p x} termine ?
    \end{itemize}
\end{definition}

\begin{theoreme}{}{indécidabilité du problème de l'arrêt}
    Le problème de l'arrêt est indécidable.
\end{theoreme}
La démonstration est la suivante. Elle est à connaître impérativement.\\\\
\begin{demonstration}
    Supposons par l'absurde qu'il existe $\mc{A}$ un algorithme implément par une fonction \code{arret} qui résout le problème de l'arrêt. Par définition de décidabilité~:
    \begin{enumerate}
        \item \code{arret code\_p x} renvoie \code{true} lorsque \code{p x} termine.
        \item \code{arret code\_p code\_x} renvoie \code{false} lorsque \code{p x} ne termine pas.
    \end{enumerate}
    On écrit~:
    \begin{lstOCaml}
        let rec boucle (b:bool) :int =
            match b with
            | true -> boucle true
            | false -> 0
    
        let absurde code_p = boucle (arret code_p code_p)
    \end{lstOCaml}
    
    On s'intéresse à l'appel \code{arret code\_absurde code\_absurde}.
    \begin{itemize}
        \item Si \code{arret code\_absurde code\_absurde} renvoie \code{true}, par définition de \code{arret} (1.), \code{absurde code\_absurde} termine.
        Or cet appel \code{absurde code\_absurde} correspond à \code{boucle (arret code\_absurde code\_absurde)} qui ne termine pas par construction, alors que \code{arret code\_absurde code\_absurde} renvoie \code{true} : ceci constitue une contradiction.
        \item Si \code{arret code\_absurde code\_absurde} renvoie \code{false}. Par définition de \code{arret} (2.), \code{absurde code\_absurde} ne termine pas.
        Or \code{absurde code\_absurde} correspond à \code{boucle (arret code\_absurde code\_absurde)} qui termine par construction, alors que \notion{arret code\_absurde code\_absurde} renvoie \code{false}. Contradiction.
    \end{itemize}
\end{demonstration}



\begin{remarque}{}{usage de chaînes de caractères}
    Dans les démonstrations, on préfèrera la manipulation de chaînes de caractères associées à ce qui n'en est pas. En effet, en machine, tout est représenté par des chaînes et en particulier les machines de Turing ne manipulent que ça. C'est plus formel.
\end{remarque}

\end{adjustwidth}
\end{document}