\documentclass{article}
\usepackage{amsmath,amssymb,mathtools}
\usepackage{xcolor}
\usepackage{minted}
\usepackage{enumitem}
\usepackage{multicol}
\usepackage{changepage}
\usepackage{stmaryrd}
\usepackage{graphicx}
\graphicspath{ {./images/} }
\usepackage[framemethod=tikz]{mdframed}
\usepackage{tikz,pgfplots}
\pgfplotsset{compat=1.18}

% physique
\definecolor{oranges}{RGB}{255, 242, 230}
\definecolor{rouges}{RGB}{255, 230, 230}
\definecolor{rose}{RGB}{255, 204, 204}

% maths - info
\definecolor{rouge_fonce}{RGB}{204, 0, 0}
\definecolor{rouge}{RGB}{255, 0, 0}
\definecolor{bleu_fonce}{RGB}{0, 0, 255}
\definecolor{vert_fonce}{RGB}{0, 69, 33}
\definecolor{vert}{RGB}{0,255,0}

\definecolor{orange_foncee}{RGB}{255, 153, 0}
\definecolor{myrtille}{RGB}{225, 225, 255}
\definecolor{mayonnaise}{RGB}{255, 253, 233}
\definecolor{magenta}{RGB}{224, 209, 240}
\definecolor{pomme}{RGB}{204, 255, 204}
\definecolor{mauve}{RGB}{255, 230, 255}


% Cours

\newmdenv[
  nobreak=true,
  topline=true,
  bottomline=true,
  rightline=true,
  leftline=true,
  linewidth=0.5pt,
  linecolor=black,
  backgroundcolor=mayonnaise,
  innerleftmargin=10pt,
  innerrightmargin=10pt,
  innertopmargin=5pt,
  innerbottommargin=5pt,
  skipabove=\topsep,
  skipbelow=\topsep,
]{boite_definition}

\newmdenv[
  nobreak=false,
  topline=true,
  bottomline=true,
  rightline=true,
  leftline=true,
  linewidth=0.5pt,
  linecolor=white,
  backgroundcolor=white,
  innerleftmargin=10pt,
  innerrightmargin=10pt,
  innertopmargin=5pt,
  innerbottommargin=5pt,
  skipabove=\topsep,
  skipbelow=\topsep,
]{boite_exemple}

\newmdenv[
  nobreak=true,
  topline=true,
  bottomline=true,
  rightline=true,
  leftline=true,
  linewidth=0.5pt,
  linecolor=black,
  backgroundcolor=magenta,
  innerleftmargin=10pt,
  innerrightmargin=10pt,
  innertopmargin=5pt,
  innerbottommargin=5pt,
  skipabove=\topsep,
  skipbelow=\topsep,
]{boite_proposition}

\newmdenv[
  nobreak=true,
  topline=true,
  bottomline=true,
  rightline=true,
  leftline=true,
  linewidth=0.5pt,
  linecolor=black,
  backgroundcolor=white,
  innerleftmargin=10pt,
  innerrightmargin=10pt,
  innertopmargin=5pt,
  innerbottommargin=5pt,
  skipabove=\topsep,
  skipbelow=\topsep,
]{boite_demonstration}

\newmdenv[
  nobreak=true,
  topline=true,
  bottomline=true,
  rightline=true,
  leftline=true,
  linewidth=0.5pt,
  linecolor=white,
  backgroundcolor=white,
  innerleftmargin=10pt,
  innerrightmargin=10pt,
  innertopmargin=5pt,
  innerbottommargin=5pt,
  skipabove=\topsep,
  skipbelow=\topsep,
]{boite_remarque}


\newenvironment{definition}[2]
{
    \vspace{15pt}
    \begin{boite_definition}
    \textbf{\textcolor{rouge}{Définition #1}}
    \if\relax\detokenize{#2}\relax
    \else
        \textit{ - #2}
    \fi \\ \\
}
{
    \end{boite_definition}
    
}

\newenvironment{exemple}[2]
{
    \vspace{15pt}
    \begin{boite_exemple}
    \textbf{\textcolor{bleu_fonce}{Exemple #1}}
    \if\relax\detokenize{#2}\relax
    \else
        \textit{ - #2}
    \fi \\ \\ 
}
{   
    \end{boite_exemple}
    
}

\newenvironment{proposition}[2]
{
    \vspace{15pt}
    \begin{boite_proposition}
    \textbf{\textcolor{rouge}{Proposition #1}}
    \if\relax\detokenize{#2}\relax
    \else
        \textit{ - #2}
    \fi \\ \\
}
{
    \end{boite_proposition}
    
}

\newenvironment{theoreme}[2]
{
    \vspace{15pt}
    \begin{boite_proposition}
    \textbf{\textcolor{rouge}{Théorème #1}} 
    \if\relax\detokenize{#2}\relax
    \else
        \textit{ - #2}
    \fi \\ \\
}
{
    \end{boite_proposition}
    
}

\newenvironment{demonstration}
{
    \vspace{15pt}
    \begin{boite_demonstration}
    \textbf{\textcolor{rouge}{Démonstration}}\\ \\
}
{
    \end{boite_demonstration}
    
}

\newenvironment{remarque}[2]
{
    \vspace{15pt}
    \begin{boite_remarque}
    \textbf{\textcolor{bleu_fonce}{Remarque #1}}
    \if\relax\detokenize{#2}\relax
    \else
        \textit{ - #2}
    \fi \\ \\   
}
{  
    \end{boite_remarque}
    
}



%Corrections
\newmdenv[
  nobreak=true,
  topline=true,
  bottomline=true,
  rightline=true,
  leftline=true,
  linewidth=0.5pt,
  linecolor=black,
  backgroundcolor=mayonnaise,
  innerleftmargin=10pt,
  innerrightmargin=10pt,
  innertopmargin=5pt,
  innerbottommargin=5pt,
  skipabove=\topsep,
  skipbelow=\topsep,
]{boite_question}


\newenvironment{question}[2]
{
    \vspace{15pt}
    \begin{boite_question}
    \textbf{\textcolor{rouge}{Question #1}}
    \if\relax\detokenize{#2}\relax
    \else
        \textit{ - #2}
    \fi \\ \\
}
{
    \end{boite_question}
    
}

\newenvironment{enumeratebf}{
    \begin{enumerate}[label=\textbf{\arabic*.}]
}
{
    \end{enumerate}
}

\begin{document}
\begin{adjustwidth}{-3cm}{-3cm}
\begin{document}
\begin{adjustwidth}{-3cm}{-3cm}
% commandes
\newcommand{\notion}[1]{\textcolor{vert_fonce}{\textit{#1}}}
\newcommand{\mb}[1]{\mathbb{#1}}
\newcommand{\mc}[1]{\mathcal{#1}}
\newcommand{\mr}[1]{\mathrm{#1}}
\newcommand{\code}[1]{\texttt{#1}}
\newcommand{\ccode}[1]{\texttt{|#1|}}
\newcommand{\ov}[1]{\overline{#1}}
\newcommand{\abs}[1]{|#1|}
\newcommand{\rev}[1]{\texttt{reverse(#1)}}
\newcommand{\crev}[1]{\texttt{|reverse(#1)|}}

\newcommand{\ie}{\textit{i.e.} }

\newcommand{\N}{\mathbb{N}}
\newcommand{\R}{\mathbb{R}}
\newcommand{\C}{\mathbb{C}}
\newcommand{\K}{\mathbb{K}}
\newcommand{\Z}{\mathbb{Z}}

\newcommand{\A}{\mathcal{A}}
\newcommand{\bigO}{\mathcal{O}}
\renewcommand{\L}{\mathcal{L}}

\newcommand{\rg}[0]{\mathrm{rg}}
\newcommand{\re}[0]{\mathrm{Re}}
\newcommand{\im}[0]{\mathrm{Im}}
\newcommand{\cl}[0]{\mathrm{cl}}
\newcommand{\grad}[0]{\vec{\mathrm{grad}}}
\renewcommand{\div}[0]{\mathrm{div}\,}
\newcommand{\rot}[0]{\vec{\mathrm{rot}}\,}
\newcommand{\vnabla}[0]{\vec{\nabla}}
\renewcommand{\vec}[1]{\overrightarrow{#1}}
\newcommand{\mat}[1]{\mathrm{Mat}_{#1}}
\newcommand{\matrice}[1]{\mathcal{M}_{#1}}
\newcommand{\sgEngendre}[1]{\left\langle #1 \right\rangle}
\newcommand{\gpquotient}[1]{\mathbb{Z} / #1\mathbb{Z}}
\newcommand{\norme}[1]{||#1||}
\renewcommand{\d}[1]{\,\mathrm{d}#1}
\newcommand{\adh}[1]{\overline{#1}}
\newcommand{\intint}[2]{\llbracket #1 ,\, #2 \rrbracket}
\newcommand{\seg}[2]{[#1\, ; \, #2]}
\newcommand{\scal}[2]{( #1 | #2 )}
\newcommand{\distance}[2]{\mathrm{d}(#1,\,#2)}
\newcommand{\inte}[2]{\int_{#1}^{#2}}
\newcommand{\somme}[2]{\sum_{#1}^{#2}}
\newcommand{\deriveref}[4]{\biggl( \frac{\text{d}^{#1}#2}{\text{d}#3^{#1}} \biggr)_{#4}}





\newcounter{chapitre}
\setcounter{chapitre}{9}

\title{Chapitre 9 : Grammaires non contextuelles}
\maketitle

\section{Grammaire non contextuelle}

\subsection{Vocabulaire}

\begin{definition}{}{grammaire au sens général, grammaire de type 0}
    Une \notion{grammaire} est défini par un quadruplet $(\Sigma,\,V,\,P,\,S)$ où~:
    \begin{itemize}
        \item $\Sigma$ est un alphabet fini de \notion{symboles terminaux}, dit aussi \notion{alphabet terminal}
        \item $V$ est un alphabet fini de \notion{symboles non terminaux} (ou \notion{variables}), dit aussi \notion{alphabet non terminal}
        \item $P \subset (\Sigma \cup V)^* \times (\Sigma \cup V)^* $ est un ensemble de \notion{règles de production}.\\
        Une règle de production $(w_1,w_2) \in P$, notée $w_1 \to w_2$ est un couple de mots écrits avec des symboles terminaux et non terminaux.
        \item $S \in V$ est un symbole non terminal avec un statut particulier de \notion{symbole initial} (ou \notion{axiome}, \notion{variable initiale})
    \end{itemize}
    Une grammaire sans propriété particulière est dite \notion{type 0}.
\end{definition}

\begin{remarque}{}{grammaires}
    On note usuellement par des majuscules les symboles non terminaux, et en minuscule les terminaux.
\end{remarque}

\newcommand{\rp}[2]{\code{#1}\to\code{#2}}


\begin{exemple}{}{de grammaire de type 0}
    Pour $\Sigma = \{a\}$ et $V = \{S,D,F,X,Y,Z\}$
    et $P = \{\rp{S}{DXaD},\rp{Xa}{aaX},\rp{XF}{YF},\rp{aY}{Ya},\rp{DY}{DX},\rp{XZ}{Z},\rp{aZ}{Za},\code{DZ}\to \epsilon\}$, $G = (\Sigma, V, P, S)$ est une grammaire de type 0.
\end{exemple}

\begin{definition}{}{dérivation immédiate}
    Soit $G = (\Sigma, V, P, S)$ une grammaire. On dit que $\alpha \in (\Sigma \cup V)^*$ se dérive immédiatement en $\beta \in (\Sigma \cup V)^*$ lorsqu'il existe $(\alpha_2,\beta_2)\in P$ tel que~:
    $$\exists (\alpha_1, \alpha_3) \in \Big((\Sigma \cup V)^*\Big)^2,\, \begin{cases*}
        \alpha = \alpha_1\alpha_2\alpha_3\\
        \beta = \alpha_1 \beta_2 \alpha_3
    \end{cases*}$$
    Le cas échéant, on note $\alpha \Rightarrow  \beta$. On parle de \notion{dérivation immédiate}. Moralement, le facteur $\alpha_2$ est remplacé par le facteur $\beta_2$.
\end{definition}

\newcommand{\crt}[]{\Rightarrow ^*}


\begin{definition}{}{clôture reflexive et transitive}
    On note $\crt$ la \notion{clôture reflexive et transitive} de la relation $\Rightarrow $ de dérivabilité immédiate.\\\\
    $\crt$ est définie comme la plus petite relation au sens de l'inclusion tel que~:
    \begin{itemize}
        \item $\forall \alpha \in (\Sigma \cup V)^*,\, \alpha \crt \alpha$
        \item $\forall (\alpha,\beta) \in \Big((\Sigma \cup V)^*\Big)^2, (\alpha \Rightarrow  \beta) \implies (\alpha \crt \beta)$
        \item $\forall (\alpha,\beta, \gamma) \in \Big((\Sigma \cup V)^*\Big)^3,\, (\alpha \crt \beta \text{ et } \beta \crt \gamma) \implies (\alpha \crt \gamma)$
    \end{itemize}
    Autrement dit, $\alpha \crt \beta$ lorsqu'il existe $(\alpha=\alpha_0,\, \dots,\, \alpha_k = \beta)$ une suite de mots dans $(\Sigma \cup V)^*$ telle que~: 
    $$\forall i \in \intint{0}{k-1}, \alpha_i \Rightarrow  \alpha_{i+1}$$
\end{definition}

\begin{exemple}{}{de dérivation}
    Dans la grammaire précédemment introduite~:\\
    Pour $\Sigma = \{a\}$ et $V = \{S,D,F,X,Y,Z\}$
    et $P = \{\rp{S}{DXaD},\rp{Xa}{aaX},\rp{XF}{YF},\rp{aY}{Ya},\rp{DY}{DX},\rp{XZ}{Z},\rp{aZ}{Za},\code{DZ}\to \epsilon\}$, $G = (\Sigma, V, P, S)$ est une grammaire de type 0.
    \begin{align*}
        S &\Rightarrow \code{DXaF} \\
        &\Rightarrow \code{DaaXF}\\
        &\Rightarrow \code{DaaYF}\\
        &\Rightarrow \code{DaYaF}\\
        &\Rightarrow \code{DYaaF}\\
        &\Rightarrow \code{DXaaF}\\
        &\Rightarrow \code{DaaXaF}\\
        &\Rightarrow \code{DaaaaXF}\\
        &\Rightarrow \code{DaaaaZ}\\
        &\Rightarrow \code{DaaaZa}\\
        &\Rightarrow \code{DaaZaa}\\
        &\Rightarrow \code{DaZaaa}\\
        &\Rightarrow \code{DZaaaa}\\
        &\Rightarrow \code{aaaa}
    \end{align*}
    D'où $S \crt \code{aaaa}$
\end{exemple}

\begin{definition}{}{langage engendré par une grammaire depuis un mot}
    Soit $G = (\Sigma, V, , S)$ une grammaire et $\alpha \in (\Sigma\cup V)^*$.\\
    On définit le \notion{langage engendré par $G$ depuis $\alpha$} comme l'ensemble des mots de $\Sigma^*$ que l'on peut obtenir par dérivation depuis $\alpha$ en utilisant les règles de production de $G$.\\
    $$\mc{L}_G = \{u \in \Sigma^*,\, \alpha \crt u\}$$
    Le \notion{langage élargi engendré par $G$ depuis $\alpha$} est~:
    $$\widehat{\mc{L}_G(\alpha)} = \{\beta \in (\Sigma \cup V)^*,\, \alpha \crt \beta\}$$
    Le \notion{langage engendré par $G$} désigne $\mc{L}_G(S)$ le langage engendré par $G$ depuis le symbole initial.
\end{definition}

\begin{exemple}{}{}
    Pour la grammaire de l'exemple, on pourrait montrer que $\mc{L}_G(S) = \{a^{2^n},\, n \in \N^*\}$
    On montre que ce langage n'est pas régulier (absurde + lemme de l'étoile tmtc)
\end{exemple}

\begin{definition}{}{langage de type 0}
    On dit qu'un \notion{langage est de type 0} s'il peut être engendré par une grammaire de type 0.
\end{definition}

\begin{theoreme}{}{de Chomsky (HP)}
    Les langages de type 0 sont exactement les langages récursivement énumérables, c'est-à-dire les langages reconnaissables par une machine de Turing.
\end{theoreme}



\end{adjustwidth}
\end{document}