\documentclass{article}
\usepackage{amsmath,amssymb,mathtools}
\usepackage{xcolor}
\usepackage{minted}
\usepackage{enumitem}
\usepackage{multicol}
\usepackage{changepage}
\usepackage{stmaryrd}
\usepackage{graphicx}
\graphicspath{ {./images/} }
\usepackage[framemethod=tikz]{mdframed}
\usepackage{tikz,pgfplots}
\pgfplotsset{compat=1.18}

% physique
\definecolor{oranges}{RGB}{255, 242, 230}
\definecolor{rouges}{RGB}{255, 230, 230}
\definecolor{rose}{RGB}{255, 204, 204}

% maths - info
\definecolor{rouge_fonce}{RGB}{204, 0, 0}
\definecolor{rouge}{RGB}{255, 0, 0}
\definecolor{bleu_fonce}{RGB}{0, 0, 255}
\definecolor{vert_fonce}{RGB}{0, 69, 33}
\definecolor{vert}{RGB}{0,255,0}

\definecolor{orange_foncee}{RGB}{255, 153, 0}
\definecolor{myrtille}{RGB}{225, 225, 255}
\definecolor{mayonnaise}{RGB}{255, 253, 233}
\definecolor{magenta}{RGB}{224, 209, 240}
\definecolor{pomme}{RGB}{204, 255, 204}
\definecolor{mauve}{RGB}{255, 230, 255}


% Cours

\newmdenv[
  nobreak=true,
  topline=true,
  bottomline=true,
  rightline=true,
  leftline=true,
  linewidth=0.5pt,
  linecolor=black,
  backgroundcolor=mayonnaise,
  innerleftmargin=10pt,
  innerrightmargin=10pt,
  innertopmargin=5pt,
  innerbottommargin=5pt,
  skipabove=\topsep,
  skipbelow=\topsep,
]{boite_definition}

\newmdenv[
  nobreak=false,
  topline=true,
  bottomline=true,
  rightline=true,
  leftline=true,
  linewidth=0.5pt,
  linecolor=white,
  backgroundcolor=white,
  innerleftmargin=10pt,
  innerrightmargin=10pt,
  innertopmargin=5pt,
  innerbottommargin=5pt,
  skipabove=\topsep,
  skipbelow=\topsep,
]{boite_exemple}

\newmdenv[
  nobreak=true,
  topline=true,
  bottomline=true,
  rightline=true,
  leftline=true,
  linewidth=0.5pt,
  linecolor=black,
  backgroundcolor=magenta,
  innerleftmargin=10pt,
  innerrightmargin=10pt,
  innertopmargin=5pt,
  innerbottommargin=5pt,
  skipabove=\topsep,
  skipbelow=\topsep,
]{boite_proposition}

\newmdenv[
  nobreak=true,
  topline=true,
  bottomline=true,
  rightline=true,
  leftline=true,
  linewidth=0.5pt,
  linecolor=black,
  backgroundcolor=white,
  innerleftmargin=10pt,
  innerrightmargin=10pt,
  innertopmargin=5pt,
  innerbottommargin=5pt,
  skipabove=\topsep,
  skipbelow=\topsep,
]{boite_demonstration}

\newmdenv[
  nobreak=true,
  topline=true,
  bottomline=true,
  rightline=true,
  leftline=true,
  linewidth=0.5pt,
  linecolor=white,
  backgroundcolor=white,
  innerleftmargin=10pt,
  innerrightmargin=10pt,
  innertopmargin=5pt,
  innerbottommargin=5pt,
  skipabove=\topsep,
  skipbelow=\topsep,
]{boite_remarque}


\newenvironment{definition}[2]
{
    \vspace{15pt}
    \begin{boite_definition}
    \textbf{\textcolor{rouge}{Définition #1}}
    \if\relax\detokenize{#2}\relax
    \else
        \textit{ - #2}
    \fi \\ \\
}
{
    \end{boite_definition}
    
}

\newenvironment{exemple}[2]
{
    \vspace{15pt}
    \begin{boite_exemple}
    \textbf{\textcolor{bleu_fonce}{Exemple #1}}
    \if\relax\detokenize{#2}\relax
    \else
        \textit{ - #2}
    \fi \\ \\ 
}
{   
    \end{boite_exemple}
    
}

\newenvironment{proposition}[2]
{
    \vspace{15pt}
    \begin{boite_proposition}
    \textbf{\textcolor{rouge}{Proposition #1}}
    \if\relax\detokenize{#2}\relax
    \else
        \textit{ - #2}
    \fi \\ \\
}
{
    \end{boite_proposition}
    
}

\newenvironment{theoreme}[2]
{
    \vspace{15pt}
    \begin{boite_proposition}
    \textbf{\textcolor{rouge}{Théorème #1}} 
    \if\relax\detokenize{#2}\relax
    \else
        \textit{ - #2}
    \fi \\ \\
}
{
    \end{boite_proposition}
    
}

\newenvironment{demonstration}
{
    \vspace{15pt}
    \begin{boite_demonstration}
    \textbf{\textcolor{rouge}{Démonstration}}\\ \\
}
{
    \end{boite_demonstration}
    
}

\newenvironment{remarque}[2]
{
    \vspace{15pt}
    \begin{boite_remarque}
    \textbf{\textcolor{bleu_fonce}{Remarque #1}}
    \if\relax\detokenize{#2}\relax
    \else
        \textit{ - #2}
    \fi \\ \\   
}
{  
    \end{boite_remarque}
    
}



%Corrections
\newmdenv[
  nobreak=true,
  topline=true,
  bottomline=true,
  rightline=true,
  leftline=true,
  linewidth=0.5pt,
  linecolor=black,
  backgroundcolor=mayonnaise,
  innerleftmargin=10pt,
  innerrightmargin=10pt,
  innertopmargin=5pt,
  innerbottommargin=5pt,
  skipabove=\topsep,
  skipbelow=\topsep,
]{boite_question}


\newenvironment{question}[2]
{
    \vspace{15pt}
    \begin{boite_question}
    \textbf{\textcolor{rouge}{Question #1}}
    \if\relax\detokenize{#2}\relax
    \else
        \textit{ - #2}
    \fi \\ \\
}
{
    \end{boite_question}
    
}

\newenvironment{enumeratebf}{
    \begin{enumerate}[label=\textbf{\arabic*.}]
}
{
    \end{enumerate}
}

\begin{document}
\begin{adjustwidth}{-3cm}{-3cm}
\begin{document}
\begin{adjustwidth}{-3cm}{-3cm}
% commandes
\newcommand{\notion}[1]{\textcolor{vert_fonce}{\textit{#1}}}
\newcommand{\mb}[1]{\mathbb{#1}}
\newcommand{\mc}[1]{\mathcal{#1}}
\newcommand{\mr}[1]{\mathrm{#1}}
\newcommand{\code}[1]{\texttt{#1}}
\newcommand{\ccode}[1]{\texttt{|#1|}}
\newcommand{\ov}[1]{\overline{#1}}
\newcommand{\abs}[1]{|#1|}
\newcommand{\rev}[1]{\texttt{reverse(#1)}}
\newcommand{\crev}[1]{\texttt{|reverse(#1)|}}

\newcommand{\ie}{\textit{i.e.} }

\newcommand{\N}{\mathbb{N}}
\newcommand{\R}{\mathbb{R}}
\newcommand{\C}{\mathbb{C}}
\newcommand{\K}{\mathbb{K}}
\newcommand{\Z}{\mathbb{Z}}

\newcommand{\A}{\mathcal{A}}
\newcommand{\bigO}{\mathcal{O}}
\renewcommand{\L}{\mathcal{L}}

\newcommand{\rg}[0]{\mathrm{rg}}
\newcommand{\re}[0]{\mathrm{Re}}
\newcommand{\im}[0]{\mathrm{Im}}
\newcommand{\cl}[0]{\mathrm{cl}}
\newcommand{\grad}[0]{\vec{\mathrm{grad}}}
\renewcommand{\div}[0]{\mathrm{div}\,}
\newcommand{\rot}[0]{\vec{\mathrm{rot}}\,}
\newcommand{\vnabla}[0]{\vec{\nabla}}
\renewcommand{\vec}[1]{\overrightarrow{#1}}
\newcommand{\mat}[1]{\mathrm{Mat}_{#1}}
\newcommand{\matrice}[1]{\mathcal{M}_{#1}}
\newcommand{\sgEngendre}[1]{\left\langle #1 \right\rangle}
\newcommand{\gpquotient}[1]{\mathbb{Z} / #1\mathbb{Z}}
\newcommand{\norme}[1]{||#1||}
\renewcommand{\d}[1]{\,\mathrm{d}#1}
\newcommand{\adh}[1]{\overline{#1}}
\newcommand{\intint}[2]{\llbracket #1 ,\, #2 \rrbracket}
\newcommand{\seg}[2]{[#1\, ; \, #2]}
\newcommand{\scal}[2]{( #1 | #2 )}
\newcommand{\distance}[2]{\mathrm{d}(#1,\,#2)}
\newcommand{\inte}[2]{\int_{#1}^{#2}}
\newcommand{\somme}[2]{\sum_{#1}^{#2}}
\newcommand{\deriveref}[4]{\biggl( \frac{\text{d}^{#1}#2}{\text{d}#3^{#1}} \biggr)_{#4}}





\newcounter{chapitre}
\setcounter{chapitre}{9}
\title{Chapitre 9 : Grammaires non contextuelles}
\maketitle

\section{Grammaire non contextuelle (ou hors-contexte)}

\subsection{Vocabulaire}

\begin{definition}{}{grammaire au sens général, grammaire de type 0}
    Une \notion{grammaire} est défini par un quadruplet $(\Sigma,\,V,\,P,\,S)$ où~:
    \begin{itemize}
        \item $\Sigma$ est un alphabet fini de \notion{symboles terminaux}, dit aussi \notion{alphabet terminal}
        \item $V$ est un alphabet fini de \notion{symboles non terminaux} (ou \notion{variables}), dit aussi \notion{alphabet non terminal}
        \item $P \subset (\Sigma \cup V)^* \times (\Sigma \cup V)^* $ est un ensemble de \notion{règles de production}.\\
        Une règle de production $(w_1,w_2) \in P$, notée $w_1 \to w_2$ est un couple de mots écrits avec des symboles terminaux et non terminaux.
        \item $S \in V$ est un symbole non terminal avec un statut particulier de \notion{symbole initial} (ou \notion{axiome}, \notion{variable initiale})
    \end{itemize}
    Une grammaire sans propriété particulière est dite \notion{type 0}.
\end{definition}

\begin{remarque}{}{grammaires}
    On note usuellement par des majuscules les symboles non terminaux, et en minuscule les terminaux.
\end{remarque}

\newcommand{\rp}[2]{\code{#1}\to\code{#2}}


\begin{exemple}{}{de grammaire de type 0}
    Pour $\Sigma = \{\code{a}\}$, $S = \code{S}$, $V = \{S,\code{D},\code{F},\code{X},\code{Y},\code{Z}\}$
    et $P = \{\rp{S}{DXaD},\,\rp{Xa}{aaX},\,\rp{XF}{YF},\,\rp{aY}{Ya},\,\rp{DY}{DX},\,\rp{XZ}{Z},\,\rp{aZ}{Za},\,\code{DZ}\to \epsilon\}$, $G = (\Sigma, V, P, S)$ est une grammaire de type 0.
\end{exemple}

\begin{definition}{}{dérivabilité immédiate}
    Soit $G = (\Sigma, V, P, S)$ une grammaire. Soit $\alpha$ et $\beta$ deux mots de $(\Sigma \cup V)^*$. On dit que \notion{$\alpha$ se dérive immédiatement en $\beta$} lorsqu'il existe $(\alpha_1,\beta_1)\in P$ tel que~:
    $$\exists (u, v) \in \Big((\Sigma \cup V)^*\Big)^2,\, \begin{cases*}
        \alpha = u\alpha_1v\\
        \beta = u \beta_1 v
    \end{cases*}$$
    Le cas échéant, on note $\alpha \Rightarrow  \beta$. On parle de \notion{dérivation immédiate}. Moralement, la règle de production $(\alpha_1,\beta_1)$ remplace le facteur $\alpha_1$ par le facteur $\beta_1$.
\end{definition}

\newcommand{\crt}{\Rightarrow ^*}


\begin{definition}{}{clôture reflexive et transitive}
    On note $\crt$ la \notion{clôture reflexive et transitive} de la relation $\Rightarrow $ de dérivabilité immédiate.\\\\
    $\crt$ est définie comme la plus petite relation au sens de l'inclusion tel que~:
    \begin{itemize}
        \item $\forall \alpha \in (\Sigma \cup V)^*,\, \alpha \crt \alpha$
        \item $\forall (\alpha,\beta) \in \Big((\Sigma \cup V)^*\Big)^2, (\alpha \Rightarrow  \beta) \implies (\alpha \crt \beta)$
        \item $\forall (\alpha,\beta, \gamma) \in \Big((\Sigma \cup V)^*\Big)^3,\, (\alpha \crt \beta \text{ et } \beta \crt \gamma) \implies (\alpha \crt \gamma)$
    \end{itemize}
    Autrement dit, $\alpha \crt \beta$ lorsqu'il existe $(\alpha=\alpha_0,\, \dots,\, \alpha_k = \beta)$ une suite de mots dans $(\Sigma \cup V)^*$ telle que~: 
    $$\forall i \in \intint{0}{k-1}, \alpha_i \Rightarrow  \alpha_{i+1}$$
\end{definition}

\begin{definition}{}{clôture reflexive et transitive de $\Rightarrow$}
    Soit $G = (\Sigma, V, P, S)$ une grammaire. Soit $\alpha$ et $\beta$ deux mots de $(\Sigma \cup V)^*$. On note \notion{$\crt$ la clôture reflexive et transitive de la relation $\Rightarrow $}. Ainsi $\alpha \crt \beta$ lorsqu'il existe $(\alpha=\alpha_0,\, \dots,\, \alpha_k = \beta)$ une suite de mots dans $(\Sigma \cup V)^*$ telle que~: 
    $$\forall i \in \intint{0}{k-1}, \alpha_i \Rightarrow  \alpha_{i+1}$$
\end{definition}

\begin{exemple}{}{de dérivation}
    Dans la grammaire précédemment introduite~:\\
    Pour $\Sigma = \{a\}$ et $V = \{S,D,F,X,Y,Z\}$
    et $P = \{\rp{S}{DXaD},\rp{Xa}{aaX},\rp{XF}{YF},\rp{aY}{Ya},\rp{DY}{DX},\rp{XZ}{Z},\rp{aZ}{Za},\code{DZ}\to \epsilon\}$, $G = (\Sigma, V, P, S)$ est une grammaire de type 0.
    \begin{align*}
        S &\Rightarrow \code{DXaF} \\
        &\Rightarrow \code{DaaXF}\\
        &\Rightarrow \code{DaaYF}\\
        &\Rightarrow \code{DaYaF}\\
        &\Rightarrow \code{DYaaF}\\
        &\Rightarrow \code{DXaaF}\\
        &\Rightarrow \code{DaaXaF}\\
        &\Rightarrow \code{DaaaaXF}\\
        &\Rightarrow \code{DaaaaZ}\\
        &\Rightarrow \code{DaaaZa}\\
        &\Rightarrow \code{DaaZaa}\\
        &\Rightarrow \code{DaZaaa}\\
        &\Rightarrow \code{DZaaaa}\\
        &\Rightarrow \code{aaaa}
    \end{align*}
    D'où $S \crt \code{aaaa}$
\end{exemple}

\begin{definition}{}{langage engendré, langage élargi engendré par une grammaire depuis un mot}
    Soit $G = (\Sigma, V, P , S)$ une grammaire et $\alpha \in (\Sigma\cup V)^*$.
    \begin{itemize}
        \item le \notion{langage engendré par $G$ depuis $\alpha$} est l'ensemble des mots de $\Sigma^*$ que l'on peut obtenir par dérivation de $\alpha$ en utilisant les règles de production de $G$~:
        $$\mc{L}_G(\alpha) = \{\beta \in \Sigma^*,\, \alpha \crt \beta\}$$
        \item le \notion{langage élargi engendré par $G$ depuis $\alpha$} est l'ensemble des mots de $(\Sigma \cup V)^*$ que l'on peut obtenir par dérivation de $\alpha$ en utilisant les règles de production de $G$~:
        $$\widehat{\mc{L}_G(\alpha)} = \{\beta \in (\Sigma \cup V)^*,\, \alpha \crt \beta\}$$
    \end{itemize}
\end{definition}

\begin{definition}{}{langage engendré par une grammaire}
    Soit $G = (\Sigma, V, P , S)$ une grammaire et $\alpha \in (\Sigma\cup V)^*$. Le \notion{langage engendré par $G$} désigne $\mc{L}_G(S)$ le langage engendré par $G$ depuis son symbole initial $S$.
\end{definition}

\begin{exemple}{}{}
    Pour la grammaire de l'exemple, on pourrait montrer que $\mc{L}_G(S) = \{a^{2^n},\, n \in \N^*\}$
    On montre que ce langage n'est pas régulier (absurde + lemme de l'étoile tmtc)
\end{exemple}

\begin{definition}{}{langage de type 0}
    On dit qu'un \notion{langage est de type 0} s'il peut être engendré par une grammaire de type 0.
\end{definition}

\begin{theoreme}{}{de Chomsky (HP)}
    Les langages de type 0 sont exactement les langages récursivement énumérables, c'est-à-dire les langages reconnaissables par une machine de Turing.
\end{theoreme}

\begin{definition}{}{grammaire contextuelle (HP)}
    Une grammaire $G=(\Sigma, V,P,S)$ de type 0 est appelée \notion{grammaire contextuelle (ou de type 1 ou monotone)} lorsque~:
    $$\forall (\alpha,\beta) \in P \setminus \{(S,\epsilon)\},\, \abs{\alpha} \leq \abs{\beta}$$
    Moralement, tous les facteurs "produits" sont plus long que les facteurs remplacés.
\end{definition}

\begin{exemple}{}{de grammaire qui n'est pas "contextuelle"}
    la grammaire exemple définie par $\Sigma = \{a\}$ et $V = \{S,D,F,X,Y,Z\}$ et $P = \{\rp{S}{DXaD},\rp{Xa}{aaX},\rp{XF}{YF},\rp{aY}{Ya},\rp{DY}{DX},\rp{XZ}{Z},\rp{aZ}{Za},\code{DZ}\to \epsilon\}$ puis $G = (\Sigma, V, P, S)$ n'est pas contextuelle ~:
    $$(\code{DZ}\rightarrow \epsilon) \in P$$
    mais~:
    $$\abs{DZ} = 2 > \abs{\epsilon} = 0$$
\end{exemple}

\begin{remarque}{}{indépendance des caractères "contextuel" et "non contextuel"}
    Nous le reverrons, mais une grammaire est "contextuelle" indépendamment de son caractère "non contextuel".
\end{remarque}

\begin{remarque}{}{}
    Sans l'autorisation d'avoir $(S,\epsilon) \in P$,\, les langages engendrés par des grammaires contextuelles ne peuvent pas contenir $\epsilon$.\\
    En effet, on a initialement $S$ de longueur 1 et les règles de productions ne peuvent qu'augmenter la longueur du mot. 0 serait donc une longueur inexistente.
\end{remarque}

\begin{definition}{}{grammaire non contextuelle}
    Une grammaire $G = (\Sigma, V, P, S)$ est dite \notion{non contextuelle} (ou \notion{hors contexte} ou encore \notion{algébrique} ou \notion{de type 2}) lorsque~:
    $$P \subset V \times (\Sigma \cup V)^*$$
    Moralement, les règles de production ne permettent de remplacer que des symboles non terminaux (et pas des mots) : des majuscules.
\end{definition}

\begin{exemple}{}{de règle de production non valide pour une grammaire "non contextuelle"}
    $\code{DZ}\rightarrow \epsilon$ n'est pas une règle possible pour une grammaire hors contexte.
\end{exemple}

\begin{exemple}{}{de grammaire "non contextuelle"}
    $G_1 = (\Sigma_1, V_1, P_1, S_1)$ définie par~:
    \begin{itemize}
        \item $\Sigma_1 = \{a,b\}$
        \item $V_1 = \{S_1\}$
        \item $P_1 = \{S_1 \rightarrow a S_1 b,\,S_1\rightarrow \epsilon\}$
    \end{itemize}
    est une grammaire hors contexte.
\end{exemple}

\begin{remarque}{}{}
    On dit "hors contexte" car les symboles non terminaux sont considérés seuls par les règles de production : on ne regarde pas les lettres autour.
\end{remarque}



PASSE 1
ajouter à l'exemple : 

$\widehat{\mc{L}_{G_1}(S_1)} = \{a^n S_1 b^n,\, n \in \N\} \cup \{a^n b^n ,\, n \in \N\}$ pour comprendre le chapeau.


Voir démo 1

\begin{exemple}{}{Langage de Dyck}
    Le langage de Dyck (bon parenthésage) est engendré par~:
    \begin{itemize}
        \item $\Sigma = \{\code{(}, \code{)}\}$
        \item $S \rightarrow {(S)S}$
        \item $S \rightarrow \epsilon$
    \end{itemize}
\end{exemple}

\begin{remarque}{}{grammaires "non contextuelles" uniquement définies par les règles de production}
    En général, on donne une grammaire hors-contexte uniquement avec les règles de production.\\
    On ne précise le symbole initial que si ce n'est pas $S$.\\
    Les symboles non terminaux sont exactement ceux que l'on rencontre à gauche des règles et les terminaux sont les autres qu'on rencontre.
\end{remarque}

\begin{remarque}{}{"hors-contexte" $\nRightarrow$ "régulier"}
    Les exemples montrent qu'il existe des langages hors contexte qui ne sont pas réguliers.
\end{remarque}

\begin{remarque}{}{abréviation d'une famille de règles de production}
    Pour noter rapidement un ensemble de règles de production utilisant le même symbole non terminal comme départ~:
    $$\{\rp{X,$\alpha_1$},\, \dots,\, \rp{X}{$\alpha_k$}\} \subset P$$
    avec $X\in V$,\, on note~:
    $$X \rightarrow \alpha_1 \vert \cdots \vert \alpha_k$$
    Ce n'est qu'une notation, on a bien $k$ règles de production derrière ça.
\end{remarque}

\begin{exemple}{}{exploitant ce raccourci}
    $G_1$ s'écrit~:
    $$S_1 \rightarrow aS_1b \vert \epsilon$$
\end{exemple}

\subsection{Langages réguliers et langages hors-contexte}

Un résultat à connaître et savoir redémontrer.
\begin{proposition}{}{"régulier" $\implies$ "hors-contexte"}
    Soit $\Sigma$ un alphabet, Tout langage sur $\Sigma$ régulier est "hors-contexte".
\end{proposition}
Voir Démo 2 (elle est incomplète)

\begin{remarque}{}{concernant ce résultat}
    L'inclusion réciproque est fausse si $\abs{Sigma} \geq 2$~:
    $$\{a^nb^n,\, n \in \N\}$$
    est un langage hors-contexte mais non régulier.
\end{remarque}

cette définition sera rappelée, mais il faut vite la comprendre ! FC où on donne la déf, et on doit interpréter !
\begin{definition}{}{grammaire linéaire (HP)}
    $G = (\Sigma, V, P, S)$ est dite~:
    \begin{itemize}
        \item \notion{linéaire} si~:
        $$P \subset V \times (\Sigma^* \cup \Sigma^*V\Sigma^*)$$
        Moralement, on a toujours au plus un symbole non terminal (voir Fig.1)
        \item \notion{linéaire gauche} lorsque~:
        $$P \subset V \times \bigg( \Sigma^* \cup (V \Sigma^*) \bigg)$$
        Le facteur "produit" commence par un unique symbole non terminal ou n'en a aucun.
        \item \notion{linéaire droit} lorsque~:
        $$P \subset V \times \bigg( \Sigma^* \cup (Sigma^*V) \bigg)$$
    \end{itemize}
\end{definition}

\begin{remarque}{}{chaine d'implications}

    linéaire droit implqiue linéaire implique hors contexte
    linéaire gauche implique linéaire implique hors contexte.
    
\end{remarque}

\begin{definition}{}{langages linéaire, linéaire droit, linéaire gauche}
    Un langage est dit linéaire (resp. simple, droit, gauche) lorsqu'il peut être engendré par une grammaire linéaire (resp. simple, linéaire droite).
\end{definition}

\begin{remarque}{}{existence langages linéaires non réguliers}
    $$\code{S} \rightarrow \code{aSb} \vert \epsilon$$ est linéaire et engendre $\{a^n b^n,\, n \in \N\}$
\end{remarque}

\begin{proposition}{}{pour un langage, régulier $\ssi$ linéaire droit $\ssi$ linéaire gauche} 
    Soit $L$ un langage. Les propriétés suivantes sont équivalentes.
    \begin{enumeratebf}
        \item $L$ est régulier
        \item $L$ est linéaire droit
        \item $L$ est linéaire gauche
    \end{enumeratebf}
\end{proposition}

\begin{remarque}{}{sur la démo}
    En montrant régulier implqiue linéaire droit, on montre à nouveau que régulier implqiue hors contexte, ceci constitue une autre démo que la propriété précédente.
\end{remarque}

Démo 3.


\end{adjustwidth}
\end{document}