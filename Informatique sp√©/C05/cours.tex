\documentclass{article}
\usepackage{amsmath,amssymb,mathtools}
\usepackage{xcolor}
\usepackage{minted}
\usepackage{enumitem}
\usepackage{multicol}
\usepackage{changepage}
\usepackage{stmaryrd}
\usepackage{graphicx}
\graphicspath{ {./images/} }
\usepackage[framemethod=tikz]{mdframed}
\usepackage{tikz,pgfplots}
\pgfplotsset{compat=1.18}

% physique
\definecolor{oranges}{RGB}{255, 242, 230}
\definecolor{rouges}{RGB}{255, 230, 230}
\definecolor{rose}{RGB}{255, 204, 204}

% maths - info
\definecolor{rouge_fonce}{RGB}{204, 0, 0}
\definecolor{rouge}{RGB}{255, 0, 0}
\definecolor{bleu_fonce}{RGB}{0, 0, 255}
\definecolor{vert_fonce}{RGB}{0, 69, 33}
\definecolor{vert}{RGB}{0,255,0}

\definecolor{orange_foncee}{RGB}{255, 153, 0}
\definecolor{myrtille}{RGB}{225, 225, 255}
\definecolor{mayonnaise}{RGB}{255, 253, 233}
\definecolor{magenta}{RGB}{224, 209, 240}
\definecolor{pomme}{RGB}{204, 255, 204}
\definecolor{mauve}{RGB}{255, 230, 255}


% Cours

\newmdenv[
  nobreak=true,
  topline=true,
  bottomline=true,
  rightline=true,
  leftline=true,
  linewidth=0.5pt,
  linecolor=black,
  backgroundcolor=mayonnaise,
  innerleftmargin=10pt,
  innerrightmargin=10pt,
  innertopmargin=5pt,
  innerbottommargin=5pt,
  skipabove=\topsep,
  skipbelow=\topsep,
]{boite_definition}

\newmdenv[
  nobreak=false,
  topline=true,
  bottomline=true,
  rightline=true,
  leftline=true,
  linewidth=0.5pt,
  linecolor=white,
  backgroundcolor=white,
  innerleftmargin=10pt,
  innerrightmargin=10pt,
  innertopmargin=5pt,
  innerbottommargin=5pt,
  skipabove=\topsep,
  skipbelow=\topsep,
]{boite_exemple}

\newmdenv[
  nobreak=true,
  topline=true,
  bottomline=true,
  rightline=true,
  leftline=true,
  linewidth=0.5pt,
  linecolor=black,
  backgroundcolor=magenta,
  innerleftmargin=10pt,
  innerrightmargin=10pt,
  innertopmargin=5pt,
  innerbottommargin=5pt,
  skipabove=\topsep,
  skipbelow=\topsep,
]{boite_proposition}

\newmdenv[
  nobreak=true,
  topline=true,
  bottomline=true,
  rightline=true,
  leftline=true,
  linewidth=0.5pt,
  linecolor=black,
  backgroundcolor=white,
  innerleftmargin=10pt,
  innerrightmargin=10pt,
  innertopmargin=5pt,
  innerbottommargin=5pt,
  skipabove=\topsep,
  skipbelow=\topsep,
]{boite_demonstration}

\newmdenv[
  nobreak=true,
  topline=true,
  bottomline=true,
  rightline=true,
  leftline=true,
  linewidth=0.5pt,
  linecolor=white,
  backgroundcolor=white,
  innerleftmargin=10pt,
  innerrightmargin=10pt,
  innertopmargin=5pt,
  innerbottommargin=5pt,
  skipabove=\topsep,
  skipbelow=\topsep,
]{boite_remarque}


\newenvironment{definition}[2]
{
    \vspace{15pt}
    \begin{boite_definition}
    \textbf{\textcolor{rouge}{Définition #1}}
    \if\relax\detokenize{#2}\relax
    \else
        \textit{ - #2}
    \fi \\ \\
}
{
    \end{boite_definition}
    
}

\newenvironment{exemple}[2]
{
    \vspace{15pt}
    \begin{boite_exemple}
    \textbf{\textcolor{bleu_fonce}{Exemple #1}}
    \if\relax\detokenize{#2}\relax
    \else
        \textit{ - #2}
    \fi \\ \\ 
}
{   
    \end{boite_exemple}
    
}

\newenvironment{proposition}[2]
{
    \vspace{15pt}
    \begin{boite_proposition}
    \textbf{\textcolor{rouge}{Proposition #1}}
    \if\relax\detokenize{#2}\relax
    \else
        \textit{ - #2}
    \fi \\ \\
}
{
    \end{boite_proposition}
    
}

\newenvironment{theoreme}[2]
{
    \vspace{15pt}
    \begin{boite_proposition}
    \textbf{\textcolor{rouge}{Théorème #1}} 
    \if\relax\detokenize{#2}\relax
    \else
        \textit{ - #2}
    \fi \\ \\
}
{
    \end{boite_proposition}
    
}

\newenvironment{demonstration}
{
    \vspace{15pt}
    \begin{boite_demonstration}
    \textbf{\textcolor{rouge}{Démonstration}}\\ \\
}
{
    \end{boite_demonstration}
    
}

\newenvironment{remarque}[2]
{
    \vspace{15pt}
    \begin{boite_remarque}
    \textbf{\textcolor{bleu_fonce}{Remarque #1}}
    \if\relax\detokenize{#2}\relax
    \else
        \textit{ - #2}
    \fi \\ \\   
}
{  
    \end{boite_remarque}
    
}



%Corrections
\newmdenv[
  nobreak=true,
  topline=true,
  bottomline=true,
  rightline=true,
  leftline=true,
  linewidth=0.5pt,
  linecolor=black,
  backgroundcolor=mayonnaise,
  innerleftmargin=10pt,
  innerrightmargin=10pt,
  innertopmargin=5pt,
  innerbottommargin=5pt,
  skipabove=\topsep,
  skipbelow=\topsep,
]{boite_question}


\newenvironment{question}[2]
{
    \vspace{15pt}
    \begin{boite_question}
    \textbf{\textcolor{rouge}{Question #1}}
    \if\relax\detokenize{#2}\relax
    \else
        \textit{ - #2}
    \fi \\ \\
}
{
    \end{boite_question}
    
}

\newenvironment{enumeratebf}{
    \begin{enumerate}[label=\textbf{\arabic*.}]
}
{
    \end{enumerate}
}

\begin{document}
\begin{adjustwidth}{-3cm}{-3cm}
\begin{document}
\begin{adjustwidth}{-3cm}{-3cm}
% commandes
\newcommand{\notion}[1]{\textcolor{vert_fonce}{\textit{#1}}}
\newcommand{\mb}[1]{\mathbb{#1}}
\newcommand{\mc}[1]{\mathcal{#1}}
\newcommand{\mr}[1]{\mathrm{#1}}
\newcommand{\code}[1]{\texttt{#1}}
\newcommand{\ccode}[1]{\texttt{|#1|}}
\newcommand{\ov}[1]{\overline{#1}}
\newcommand{\abs}[1]{|#1|}
\newcommand{\rev}[1]{\texttt{reverse(#1)}}
\newcommand{\crev}[1]{\texttt{|reverse(#1)|}}

\newcommand{\ie}{\textit{i.e.} }

\newcommand{\N}{\mathbb{N}}
\newcommand{\R}{\mathbb{R}}
\newcommand{\C}{\mathbb{C}}
\newcommand{\K}{\mathbb{K}}
\newcommand{\Z}{\mathbb{Z}}

\newcommand{\A}{\mathcal{A}}
\newcommand{\bigO}{\mathcal{O}}
\renewcommand{\L}{\mathcal{L}}

\newcommand{\rg}[0]{\mathrm{rg}}
\newcommand{\re}[0]{\mathrm{Re}}
\newcommand{\im}[0]{\mathrm{Im}}
\newcommand{\cl}[0]{\mathrm{cl}}
\newcommand{\grad}[0]{\vec{\mathrm{grad}}}
\renewcommand{\div}[0]{\mathrm{div}\,}
\newcommand{\rot}[0]{\vec{\mathrm{rot}}\,}
\newcommand{\vnabla}[0]{\vec{\nabla}}
\renewcommand{\vec}[1]{\overrightarrow{#1}}
\newcommand{\mat}[1]{\mathrm{Mat}_{#1}}
\newcommand{\matrice}[1]{\mathcal{M}_{#1}}
\newcommand{\sgEngendre}[1]{\left\langle #1 \right\rangle}
\newcommand{\gpquotient}[1]{\mathbb{Z} / #1\mathbb{Z}}
\newcommand{\norme}[1]{||#1||}
\renewcommand{\d}[1]{\,\mathrm{d}#1}
\newcommand{\adh}[1]{\overline{#1}}
\newcommand{\intint}[2]{\llbracket #1 ,\, #2 \rrbracket}
\newcommand{\seg}[2]{[#1\, ; \, #2]}
\newcommand{\scal}[2]{( #1 | #2 )}
\newcommand{\distance}[2]{\mathrm{d}(#1,\,#2)}
\newcommand{\inte}[2]{\int_{#1}^{#2}}
\newcommand{\somme}[2]{\sum_{#1}^{#2}}
\newcommand{\deriveref}[4]{\biggl( \frac{\text{d}^{#1}#2}{\text{d}#3^{#1}} \biggr)_{#4}}






\begin{definition}{5.1}{apprentissage}
    L'\notion{apprentissage} en informatique permet l'approche de différents problèmes~:
    \begin{itemize}
        \item la \notion{classification}, par exemple déterminer un objet sur une image, ou un son sur un flux audio ;
        \item la \notion{régression}, par exemple prévoir la valeur du cours de la bourse.
    \end{itemize}
\end{definition}

\begin{definition}{5.2}{apprentissage supervisé}
    L'\notion{apprentissage supervisé} consiste à entraîner un modèle ou un algorithme à l'aide d'un \notion{ensemble d'apprentissage}~:
    $$S = \{(x_i,\,  y_i),\, i \in \intint{1}{n}\} \subset X \times Y \quad \text{avec}\,\begin{cases*}
        X &l'ensemble des \notion{objets} manipulés par le modèle ou l'algorithme\\
        Y &l'ensemble des \notion{classes} ou \notion{valeurs} associées aux objets de $X$
    \end{cases*}$$
    $(x,\, y) \in S$ signifie que \notion{l'objet $x$ est dans la classe $y$} ou \notion{a pour valeur $y$}.\\ 
    On cherche pour un objet inconnu $x$ à déterminer une classe ou une valeur $y$ convenable en s'appuyant sur l'ensemble d'apprentissage $S$.
\end{definition}

\begin{definition}{5.3}{fonction de prédiction}
    À tout modèle ou algorithme d'apprentissage supervisé on peut associer une \notion{fonction $f:X \to Y$ dite de prédiction} qui à un objet associe la \notion{classe estimée raisonnable par le modèle ou l'algorithme}.
\end{definition}

\begin{definition}{5.4}{fonction de perte}
    À tout modèle ou algorithme d'apprentissage supervisé on peut associer une fonction $L:Y^2 \to \R_+$ qui à une prédiction associe une \notion{valeur mesurant son inexactitude}. On a~:
    $$\forall y \in Y,\, L(y,\, y) = 0$$
\end{definition}

\begin{definition}{5.5}{fonction de risque empirique}
    Pour tout modèle ou algorithme d'apprentissage supervisé on peut associer à toute fonction $f$ de prédiction une espérance appelée \notion{risque $R$} par~:
    \begin{align*}
        R(f) &= \mb{E}_{X,Y}\Big( L\big( Y,\, f(X) \big) \Big) \\
        &= \sum_{(x,y) \in X\times Y} L \big( y, f(x) \big) \mb{P}_{X,Y}(x,y)
    \end{align*}
    En pratique, on n'a jamais accès à $\mb{P}_{X,Y}$. On définit alors le \notion{risque empirique comme étant la moyenne des pertes sur l'ensemble d'apprentissage}. Lui est calculable~:
    $$R_\mr{emp}(f) = \frac{1}{\abs{S}} \sum_{(x,y) \in S}L(y,f(x))$$
    Dès lors, un algorithme d'apprentissage supervisé mettra en œuvre des algorithmes d'optimisation afin de trouver une fonction $f$ qui minimise le risque empirique.
\end{definition}

\begin{implementation}{algorithme de classification des $k$ plus proches voisins - classique}
    \begin{itemize}
        \item \textbf{Entrée}~: \begin{itemize}
            \item un ensemble d'apprentissage $S \subset X\times Y$ indexé sur $\intint{0}{n}$
            \item une distance $\mr{d} : X^2 \to \R_+$
            \item un objet $x$ de classe inconnue
            \item $k$ le nombre de voisins à considérer
            
        \end{itemize}
        \item \textbf{Sortie}~: la classe $y$ du voisin majoritairement présent parmi les $k$ plus proches de $x$
    \end{itemize}
    \begin{lstLNat}
    file = file de priorité max vide
    pour tout $i \in \intint{0}{n}$ :
        si $\abs{\code{file}}$ < $k$ :
            ajouter $x_i$ à file avec la priorité $\mr{d}(x,x_i)$
        sinon :
            si $d(x,x_i)$ < file[0] :
                supprimer le maximum de file
                ajouter $x_i$ à file avec la priorité $\mr{d}(x,x_i)$
    renvoyer la classe majoritaire parmi les éléments de file
    \end{lstLNat}
\end{implementation}

\end{adjustwidth}
\end{document}