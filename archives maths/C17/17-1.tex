% commandes
\newcommand{\notion}[1]{\textcolor{vert_fonce}{\textit{#1}}}
\newcommand{\mb}[1]{\mathbb{#1}}
\newcommand{\mc}[1]{\mathcal{#1}}
\newcommand{\code}[1]{\texttt{#1}}
\newcommand{\ccode}[1]{\texttt{|#1|}}
\newcommand{\ov}[1]{\overline{#1}}
\newcommand{\abs}[1]{|#1|}
\newcommand{\rev}[1]{\texttt{reverse(#1)}}
\newcommand{\crev}[1]{\texttt{|reverse(#1)|}}

\newcommand{\ie}{\textit{i.e.} }

\newcommand{\N}{\mathbb{N}}
\newcommand{\R}{\mathbb{R}}
\newcommand{\C}{\mathbb{C}}
\newcommand{\K}{\mathbb{K}}

\newcommand{\A}{\mathcal{A}}
\newcommand{\bigO}{\mathcal{O}}
\renewcommand{\L}{\mathcal{L}}

\newcommand{\rg}[0]{\text{rg}}
\newcommand{\re}[0]{\text{Re}}
\newcommand{\im}[0]{\text{Im}}
\newcommand{\cl}[0]{\text{cl}}
\newcommand{\mat}[1]{\text{Mat}_{#1}}
\newcommand{\matrice}[1]{\mathcal{M}_{#1}}
\newcommand{\sgEngendre}[1]{\left\langle #1 \right\rangle}
\newcommand{\norme}[1]{||#1||}
\renewcommand{\d}[1]{\,\text{d}#1}
\newcommand{\intint}[2]{\llbracket #1 ,\, #2 \rrbracket}
\newcommand{\seg}[2]{[#1\, ; \, #2]}
\newcommand{\scal}[2]{\left\langle #1 ,\, #2 \right\rangle}
\newcommand{\inte}[2]{\int_{#1}^{#2}}
\newcommand{\somme}[2]{\sum_{#1}^{#2}}





\input{../../stock/en-tete.tex}
\everymath{\displaystyle}

    \begin{definition}{17.1}{ - fraction rationnelle}
        Dans $\mb{K}[X] \times (\mb{K}[X] \backslash \{0\})$ On définit la relation d'équivalence $\mc{R}$ en posant : \begin{align*}
            &(P,Q)\mc{R}(R,S) &&\\
            \Leftrightarrow \quad& P/Q = R/S && \text{(Cette étape n'est qu'à titre explicatif dans la mesure où l'opération / n'est pas définie)} \\
            \Leftrightarrow \quad& PS = RQ &&\\
        \end{align*}
    On appelle \notion{fraction rationnelle} à coefficients dans $\mb{K}$ toute classe d'équivalence pour la relation $\mc{R}$. La classe de $(P,Q)$ est alors notée $\displaystyle \frac{P}{Q}$. On a donc : 
    $$\frac{P}{Q} = \{(R,S) \in \mb{K}[X] \times (\mb{K}[X] \backslash \{0\}), PS = RQ\}$$
    On dit que $(P,Q)$ est un \notion{représentant} de la fraction $\displaystyle\frac{P}{Q}$. L'ensemble des fractions rationnelles est noté $\mb{K}(X)$ et la relation $\mc{R}$ est appelée \notion{égalité des fractions rationnelles}.

    \end{definition} 

    \begin{proposition}{16.4}{ - structure de $\mb{K}(X)$}
        $(\mb{K}(X),+,\times)$ est un corps commutatif et $(\mb{K}(X),+,\times,\cdot)$ (où $\cdot$ est la loi externe) est une $\mb{K}$-algèbre commutative. \\ \\
        L'application $\varphi : \mb{K}[X] \rightarrow \mb{K}(X)$ définie par $ \varphi(P) = \frac{P}{1}$ est un morphisme d'algèbres injectif.
    \end{proposition}

    \begin{definition}{17.7}{ - représentant irréductible}
        Soit $F = \frac{P}{Q}$ une fraction. On dit que $\frac{P}{Q}$ est un \notion{représentant irréductible} lorsque $P \wedge Q = 1$ et que $Q$ est unitaire. Toute fration rationnelle de $\mb{K}(X)$ admet un \underline{unique} (dénominateur unitaire) représentant irréductible.
    \end{definition}

    \begin{theoreme}{17.34}{ - décomposition en éléments simples}
        Soit $F = \frac{A}{B}$ une fraction sous forme irréductible, et $B = \prod_{i=1}^{k}P_i^{\alpha_i}$ sa décomposition en produit de polynômes irréductibles. Il existe des polynômes $(U_{i})_{i \in \llbracket 1,k \rrbracket}$ tels que $$F = E + \sum_{i=1}^{k}\frac{U_{i}}{P_i^{\alpha_i}} \quad \text{avec $\deg(\frac{U_{i}}{P_i})<0$} $$ \\
        De plus, pour $n \in \mb{N}^*$, Si $T \in \mc{I}_{\mb{K}[X]} $ et $\deg(\frac{A}{T^n})<0$, alors il existe des polynômes $V_1,\dots,V_n$ tels que $$\frac{A}{T^n} = \sum_{k=1}^{n}\frac{V_k}{T^k} \quad \text{avec $\deg(\frac{V_k}{T^k})<0$}$$ \\
        Finalement, Il existe des polynômes $(U_{i,j})_{i \in \llbracket 1,k \rrbracket, j \in \llbracket 1,\alpha_i \rrbracket}$ tels que $$F = E + \sum_{i=1}^{k}\sum_{j=1}^{\alpha_i}\frac{U_{i,j}}{P_i^j} \quad \text{avec $\deg(\frac{U_{i,j}}{P_i})<0$} $$ \\ \\
        Cette décomposition est unique.
    \end{theoreme}
\input{../../stock/pied.tex}