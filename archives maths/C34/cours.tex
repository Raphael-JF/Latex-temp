\documentclass{article}
\usepackage{amsmath,amssymb,mathtools}
\usepackage{esint} % intégrale avec un round
\usepackage{xcolor}
\usepackage{listings}
% \usepackage{minted}
\usepackage{enumitem}
\usepackage{fourier-orns}
\usepackage{multicol}
\usepackage{changepage}
\usepackage{stmaryrd}
\usepackage{graphicx}
\graphicspath{ {./images/} }
\usepackage[framemethod=tikz]{mdframed}
\usepackage{tikz,pgfplots}
\pgfplotsset{compat=1.18}
\usetikzlibrary{arrows}
\usepackage{forest}
\usepackage{hyperref}

% physique
% \renewcommand*{\overrightarrow}[1]{\vbox{\halign{##\cr 
%   \tiny\rightarrowfill\cr\noalign{\nointerlineskip\vskip1pt} 
%   $#1\mskip2mu$\cr}}}

  \newenvironment{enumeratebf}{
    \begin{enumerate}[label=\textbf{\arabic*.}]
}
{
    \end{enumerate}
}
  
\definecolor{oranges}{RGB}{255, 242, 230}
\definecolor{rouges}{RGB}{255, 230, 230}
\definecolor{rose}{RGB}{255, 204, 204}

% maths - info
\definecolor{rouge_fonce}{RGB}{204, 0, 0}
\definecolor{rouge}{RGB}{255, 0, 0}
\definecolor{bleufonce}{RGB}{0, 0, 255}
\definecolor{vert_fonce}{RGB}{0, 69, 33}
\definecolor{vert}{RGB}{0,255,0}

\definecolor{orange_foncee}{RGB}{255, 153, 0}
\definecolor{myrtille}{RGB}{225, 225, 255}
\definecolor{mayonnaise}{RGB}{255, 253, 233}
\definecolor{magenta}{RGB}{224, 209, 240}
\definecolor{pomme}{RGB}{204, 255, 204}
\definecolor{mauve}{RGB}{255, 230, 255}


% Cours

\newmdenv[
  nobreak=true,
  topline=true,
  bottomline=true,
  rightline=true,
  leftline=true,
  linewidth=0.5pt,
  linecolor=black,
  backgroundcolor=mayonnaise,
  innerleftmargin=10pt,
  innerrightmargin=2.5em,
  innertopmargin=5pt,
  innerbottommargin=5pt,
  skipabove=\topsep,
  skipbelow=\topsep,
]{boite_definition}


\newenvironment{definition}[2]
{
    \vspace{15pt}
    \begin{boite_definition}
    \textbf{\textcolor{rouge}{Définition #1}}
    \if\relax\detokenize{#2}\relax
    \else
        \textit{ - #2}
    \fi \\ \\
}
{
    \end{boite_definition}
    
}

\newmdenv[
  nobreak=true,
  topline=true,
  bottomline=true,
  rightline=true,
  leftline=true,
  linewidth=0.5pt,
  linecolor=white,
  backgroundcolor=white,
  innerleftmargin=10pt,
  innerrightmargin=2.5em,
  innertopmargin=5pt,
  innerbottommargin=5pt,
  skipabove=\topsep,
  skipbelow=\topsep,
]{boite_exemple}


\newenvironment{exemple}[2]
{
    \vspace{15pt}
    \begin{boite_exemple}
    \textbf{\textcolor{bleufonce}{Exemple #1}}
    \if\relax\detokenize{#2}\relax
    \else
        \textit{ - #2}
    \fi \\ \\ 
}
{   
    \end{boite_exemple}
    
}


\newmdenv[
  nobreak=true,
  topline=true,
  bottomline=true,
  rightline=true,
  leftline=true,
  linewidth=0.5pt,
  linecolor=black,
  backgroundcolor=magenta,
  innerleftmargin=10pt,
  innerrightmargin=2.5em,
  innertopmargin=5pt,
  innerbottommargin=5pt,
  skipabove=\topsep,
  skipbelow=\topsep,
]{boite_proposition}

\newenvironment{proposition}[2]
{
    \vspace{15pt}
    \begin{boite_proposition}
    \textbf{\textcolor{rouge}{Proposition #1}}
    \if\relax\detokenize{#2}\relax
    \else
        \textit{ - #2}
    \fi \\ \\
}
{
    \end{boite_proposition}
}


\newmdenv[
  nobreak=true,
  topline=true,
  bottomline=true,
  rightline=true,
  leftline=true,
  linewidth=0.5pt,
  linecolor=black,
  backgroundcolor=magenta,
  innerleftmargin=10pt,
  innerrightmargin=2.5em,
  innertopmargin=5pt,
  innerbottommargin=5pt,
  skipabove=\topsep,
  skipbelow=\topsep,
]{boite_theoreme}


\newenvironment{theoreme}[2]
{
    \vspace{15pt}
    \begin{boite_theoreme}
    \textbf{\textcolor{rouge}{Théorème #1}} 
    \if\relax\detokenize{#2}\relax
    \else
        \textit{ - #2}
    \fi \\ \\
}
{
    \end{boite_theoreme}
    
}


\newmdenv[
  nobreak=true,
  topline=true,
  bottomline=true,
  rightline=true,
  leftline=true,
  linewidth=0.5pt,
  linecolor=black,
  backgroundcolor=white,
  innerleftmargin=10pt,
  innerrightmargin=2.5em,
  innertopmargin=5pt,
  innerbottommargin=5pt,
  skipabove=\topsep,
  skipbelow=\topsep,
]{boite_demonstration}


\newenvironment{demonstration}
{
    \vspace{15pt}
    \begin{boite_demonstration}
    \textbf{\textcolor{rouge}{Démonstration}}\\ \\
}
{
    \end{boite_demonstration}
    
}


\newmdenv[
  nobreak=true,
  topline=true,
  bottomline=true,
  rightline=true,
  leftline=true,
  linewidth=0.5pt,
  linecolor=white,
  backgroundcolor=white,
  innerleftmargin=10pt,
  innerrightmargin=2.5em,
  innertopmargin=5pt,
  innerbottommargin=5pt,
  skipabove=\topsep,
  skipbelow=\topsep,
]{boite_remarque}


\newenvironment{remarque}[2]
{
    \vspace{15pt}
    \begin{boite_remarque}
    \textbf{\textcolor{bleufonce}{Remarque #1}}
    \if\relax\detokenize{#2}\relax
    \else
        \textit{ - #2}
    \fi \\ \\   
}
{  
    \end{boite_remarque}
    
}

\newmdenv[
  nobreak=true,
  topline=true,
  bottomline=true,
  rightline=true,
  leftline=true,
  linewidth=0.5pt,
  linecolor=bleufonce,
  backgroundcolor=white,
  innerleftmargin=10pt,
  innerrightmargin=2.5em,
  innertopmargin=5pt,
  innerbottommargin=5pt,
  skipabove=\topsep,
  skipbelow=\topsep,
]{boite_OCaml}


\definecolor{keywordcolor}{RGB}{133, 153, 0}  % les mots-clés
\definecolor{commentcolor}{RGB}{147, 161, 161} % les commentaires
\definecolor{stringcolor}{RGB}{42, 161, 152}  % les chaînes de caractères
\newenvironment{OCaml}[2]
{   
    \vspace{15pt}
    \begin{boite_OCaml}
    \textbf{\textcolor{bleufonce}{Implémentation #1}}
    \if\relax\detokenize{#2}\relax
    \else
        \textit{ - #2}
    \fi \\ \\   

    \lstset{
    language=[Objective]Caml,
    basicstyle=\ttfamily,          % Police par défaut pour le code
    keywordstyle=\color{keywordcolor}, % Mots-clés en bleu doux
    commentstyle=\color{commentcolor}, % Commentaires en vert pâle
    stringstyle=\color{stringcolor},   % Chaînes en orange léger
    backgroundcolor=\color{white},   % Fond très clair
    numbers=left,                  % Numérotation à gauche
    numberstyle=\ttfamily,             % Taille des numéros de ligne
    stepnumber=1,                  % Numérotation de chaque ligne
    frame=single,                  % Cadre autour du code
    breaklines=true,               % Retour à la ligne automatique
    tabsize=2,                        % Taille des tabulations          
    }

    \begin{lstlisting}
}
{      
    \end{lstlisting}
    \end{boite_OCaml}
}

\newmdenv[
  nobreak=true,
  topline=true,
  bottomline=true,
  rightline=true,
  leftline=true,
  linewidth=0.5pt,
  linecolor=black,
  backgroundcolor=mayonnaise,
  innerleftmargin=10pt,
  innerrightmargin=2.5em,
  innertopmargin=5pt,
  innerbottommargin=5pt,
  skipabove=\topsep,
  skipbelow=\topsep,
]{boite_question}


\newenvironment{question}[2]
{
    \vspace{15pt}
    \begin{boite_question}
    \textbf{\textcolor{rouge}{Question #1}}
    \if\relax\detokenize{#2}\relax
    \else
        \textit{ - #2}
    \fi \\ \\
}
{
    \end{boite_question}
    
}

\newmdenv[
  nobreak=true,
  topline=true,
  bottomline=true,
  rightline=true,
  leftline=true,
  linewidth=0.5pt,
  linecolor=black,
  backgroundcolor=white,
  innerleftmargin=10pt,
  innerrightmargin=2.5em,
  innertopmargin=5pt,
  innerbottommargin=5pt,
  skipabove=\topsep,
  skipbelow=\topsep,
]{boite_corollaire}



\newenvironment{corollaire}[2]
{
    \vspace{15pt}
    \begin{boite_corollaire}
    \textbf{\textcolor{rouge}{Corollaire #1}}
    \if\relax\detokenize{#2}\relax
    \else
        \textit{ - #2}
    \fi \\ \\   
}
{
    \end{boite_corollaire}
    
}

\begin{document}
\begin{adjustwidth}{-3cm}{-3cm}
% commandes
\newcommand{\notion}[1]{\textcolor{vert_fonce}{\textit{#1}}}
\newcommand{\mb}[1]{\mathbb{#1}}
\newcommand{\mc}[1]{\mathcal{#1}}
\newcommand{\code}[1]{\texttt{#1}}
\newcommand{\ccode}[1]{\texttt{|#1|}}
\newcommand{\ov}[1]{\overline{#1}}
\newcommand{\abs}[1]{|#1|}
\newcommand{\rev}[1]{\texttt{reverse(#1)}}
\newcommand{\crev}[1]{\texttt{|reverse(#1)|}}

\newcommand{\ie}{\textit{i.e.} }

\newcommand{\N}{\mathbb{N}}
\newcommand{\R}{\mathbb{R}}
\newcommand{\C}{\mathbb{C}}
\newcommand{\K}{\mathbb{K}}

\newcommand{\A}{\mathcal{A}}
\newcommand{\bigO}{\mathcal{O}}
\renewcommand{\L}{\mathcal{L}}

\newcommand{\rg}[0]{\text{rg}}
\newcommand{\re}[0]{\text{Re}}
\newcommand{\im}[0]{\text{Im}}
\newcommand{\cl}[0]{\text{cl}}
\newcommand{\mat}[1]{\text{Mat}_{#1}}
\newcommand{\matrice}[1]{\mathcal{M}_{#1}}
\newcommand{\sgEngendre}[1]{\left\langle #1 \right\rangle}
\newcommand{\norme}[1]{||#1||}
\renewcommand{\d}[1]{\,\text{d}#1}
\newcommand{\intint}[2]{\llbracket #1 ,\, #2 \rrbracket}
\newcommand{\seg}[2]{[#1\, ; \, #2]}
\newcommand{\scal}[2]{\left\langle #1 ,\, #2 \right\rangle}
\newcommand{\inte}[2]{\int_{#1}^{#2}}
\newcommand{\somme}[2]{\sum_{#1}^{#2}}






\begin{definition}{34.11 (1)}{norme associée à un produit scalaire}
    Soit $E$ un espace préhilbertien réel. On appelle \notion{norme euclidienne sur $E$} l'application :
    \begin{align*}
        \norme{.}\quad:\quad&E \to \R_+ \\
        &x \mapsto \sqrt{\scal{x}{x}}
    \end{align*}
    
\end{definition}

\begin{definition}{34.11 (2)}{vecteur unitaire}
    Soit $E$ un espace préhilbertien réel. On dit qu'un vecteur $x \in E$ est \notion{unitaire} si $\norme{x} = 1$. 
    
\end{definition}

\begin{definition}{34.11 (3)}{distance euclidienne}
    Soit $E$ un espace préhilbertien réel. On appelle \notion{distance euclidienne sur $E$} l'application :
    \begin{align*}
        d\quad:\quad&E^2 \to \R_+ \\
        &(x,y) \mapsto \norme{x-y} = \sqrt{\scal{x-y}{x-y}}
    \end{align*}
\end{definition}

\begin{proposition}{34.15}{identité de polarisation}
    Soit $E$ un espace vectoriel. Si existence, le produit vectoriel associé à une norme sur $E$ vérifie :\\
    $$\forall (x,y) \in E^2,\, \scal{x}{y} = \frac{\norme{x+y}^2 - \norme{x}^2 - \norme{y}^2}{2}$$
\end{proposition}

\begin{theoreme}{34.16 (0)}{inégalité de Cauchy-Schwarz}
    Soit $E$ un espace préhilbertien réel, $(x,y) \in E^2$.
    $$\abs{\scal{x}{y}} \leq \norme{x} \times \norme{y}$$
    avec égalité si et seulement si $x$ et $y$ sont colinéaires.
\end{theoreme}

\begin{theoreme}{34.16 (1)}{inégalité triangulaire}
    Soit $E$ un espace préhilbertien réel, $(x,y) \in E^2$.
    $$\abs{\norme{x}-{\norme{y}}} \leq \norme{x+y} \leq \norme{x} + \norme{y}$$
    De plus, 
    $$\norme{x+y} = \norme{x} + \norme{y} \Leftrightarrow \exists \alpha \in \R_+, \,x = \alpha y$$
\end{theoreme}

\begin{definition}{34.19 (1)}{vecteurs orthogonaux}
    Soit $E$ un espace préhilbertien réel, $(x,y) \in E^2$.
    On dit que $x$ et $y$ sont \notion{orthogonaux} si $\scal{x}{y} = 0$. On note alors $x \perp y$.
\end{definition}

\begin{definition}{34.19 (2)}{parties orthogonales}
    Soit $E$ un espace préhilbertien réel, $(X,Y) \in \mc{P}(E)^2$.
    On dit que $X$ et $Y$ sont \notion{orthogonales} si :
    $$\forall (x,y) \in X \times Y, \,\scal{x}{y} = 0$$
    On note alors $X \perp Y$.
\end{definition}

\begin{definition}{34.19 (3)}{famille orthogonale}
    Soit $E$ un espace préhilbertien réel, $(x_i)_{i \in I}$ une famille d'éléments de $E$.
    On dit que $(x_i)_{i \in I}$ est \notion{orthogonale} si :
    $$\forall (i,j) \in I^2,\, i \neq j \implies \scal{x_i}{x_j} = 0$$
\end{definition}

\begin{definition}{34.19 (4)}{famille orthonormée}
    Soit $E$ un espace préhilbertien réel, $(x_i)_{i \in I}$ une famille d'éléments de $E$.
    On dit que $(x_i)_{i \in I}$ est \notion{orthonormée} si elle est \underline{orthogonale} et \underline{constituée de vecteurs unitaires}, \ie :
    $$\forall (i,j) \in I^2, \,\scal{x_i}{x_j} = \delta_{ij}$$
\end{definition}

\begin{proposition}{34.bonus}{caractérisation de norme}
    Soit $E$ un espace préhilbertien réel. Une application $\varphi : E \to \R_+$ est une norme sur $E$ si et seulement si elle vérifie pour tout $(x,y) \in E^2$ :  
    \begin{enumeratebf}
        \item $\forall \lambda \in \K,\, \varphi(\lambda x) = \abs{\lambda} \varphi(x)$
        \item $\varphi(x) = 0 \Leftrightarrow x = 0$
        \item $\varphi(x) + \varphi(y) \geq \varphi(x + y)$
    \end{enumeratebf}
    
\end{proposition}

\begin{theoreme}{34.25}{CS de liberté}
    Soit $E$ un espace préhilbertien réel. Toute famille orthogonale de vecteurs non nuls de $E$ est libre.
\end{theoreme}

\begin{theoreme}{34.26}{coordonnées dans une base orthonormale}
    Soit $E$ un espace euclidien de dimension $n \neq 0$ et $(e_1, \dots, e_n)$ une base orthonormale de $E$ (une famille orthonormale de $n$ éléments).\\
    Les coordonnées de tout vecteur $x \in E$  dans la base $(e_1, \dots, e_n)$ sont $(\scal{x}{e_1}, \dots, \scal{x}{e_n})$.
\end{theoreme}

\begin{theoreme}{34.27}{expression du produit scalaire dans une base orthonormale}
    Soit $E$ un espace euclidien de dimension $n \neq 0$. Soit $x,y \in E$ de coordonnées respectives $X = (x_1, \dots, x_n)$ et $Y = (y_1, \dots, y_n)$ dans une certaine base orthonormale de $E$. On a alors :
    $$\scal{x}{y} = \som{k=1}{n}x_k y_k$$
\end{theoreme}

\begin{theoreme}{34.28}{algorithme d'orthonormalisation de Gram-Schmidt}
    Soit $E$ un espace préhilbertien réel et $F = (f_1, \dots, f_n)$ une famille libre de $E$. À partir de $F$, il est possible de construire une famille orthonormale $(u_1, \dots u_n)$, telle que :
    $$\forall k \in \intint{1}{n},\, \text{Vect}(f_1, \dots, f_k) = \text{Vect}(u_1, \dots, u_k)$$
    Pour tout $k \in \intint{1}{n}$, le vecteur $u_k$ est donné par :
    $$  u_k = \pm \frac{\displaystyle f_k - \som{i=1}{k-1}\scal{f_k}{u_i}u_i}{\displaystyle \norme{f_k - \som{i=1}{k-1}\scal{f_k}{u_i}u_i}}$$
\end{theoreme}

\begin{theoreme}{34.38 (0)}{supplémentaire orthogonal d’un sev de dimension finie}
    Soit $E$ un espace préhilbertien réel et $F$ un sev de dimension finie de $E$.\\
    $F^\bot$ est l'unique supplémentaire de $F$ dans $E$, orthogonal à $E$. On l'appelle le \notion{supplémentaire orthogonal} de $F$ dans $E$.
\end{theoreme}


\begin{theoreme}{34.38 (1)}{supplémentaire orthogonal d’un sev de dimension finie}
    Soit $E$ un espace préhilbertien réel et $F$ un sev de dimension finie de $E$. On a :
    $$F = F^{\bot \bot}$$
\end{theoreme}

\input{../../stock/pied.tex}