\documentclass{article}
\usepackage{amsmath,amssymb}
\usepackage{xcolor}
\usepackage{enumitem}
\usepackage{multicol}
\usepackage{pdflscape}
\usepackage{changepage}

\definecolor{oranges}{RGB}{255, 242, 230}
\definecolor{rouges}{RGB}{255, 230, 230}

\definecolor{saumon}{RGB}{224, 209, 240}
\renewcommand{\labelitemi}{--}
\begin{document}
\begin{adjustwidth}{-3cm}{-3cm}

% \begin{landscape}
    \pagecolor{saumon}
    \noindent Soit $D \subset \mathbb{R}$, $a \in D$, $f:D \rightarrow D$ une fonction et $(u_n) \in D^{\mathbb{N}}$ l'unique suite telle que $u_0 = a$ et $\forall n \in \mathbb{N}, u_{n+1} = f(u_n)$.
    \begin{enumerate}[label=\textbf{\arabic*.}]

        \item Le signe de $x \mapsto f(x) - x$ renseigne sur la monotonie de $(u_n)$ :\begin{equation*}\begin{cases}
            \forall x \in D, f(x) \geq x \implies \forall n \in \mathbb{N}, u_{n+1} \geq u_n \\
            \forall x \in D, f(x) \leq x \implies \forall n \in \mathbb{N}, u_{n+1} \leq u_n 
            \end{cases}\end{equation*}

        \item  Si $f$ est croissante, alors $(u_n)$ est :
        \begin{itemize}
            \item croissante si $u_1 \geq u_0$
            \item décroissante si $u_1 \leq u_0$
        \end{itemize}

        \item si $f$ est décroissante, alors $(u_{2n})$ et $(u_{2n+1})$ sont monotones et de sens contraire :
        \begin{itemize}
            \item si $u_2 \geq u_0$ alors $(u_{2n})$ est croissante et $(u_{2n+1})$ est décroissante
            \item si $u_2 \leq u_0$ alors $(u_{2n})$ est décroissante et $(u_{2n+1})$ est croissante
        \end{itemize}


        
        \end{enumerate}


% \end{landscape}
\end{adjustwidth}
\end{document}