\documentclass{article}
\usepackage{titling} % Personnalisation du titre
\usepackage[left=20mm, right=20mm]{geometry}
\usepackage{amsmath,amssymb,mathtools}
\usepackage{esint} % intégrale avec un round
\usepackage{xcolor}
\usepackage[utf8]{inputenc}
\usepackage{listings}
\usepackage{enumitem}
\usepackage{multicol}
\usepackage{stmaryrd}
\usepackage{graphicx}
\graphicspath{ {./images/} }
\usepackage[framemethod=tikz]{mdframed}
\usepackage{tikz,pgfplots}
\pgfplotsset{compat=1.18}
\usetikzlibrary{arrows}
\usepackage{forest}
\usepackage{titlesec}
\setlength{\parindent}{0pt}

\pretitle{\begin{center}\Huge\bfseries}
\posttitle{\end{center}}
\date{}
\renewcommand{\thesection}{\Roman{section}} 
\titleformat{\section}
  {\Large\bfseries} % Style du titre
  {\thesection} % Numéro de section
  {0.5cm} % Espacement entre numéro et titre
  {} % Pas de préfixe supplémentaire
\titleformat{\subsection}
  {\large\bfseries} % Style du titre
  {\thesubsection} % Numéro de section
  {0.4cm} % Espacement entre numéro et titre
  {} % Pas de préfixe supplémentair


\newenvironment{enumeratebf}{
    \begin{enumerate}[label=\textbf{\arabic*.}]
}
{
    \end{enumerate}
}
  
\definecolor{oranges}{RGB}{255, 242, 230}
\definecolor{rouges}{RGB}{255, 230, 230}
\definecolor{rose}{RGB}{255, 204, 204}

% maths - info
\definecolor{rouge_fonce}{RGB}{204, 0, 0}
\definecolor{rouge}{RGB}{255, 0, 0}
\definecolor{bleufonce}{RGB}{0, 0, 255}
\definecolor{vert_fonce}{RGB}{0, 69, 33}
\definecolor{vert}{RGB}{0,255,0}

\definecolor{orange_foncee}{RGB}{255, 153, 0}
\definecolor{myrtille}{RGB}{225, 225, 255}
\definecolor{mayonnaise}{RGB}{255, 253, 233}
\definecolor{magenta}{RGB}{224, 209, 240}
\definecolor{pomme}{RGB}{204, 255, 204}
\definecolor{mauve}{RGB}{255, 230, 255}


% Cours

\newmdenv[
    nobreak=true,
    topline=true,
    bottomline=true,
    rightline=true,
    leftline=true,
    linewidth=0.5pt,
    linecolor=black,
    backgroundcolor=mayonnaise,
    innerleftmargin=10pt,
    innerrightmargin=2.5em,
    innertopmargin=5pt,
    innerbottommargin=5pt,
    skipabove=\topsep,
    skipbelow=\topsep,
]{boite_definition}

\newcounter{boite}
\setcounter{boite}{1}
\newenvironment{definition}[2]
{
    \vspace{15pt}
    \begin{boite_definition}
    \if\relax\detokenize{#1}\relax
        \textbf{\textcolor{rouge}{Définition \arabic{chapitre}.\arabic{boite}}}%
        \if\relax\detokenize{#2}\relax
        \else
            \textit{ - #2}
        \fi
        \stepcounter{boite}
    \else
        \textbf{\textcolor{rouge}{Définition #1}}%
        \if\relax\detokenize{#2}\relax
        \else
            \textit{ - #2}
        \fi
    \fi \\
    
    
}
{
    \end{boite_definition}
    \vspace{10pt}
}

\newmdenv[
  nobreak=true,
  topline=true,
  bottomline=true,
  rightline=true,
  leftline=true,
  linewidth=0.5pt,
  linecolor=white,
  backgroundcolor=white,
  innerleftmargin=10pt,
  innerrightmargin=2.5em,
  innertopmargin=5pt,
  innerbottommargin=5pt,
  skipabove=\topsep,
  skipbelow=\topsep,
]{boite_exemple}


\newenvironment{exemple}[2]
{
    \vspace{15pt}
    \begin{boite_exemple}
    \if\relax\detokenize{#1}\relax
        \textbf{\textcolor{bleufonce}{Exemple \arabic{chapitre}.\arabic{boite}}}%
        \if\relax\detokenize{#2}\relax
        \else
            \textit{ - #2}
        \fi
        \stepcounter{boite}
    \else
        \textbf{\textcolor{bleufonce}{Exemple #1}}%
        \if\relax\detokenize{#2}\relax
        \else
            \textit{ - #2}
        \fi
    \fi \\
    
    
}
{
    \end{boite_exemple}
    \vspace{10pt}
}

\newmdenv[
  nobreak=true,
  topline=true,
  bottomline=true,
  rightline=true,
  leftline=true,
  linewidth=0.5pt,
  linecolor=black,
  backgroundcolor=magenta,
  innerleftmargin=10pt,
  innerrightmargin=2.5em,
  innertopmargin=5pt,
  innerbottommargin=5pt,
  skipabove=\topsep,
  skipbelow=\topsep,
]{boite_proposition}

\newenvironment{proposition}[2]
{
    \vspace{15pt}
    \begin{boite_proposition}
    \if\relax\detokenize{#1}\relax
        \textbf{\textcolor{rouge}{Proposition \arabic{chapitre}.\arabic{boite}}}%
        \if\relax\detokenize{#2}\relax
        \else
            \textit{ - #2}
        \fi
        \stepcounter{boite}
    \else
        \textbf{\textcolor{rouge}{Proposition #1}}%
        \if\relax\detokenize{#2}\relax
        \else
            \textit{ - #2}
        \fi
    \fi \\
    
    
}
{
    \end{boite_proposition}
}

\newmdenv[
  nobreak=true,
  topline=true,
  bottomline=true,
  rightline=true,
  leftline=true,
  linewidth=0.5pt,
  linecolor=black,
  backgroundcolor=magenta,
  innerleftmargin=10pt,
  innerrightmargin=2.5em,
  innertopmargin=5pt,
  innerbottommargin=5pt,
  skipabove=\topsep,
  skipbelow=\topsep,
]{boite_theoreme}


\newenvironment{theoreme}[2]
{
    \vspace{15pt}
    \begin{boite_theoreme}
    \if\relax\detokenize{#1}\relax
        \textbf{\textcolor{rouge}{Théorème \arabic{chapitre}.\arabic{boite}}}%
        \if\relax\detokenize{#2}\relax
        \else
            \textit{ - #2}
        \fi
        \stepcounter{boite}
    \else
        \textbf{\textcolor{rouge}{Théorème #1}}%
        \if\relax\detokenize{#2}\relax
        \else
            \textit{ - #2}
        \fi
    \fi \\
    
    
}
{
    \end{boite_theoreme}
}


\newmdenv[
  nobreak=true,
  topline=true,
  bottomline=true,
  rightline=true,
  leftline=true,
  linewidth=0.5pt,
  linecolor=black,
  backgroundcolor=white,
  innerleftmargin=10pt,
  innerrightmargin=2.5em,
  innertopmargin=5pt,
  innerbottommargin=5pt,
  skipabove=\topsep,
  skipbelow=\topsep,
]{boite_demonstration}


\newenvironment{demonstration}
{
    \vspace{15pt}
    \begin{boite_demonstration}
    \textbf{\textcolor{rouge}{Démonstration}}\\ \\
}
{
    \end{boite_demonstration}
    
}


\newmdenv[
  nobreak=true,
  topline=true,
  bottomline=true,
  rightline=true,
  leftline=true,
  linewidth=0.5pt,
  linecolor=white,
  backgroundcolor=white,
  innerleftmargin=10pt,
  innerrightmargin=2.5em,
  innertopmargin=5pt,
  innerbottommargin=5pt,
  skipabove=\topsep,
  skipbelow=\topsep,
]{boite_remarque}


\newenvironment{remarque}[2]
{
    \vspace{15pt}
    \begin{boite_remarque}
    \if\relax\detokenize{#1}\relax
        \textbf{\textcolor{bleufonce}{Remarque \arabic{chapitre}.\arabic{boite}}}%
        \if\relax\detokenize{#2}\relax
        \else
            \textit{ - #2}
        \fi
        \stepcounter{boite}
    \else
        \textbf{\textcolor{bleufonce}{Remarque #1}}%
        \if\relax\detokenize{#2}\relax
        \else
            \textit{ - #2}
        \fi
    \fi \\
    
    
}
{
    \end{boite_remarque}
}

\newmdenv[
  nobreak=true,
  topline=true,
  bottomline=true,
  rightline=true,
  leftline=true,
  linewidth=0.5pt,
  linecolor=bleufonce,
  backgroundcolor=white,
  innerleftmargin=10pt,
  innerrightmargin=2.5em,
  innertopmargin=5pt,
  innerbottommargin=5pt,
  skipabove=\topsep,
  skipbelow=\topsep,
]{boite_implementation}


\definecolor{keywordcolor}{RGB}{133, 153, 0}  % les mots-clés
\definecolor{commentcolor}{RGB}{147, 161, 161} % les commentaires
\definecolor{stringcolor}{RGB}{42, 161, 152}  % les chaînes de caractères

\lstnewenvironment{lstOCaml}
{\lstset{
    language=[Objective]Caml,
    basicstyle=\ttfamily,
    keywordstyle=\color{keywordcolor},
    commentstyle=\color{commentcolor},
    stringstyle=\color{stringcolor},
    backgroundcolor=\color{white},
    numbers=left,
    numberstyle=\ttfamily,
    numbersep=-1.5em,
    stepnumber=1,
    frame=l,
    framexleftmargin=-2.25em,
    tabsize=2,
    literate=%
    {é}{{\'e}}{1}%
    {è}{{\`e}}{1}%
    {à}{{\`a}}{1}%
    {ç}{{\c{c}}}{1}%
    {œ}{{\oe}}{1}%
    {ù}{{\`u}}{1}%
    {É}{{\'E}}{1}%
    {È}{{\`E}}{1}%
    {À}{{\`A}}{1}%
    {Ç}{{\c{C}}}{1}%
    {Œ}{{\OE}}{1}%
    {Ê}{{\^E}}{1}%
    {ê}{{\^e}}{1}%
    {î}{{\^i}}{1}%
    {ô}{{\^o}}{1}%
    {û}{{\^u}}{1}%
    {ä}{{\"{a}}}1
    {ë}{{\"{e}}}1
    {ï}{{\"{i}}}1
    {ö}{{\"{o}}}1
    {ü}{{\"{u}}}1
    {û}{{\^{u}}}1
    {â}{{\^{a}}}1
    {Â}{{\^{A}}}1
    {Î}{{\^{I}}}1
}}{}
 
\lstnewenvironment{lstC}
{\lstset{
    language=C,
    basicstyle=\ttfamily,
    keywordstyle=\color{keywordcolor},
    commentstyle=\color{commentcolor},
    stringstyle=\color{stringcolor},
    backgroundcolor=\color{white},
    numbers=left,
    numberstyle=\ttfamily,
    numbersep=-1.5em,
    stepnumber=1,
    frame=l,
    framexleftmargin=-2.25em,
    tabsize=2,
    literate=%
    {é}{{\'e}}{1}%
    {è}{{\`e}}{1}%
    {à}{{\`a}}{1}%
    {ç}{{\c{c}}}{1}%
    {œ}{{\oe}}{1}%
    {ù}{{\`u}}{1}%
    {É}{{\'E}}{1}%
    {È}{{\`E}}{1}%
    {À}{{\`A}}{1}%
    {Ç}{{\c{C}}}{1}%
    {Œ}{{\OE}}{1}%
    {Ê}{{\^E}}{1}%
    {ê}{{\^e}}{1}%
    {î}{{\^i}}{1}%
    {ô}{{\^o}}{1}%
    {û}{{\^u}}{1}%
    {ä}{{\"{a}}}1
    {ë}{{\"{e}}}1
    {ï}{{\"{i}}}1
    {ö}{{\"{o}}}1
    {ü}{{\"{u}}}1
    {û}{{\^{u}}}1
    {â}{{\^{a}}}1
    {Â}{{\^{A}}}1
    {Î}{{\^{I}}}1
}}{}


\lstdefinelanguage{LNat}{
    morekeywords={tant,que,pour,tout,si,sinon,initialiser,renvoyer,attendre la fin, afficher},
    sensitive=false,
    morecomment=[l]{//},
}

\lstnewenvironment{lstLNat}
{\lstset{
    language=LNat,
    basicstyle=\ttfamily,
    keywordstyle=\color{keywordcolor},
    commentstyle=\color{commentcolor},
    stringstyle=\color{stringcolor},
    backgroundcolor=\color{white},
    numbers=left,
    numberstyle=\ttfamily,
    numbersep=-1.5em,
    stepnumber=1,
    frame=l,
    mathescape=true,
    framexleftmargin=-2.25em,
    tabsize=2,
    literate=%
    {é}{{\'e}}{1}%
    {è}{{\`e}}{1}%
    {à}{{\`a}}{1}%
    {ç}{{\c{c}}}{1}%
    {œ}{{\oe}}{1}%
    {ù}{{\`u}}{1}%
    {É}{{\'E}}{1}%
    {È}{{\`E}}{1}%
    {À}{{\`A}}{1}%
    {Ç}{{\c{C}}}{1}%
    {Œ}{{\OE}}{1}%
    {Ê}{{\^E}}{1}%
    {ê}{{\^e}}{1}%
    {î}{{\^i}}{1}%
    {ô}{{\^o}}{1}%
    {û}{{\^u}}{1}%
    {ä}{{\"{a}}}1
    {ë}{{\"{e}}}1
    {ï}{{\"{i}}}1
    {ö}{{\"{o}}}1
    {ü}{{\"{u}}}1
    {û}{{\^{u}}}1
    {â}{{\^{a}}}1
    {Â}{{\^{A}}}1
    {Î}{{\^{I}}}1}
}{}

\newenvironment{implementation}[1]
{   
    \vspace{15pt}
    \begin{boite_implementation}
    \textbf{\textcolor{bleufonce}{Implémentation}}\textit{ - #1}
     \\ \\
}
{    
    \end{boite_implementation}
}

\newmdenv[
  nobreak=true,
  topline=true,
  bottomline=true,
  rightline=true,
  leftline=true,
  linewidth=0.5pt,
  linecolor=black,
  backgroundcolor=mayonnaise,
  innerleftmargin=10pt,
  innerrightmargin=2.5em,
  innertopmargin=5pt,
  innerbottommargin=5pt,
  skipabove=\topsep,
  skipbelow=\topsep,
]{boite_question}


\newenvironment{question}[2]
{
    \vspace{15pt}
    \begin{boite_question}
    \if\relax\detokenize{#1}\relax
        \textbf{\textcolor{rouge}{Question \arabic{chapitre}.\arabic{boite}}}%
        \if\relax\detokenize{#2}\relax
        \else
            \textit{ - #2}
        \fi
        \stepcounter{boite}
    \else
        \textbf{\textcolor{rouge}{Question #1}}%
        \if\relax\detokenize{#2}\relax
        \else
            \textit{ - #2}
        \fi
    \fi \\
    
    
}
{
    \end{boite_question}
}

\newmdenv[
  nobreak=true,
  topline=true,
  bottomline=true,
  rightline=true,
  leftline=true,
  linewidth=0.5pt,
  linecolor=black,
  backgroundcolor=white,
  innerleftmargin=10pt,
  innerrightmargin=2.5em,
  innertopmargin=5pt,
  innerbottommargin=5pt,
  skipabove=\topsep,
  skipbelow=\topsep,
]{boite_corollaire}



\newenvironment{corollaire}[2]
{
    \vspace{15pt}
    \begin{boite_corollaire}
    \if\relax\detokenize{#1}\relax
        \textbf{\textcolor{rouge}{Corollaire \arabic{chapitre}.\arabic{boite}}}%
        \if\relax\detokenize{#2}\relax
        \else
            \textit{ - #2}
        \fi
        \stepcounter{boite}
    \else
        \textbf{\textcolor{rouge}{Corollaire #1}}%
        \if\relax\detokenize{#2}\relax
        \else
            \textit{ - #2}
        \fi
    \fi \\
    
    
}
{
    \end{boite_corollaire}
}


\title{\Large Étude d'un texte \\ \Huge Michel Deneken}
\date{14 mars 2025}
\begin{document}
% commandes
\newcommand{\notion}[1]{\textcolor{vert_fonce}{\textit{#1}}}
\newcommand{\mb}[1]{\mathbb{#1}}
\newcommand{\mc}[1]{\mathcal{#1}}
\newcommand{\code}[1]{\texttt{#1}}
\newcommand{\ccode}[1]{\texttt{|#1|}}
\newcommand{\ov}[1]{\overline{#1}}
\newcommand{\abs}[1]{|#1|}
\newcommand{\rev}[1]{\texttt{reverse(#1)}}
\newcommand{\crev}[1]{\texttt{|reverse(#1)|}}

\newcommand{\ie}{\textit{i.e.} }

\newcommand{\N}{\mathbb{N}}
\newcommand{\R}{\mathbb{R}}
\newcommand{\C}{\mathbb{C}}
\newcommand{\K}{\mathbb{K}}

\newcommand{\A}{\mathcal{A}}
\newcommand{\bigO}{\mathcal{O}}
\renewcommand{\L}{\mathcal{L}}

\newcommand{\rg}[0]{\text{rg}}
\newcommand{\re}[0]{\text{Re}}
\newcommand{\im}[0]{\text{Im}}
\newcommand{\cl}[0]{\text{cl}}
\newcommand{\mat}[1]{\text{Mat}_{#1}}
\newcommand{\matrice}[1]{\mathcal{M}_{#1}}
\newcommand{\sgEngendre}[1]{\left\langle #1 \right\rangle}
\newcommand{\norme}[1]{||#1||}
\renewcommand{\d}[1]{\,\text{d}#1}
\newcommand{\intint}[2]{\llbracket #1 ,\, #2 \rrbracket}
\newcommand{\seg}[2]{[#1\, ; \, #2]}
\newcommand{\scal}[2]{\left\langle #1 ,\, #2 \right\rangle}
\newcommand{\inte}[2]{\int_{#1}^{#2}}
\newcommand{\somme}[2]{\sum_{#1}^{#2}}





\maketitle

\section{Préambule}

Texte de 1200 mots, à résumer en 200. C'est du type Centrale-Supélec. Texte de Michel Deneken. Deneken parle d'un penseur : Emmanuel Mounier, auteur du vingtième siècle. Le but n'est pas de faire une réinterprétation mais de réduire le texte. Ce doit être comme si l'auteur résumait son propre texte.\\\\

Thèse du texte : L'individualisme et la tyranie collective, bien qu'opposés vont détruire la personne (terme qu'on voit plus employé que l'individu).\\\\

\section{Méthode}
Pas plus de 4 paragraphes~:
\begin{enumeratebf}
    \item I - introduction, plus court que II et III
    \item II - plus long
    \item III - plus long
    \item IV - conclusion, plus court que II et III
\end{enumeratebf}

\begin{remarque}{FONDAMENTALE}{forme}
    Il est possible de présenter un "//" tous les 50 mots, aux deux concours. Toujours indiquer le nombre de mots à la fin.
\end{remarque}

\begin{remarque}{1}{concernant cette structure}
    On conserve la présence d'une introduction et d'une conclusion.
\end{remarque}

\begin{remarque}{2}{Temps alloué au résumé}
    Ne pas allouer au plus 1h30 au résumé, 2h30 c'est le strict minimum pour une dissertation. Traiter le résumé avant la dissertation.
\end{remarque}

\begin{remarque}{3}{}
    Il est intéressant de chercher les paragraphe qui se répètent pour sabrer au mieux.
\end{remarque}


\begin{remarque}{4}{l'importance de la nuance}
    Les quatre paragraphes doivent avoir un lien. Le III doit faire une allusion au II pour montrer l'enchaînement logique, la \textbf{nuance} du texte.
\end{remarque}

\begin{remarque}{5}{citations}
    Au lieu d'écrire E. Macron, écrire Macron.
\end{remarque}

\section{lecture}
\subsection{Partie 1 dans le texte}
communauté, personne humaine minée par l'individualisme et la tyrannie collective~:
\begin{itemize}
    \item individualisme : primauté de l'individu sur la communauté, il est au centre de la société~:
    \begin{itemize}
        \item libéralisme économique
        \item libéralisme politique
    \end{itemize}
    Paradoxalement (car l'individualisme pourrait permettre plus de liberté), cette liberté détruit l'individu selon Mounier
    \item tyrannie collective : systèmes totalitaires. Il est assez évident que ça détruit la personne car elle n'est plus prise en compte et mise à mal dans son intégrité.
\end{itemize}

\subsection{partie 2}
Elle termine un paragraphe qui met bien l'introduction.\\
Notion de société vitale à la personne humaine : contractualisme (contrat social de Rousseau). On la voit beaucoup où le droit et la loi sont fondamentaux dans la société, quitte à déshumaniser les relations en les contractualisant.\\
On avance sur la notion de personne humaine~:
\begin{itemize}
    \item digne de respect
    \item qui entre en relation avec d'autres personnes, sans contract
\end{itemize}

\subsection{partie 3 : tout un paragraphe.}
Aide à comprendre la personne humaine, mais moins précieux dans le résumé. On découvre une dérive de l'individualisme : le narcissisme et l'égocentrique.\\
"Sortir de soi"

\subsection{partie 4}
On se répète avec le dernier paragraphe

\subsection{partie 5}

Mounier n'attaque pas complètement l'individualisme mais il reconnaît que le libéralisme philosophique a fait reconnaître comme individu à part entière des entités précédemment infériosées, comme la femme qui n'existait pas autrefois comme individu autonome, ou encore l'enfant, qui était considéré comme la partie négligeable de la famille. On doit tout ce progrès ça à l'individualisme.

\subsection{partie 6 : une citation de Mounier}
Pas question bien sûr de refaire la citation, mais ça aide à comprendre la personne humaine~:
\begin{itemize}
    \item vivre avec un maximum d'initiative
    \item vivre avec un maximum de responsabilité
    \item s'élever spirituellement
\end{itemize}
On parle de "communauté personnaliste" (on appelle ce courant de pensée d'Emmanuel Mounier).

\subsection{partie 7}
Mounier par le de "communauté totalitaire" : le personnalisme constitue une attitude intégrale. Ne pas confondre avec les régimes totalitaires. C'est différent.

\subsection{partie 8}
On voit qu'il faut souligner cette idée de cohésion communautaire qui s'oppose à cette atomisation des individus dans la pensée libérale

\subsection{partie 9}
Toute communauté aspire à s'ériger comme une personne.

\subsection{partie 10}
Une idée intéressante : séparation entre communauté et communautarisme.
\begin{itemize}
    \item la communauté promeut l'engagement et l'ouverture, le naturel
    \item le communautarisme promeut le repli sur soi, l'artificiel
\end{itemize}
Le terme d'utopie apparaît pour la première fois.

\subsection{partie 11}

Lien avec le paragraphe précédent. Celui-ci précise le précédent.\\
La communauté serait plutôt une orientation de vie, une façon pour chacun d'orienter son existence, une sagesse de vie, plutôt  qu'un projet politique ou religieux. On retrouve la discussion à la place du contrat.

\subsection{partie 12 : conclusion}
référence à l'existentialisme : à la différence de l'existentialisme, la personne est envisagée comme une communion.


\section{Corrigée}

\subsection{introduction}
Raconte comment l'esprit de communauté s'est perdu dans la société contemporaine

\subsection{Partie 1}
Les mots en gras (y, inversement) désignent un lien avec le paragraphe précédent : ça montre au correcteur qu'un lien a bien été fait et qu'on évite la compilation de paragraphes. le "inversement" désigne l'opposition entre communauté et individualisme, totalitarisme.\\
Le terme chrétien doit appraître.\\
La dernière ligne montre qu'avoir le choix ne suffit pas à faire une personne.

\subsection{partie 2}
lien avec le paragraphe précédent

\subsection{3}


\end{document}