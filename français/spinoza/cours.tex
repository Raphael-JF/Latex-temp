\documentclass[a4paper,12pt]{article}

% Encodage et langue
\usepackage[utf8]{inputenc}
\usepackage[T1]{fontenc}
\usepackage[french]{babel}

\newcommand{\ttp}[0]{\underline{\textit{Traité Théologico-politique}} }


% Marges et mise en page
\usepackage[margin=2.5cm]{geometry}

% Titre du document
\title{Cours de Français\\ \Large \textit{\ttp de Baruch Spinoza (1632-1677)}}
\author{Raphaël JONTEF}
\date{\today}

\begin{document}

\maketitle

% \section*{Introduction : \ttp de Baruch Spinoza (1632-1677)}
% Ce document présente un cours de français sur la structure et les éléments clés d'un texte.

\section{Contexte de l'oeuvre}
Le \ttp est publié anonymement, avec une adresse d'éditeur fictive. Il s'agit de précautions pour éviter à Spinoza de s'exposer à des poursuites. \\
Malgré ces précautions, le livre cause un scandale immédiat. Rapidement, on accuse Spinoza en le qualifiant du "Juif athée de Voorburg". \\
Dès 1674, le \ttp est banni dans les Provinces Unies (actuel Pays-bas), dans la foulée de l'interdition du Léviathan de Thomas Hobbes. Pourtant le livre se fait passer sous le manteau. \\ \\
On accuse Johan de Witt, protecteur de l'oeuvre de Spinoza, de promouvoir une oeuvre si scandaleuse, il est battu à mort en public. \\
\vspace{1em}
Le \ttp invite à lutter contre deux phénomènes~:
\begin{itemize}
    \item la dégénérescence des religions en superstitions : le fait que la religion débouche sur une hiérarchie sociale ;
    \item la tendance dogmatique des Églises. (En philosophie, le dogmatisme est péjoratif, il décrit la tendance à constamment affirmer sans démontrer)
\end{itemize}
\vspace{1em}
Pour éviter ce déclin de la religion, Spinoza distingue deux types de connaissances~:
\begin{itemize}
    \item la \textit{connaissance révélée} : ce qui est censé être transmis directement par Dieu (révélé a éthymologiquement un sens théologique). La \textit{connaissance révélée} est essentielle car elle permet la concorde parmi les individus.
    \item la \textit{connaissance naturelle} : ce qui découle de la science et de l'usage de la raison. C'est ce qui permet d'établir les lois de la nature.
\end{itemize}
\\La séparation de ces connaissances permet de libérer les opinions individuelles sur le plan de la foi et de l'état. La connaissance révélée n'a aucun impact sur la connaissance naturelle et inversement.

Spinoza pose deux seules limites à la liberté de philopher, la première étant que les opinions ne nuisent à personne et la deuxième est que cette opinion ne trouble pas l'ordre public, qu'elle n'engendre pas de séditions infondées.
\\Spinoza veut libérer la communauté des dominations stériles. Il veut au contraire favoriser la \textit{nature des individus} : la volonté de toujours persévérer dans son être (penser aux plantes qui poussent).
\\\\Chacun est poussé à exercer sa pleine puissance, individuellement et collectivement. C'est dans cette perspective que le pouvoir des prêtres peut être si toxique. D'une part ils vont chercher à limiter les droits de l'individu, d'autre part il vont agiter les foules contre les responsables politiques pour espérer prendre leur place.\\\\

\newcommand{\etat}[]{État }


Interlude concernant le titre du \ttp : Théologico-politique et pourtant si clivant entre ces deux notions. Spinoza avance que la religion et l'État sont entièrement à distinguer, avec primat de ce dernier sur le religieux. Spinoza établit enfin une tension entre la souvraineté de l'\etat qui pour lui doit être absolue : l'\etat doit être seul chef de décision.\\\\

À l'époque de Spinoza, la Libre République d'Amsterdam et plus généralement les Provinces Unies sont les lieu le plus ouverts aux courants de pensée progressistes. C'est d'ailleurs là que s'est réfugié Descartes, pour écrire librement. Une génération après est venu Spinoza au moment où cette liberté commençait à s'effriter, à cause d'une alliance entre les monarchistes absolutistes et les théologiens dogmatiques (intégristes calvinistes, protestants des plus âpres) luttant contre la République d'Amsterdam. Spinoza est persécuté par cette alliance. Quand il écrit le \ttp, c'est justement pour défendre le parti adverse. C'est aussi à titre personnel pour lutter contre les soupçons d'athéisme qui pesaient sur lui.

\section{L'œuvre}

Un des objectifs de l'oeuvre est d'interroger la présence du Christianisme à travers les saintes écritures. Il les aborde de façon rationnelle, chose inédite. Le livre commence par 15 chapitres plus théoligiques portant sur la doctrine (HP). Dans la seconde partie du \ttp, Spinoza défend la thèse selon laquelle la liberté de pensée de l'individu rationnel (pas du ravagé) est nécessaire à l'obéissance de chacun aux lois de l'\etat.\\(

Le \ttp s'oppose à l'idéologie du théocratisme : fait que les pouvoirs politique et religieux coïncident (comme dans le cas du peuple Hébreu : principe théocratique enseigné par Dieu, transmis par Moïse). Au cours du livre, Spinoza tente de démontrer que le souverain doit décider de la loi religieuse, comme de la loi civile : que le politique doit avoir le primat sur le religieux afin d'éviter que ce dernier n'engendre un \etat dans l'\etat.\\

Rappelons la \textbf{progression de l'oeuvre} :
\begin{itemize}
    \item Chapitre 16 : Spinoza montre les limites du pouvoir étatique (de l'état). C'est le chapitre le plus riche de l'œuvre.
    \item Chapitres 17 et 18 : Spinoza abordent le cas de l'\etat de l'\etat Hébreu dans l'Ancien Testament. Il évoque ce cas de figure car c'est un cas de theocratie dont Spinoza va diagnostiquer la chute, et pourquoi elle était inévitable car intrinsèque au principe théocratique.
    \item Chapitre 19 : Spinoza distingue deux formes de cultes~: \begin{itemize}
        \item le culte intérieur, la foi individuelle ;
        \item le culte extérieur, celui pris en main par les églises (à quel endroit, selon quel rite, dans quelle mesure). Contrairement au culte intérieur, Spinoza pense que le culte extérieur doit être encadré par l'\etat pour empêcher le sectarisme, la prise du contrôle par le religieux du politique.
    \end{itemize}
    \item Chapitre 20 : Spinoza promeut une république libre, où chacun peut penser ce qu'il veut, et dire ce qu'il pense. Le tout, sans s'exposer à des sanctions.
\end{itemize}


\end{document}
