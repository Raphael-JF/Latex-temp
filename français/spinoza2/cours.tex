\documentclass[a4paper,12pt]{article}

% Encodage et langue
\usepackage[utf8]{inputenc}
\usepackage[T1]{fontenc}
\usepackage[french]{babel}

\newcommand{\ttp}[0]{\underline{\textit{Traité Théologico-politique}} }


% Marges et mise en page
\usepackage[margin=2.5cm]{geometry}

% Titre du document
\title{Cours de Français\\ \Large \textit{La communauté "prisonnière de la superstition"}}
\author{Raphaël JONTEF}
\date{\today}

\begin{document}

\section{La préface (p.41)}
Spinoza affirme qu'on peut facilement contrôler les corps, mais personne ne peut penser pour quelqu'un d'autre. Chacun a donc la liberté de penser comme il veut. Néanmoins, il existe un moyen d'influencer les esprits : la peur. La peur prospère dans les situations de crise.\\

Spinoza montre dans cette préface que l'âme humaine est encline à la crédulité, particulièrement dans les crises, car elle projette sur ces craintes ses espoirs et désirs. L'homme préfère spontanément plonger dans des superstitions car c'est plus facile. \\L'individu moyen constitue donc une cible idéale pour la superstition, activement soutenue par les faux profètes, les ministres du culte. Il distingue deux types de religion~:
\begin{itemize}
    \item la part essentielle de la religion, notamment en terme de justice et de piété (Eschyle : justice et piété, ce qui caractérise la cité)
    \item en contrepartie, la religion comme institution autoritaire qui gouverne les foules et propage l'intolérance 
\end{itemize}\\
Spinoza explique que la religion cherche à discréditer la politique

\end{document}