\documentclass{beamer}
\usepackage{color}
\usepackage[utf8]{inputenc}
\usepackage[T1]{fontenc}
\usepackage{multicol}
\usepackage{tikz}
\usepackage{xcolor}
\usepackage{listings}
\usepackage{pgfplots}
\usepackage{amsmath,mathtools}
\usetikzlibrary{plotmarks}
\usefonttheme[onlymath]{serif}

\title{TP 16 \\ Résonance en intensité du circuit RLC série}
\author{Raphaël J, Romain}

\beamertemplatenavigationsymbolsempty
\date{}
\everymath{\displaystyle}
\pgfplotsset{compat=1.15}%ligne que je ne comprends pas trop

\definecolor{vert_fonce}{RGB}{0, 69, 33}



\begin{document}

\frame{\titlepage}

\begin{frame}
    \frametitle{Sommaire}
    \tableofcontents
\end{frame}

\section{I Déterminer la fréquence de résonance en intensité}

\begin{frame}
    \frametitle{I Déterminer la fréquence de résonance en intensité}
    \subsection{I.1 Théorie}
    \framesubtitle{I.1 Théorie}

        \begin{align*}
            &\forall f, U_{Rm}(f) = RI_m(f) \\
            \implies \quad \Aboxed{&U_{Rm}(f)= \frac{E_m}{\sqrt{1+Q^2(\frac{f}{f_0}-\frac{f_0}{f})^2}}}
        \end{align*} \vspace{15pt}
        \begin{align*}
            &U_{Rm}(f_r) = \max(U_{Rm}) \\
            \implies \quad &f_r=f_0 \\
            \implies \quad &f_r = \frac{\omega_0}{2\pi} \\
            \implies \quad \Aboxed{&f_r = \frac{1}{2\pi \sqrt{LC}}}
        \end{align*}
    \end{frame}

    \begin{frame}
        \frametitle{I Déterminer la fréquence de résonance en intensité}
        \subsection{I.2 Pratique}
        \framesubtitle{I.2 Pratique}
        \begin{itemize}
            \item Tâtonnement : Adapter la fréquence de $e(t)$ de sorte que $u_R(t)$ soit maximale ($i(t)$ suivra).
            \item Mode $XY$ : Adapter la fréquence de $e(t)$ de sorte que $u_R$ (et donc $i$) soit en phase avec $u_g$ :
        \end{itemize}
        \begin{center}
            \begin{tikzpicture}
                \begin{axis}[
                    width = 0.6\textwidth,
                    xmax = 5,
                    xmin = 0,
                    ymax = 5,
                    ymin = 0,
                    ytick = {0},
                    xtick = {0},
                    samples=2,
                    xlabel={$u_g$},
                    ylabel={$u_R$},
                    axis lines=middle,
                    grid=major,
                    ]
                    \addplot[purple,thick,domain =1:3] {x};
                \end{axis}
            \end{tikzpicture}
        \end{center}
    \end{frame}


    \section{II Déterminer le facteur qualité}
    \begin{frame}
        \frametitle{II Déterminer le facteur qualité}
        \subsection{II.1 Graphe du gain}
        \framesubtitle{II.1 Graphe du gain}

        \begin{center}
        \begin{tikzpicture}
            \begin{axis}[
            ,axis x line=bottom,axis y line=left
            ,xmin=0,xmax=6.24
            ,ymin=0,ymax=0.8805
            ,grid=major
            ,title={Graphe ajusté de $\displaystyle G(f) = \frac{U_R}{U_g}$}
            ,xlabel={$f$/kHz}
            ,ylabel={$G$}
            ]
            \addplot[draw=black,only marks,mark=*,mark options={fill=black}] file {courbes/G10.txt};
            \addplot[draw=blue,mark=none,smooth] file {courbes/G11.txt};
            \addplot[draw=black,only marks,mark=*,mark options={fill=black}] file {courbes/G12.txt};

            \addplot[black,dashed,domain=0:6.5] {0.5982} node [pos=0.8,above right,text=black]{$\displaystyle \frac{G_{max}}{\sqrt{2}}$};
            \draw[vert_fonce] (axis cs:1.884,\pgfkeysvalueof{/pgfplots/ymin}) -- (axis cs:1.884,0.5982) node [pos=0.135,below left]{$f_{c1}$};
            \draw[vert_fonce] (axis cs:2.687,\pgfkeysvalueof{/pgfplots/ymin}) -- (axis cs:2.687,0.5982) node [pos=0.135,below right]{$f_{c2}$};
            \end{axis}
        \end{tikzpicture}
        \end{center}

    \end{frame}

    \begin{frame}
        \frametitle{II Déterminer le facteur qualité}
        \subsection{II.2 Calculs}
        \framesubtitle{II.2 Calculs}

        \begin{align*}
            &\Delta \omega = \frac{\omega_0}{Q} \\
            \implies &\omega_2 - \omega_1 = \frac{\omega_0}{Q} \\
            \implies &2\pi(f_{c2} - f_{c1}) = \frac{\omega_0}{Q} \\ 
            \implies &\begin{cases}
                Q = \textstyle \frac{\omega_0}{2\pi(f_{c2} - f_{c1})} \\
                \omega_0 = \omega_r = 2\pi f_r
            \end{cases} \\
            \implies \Aboxed{&Q = \frac{f_r}{f_{c2} - f_{c1}}}
        \end{align*}
    \end{frame}

    \begin{frame}
        \frametitle{Bonus !}
        \framesubtitle{Graphe du déphasage $\varphi_{R/g}$ de $u_R(t)$ par rapport à $u_g(t)$}
        \begin{center}
            \begin{tikzpicture}
                \begin{axis}[
                ,axis x line=bottom,axis y line=left
                ,xmin=0,xmax=6.24
                ,ymin=-1.49,ymax=1.507
                ,grid=major
                ,title={Graphe ajusté de $\varphi_{R/g}$}
                ,xlabel={$f$/kHz}
                ,ylabel={$\varphi_{R/g}$/rad}
                ]
                \addplot[draw=black,only marks,mark=*,mark options={fill=black}] file {courbes/varphi10.txt};
                \addplot[draw=blue,mark=none,smooth] file {courbes/varphi11.txt};
                \addplot[draw=black,only marks,mark=*,mark options={fill=black}] file {courbes/varphi12.txt};
                \end{axis}
            \end{tikzpicture}
        \end{center}

    \end{frame}

    \begin{frame}
        \frametitle{Conclusion}
        \tableofcontents
    \end{frame}
    
\end{document}
