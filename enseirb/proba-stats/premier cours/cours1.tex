\documentclass{article}
\usepackage{titling} % Personnalisation du titre
\usepackage[left=20mm, right=20mm]{geometry}
\usepackage{amsmath,amssymb,mathtools}
\usepackage{esint} % intégrale avec un round
\usepackage{xcolor}
\usepackage[utf8]{inputenc}
\usepackage{listings}
\usepackage{enumitem}
\usepackage{multicol}
\usepackage{stmaryrd}
\usepackage{graphicx}
\graphicspath{ {./images/} }
\usepackage[framemethod=tikz]{mdframed}
\usepackage{tikz,pgfplots}
\pgfplotsset{compat=1.18}
\usetikzlibrary{arrows}
\usepackage{forest}
\usepackage{titlesec}
\setlength{\parindent}{0pt}

\pretitle{\begin{center}\Huge\bfseries}
\posttitle{\end{center}}
\date{}
\renewcommand{\thesection}{\Roman{section}} 
\titleformat{\section}
  {\Large\bfseries} % Style du titre
  {\thesection} % Numéro de section
  {0.5cm} % Espacement entre numéro et titre
  {} % Pas de préfixe supplémentaire
\titleformat{\subsection}
  {\large\bfseries} % Style du titre
  {\thesubsection} % Numéro de section
  {0.4cm} % Espacement entre numéro et titre
  {} % Pas de préfixe supplémentair


\newenvironment{enumeratebf}{
    \begin{enumerate}[label=\textbf{\arabic*.}]
}
{
    \end{enumerate}
}
  
\definecolor{oranges}{RGB}{255, 242, 230}
\definecolor{rouges}{RGB}{255, 230, 230}
\definecolor{rose}{RGB}{255, 204, 204}

% maths - info
\definecolor{rouge_fonce}{RGB}{204, 0, 0}
\definecolor{rouge}{RGB}{255, 0, 0}
\definecolor{bleufonce}{RGB}{0, 0, 255}
\definecolor{vert_fonce}{RGB}{0, 69, 33}
\definecolor{vert}{RGB}{0,255,0}

\definecolor{orange_foncee}{RGB}{255, 153, 0}
\definecolor{myrtille}{RGB}{225, 225, 255}
\definecolor{mayonnaise}{RGB}{255, 253, 233}
\definecolor{magenta}{RGB}{224, 209, 240}
\definecolor{pomme}{RGB}{204, 255, 204}
\definecolor{mauve}{RGB}{255, 230, 255}


% Cours

\newmdenv[
    nobreak=true,
    topline=true,
    bottomline=true,
    rightline=true,
    leftline=true,
    linewidth=0.5pt,
    linecolor=black,
    backgroundcolor=mayonnaise,
    innerleftmargin=10pt,
    innerrightmargin=2.5em,
    innertopmargin=5pt,
    innerbottommargin=5pt,
    skipabove=\topsep,
    skipbelow=\topsep,
]{boite_definition}

\newcounter{boite}
\setcounter{boite}{1}
\newenvironment{definition}[2]
{
    \vspace{15pt}
    \begin{boite_definition}
    \if\relax\detokenize{#1}\relax
        \textbf{\textcolor{rouge}{Définition \arabic{chapitre}.\arabic{boite}}}%
        \if\relax\detokenize{#2}\relax
        \else
            \textit{ - #2}
        \fi
        \stepcounter{boite}
    \else
        \textbf{\textcolor{rouge}{Définition #1}}%
        \if\relax\detokenize{#2}\relax
        \else
            \textit{ - #2}
        \fi
    \fi \\
    
    
}
{
    \end{boite_definition}
    \vspace{10pt}
}

\newmdenv[
  nobreak=true,
  topline=true,
  bottomline=true,
  rightline=true,
  leftline=true,
  linewidth=0.5pt,
  linecolor=white,
  backgroundcolor=white,
  innerleftmargin=10pt,
  innerrightmargin=2.5em,
  innertopmargin=5pt,
  innerbottommargin=5pt,
  skipabove=\topsep,
  skipbelow=\topsep,
]{boite_exemple}


\newenvironment{exemple}[2]
{
    \vspace{15pt}
    \begin{boite_exemple}
    \if\relax\detokenize{#1}\relax
        \textbf{\textcolor{bleufonce}{Exemple \arabic{chapitre}.\arabic{boite}}}%
        \if\relax\detokenize{#2}\relax
        \else
            \textit{ - #2}
        \fi
        \stepcounter{boite}
    \else
        \textbf{\textcolor{bleufonce}{Exemple #1}}%
        \if\relax\detokenize{#2}\relax
        \else
            \textit{ - #2}
        \fi
    \fi \\
    
    
}
{
    \end{boite_exemple}
    \vspace{10pt}
}

\newmdenv[
  nobreak=true,
  topline=true,
  bottomline=true,
  rightline=true,
  leftline=true,
  linewidth=0.5pt,
  linecolor=black,
  backgroundcolor=magenta,
  innerleftmargin=10pt,
  innerrightmargin=2.5em,
  innertopmargin=5pt,
  innerbottommargin=5pt,
  skipabove=\topsep,
  skipbelow=\topsep,
]{boite_proposition}

\newenvironment{proposition}[2]
{
    \vspace{15pt}
    \begin{boite_proposition}
    \if\relax\detokenize{#1}\relax
        \textbf{\textcolor{rouge}{Proposition \arabic{chapitre}.\arabic{boite}}}%
        \if\relax\detokenize{#2}\relax
        \else
            \textit{ - #2}
        \fi
        \stepcounter{boite}
    \else
        \textbf{\textcolor{rouge}{Proposition #1}}%
        \if\relax\detokenize{#2}\relax
        \else
            \textit{ - #2}
        \fi
    \fi \\
    
    
}
{
    \end{boite_proposition}
}

\newmdenv[
  nobreak=true,
  topline=true,
  bottomline=true,
  rightline=true,
  leftline=true,
  linewidth=0.5pt,
  linecolor=black,
  backgroundcolor=magenta,
  innerleftmargin=10pt,
  innerrightmargin=2.5em,
  innertopmargin=5pt,
  innerbottommargin=5pt,
  skipabove=\topsep,
  skipbelow=\topsep,
]{boite_theoreme}


\newenvironment{theoreme}[2]
{
    \vspace{15pt}
    \begin{boite_theoreme}
    \if\relax\detokenize{#1}\relax
        \textbf{\textcolor{rouge}{Théorème \arabic{chapitre}.\arabic{boite}}}%
        \if\relax\detokenize{#2}\relax
        \else
            \textit{ - #2}
        \fi
        \stepcounter{boite}
    \else
        \textbf{\textcolor{rouge}{Théorème #1}}%
        \if\relax\detokenize{#2}\relax
        \else
            \textit{ - #2}
        \fi
    \fi \\
    
    
}
{
    \end{boite_theoreme}
}


\newmdenv[
  nobreak=true,
  topline=true,
  bottomline=true,
  rightline=true,
  leftline=true,
  linewidth=0.5pt,
  linecolor=black,
  backgroundcolor=white,
  innerleftmargin=10pt,
  innerrightmargin=2.5em,
  innertopmargin=5pt,
  innerbottommargin=5pt,
  skipabove=\topsep,
  skipbelow=\topsep,
]{boite_demonstration}


\newenvironment{demonstration}
{
    \vspace{15pt}
    \begin{boite_demonstration}
    \textbf{\textcolor{rouge}{Démonstration}}\\ \\
}
{
    \end{boite_demonstration}
    
}


\newmdenv[
  nobreak=true,
  topline=true,
  bottomline=true,
  rightline=true,
  leftline=true,
  linewidth=0.5pt,
  linecolor=white,
  backgroundcolor=white,
  innerleftmargin=10pt,
  innerrightmargin=2.5em,
  innertopmargin=5pt,
  innerbottommargin=5pt,
  skipabove=\topsep,
  skipbelow=\topsep,
]{boite_remarque}


\newenvironment{remarque}[2]
{
    \vspace{15pt}
    \begin{boite_remarque}
    \if\relax\detokenize{#1}\relax
        \textbf{\textcolor{bleufonce}{Remarque \arabic{chapitre}.\arabic{boite}}}%
        \if\relax\detokenize{#2}\relax
        \else
            \textit{ - #2}
        \fi
        \stepcounter{boite}
    \else
        \textbf{\textcolor{bleufonce}{Remarque #1}}%
        \if\relax\detokenize{#2}\relax
        \else
            \textit{ - #2}
        \fi
    \fi \\
    
    
}
{
    \end{boite_remarque}
}

\newmdenv[
  nobreak=true,
  topline=true,
  bottomline=true,
  rightline=true,
  leftline=true,
  linewidth=0.5pt,
  linecolor=bleufonce,
  backgroundcolor=white,
  innerleftmargin=10pt,
  innerrightmargin=2.5em,
  innertopmargin=5pt,
  innerbottommargin=5pt,
  skipabove=\topsep,
  skipbelow=\topsep,
]{boite_implementation}


\definecolor{keywordcolor}{RGB}{133, 153, 0}  % les mots-clés
\definecolor{commentcolor}{RGB}{147, 161, 161} % les commentaires
\definecolor{stringcolor}{RGB}{42, 161, 152}  % les chaînes de caractères

\lstnewenvironment{lstOCaml}
{\lstset{
    language=[Objective]Caml,
    basicstyle=\ttfamily,
    keywordstyle=\color{keywordcolor},
    commentstyle=\color{commentcolor},
    stringstyle=\color{stringcolor},
    backgroundcolor=\color{white},
    numbers=left,
    numberstyle=\ttfamily,
    numbersep=-1.5em,
    stepnumber=1,
    frame=l,
    framexleftmargin=-2.25em,
    tabsize=2,
    literate=%
    {é}{{\'e}}{1}%
    {è}{{\`e}}{1}%
    {à}{{\`a}}{1}%
    {ç}{{\c{c}}}{1}%
    {œ}{{\oe}}{1}%
    {ù}{{\`u}}{1}%
    {É}{{\'E}}{1}%
    {È}{{\`E}}{1}%
    {À}{{\`A}}{1}%
    {Ç}{{\c{C}}}{1}%
    {Œ}{{\OE}}{1}%
    {Ê}{{\^E}}{1}%
    {ê}{{\^e}}{1}%
    {î}{{\^i}}{1}%
    {ô}{{\^o}}{1}%
    {û}{{\^u}}{1}%
    {ä}{{\"{a}}}1
    {ë}{{\"{e}}}1
    {ï}{{\"{i}}}1
    {ö}{{\"{o}}}1
    {ü}{{\"{u}}}1
    {û}{{\^{u}}}1
    {â}{{\^{a}}}1
    {Â}{{\^{A}}}1
    {Î}{{\^{I}}}1
}}{}
 
\lstnewenvironment{lstC}
{\lstset{
    language=C,
    basicstyle=\ttfamily,
    keywordstyle=\color{keywordcolor},
    commentstyle=\color{commentcolor},
    stringstyle=\color{stringcolor},
    backgroundcolor=\color{white},
    numbers=left,
    numberstyle=\ttfamily,
    numbersep=-1.5em,
    stepnumber=1,
    frame=l,
    framexleftmargin=-2.25em,
    tabsize=2,
    literate=%
    {é}{{\'e}}{1}%
    {è}{{\`e}}{1}%
    {à}{{\`a}}{1}%
    {ç}{{\c{c}}}{1}%
    {œ}{{\oe}}{1}%
    {ù}{{\`u}}{1}%
    {É}{{\'E}}{1}%
    {È}{{\`E}}{1}%
    {À}{{\`A}}{1}%
    {Ç}{{\c{C}}}{1}%
    {Œ}{{\OE}}{1}%
    {Ê}{{\^E}}{1}%
    {ê}{{\^e}}{1}%
    {î}{{\^i}}{1}%
    {ô}{{\^o}}{1}%
    {û}{{\^u}}{1}%
    {ä}{{\"{a}}}1
    {ë}{{\"{e}}}1
    {ï}{{\"{i}}}1
    {ö}{{\"{o}}}1
    {ü}{{\"{u}}}1
    {û}{{\^{u}}}1
    {â}{{\^{a}}}1
    {Â}{{\^{A}}}1
    {Î}{{\^{I}}}1
}}{}


\lstdefinelanguage{LNat}{
    morekeywords={tant,que,pour,tout,si,sinon,initialiser,renvoyer,attendre la fin, afficher},
    sensitive=false,
    morecomment=[l]{//},
}

\lstnewenvironment{lstLNat}
{\lstset{
    language=LNat,
    basicstyle=\ttfamily,
    keywordstyle=\color{keywordcolor},
    commentstyle=\color{commentcolor},
    stringstyle=\color{stringcolor},
    backgroundcolor=\color{white},
    numbers=left,
    numberstyle=\ttfamily,
    numbersep=-1.5em,
    stepnumber=1,
    frame=l,
    mathescape=true,
    framexleftmargin=-2.25em,
    tabsize=2,
    literate=%
    {é}{{\'e}}{1}%
    {è}{{\`e}}{1}%
    {à}{{\`a}}{1}%
    {ç}{{\c{c}}}{1}%
    {œ}{{\oe}}{1}%
    {ù}{{\`u}}{1}%
    {É}{{\'E}}{1}%
    {È}{{\`E}}{1}%
    {À}{{\`A}}{1}%
    {Ç}{{\c{C}}}{1}%
    {Œ}{{\OE}}{1}%
    {Ê}{{\^E}}{1}%
    {ê}{{\^e}}{1}%
    {î}{{\^i}}{1}%
    {ô}{{\^o}}{1}%
    {û}{{\^u}}{1}%
    {ä}{{\"{a}}}1
    {ë}{{\"{e}}}1
    {ï}{{\"{i}}}1
    {ö}{{\"{o}}}1
    {ü}{{\"{u}}}1
    {û}{{\^{u}}}1
    {â}{{\^{a}}}1
    {Â}{{\^{A}}}1
    {Î}{{\^{I}}}1}
}{}

\newenvironment{implementation}[1]
{   
    \vspace{15pt}
    \begin{boite_implementation}
    \textbf{\textcolor{bleufonce}{Implémentation}}\textit{ - #1}
     \\ \\
}
{    
    \end{boite_implementation}
}

\newmdenv[
  nobreak=true,
  topline=true,
  bottomline=true,
  rightline=true,
  leftline=true,
  linewidth=0.5pt,
  linecolor=black,
  backgroundcolor=mayonnaise,
  innerleftmargin=10pt,
  innerrightmargin=2.5em,
  innertopmargin=5pt,
  innerbottommargin=5pt,
  skipabove=\topsep,
  skipbelow=\topsep,
]{boite_question}


\newenvironment{question}[2]
{
    \vspace{15pt}
    \begin{boite_question}
    \if\relax\detokenize{#1}\relax
        \textbf{\textcolor{rouge}{Question \arabic{chapitre}.\arabic{boite}}}%
        \if\relax\detokenize{#2}\relax
        \else
            \textit{ - #2}
        \fi
        \stepcounter{boite}
    \else
        \textbf{\textcolor{rouge}{Question #1}}%
        \if\relax\detokenize{#2}\relax
        \else
            \textit{ - #2}
        \fi
    \fi \\
    
    
}
{
    \end{boite_question}
}

\newmdenv[
  nobreak=true,
  topline=true,
  bottomline=true,
  rightline=true,
  leftline=true,
  linewidth=0.5pt,
  linecolor=black,
  backgroundcolor=white,
  innerleftmargin=10pt,
  innerrightmargin=2.5em,
  innertopmargin=5pt,
  innerbottommargin=5pt,
  skipabove=\topsep,
  skipbelow=\topsep,
]{boite_corollaire}



\newenvironment{corollaire}[2]
{
    \vspace{15pt}
    \begin{boite_corollaire}
    \if\relax\detokenize{#1}\relax
        \textbf{\textcolor{rouge}{Corollaire \arabic{chapitre}.\arabic{boite}}}%
        \if\relax\detokenize{#2}\relax
        \else
            \textit{ - #2}
        \fi
        \stepcounter{boite}
    \else
        \textbf{\textcolor{rouge}{Corollaire #1}}%
        \if\relax\detokenize{#2}\relax
        \else
            \textit{ - #2}
        \fi
    \fi \\
    
    
}
{
    \end{boite_corollaire}
}


\newcounter{chapitre}
\setcounter{chapitre}{1}

\title{\Large Chapitre 1 \\ \Huge Probabilités}

\begin{document}
% commandes
\newcommand{\notion}[1]{\textcolor{vert_fonce}{\textit{#1}}}
\newcommand{\mb}[1]{\mathbb{#1}}
\newcommand{\mc}[1]{\mathcal{#1}}
\newcommand{\code}[1]{\texttt{#1}}
\newcommand{\ccode}[1]{\texttt{|#1|}}
\newcommand{\ov}[1]{\overline{#1}}
\newcommand{\abs}[1]{|#1|}
\newcommand{\rev}[1]{\texttt{reverse(#1)}}
\newcommand{\crev}[1]{\texttt{|reverse(#1)|}}

\newcommand{\ie}{\textit{i.e.} }

\newcommand{\N}{\mathbb{N}}
\newcommand{\R}{\mathbb{R}}
\newcommand{\C}{\mathbb{C}}
\newcommand{\K}{\mathbb{K}}

\newcommand{\A}{\mathcal{A}}
\newcommand{\bigO}{\mathcal{O}}
\renewcommand{\L}{\mathcal{L}}

\newcommand{\rg}[0]{\text{rg}}
\newcommand{\re}[0]{\text{Re}}
\newcommand{\im}[0]{\text{Im}}
\newcommand{\cl}[0]{\text{cl}}
\newcommand{\mat}[1]{\text{Mat}_{#1}}
\newcommand{\matrice}[1]{\mathcal{M}_{#1}}
\newcommand{\sgEngendre}[1]{\left\langle #1 \right\rangle}
\newcommand{\norme}[1]{||#1||}
\renewcommand{\d}[1]{\,\text{d}#1}
\newcommand{\intint}[2]{\llbracket #1 ,\, #2 \rrbracket}
\newcommand{\seg}[2]{[#1\, ; \, #2]}
\newcommand{\scal}[2]{\left\langle #1 ,\, #2 \right\rangle}
\newcommand{\inte}[2]{\int_{#1}^{#2}}
\newcommand{\somme}[2]{\sum_{#1}^{#2}}





\maketitle

\newcommand{\tribu}[0]{\mathcal{T}}
\newcommand{\univ}[0]{\Omega}
\newcommand{\proba}[0]{\mathbb{P}}

mail du prof : francois.dufour@math.u-bordeaux.fr
\\ \\
\textbf{Pourquoi des probas ?}
\begin{itemize}
    \item modéliser des réseaux de télécommunication de façon aléatoires (nombre de clients, temps de traitement, etc.).
    \item théorie des algorithmes stochastiques~: une grande partie des modèles d'IA ont un lien avec ceux-là.
\end{itemize}

\section*{Espace de probabilité}

\begin{definition}{}{ensemble des éventualités}
    Dans la modélisation mathématique d'une expérience aléatoire, la première étape consiste à lister l'ensemble des résultats possibles de cette expérience. On note $\Omega$ un tel ensemble, appelé \notion{univers}, ou \notion{ensemble des réalisations}.
\end{definition}

\begin{exemple}{}{Pile ou Face}
    Dans un jeu de Pile ou Face, on a $\Omega = \{\text{Pile},\, \text{Face}\}$.
\end{exemple}

\begin{exemple}{}{Jeu à dé à six faces}
    Dans un jeu impliquant un dé à six faces, on a $\Omega = \{1,\, 2,\, 3,\, 4,\, 5,\, 6\}$
    
\end{exemple}

\section{Notions d'évènements et de tribu}

\begin{definition}{}{évènement}
    On appelle \notion{évènement} une partie $A \subset \Omega$.
\end{definition}

\begin{exemple}{}{d'évènement}
    Au jeu de dé à six faces, $A$ décrit par "le résultat du jeu de dé vaut 4 ou 5" est un évènement de $\Omega$.
\end{exemple}

\begin{proposition}{}{évènements axiomatiques}
    Soit $\Omega$ un univers.
    \begin{itemize}
        \item l'évènement $\varnothing$ est un évènement de $\Omega$ appelé \notion{évènement impossible}
        \item l'évènement $\Omega$ entier est un évènement.
    \end{itemize}
\end{proposition}

\begin{definition}{}{tribu}
    Pour un univers $\univ$ au plus dénombrable, on appelle \notion{tribu sur $\univ$} une partie $\tribu \subset \mc{P}(\univ)$ vérifiant~:
    \begin{enumeratebf}
        \item $\univ \in \tribu$
        \item Pour tout $A \in \tribu, \overline{A} \in \tribu$
        \item Pour toute suite $(A_n)_{n \in \N}$ d'éléments de $\tribu$, $\displaystyle \bigcup_{n \in \N}A_n \in \tribu$
    \end{enumeratebf}
    Le couple $(\univ, \tribu)$ est alors appelé \notion{espace mesurable} (ou \notion{espace probabilisable}).
\end{definition}

\begin{exemple}{}{de tribus}
    Pour $\Omega$ un univers quelconque~:
    \begin{itemize}
        \item $\{\varnothing, \Omega\}$ est une tribu de $\Omega$
        \item $\mc{P}(\Omega)$ est une tribu, dite \notion{tribu pleine de $\Omega$}
        \item $\{\varnothing, A, \overline{A}, \Omega\}$ est une tribu pour $A \subset \Omega$. C'est la plus petite tribu de $\Omega$ qui contient $A$ pour l'inclusion.
    \end{itemize}
\end{exemple}

\section{Probabilité}

\begin{definition}{}{probabilité}
    Soit $(\univ, \tribu)$ un espace mesurable. On appelle \notion{probabilité sur $(\univ, \tribu)$} une application $\proba : \tribu \to \seg{0}{1}$ telle que~:
    \begin{enumeratebf}
        \item \notion{principe de normalisation}$~: \proba(\univ) = 1$
        \item \notion{$\sigma$-additivité}~: Pour toute suite $(A_n)_{n \in \N}$ d'évènements deux à deux incompatibles, la série de terme général $\proba(A_n)$ converge et~:
        $$\proba\Bigg(\bigsqcup_{n \in \N}A_n\Bigg) = \sum_{n=0}^{+ \infty} \proba(A_n)$$
    \end{enumeratebf}
    On dit alors que $(\univ, \tribu, \proba)$ constitue un \notion{espace de probabilité} (ou \notion{espace de probabilité   }).
\end{definition}

\begin{definition}{}{événements négligeable, certain}
    Soit $(\univ, \tribu, \proba)$ un espace de probabilité.
    \begin{itemize}
        \item Un événement $A$ est dit \notion{négligeable} si $\proba(A) = 0$.
        \item Un événement $A$ est dit \notion{certain} si $\proba(A) = 1$.
    \end{itemize}
\end{definition}


\begin{remarque}{}{}
    On munira souvent les univers $\univ$ finis de leur tribu pleine $\mc{P}(\univ)$
\end{remarque}

\begin{exemple}{}{}
    Soit $\Omega = \{\omega_1,\, \dots,\, \omega_n\}$ un univers fini qu'on munit de la tribu pleine. On considère $\proba$ une probabilité sur cet espace mesurable telle que~:
    $$\forall i \in \intint{1}{n},\, \proba(\omega_i) = p \in ]0,\,1]$$
    $\proba$ est alors appelée probabilité uniforme. On vérifie que nécessairement~:
    $$p = \frac{1}{\abs{\univ}}$$
    et ainsi pour $A \in \tribu = \mc{P(A)}$~:
    $$\proba(A) = \frac{\abs(A)}{\abs{\univ}}$$
\end{exemple}

\begin{proposition}{}{règles de calcul}
    Soit $(\univ, \tribu, \proba)$ un espace de probabilité. Pour $A$ et $B$ dans $\tribu$~:
    \begin{enumeratebf}
        \item $\proba(\overline{A}) = 1 - \proba(A)$
        \item $\proba(A\cup B) = \proba(A) + \proba(B) - \proba(A \cap B)$
        \item $\proba(A \cap B) = $ ?
        \item Si $A \subset B$, alors $\proba(A) \leq \proba(B)$
    \end{enumeratebf}
\end{proposition}

\section{Probabilité conditionnelle}

On cherche à évaluer la probabilité d'un évènement $A$ sachant qu'un évènement $B$ est réalisé.

\begin{definition}{}{probabilité conditionnelle}
    Soit $(\univ, \tribu, \proba)$ un espace de probabilité. Soit $A \in \tribu$ un événement non négligeable. Alors l'application~:
    \fonction{\proba_A}{\tribu}{\seg{0}{1}}{B}{\frac{\proba(A \cap B)}{\proba(A)}}
    est une probabilité, appelée \notion{probabilité conditionnelle sachant $A$}.
\end{definition}

\begin{exemple}{}{des deux enfants}
    Un voisin a deux enfants. On note~:
    \begin{itemize}
        \item $A$ : "Il a au moins un garçon."
        \item $B$ : "Il a au moins une fille."
        \item $C$ : "Son deuxième enfant est une fille."
    \end{itemize}
    On a~:
    $$\proba_B(A) = \frac{\proba(A \cap B)}{\proba(B)} = \frac{\abs{A \cap B}}{\abs{\univ}} \frac{\abs{\univ}}{\abs{B}} = \frac{2}{3}$$
    De même, 
    $$\proba_C(A) = \frac{\abs{A \cap C}}{\abs{C}} = \frac{1}{2}$$
\end{exemple}


\section{Indépendance d'évènements}

\begin{definition}{}{évènements indépendants}
    Soit $(\univ, \tribu, \proba)$ un espace de probabilité. Deux évènements $A$ et $B$ sont indépendants si~:
    $$\proba(A \cap B) = \proba(A) \proba(B)$$
\end{definition}

\begin{definition}{15.37}{évènements indépendants}
    Soit $(\univ, \tribu, \proba)$ un espace de probabilité. Soit $(A_i)_{i \in I}$ un système complet d'évènement, où $I$ est au plus dénombrable. $(A_i)_{i \in I}$ est une \notion{famille d'évènements indépendants} (aussi appelée \notion{famille d'évènements mutuellement indépendants}) si~:
    $$\forall J \in \mc{P}_f(I),\, \proba\Bigg(\bigcap_{j \in J} A_j\Bigg) = \prod_{j \in J}\proba(A_j)$$
\end{definition}

\begin{proposition}{}{relation entre indépendance d'évènements et probabilité conditionnelle}
    Soit $(\univ, \tribu, \proba)$ un espace de probabilité. Soit $A$ et $B$ deux évènements. Les propriétés suivantes sont équivalentes~:
    \begin{enumeratebf}
        \item $A$ et $B$ sont indépendants.
        \item $\proba_B(A) = \proba(A)$
        \item $\proba_A(B) = \proba(B)$
    \end{enumeratebf}
    
\end{proposition}

\section{Identités et définitions}

\begin{definition}{}{partition}
    Soit $E$ un ensemble. On dit qu'une famille quelconque $(A_i)_{i\in I}$ de parties de $E$ est une \notion{partition} de $E$ si~:
    $$E = \bigsqcup_{i \in I}A_i$$
\end{definition}

\begin{theoreme}{}{formule de Bayes}
    Soit $(\univ, \tribu, \proba)$ un espace de probabilité. Soit $(A_i)_{i \in I}$ un système complet d'évènement de probabilités non nulles, où $I$ est au plus dénombrable. Pour tout $B \in \tribu$ de probabilité non nulle~: 
    $$\forall j \in I,\, \proba_B(A_j) = \frac{\proba_{A_j}(B)}{\displaystyle \sum_{i \in I} \proba_{A_i}(B)\proba(A_i)}$$
\end{theoreme}

\begin{theoreme}{}{formule de probabilités totales}
    Soit $(\univ, \tribu, \proba)$ un espace de probabilité. Soit $(A_i)_{i \in I}$ un système complet d'évènement de probabilités non nulles, où $I$ est au plus dénombrable. Pour tout $B \in \tribu$~:
    $$\proba(B) = \sum_{i \in I} \proba(B \cap A_i)$$
\end{theoreme}

\begin{remarque}{}{formule des probabilités totales}
    le système complet d'évènements peut n'être qu'une partition d'un évènement quelconque ! En effet, on peut écrire pour un tel évènement $A$~:
    $$A = \bigsqcup_{n\in \N} (B_n \cap A)$$
    La démonstration en découle alors.
\end{remarque}

\end{document}