\documentclass{article}
\usepackage{amsmath,amssymb,mathtools}
\usepackage{xcolor}
\usepackage{minted}
\usepackage{enumitem}
\usepackage{multicol}
\usepackage{changepage}
\usepackage{stmaryrd}
\usepackage{graphicx}
\graphicspath{ {./images/} }
\usepackage[framemethod=tikz]{mdframed}
\usepackage{tikz,pgfplots}
\pgfplotsset{compat=1.18}

% physique
\definecolor{oranges}{RGB}{255, 242, 230}
\definecolor{rouges}{RGB}{255, 230, 230}
\definecolor{rose}{RGB}{255, 204, 204}

% maths - info
\definecolor{rouge_fonce}{RGB}{204, 0, 0}
\definecolor{rouge}{RGB}{255, 0, 0}
\definecolor{bleu_fonce}{RGB}{0, 0, 255}
\definecolor{vert_fonce}{RGB}{0, 69, 33}
\definecolor{vert}{RGB}{0,255,0}

\definecolor{orange_foncee}{RGB}{255, 153, 0}
\definecolor{myrtille}{RGB}{225, 225, 255}
\definecolor{mayonnaise}{RGB}{255, 253, 233}
\definecolor{magenta}{RGB}{224, 209, 240}
\definecolor{pomme}{RGB}{204, 255, 204}
\definecolor{mauve}{RGB}{255, 230, 255}


% Cours

\newmdenv[
  nobreak=true,
  topline=true,
  bottomline=true,
  rightline=true,
  leftline=true,
  linewidth=0.5pt,
  linecolor=black,
  backgroundcolor=mayonnaise,
  innerleftmargin=10pt,
  innerrightmargin=10pt,
  innertopmargin=5pt,
  innerbottommargin=5pt,
  skipabove=\topsep,
  skipbelow=\topsep,
]{boite_definition}

\newmdenv[
  nobreak=false,
  topline=true,
  bottomline=true,
  rightline=true,
  leftline=true,
  linewidth=0.5pt,
  linecolor=white,
  backgroundcolor=white,
  innerleftmargin=10pt,
  innerrightmargin=10pt,
  innertopmargin=5pt,
  innerbottommargin=5pt,
  skipabove=\topsep,
  skipbelow=\topsep,
]{boite_exemple}

\newmdenv[
  nobreak=true,
  topline=true,
  bottomline=true,
  rightline=true,
  leftline=true,
  linewidth=0.5pt,
  linecolor=black,
  backgroundcolor=magenta,
  innerleftmargin=10pt,
  innerrightmargin=10pt,
  innertopmargin=5pt,
  innerbottommargin=5pt,
  skipabove=\topsep,
  skipbelow=\topsep,
]{boite_proposition}

\newmdenv[
  nobreak=true,
  topline=true,
  bottomline=true,
  rightline=true,
  leftline=true,
  linewidth=0.5pt,
  linecolor=black,
  backgroundcolor=white,
  innerleftmargin=10pt,
  innerrightmargin=10pt,
  innertopmargin=5pt,
  innerbottommargin=5pt,
  skipabove=\topsep,
  skipbelow=\topsep,
]{boite_demonstration}

\newmdenv[
  nobreak=true,
  topline=true,
  bottomline=true,
  rightline=true,
  leftline=true,
  linewidth=0.5pt,
  linecolor=white,
  backgroundcolor=white,
  innerleftmargin=10pt,
  innerrightmargin=10pt,
  innertopmargin=5pt,
  innerbottommargin=5pt,
  skipabove=\topsep,
  skipbelow=\topsep,
]{boite_remarque}


\newenvironment{definition}[2]
{
    \vspace{15pt}
    \begin{boite_definition}
    \textbf{\textcolor{rouge}{Définition #1}}
    \if\relax\detokenize{#2}\relax
    \else
        \textit{ - #2}
    \fi \\ \\
}
{
    \end{boite_definition}
    
}

\newenvironment{exemple}[2]
{
    \vspace{15pt}
    \begin{boite_exemple}
    \textbf{\textcolor{bleu_fonce}{Exemple #1}}
    \if\relax\detokenize{#2}\relax
    \else
        \textit{ - #2}
    \fi \\ \\ 
}
{   
    \end{boite_exemple}
    
}

\newenvironment{proposition}[2]
{
    \vspace{15pt}
    \begin{boite_proposition}
    \textbf{\textcolor{rouge}{Proposition #1}}
    \if\relax\detokenize{#2}\relax
    \else
        \textit{ - #2}
    \fi \\ \\
}
{
    \end{boite_proposition}
    
}

\newenvironment{theoreme}[2]
{
    \vspace{15pt}
    \begin{boite_proposition}
    \textbf{\textcolor{rouge}{Théorème #1}} 
    \if\relax\detokenize{#2}\relax
    \else
        \textit{ - #2}
    \fi \\ \\
}
{
    \end{boite_proposition}
    
}

\newenvironment{demonstration}
{
    \vspace{15pt}
    \begin{boite_demonstration}
    \textbf{\textcolor{rouge}{Démonstration}}\\ \\
}
{
    \end{boite_demonstration}
    
}

\newenvironment{remarque}[2]
{
    \vspace{15pt}
    \begin{boite_remarque}
    \textbf{\textcolor{bleu_fonce}{Remarque #1}}
    \if\relax\detokenize{#2}\relax
    \else
        \textit{ - #2}
    \fi \\ \\   
}
{  
    \end{boite_remarque}
    
}



%Corrections
\newmdenv[
  nobreak=true,
  topline=true,
  bottomline=true,
  rightline=true,
  leftline=true,
  linewidth=0.5pt,
  linecolor=black,
  backgroundcolor=mayonnaise,
  innerleftmargin=10pt,
  innerrightmargin=10pt,
  innertopmargin=5pt,
  innerbottommargin=5pt,
  skipabove=\topsep,
  skipbelow=\topsep,
]{boite_question}


\newenvironment{question}[2]
{
    \vspace{15pt}
    \begin{boite_question}
    \textbf{\textcolor{rouge}{Question #1}}
    \if\relax\detokenize{#2}\relax
    \else
        \textit{ - #2}
    \fi \\ \\
}
{
    \end{boite_question}
    
}

\newenvironment{enumeratebf}{
    \begin{enumerate}[label=\textbf{\arabic*.}]
}
{
    \end{enumerate}
}

\begin{document}
\begin{adjustwidth}{-3cm}{-3cm}
\begin{document}
\begin{adjustwidth}{-3cm}{-3cm}
% commandes
\newcommand{\notion}[1]{\textcolor{vert_fonce}{\textit{#1}}}
\newcommand{\mb}[1]{\mathbb{#1}}
\newcommand{\mc}[1]{\mathcal{#1}}
\newcommand{\mr}[1]{\mathrm{#1}}
\newcommand{\code}[1]{\texttt{#1}}
\newcommand{\ccode}[1]{\texttt{|#1|}}
\newcommand{\ov}[1]{\overline{#1}}
\newcommand{\abs}[1]{|#1|}
\newcommand{\rev}[1]{\texttt{reverse(#1)}}
\newcommand{\crev}[1]{\texttt{|reverse(#1)|}}

\newcommand{\ie}{\textit{i.e.} }

\newcommand{\N}{\mathbb{N}}
\newcommand{\R}{\mathbb{R}}
\newcommand{\C}{\mathbb{C}}
\newcommand{\K}{\mathbb{K}}
\newcommand{\Z}{\mathbb{Z}}

\newcommand{\A}{\mathcal{A}}
\newcommand{\bigO}{\mathcal{O}}
\renewcommand{\L}{\mathcal{L}}

\newcommand{\rg}[0]{\mathrm{rg}}
\newcommand{\re}[0]{\mathrm{Re}}
\newcommand{\im}[0]{\mathrm{Im}}
\newcommand{\cl}[0]{\mathrm{cl}}
\newcommand{\grad}[0]{\vec{\mathrm{grad}}}
\renewcommand{\div}[0]{\mathrm{div}\,}
\newcommand{\rot}[0]{\vec{\mathrm{rot}}\,}
\newcommand{\vnabla}[0]{\vec{\nabla}}
\renewcommand{\vec}[1]{\overrightarrow{#1}}
\newcommand{\mat}[1]{\mathrm{Mat}_{#1}}
\newcommand{\matrice}[1]{\mathcal{M}_{#1}}
\newcommand{\sgEngendre}[1]{\left\langle #1 \right\rangle}
\newcommand{\gpquotient}[1]{\mathbb{Z} / #1\mathbb{Z}}
\newcommand{\norme}[1]{||#1||}
\renewcommand{\d}[1]{\,\mathrm{d}#1}
\newcommand{\adh}[1]{\overline{#1}}
\newcommand{\intint}[2]{\llbracket #1 ,\, #2 \rrbracket}
\newcommand{\seg}[2]{[#1\, ; \, #2]}
\newcommand{\scal}[2]{( #1 | #2 )}
\newcommand{\distance}[2]{\mathrm{d}(#1,\,#2)}
\newcommand{\inte}[2]{\int_{#1}^{#2}}
\newcommand{\somme}[2]{\sum_{#1}^{#2}}
\newcommand{\deriveref}[4]{\biggl( \frac{\text{d}^{#1}#2}{\text{d}#3^{#1}} \biggr)_{#4}}






\begin{definition}{11.1}{série entière}
    On appelle \notion{série entière }de la variable complexe $x$ de coefficients $(a_n)_{n \in \N}$ la série de fonctions $\sum_{n}a_n x^n$.
\end{definition}

\begin{theoreme}{11.3}{lemme d'Abel}
    Soit $\sum_n a_nz^n$ une série entière et $z_0$ un nombre complexe non nul tel que $(a_nz_0^n)_{n \in \N}$ est bornée. Pour tout $z \in \C$ tel que $\abs{z} < \abs{z_0}$, la série $\sum_n a_n z^n$ est absolument convergente.
\end{theoreme}

\begin{definition}{11.4}{rayon de convergence d'une série entière}
    On appelle \notion{rayon de convergence de la série entière $\sum_n a_n z^n$} la borne supérieure (au sens large) de cet intervalle~:
    $$R = \sup\{r \geq 0,\, (a_nr^n)_{n \in \N}\text{ est bornée}\}$$

\end{definition}

\begin{theoreme}{11.5}{propagation sur le cercle de convergence des caractères forts}
    Soit $\sum_n a_nz^n$ une série entière de rayon de convergence. 
    \begin{enumeratebf}
        \item  Si la série converge absolument en un point du cercle, alors elle converge absolument sur tout le cercle.
        \item Si la série diverge grossièrement en un point du cercle, alors elle diverge grossièrement sur tout le cercle.
    \end{enumeratebf}
    
\end{theoreme}

\begin{proposition}{11.8}{rayon de convergence de $\sum_n n^\alpha z^n$}
    Pour tout $\alpha \in \R$, le rayon de convergence de la série entière $\sum_n n^\alpha z^n$ vaut $1$.
\end{proposition}

\begin{proposition}{11.11 (1)}{comparaison de séries entières par inégalité}
    Soit $\sum_n a_n z^n$ et $\sum_n b_n z^n$ deux séries entières de rayons de convergence respectifs $R_a$ et $R_b$. Si à partir d'un certain rang $\abs{a_n} \leq \abs{b_n}$, alors $R_a \geq R_b$.
\end{proposition}

\begin{proposition}{11.11 (2)}{comparaison de séries entières par domination}
    Soit $\sum_n a_n z^n$ et $\sum_n b_n z^n$ deux séries entières de rayons de convergence respectifs $R_a$ et $R_b$. \\Si $a_n \underset{n \to +\infty}{=}\mc{O}\Big(b_n\Big)$, alors $R_a \geq R_b$.
\end{proposition}

\begin{proposition}{11.11 (3)}{comparaison de séries entières par équivalence}
    Soit $\sum_n a_n z^n$ et $\sum_n b_n z^n$ deux séries entières de rayons de convergence respectifs $R_a$ et $R_b$. \\Si $a_n \underset{n \to +\infty}{\sim}b_n$, alors $R_a = R_b$.
\end{proposition}

\begin{theoreme}{11.13}{règle de d'Alembert pour les séries entières}
    Soit $\sum_n a_n z^n$ une série entière telle que $\displaystyle \Bigg(\bigg\vert\frac{a_{n+1}}{a_n}\bigg\vert\Bigg)_{n \in \N}$ est définie à partir d'un certain rang et admette une limite $\ell \in \overline{\R}$. Le rayon de convergence $R$ de la série $\sum_n a_n z^n$ vaut $\begin{cases*}
        0 &si $\ell = + \infty$ \\
        \ell^{-1} &si $\ell>0$\\
        + \infty &si $\ell=0$
    \end{cases*}$.
\end{theoreme}

\begin{proposition}{11.14}{rayon de convergence d'une somme de séries entières}
    Soit $\sum_n a_n z^n$ et $\sum_n b_n z^n$ deux séries entières de rayons de convergence respectifs $R_a$ et $R_b$. En notant $R_{a+b}$ le rayon de convergence de la série entière $\sum_n (a_n + b_n)z^n$, on a~:
    $$R_{a+b} \geq \min(R_a,R_b)$$
    avec égalité si et seulement si les rayons sont distincts.
\end{proposition}

\begin{proposition}{11.15}{rayon de convergence d'un produit de Cauchy de séries entières}
    Soit $\sum_n a_n z^n$ et $\sum_n b_n z^n$ deux séries entières de rayons de convergence respectifs $R_a$ et $R_b$. En notant $R_{a\star b}$ le rayon de convergence de la série entière $\displaystyle \sum_n \big((a_i)_{i \in \N}\star (b_i)_{i \in \N}\big)_n z^n$, on a~:
    $$R_{a \star b} \geq \min(R_a,R_b)$$
    sans aucun cas d'égalité.
\end{proposition}

\begin{theoreme}{11.19}{convergence normale dans tout segment de l'intervalle ouvert de convergence}
    Soit $\sum_n a_n z^n$ une série entière de rayon de convergence $R$. Pour $n \in \N$, soit~:
    \fonction{f_n}{]-R;\, R[}{\R}{x}{\sum_{k=0}^{n}a_k x^k}
    La série de fonctions $(f_n)_{n\in \N}$ est normalement convergente sur tout segment de $]-R;\, R[$.
\end{theoreme}
\end{adjustwidth}
\end{document}