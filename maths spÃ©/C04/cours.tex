\documentclass{article}
\usepackage{amsmath,amssymb,mathtools}
\usepackage{xcolor}
\usepackage{minted}
\usepackage{enumitem}
\usepackage{multicol}
\usepackage{changepage}
\usepackage{stmaryrd}
\usepackage{graphicx}
\graphicspath{ {./images/} }
\usepackage[framemethod=tikz]{mdframed}
\usepackage{tikz,pgfplots}
\pgfplotsset{compat=1.18}

% physique
\definecolor{oranges}{RGB}{255, 242, 230}
\definecolor{rouges}{RGB}{255, 230, 230}
\definecolor{rose}{RGB}{255, 204, 204}

% maths - info
\definecolor{rouge_fonce}{RGB}{204, 0, 0}
\definecolor{rouge}{RGB}{255, 0, 0}
\definecolor{bleu_fonce}{RGB}{0, 0, 255}
\definecolor{vert_fonce}{RGB}{0, 69, 33}
\definecolor{vert}{RGB}{0,255,0}

\definecolor{orange_foncee}{RGB}{255, 153, 0}
\definecolor{myrtille}{RGB}{225, 225, 255}
\definecolor{mayonnaise}{RGB}{255, 253, 233}
\definecolor{magenta}{RGB}{224, 209, 240}
\definecolor{pomme}{RGB}{204, 255, 204}
\definecolor{mauve}{RGB}{255, 230, 255}


% Cours

\newmdenv[
  nobreak=true,
  topline=true,
  bottomline=true,
  rightline=true,
  leftline=true,
  linewidth=0.5pt,
  linecolor=black,
  backgroundcolor=mayonnaise,
  innerleftmargin=10pt,
  innerrightmargin=10pt,
  innertopmargin=5pt,
  innerbottommargin=5pt,
  skipabove=\topsep,
  skipbelow=\topsep,
]{boite_definition}

\newmdenv[
  nobreak=false,
  topline=true,
  bottomline=true,
  rightline=true,
  leftline=true,
  linewidth=0.5pt,
  linecolor=white,
  backgroundcolor=white,
  innerleftmargin=10pt,
  innerrightmargin=10pt,
  innertopmargin=5pt,
  innerbottommargin=5pt,
  skipabove=\topsep,
  skipbelow=\topsep,
]{boite_exemple}

\newmdenv[
  nobreak=true,
  topline=true,
  bottomline=true,
  rightline=true,
  leftline=true,
  linewidth=0.5pt,
  linecolor=black,
  backgroundcolor=magenta,
  innerleftmargin=10pt,
  innerrightmargin=10pt,
  innertopmargin=5pt,
  innerbottommargin=5pt,
  skipabove=\topsep,
  skipbelow=\topsep,
]{boite_proposition}

\newmdenv[
  nobreak=true,
  topline=true,
  bottomline=true,
  rightline=true,
  leftline=true,
  linewidth=0.5pt,
  linecolor=black,
  backgroundcolor=white,
  innerleftmargin=10pt,
  innerrightmargin=10pt,
  innertopmargin=5pt,
  innerbottommargin=5pt,
  skipabove=\topsep,
  skipbelow=\topsep,
]{boite_demonstration}

\newmdenv[
  nobreak=true,
  topline=true,
  bottomline=true,
  rightline=true,
  leftline=true,
  linewidth=0.5pt,
  linecolor=white,
  backgroundcolor=white,
  innerleftmargin=10pt,
  innerrightmargin=10pt,
  innertopmargin=5pt,
  innerbottommargin=5pt,
  skipabove=\topsep,
  skipbelow=\topsep,
]{boite_remarque}


\newenvironment{definition}[2]
{
    \vspace{15pt}
    \begin{boite_definition}
    \textbf{\textcolor{rouge}{Définition #1}}
    \if\relax\detokenize{#2}\relax
    \else
        \textit{ - #2}
    \fi \\ \\
}
{
    \end{boite_definition}
    
}

\newenvironment{exemple}[2]
{
    \vspace{15pt}
    \begin{boite_exemple}
    \textbf{\textcolor{bleu_fonce}{Exemple #1}}
    \if\relax\detokenize{#2}\relax
    \else
        \textit{ - #2}
    \fi \\ \\ 
}
{   
    \end{boite_exemple}
    
}

\newenvironment{proposition}[2]
{
    \vspace{15pt}
    \begin{boite_proposition}
    \textbf{\textcolor{rouge}{Proposition #1}}
    \if\relax\detokenize{#2}\relax
    \else
        \textit{ - #2}
    \fi \\ \\
}
{
    \end{boite_proposition}
    
}

\newenvironment{theoreme}[2]
{
    \vspace{15pt}
    \begin{boite_proposition}
    \textbf{\textcolor{rouge}{Théorème #1}} 
    \if\relax\detokenize{#2}\relax
    \else
        \textit{ - #2}
    \fi \\ \\
}
{
    \end{boite_proposition}
    
}

\newenvironment{demonstration}
{
    \vspace{15pt}
    \begin{boite_demonstration}
    \textbf{\textcolor{rouge}{Démonstration}}\\ \\
}
{
    \end{boite_demonstration}
    
}

\newenvironment{remarque}[2]
{
    \vspace{15pt}
    \begin{boite_remarque}
    \textbf{\textcolor{bleu_fonce}{Remarque #1}}
    \if\relax\detokenize{#2}\relax
    \else
        \textit{ - #2}
    \fi \\ \\   
}
{  
    \end{boite_remarque}
    
}



%Corrections
\newmdenv[
  nobreak=true,
  topline=true,
  bottomline=true,
  rightline=true,
  leftline=true,
  linewidth=0.5pt,
  linecolor=black,
  backgroundcolor=mayonnaise,
  innerleftmargin=10pt,
  innerrightmargin=10pt,
  innertopmargin=5pt,
  innerbottommargin=5pt,
  skipabove=\topsep,
  skipbelow=\topsep,
]{boite_question}


\newenvironment{question}[2]
{
    \vspace{15pt}
    \begin{boite_question}
    \textbf{\textcolor{rouge}{Question #1}}
    \if\relax\detokenize{#2}\relax
    \else
        \textit{ - #2}
    \fi \\ \\
}
{
    \end{boite_question}
    
}

\newenvironment{enumeratebf}{
    \begin{enumerate}[label=\textbf{\arabic*.}]
}
{
    \end{enumerate}
}

\begin{document}
\begin{adjustwidth}{-3cm}{-3cm}
\begin{document}
\begin{adjustwidth}{-3cm}{-3cm}
% commandes
\newcommand{\notion}[1]{\textcolor{vert_fonce}{\textit{#1}}}
\newcommand{\mb}[1]{\mathbb{#1}}
\newcommand{\mc}[1]{\mathcal{#1}}
\newcommand{\mr}[1]{\mathrm{#1}}
\newcommand{\code}[1]{\texttt{#1}}
\newcommand{\ccode}[1]{\texttt{|#1|}}
\newcommand{\ov}[1]{\overline{#1}}
\newcommand{\abs}[1]{|#1|}
\newcommand{\rev}[1]{\texttt{reverse(#1)}}
\newcommand{\crev}[1]{\texttt{|reverse(#1)|}}

\newcommand{\ie}{\textit{i.e.} }

\newcommand{\N}{\mathbb{N}}
\newcommand{\R}{\mathbb{R}}
\newcommand{\C}{\mathbb{C}}
\newcommand{\K}{\mathbb{K}}
\newcommand{\Z}{\mathbb{Z}}

\newcommand{\A}{\mathcal{A}}
\newcommand{\bigO}{\mathcal{O}}
\renewcommand{\L}{\mathcal{L}}

\newcommand{\rg}[0]{\mathrm{rg}}
\newcommand{\re}[0]{\mathrm{Re}}
\newcommand{\im}[0]{\mathrm{Im}}
\newcommand{\cl}[0]{\mathrm{cl}}
\newcommand{\grad}[0]{\vec{\mathrm{grad}}}
\renewcommand{\div}[0]{\mathrm{div}\,}
\newcommand{\rot}[0]{\vec{\mathrm{rot}}\,}
\newcommand{\vnabla}[0]{\vec{\nabla}}
\renewcommand{\vec}[1]{\overrightarrow{#1}}
\newcommand{\mat}[1]{\mathrm{Mat}_{#1}}
\newcommand{\matrice}[1]{\mathcal{M}_{#1}}
\newcommand{\sgEngendre}[1]{\left\langle #1 \right\rangle}
\newcommand{\gpquotient}[1]{\mathbb{Z} / #1\mathbb{Z}}
\newcommand{\norme}[1]{||#1||}
\renewcommand{\d}[1]{\,\mathrm{d}#1}
\newcommand{\adh}[1]{\overline{#1}}
\newcommand{\intint}[2]{\llbracket #1 ,\, #2 \rrbracket}
\newcommand{\seg}[2]{[#1\, ; \, #2]}
\newcommand{\scal}[2]{( #1 | #2 )}
\newcommand{\distance}[2]{\mathrm{d}(#1,\,#2)}
\newcommand{\inte}[2]{\int_{#1}^{#2}}
\newcommand{\somme}[2]{\sum_{#1}^{#2}}
\newcommand{\deriveref}[4]{\biggl( \frac{\text{d}^{#1}#2}{\text{d}#3^{#1}} \biggr)_{#4}}






\begin{proposition}{4.2 (6)}{intersection finie de voisinage}
    Soit $E$ un espace vectoriel normé. L'intersection finie de voisinages d'un élément $x \in E$ est un voisinage de $x$.
\end{proposition}

\begin{definition}{4.4}{ouvert}
    Soit $E$ un espace vectoriel normé. On appelle ouvert de $E$ une partie $O$ de $E$ qui est un \notion{voisinage de chacun de ses points}, \ie~:
    $$\forall x \in O,\, \exists r > 0,\, \mc{B}(x,r) \subset O$$
\end{definition}

\begin{proposition}{4.5 (5)}{intersection finie de voisinage}
    Soit $E$ un espace vectoriel normé. L'intersection finie d'ouverts de $E$ est un ouvert de $E$.
\end{proposition}

\begin{definition}{4.6}{fermé}
    Soit $E$ un espace vectoriel normé. On appelle fermé de $E$ \notion{le complémentaire dans $E$ d'un ouvert $O$ de $E$}.
\end{definition}


\begin{definition}{4.9}{point adhérent à une partie}
    Soit $E$ un espace vectoriel normé et $A$ une partie de $E$. Un point $x \in E$ est dit \notion{adhérent à $A$} si tout voisinage de $x$ rencontre $A$~:
    $$\forall V \in \mc{V}_E(x),\, V \cap A \neq \varnothing$$
    On appelle \notion{adhérence de $A$ l'ensemble $\adh{A}$ des éléments de $E$ adhérents à $A$}.
\end{definition}

\begin{proposition}{4.10}{caractérisation de l'adhérence d'une partie}
    Soit $E$ un espace vectoriel normé et $A$ une partie de $E$. Pour tout $x \in E$, les propriétés suivantes sont équivalentes.
    \begin{enumeratebf}
        \item $x$ est adhérent à $A$ ;
        \item Toute boule ouverte de centre $x$ contient au moins un élément de $A$ ;
        \item la distance de $x$ à $A$ est nulle.
    \end{enumeratebf}
\end{proposition}

\begin{definition}{4.12 (1)}{point d'accumulation d'une partie}
    Soit $E$ un espace vectoriel normé et $A$ une partie de $E$. Un point $x$ est un \notion{point d'accumulation de $A$} si tout voisinage de $x$ rencontre $A$ en au moins un autre point que $x$~:
    $$\forall V \in \mc{V}_E(x),\, (V\setminus\{x\}) \cap A \neq \varnothing$$
\end{definition}

\begin{definition}{4.12 (2)}{point isolé d'une partie}
    Soit $E$ un espace vectoriel normé et $A$ une partie de $E$. Un point $x$ est un \notion{point isolé de $A$} s'il existe un voisinage de $x$ qui ne rencontre $A$ qu'en $x$~:
    $$\exists V \in \mc{V}_E(x),\, V \cap A = \{x\}$$
\end{definition}

\begin{definition}{4.15}{partie dense}
    Soit $E$ un espace vectoriel normé et $A$ et $B$ des parties de $E$. On dit que \notion{$A$ est dense dans $B$} si $\adh{A} = B$
\end{definition} 

\begin{theoreme}{4.17}{caractérisation de l'adhérence par l'inclusion}
    Soit $E$ un espace vectoriel normé et $A$ une partie de $E$. $\adh{A}$ est le plus petit fermé de $E$ contenant $A$.
\end{theoreme}

\begin{theoreme}{4.18}{caractérisation de l'adhérence par l'intersection}
    Soit $E$ un espace vectoriel normé et $A$ une partie de $E$. $\adh{A}$ est l'intersection de tous les fermés de $E$ contenant $A$.
\end{theoreme}

\begin{theoreme}{4.19}{caractérisation du caractère fermé}
    Soit $E$ un espace vectoriel normé et $A$ une partie de $E$. $A$ est fermée si et seulement si elle contient tous ses points adhérents~:
    $$\ov{A} = A$$
\end{theoreme}

\begin{theoreme}{4.20}{caractérisation séquentielle de l'adhérence}
    Soit $E$ un espace vectoriel normé et $A$ une partie de $E$. $\adh{A}$ est l'ensemble des limites des suites convergentes d'éléments de $A$.
\end{theoreme}

\begin{definition}{4.22}{intérieur d'une partie}
    Soit $E$ un espace vectoriel normé et $A$ une partie de $E$. \notion{$x\in A$ est intérieur à $A$} lorsque $A$ est un voisinage de $x$~:
    $$\exists \epsilon >0,\, \mc{B}(x,\epsilon) \subset A$$
    l'ensemble des éléments de $E$ intérieurs à $A$ est appelé \notion{intérieur $\mathring{A}$ de $A$}.
\end{definition}

\begin{theoreme}{4.24}{caractérisation de l'intérieur d'une partie}
    Soit $E$ un espace vectoriel normé et $A$ une partie de $E$. $\mathring{A}$ est le plus grand ouvert de $E$ inclus dans $A$.
\end{theoreme}

\begin{theoreme}{4.25}{lien entre adhérence et intérieur}
    Soit $E$ un espace vectoriel normé et $A$ une partie de $E$.
    \begin{enumeratebf}
        \item Le complémentaire de l’adhérence de $A$ est l’intérieur du complémentaire de $A$~: $\complement_E \ov{A} = \mathring{\Big(\complement_E A \Big)}$
        \item Le complémentaire de l’intérieur de $A$ est l’adhérence du complémentaire de $A$~: $\complement_E \mathring{A} = \ov{\Big(\complement_E A \Big)}$
    \end{enumeratebf}
\end{theoreme}

\begin{definition}{4.26}{frontière d'une partie}
    Soit $E$ un espace vectoriel normé et $A$ une partie de $E$. On appelle \notion{point frontière de $A$} un point adhérent à $A$ qui n’est pas intérieur à $A$. L’ensemble des points frontières
est appelé \notion{frontière $\mr{Fr}(A)$ de $A$}~:
$$\mr{Fr}(A) = \ov{A}  \setminus \mathring{A} = \ov{A} \cap \ov{\Big(\complement_E A\Big)}$$
\end{definition}

\begin{definition}{4.29}{voisinage relatif à une partie}
    Soit $E$ un espace vectoriel normé et $A$ une partie de $E$. On appelle \notion{voisinage de $a \in \ov{A}$ relatif à $A$} l'intersection de $A$ avec un voisinage de $a$.
\end{definition}

\begin{definition}{4.30}{ouvert, fermé relatifs à une partie}
    Soit $E$ un espace vectoriel normé et $A$ une partie de $E$. 
    \begin{enumeratebf}
        \item On appelle \notion{ouvert relatif de $A$} l'intersection de $A$ avec un ouvert de $E$.
        \item On appelle \notion{fermé relatif de $A$} l'intersection de $A$ avec un fermé de $E$.
    \end{enumeratebf}
\end{definition}

\begin{theoreme}{4.46}{caractérisation séquentielle de la limite d'une fonction}
    Soit $E$, $F$ deux espaces vectoriels normés, $A$ une partie de $E$ et $f : A \to F$. $f$ admet une limite $l\in F$ en $a \in \ov{A}$ si et seulement si pour toute suite $(u_n)_{n\in \N}$ d’éléments de $A$ convergeant vers $a$, la suite $\big(f(u_n)\big)_{n\in \N}$ converge vers $l$.
\end{theoreme}

\begin{theoreme}{4.47}{caractérisation séquentielle de la continuité d'une fonction}
    Soit $E$, $F$ deux espaces vectoriels normés, $A$ une partie de $E$ et $f : A \to F$. $f$ est continue en $a \in \ov{A}$ si et seulement si pour toute suite $(u_n)_{n\in \N}$ d’éléments de $A$ convergeant vers $a$, la suite $\big(f(u_n)\big)_{n\in \N}$ converge vers $f(a)$.
\end{theoreme}

\begin{theoreme}{4.55}{applications coïncidant sur une partie dense}
    Soit $E$ et $F$ deux espaces vectoriels normés et $A$ une partie de $E$. Deux applications continues de $A$ dans $F$ qui coïncident sur une partie $B$ dense dans $A$ sont égales.
\end{theoreme}

\begin{theoreme}{4.56}{caractérisation de la continuité par l'image réciproque}
    Soit $E$, $F$ deux espaces vectoriels normés, $A$ une partie de $E$ et $f : A \to F$. Les trois propriétés suivantes sont équivalentes :
    \begin{enumeratebf}
        \item $f$ est continue.
        \item L’image réciproque par $f$ de tout fermé de $F$ est un fermé relatif de $A$.
        \item  L’image réciproque par $f$ de tout ouvert de $F$ est un ouvert relatif de $A$.
    \end{enumeratebf}
\end{theoreme}

\begin{definition}{4.59}{homéomorphisme}
    Soit $E$ et $F$ deux espaces vectoriels normés. On appelle \notion{homéomorphisme de $E$ dans $F$} une application continue et bijective de $E$ dans $F$ dont la bijection réciproque est aussi continue.
\end{definition}

\begin{theoreme}{4.65}{caractérisation des applications linéaires continues}
    Soit $E$ et $F$ deux espaces vectoriels normés et $f$ une application linéaire de $E$ dans $F$. Les propositions suivantes sont équivalentes~:
    \begin{enumeratebf}
        \item $f$ est continue sur $E$.
        \item $f$ est continue en $0_E$.
        \item $f$ est bornée sur la boule unité fermée.
        \item $\exists K \in \R_+,\, \forall x \in E,\, \norme{f(x)}_F \leq K\norme{x}_E$.
        \item $f$ est lipschitzienne sur $E$.
        \item $f$ est uniformément continue sur $E$.
    \end{enumeratebf}
\end{theoreme}

\begin{theoreme}{4.69}{continuité des applications linéaires de source de dimension finie}
    Soit $E$ et $F$ deux espaces vectoriels normés. Si $E$ est de dimension finie, alors toute application linéaire de $E$ dans $F$ est continue.
\end{theoreme}

\begin{theoreme}{4.71}{caractérisation des applications bilinéaires continues}
    Soit $E$, $F$ et $G$ trois espaces vectoriels normés et $\varphi$ une application bilinéaire de $E \times F$ dans $G$. Alors l’application $\varphi$ est continue sur $E \times F$ si et seulement si~:
    $$\exists K \in \R_+,\, \forall (x,y) \in E \times F,\, \norme{\varphi(x,\, y)}_G \leq K \norme{x}_E \norme{y}_F$$
\end{theoreme}

\begin{definition}{4.75}{norme subordonnée d'une application linéaire continue}
    Soit $E$ et $F$ deux espaces vectoriels normés. Pour toute application linéaire continue de $E$ dans $F$, l’ensemble des normes des images des éléments de la boule unité fermée de $E$ est une partie non vide et majorée de $\R$ : elle admet une borne supérieure, notée~:
    $$\normet{f} = \sup_{\norme{x} \leq 1} \norme{f(x)}$$
    On appelle \notion{norme subordonnée aux normes de $E$ et $F$} l'application ~: 
    $$\fonction{\normet{\cdot}}{\mc{L}\mc{C}(E,F)}{\R_+}{f}{\normet{f}}$$
    C'est une norme.
\end{definition}

\begin{theoreme}{4.77}{caractérisation de la norme triple}
    Soit $f \in \mc{L}\mc{C}(E,F)$, où $E$ et $F$ sont deux espaces vectoriels normés. La norme triple de $f$ est le plus petit réel $K>0$ tel que pour tout $x\E$, $\norme{f(x)}_F \leq K \norme{x}_E$~:
    $$\normet{f} = \inf \{K>0,\, \forall x \in E,\, \norme{f(x)}_F \leq K \norme{x}_E \}$$
    On a aussi~:
    $$\normet{f} = \sup_{\norme{x}_E \neq 0_E} \frac{\norme{f(x)}_F}{\norme{x}_E}$$
\end{theoreme}

\begin{theoreme}{4.78}{sous-multiplicativité de la norme triple}
    Soit $E$, $F$ et $G$ trois espaces vectoriels normés. Si $f$ est une application linéaire continue de $E$ dans $F$ et $g$ une application linéaire continue de $F$ dans $G$, alors $g \circ f$ est une application linéaire continue de $E$ dans $G$ et :
    $$\normet{g \circ f} \leq \normet{f} \times \normet{g}$$
\end{theoreme}

\begin{definition}{4.79}{algèbre normée unitaire}
    On appelle \notion{algèbre normée unitaire} une algèbre $\mc{A}$ munie d’une norme $\norme{\cdot}$ telle que~:
    $$\begin{cases*}
        \norme{e} = 1\\
        \forall (x,y) \in \mc{A}^2,\, \norme{x \times y} \leq \norme{x} \norme{y}
    \end{cases*}$$
\end{definition}

\end{adjustwidth}
\end{document}