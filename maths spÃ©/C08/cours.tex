\documentclass{article}
\usepackage{amsmath,amssymb,mathtools}
\usepackage{xcolor}
\usepackage{minted}
\usepackage{enumitem}
\usepackage{multicol}
\usepackage{changepage}
\usepackage{stmaryrd}
\usepackage{graphicx}
\graphicspath{ {./images/} }
\usepackage[framemethod=tikz]{mdframed}
\usepackage{tikz,pgfplots}
\pgfplotsset{compat=1.18}

% physique
\definecolor{oranges}{RGB}{255, 242, 230}
\definecolor{rouges}{RGB}{255, 230, 230}
\definecolor{rose}{RGB}{255, 204, 204}

% maths - info
\definecolor{rouge_fonce}{RGB}{204, 0, 0}
\definecolor{rouge}{RGB}{255, 0, 0}
\definecolor{bleu_fonce}{RGB}{0, 0, 255}
\definecolor{vert_fonce}{RGB}{0, 69, 33}
\definecolor{vert}{RGB}{0,255,0}

\definecolor{orange_foncee}{RGB}{255, 153, 0}
\definecolor{myrtille}{RGB}{225, 225, 255}
\definecolor{mayonnaise}{RGB}{255, 253, 233}
\definecolor{magenta}{RGB}{224, 209, 240}
\definecolor{pomme}{RGB}{204, 255, 204}
\definecolor{mauve}{RGB}{255, 230, 255}


% Cours

\newmdenv[
  nobreak=true,
  topline=true,
  bottomline=true,
  rightline=true,
  leftline=true,
  linewidth=0.5pt,
  linecolor=black,
  backgroundcolor=mayonnaise,
  innerleftmargin=10pt,
  innerrightmargin=10pt,
  innertopmargin=5pt,
  innerbottommargin=5pt,
  skipabove=\topsep,
  skipbelow=\topsep,
]{boite_definition}

\newmdenv[
  nobreak=false,
  topline=true,
  bottomline=true,
  rightline=true,
  leftline=true,
  linewidth=0.5pt,
  linecolor=white,
  backgroundcolor=white,
  innerleftmargin=10pt,
  innerrightmargin=10pt,
  innertopmargin=5pt,
  innerbottommargin=5pt,
  skipabove=\topsep,
  skipbelow=\topsep,
]{boite_exemple}

\newmdenv[
  nobreak=true,
  topline=true,
  bottomline=true,
  rightline=true,
  leftline=true,
  linewidth=0.5pt,
  linecolor=black,
  backgroundcolor=magenta,
  innerleftmargin=10pt,
  innerrightmargin=10pt,
  innertopmargin=5pt,
  innerbottommargin=5pt,
  skipabove=\topsep,
  skipbelow=\topsep,
]{boite_proposition}

\newmdenv[
  nobreak=true,
  topline=true,
  bottomline=true,
  rightline=true,
  leftline=true,
  linewidth=0.5pt,
  linecolor=black,
  backgroundcolor=white,
  innerleftmargin=10pt,
  innerrightmargin=10pt,
  innertopmargin=5pt,
  innerbottommargin=5pt,
  skipabove=\topsep,
  skipbelow=\topsep,
]{boite_demonstration}

\newmdenv[
  nobreak=true,
  topline=true,
  bottomline=true,
  rightline=true,
  leftline=true,
  linewidth=0.5pt,
  linecolor=white,
  backgroundcolor=white,
  innerleftmargin=10pt,
  innerrightmargin=10pt,
  innertopmargin=5pt,
  innerbottommargin=5pt,
  skipabove=\topsep,
  skipbelow=\topsep,
]{boite_remarque}


\newenvironment{definition}[2]
{
    \vspace{15pt}
    \begin{boite_definition}
    \textbf{\textcolor{rouge}{Définition #1}}
    \if\relax\detokenize{#2}\relax
    \else
        \textit{ - #2}
    \fi \\ \\
}
{
    \end{boite_definition}
    
}

\newenvironment{exemple}[2]
{
    \vspace{15pt}
    \begin{boite_exemple}
    \textbf{\textcolor{bleu_fonce}{Exemple #1}}
    \if\relax\detokenize{#2}\relax
    \else
        \textit{ - #2}
    \fi \\ \\ 
}
{   
    \end{boite_exemple}
    
}

\newenvironment{proposition}[2]
{
    \vspace{15pt}
    \begin{boite_proposition}
    \textbf{\textcolor{rouge}{Proposition #1}}
    \if\relax\detokenize{#2}\relax
    \else
        \textit{ - #2}
    \fi \\ \\
}
{
    \end{boite_proposition}
    
}

\newenvironment{theoreme}[2]
{
    \vspace{15pt}
    \begin{boite_proposition}
    \textbf{\textcolor{rouge}{Théorème #1}} 
    \if\relax\detokenize{#2}\relax
    \else
        \textit{ - #2}
    \fi \\ \\
}
{
    \end{boite_proposition}
    
}

\newenvironment{demonstration}
{
    \vspace{15pt}
    \begin{boite_demonstration}
    \textbf{\textcolor{rouge}{Démonstration}}\\ \\
}
{
    \end{boite_demonstration}
    
}

\newenvironment{remarque}[2]
{
    \vspace{15pt}
    \begin{boite_remarque}
    \textbf{\textcolor{bleu_fonce}{Remarque #1}}
    \if\relax\detokenize{#2}\relax
    \else
        \textit{ - #2}
    \fi \\ \\   
}
{  
    \end{boite_remarque}
    
}



%Corrections
\newmdenv[
  nobreak=true,
  topline=true,
  bottomline=true,
  rightline=true,
  leftline=true,
  linewidth=0.5pt,
  linecolor=black,
  backgroundcolor=mayonnaise,
  innerleftmargin=10pt,
  innerrightmargin=10pt,
  innertopmargin=5pt,
  innerbottommargin=5pt,
  skipabove=\topsep,
  skipbelow=\topsep,
]{boite_question}


\newenvironment{question}[2]
{
    \vspace{15pt}
    \begin{boite_question}
    \textbf{\textcolor{rouge}{Question #1}}
    \if\relax\detokenize{#2}\relax
    \else
        \textit{ - #2}
    \fi \\ \\
}
{
    \end{boite_question}
    
}

\newenvironment{enumeratebf}{
    \begin{enumerate}[label=\textbf{\arabic*.}]
}
{
    \end{enumerate}
}

\begin{document}
\begin{adjustwidth}{-3cm}{-3cm}
\begin{document}
\begin{adjustwidth}{-3cm}{-3cm}
% commandes
\newcommand{\notion}[1]{\textcolor{vert_fonce}{\textit{#1}}}
\newcommand{\mb}[1]{\mathbb{#1}}
\newcommand{\mc}[1]{\mathcal{#1}}
\newcommand{\mr}[1]{\mathrm{#1}}
\newcommand{\code}[1]{\texttt{#1}}
\newcommand{\ccode}[1]{\texttt{|#1|}}
\newcommand{\ov}[1]{\overline{#1}}
\newcommand{\abs}[1]{|#1|}
\newcommand{\rev}[1]{\texttt{reverse(#1)}}
\newcommand{\crev}[1]{\texttt{|reverse(#1)|}}

\newcommand{\ie}{\textit{i.e.} }

\newcommand{\N}{\mathbb{N}}
\newcommand{\R}{\mathbb{R}}
\newcommand{\C}{\mathbb{C}}
\newcommand{\K}{\mathbb{K}}
\newcommand{\Z}{\mathbb{Z}}

\newcommand{\A}{\mathcal{A}}
\newcommand{\bigO}{\mathcal{O}}
\renewcommand{\L}{\mathcal{L}}

\newcommand{\rg}[0]{\mathrm{rg}}
\newcommand{\re}[0]{\mathrm{Re}}
\newcommand{\im}[0]{\mathrm{Im}}
\newcommand{\cl}[0]{\mathrm{cl}}
\newcommand{\grad}[0]{\vec{\mathrm{grad}}}
\renewcommand{\div}[0]{\mathrm{div}\,}
\newcommand{\rot}[0]{\vec{\mathrm{rot}}\,}
\newcommand{\vnabla}[0]{\vec{\nabla}}
\renewcommand{\vec}[1]{\overrightarrow{#1}}
\newcommand{\mat}[1]{\mathrm{Mat}_{#1}}
\newcommand{\matrice}[1]{\mathcal{M}_{#1}}
\newcommand{\sgEngendre}[1]{\left\langle #1 \right\rangle}
\newcommand{\gpquotient}[1]{\mathbb{Z} / #1\mathbb{Z}}
\newcommand{\norme}[1]{||#1||}
\renewcommand{\d}[1]{\,\mathrm{d}#1}
\newcommand{\adh}[1]{\overline{#1}}
\newcommand{\intint}[2]{\llbracket #1 ,\, #2 \rrbracket}
\newcommand{\seg}[2]{[#1\, ; \, #2]}
\newcommand{\scal}[2]{( #1 | #2 )}
\newcommand{\distance}[2]{\mathrm{d}(#1,\,#2)}
\newcommand{\inte}[2]{\int_{#1}^{#2}}
\newcommand{\somme}[2]{\sum_{#1}^{#2}}
\newcommand{\deriveref}[4]{\biggl( \frac{\text{d}^{#1}#2}{\text{d}#3^{#1}} \biggr)_{#4}}






\begin{definition}{8.1}{ensemble dénombrable}
    Un ensemble est dit \notion{dénombrable} s'il est en bijection avec $\N$, ce qui revient à pouvoir \notion{numéroter chacun de ses éléments} (sans pour autant manipuler de "dernier élément", ce qui supposerait qu'il soit fini).
\end{definition}

\begin{proposition}{8.5}{parties infinies de $\N$}
    Toute partie infinie de $\N$ est dénombrable.
\end{proposition}

\begin{proposition}{8.10}{réunion d'ensembles dénombrables}
    Une réunion finie ou dénombrable d’ensembles dénombrables est dénombrable.
\end{proposition}

\begin{theoreme}{8.13}{$\R$ n'est pas dénombrable}
    L’ensemble $\R$ n’est pas dénombrable.
\end{theoreme}

\begin{definition}{8.14}{famille sommable de réels positifs}
    On dit qu’une \notion{famille $(u_i)_{i \in I}$ de nombres réels positifs est sommable} lorsqu’il existe $M\geq 0$ tel que, pour toute partie finie $J \subset I$ , on ait $\sum_{j \in J}u_j \leq M$.
    On définit alors \notion{la somme de la famille} par~:
    $$\sum_{i \in I}u_i = \sup_{\substack{J \subset I \\ J\text{ finie}}} \sum_{j \in J} u_j$$
\end{definition}

\begin{proposition}{8.15}{sommabilité d'une famille de réels positifs}
    Une famille $(u_i)_{i \in \N}$ de réels positifs indexée par $\N$ est sommable si et seulement si la série $\sum_n u_n$ est convergente.
\end{proposition}

\begin{theoreme}{8.17}{de Fubini positif}
    Soit $(u_i)_{i\in I}$ une famille de nombres réels positifs et $\displaystyle I = \bigsqcup_{n\in \N} I_n$ une partition dénombrable de $I$. La famille $(u_i)_{i\in I}$ est sommable si et seulement si~:
    $$\begin{cases*}
        \forall n \in \N,\, (u_i)_{i \in I_n}& est sommable \\
        \sum_n\bigg( \sum_{i\in I_n} u_i \bigg) & est convergente
    \end{cases*}$$

    Le cas échéant,
    $$\sum_{n=0}^{+ \infty}\bigg( \sum_{i\in I_n} u_i \bigg) = \sum_{i\in I}u_i$$
\end{theoreme}

\begin{definition}{8.20}{famille sommable de réels ou de complexes}
    Une \notion{famille $(u_i)_{i \in I}$ est sommable} si la famille $(\abs{u_i})_{i \in I}$ l'est.
\end{definition}

\begin{proposition}{8.21}{inégalité triangulaire}
    Soit $(u_i)_{i \in I}$ un famille sommable de nombres complexes. On a~:
    $$\Bigg\vert \somme{i \in I}{}u_i \Bigg\vert \leq \somme{i \in I}{}\abs{u_i}$$
\end{proposition}

\begin{theoreme}{8.17}{de Fubini complexe}
    Soit $(u_i)_{i\in I}$ une famille de nombres complexes et $\displaystyle I = \bigsqcup_{n\in \N} I_n$ une partition dénombrable de $I$. La famille $(u_i)_{i\in I}$ est sommable si et seulement si~:
    $$\begin{cases*}
        \forall n \in \N,\, (u_i)_{i \in I_n}& est sommable \\
        \sum_n\bigg( \sum_{i\in I_n} \abs{u_i} \bigg) & est convergente
    \end{cases*}$$

    Le cas échéant,
    $$\sum_{n=0}^{+ \infty}\bigg( \sum_{i\in I_n} u_i \bigg) = \sum_{i\in I}u_i$$
\end{theoreme}

\begin{theoreme}{8.28}{interversion des sommations de complexes}
    Soit $(u_{p,q})_{(p,q) \in \N^2}$ est une famille nombres complexes. La famille $(u_{p,q})_{(p,q) \in \N^2}$ est sommable si et seulement si~:
    $$\begin{cases*}
        \displaystyle \forall q \in \N,\, \somme{p}{}\abs{u_{p,q}} &est convergente \\
        \displaystyle \somme{q}{}\somme{p=0}{+ \infty}\abs{u_{p,q}} &est convergente
    \end{cases*}$$
    si et seulement si~:
    $$\begin{cases*}
        \displaystyle \forall p \in \N,\, \somme{q}{}\abs{u_{p,q}} &est convergente \\
        \displaystyle \somme{p}{}\somme{q=0}{+ \infty}\abs{u_{p,q}} &est convergente
    \end{cases*}$$
    et le cas échéant~:
    $$\sum_{(p,q) \in \N^2}u_{p,q} = \somme{p=0}{+\infty}\somme{q=0}{+\infty}u_{p,q} = \somme{q=0}{+\infty}\somme{p=0}{+\infty}u_{p,q}$$
\end{theoreme}

\begin{proposition}{8.29}{produit de familles sommables de nombres complexes}
    Soit $(a_p)_{p \in \N}$ et $(b_q)_{q \in \N}$ deux familles sommables de nomnbres complexes. Alors la famille $(a_pb_q)_{(p,q) \in \N^2}$ est sommable et~:
    $$\sum_{(p,q) \in \N^2}a_pb_q = \Bigg(\sum_{p=0}^{+\infty} a_p \Bigg) \times \Bigg( \sum_{q=0}^{+\infty} b_q \Bigg) $$
\end{proposition}
\end{adjustwidth}
\end{document}