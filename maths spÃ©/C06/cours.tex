\documentclass{article}
\usepackage{amsmath,amssymb,mathtools}
\usepackage{xcolor}
\usepackage{minted}
\usepackage{enumitem}
\usepackage{multicol}
\usepackage{changepage}
\usepackage{stmaryrd}
\usepackage{graphicx}
\graphicspath{ {./images/} }
\usepackage[framemethod=tikz]{mdframed}
\usepackage{tikz,pgfplots}
\pgfplotsset{compat=1.18}

% physique
\definecolor{oranges}{RGB}{255, 242, 230}
\definecolor{rouges}{RGB}{255, 230, 230}
\definecolor{rose}{RGB}{255, 204, 204}

% maths - info
\definecolor{rouge_fonce}{RGB}{204, 0, 0}
\definecolor{rouge}{RGB}{255, 0, 0}
\definecolor{bleu_fonce}{RGB}{0, 0, 255}
\definecolor{vert_fonce}{RGB}{0, 69, 33}
\definecolor{vert}{RGB}{0,255,0}

\definecolor{orange_foncee}{RGB}{255, 153, 0}
\definecolor{myrtille}{RGB}{225, 225, 255}
\definecolor{mayonnaise}{RGB}{255, 253, 233}
\definecolor{magenta}{RGB}{224, 209, 240}
\definecolor{pomme}{RGB}{204, 255, 204}
\definecolor{mauve}{RGB}{255, 230, 255}


% Cours

\newmdenv[
  nobreak=true,
  topline=true,
  bottomline=true,
  rightline=true,
  leftline=true,
  linewidth=0.5pt,
  linecolor=black,
  backgroundcolor=mayonnaise,
  innerleftmargin=10pt,
  innerrightmargin=10pt,
  innertopmargin=5pt,
  innerbottommargin=5pt,
  skipabove=\topsep,
  skipbelow=\topsep,
]{boite_definition}

\newmdenv[
  nobreak=false,
  topline=true,
  bottomline=true,
  rightline=true,
  leftline=true,
  linewidth=0.5pt,
  linecolor=white,
  backgroundcolor=white,
  innerleftmargin=10pt,
  innerrightmargin=10pt,
  innertopmargin=5pt,
  innerbottommargin=5pt,
  skipabove=\topsep,
  skipbelow=\topsep,
]{boite_exemple}

\newmdenv[
  nobreak=true,
  topline=true,
  bottomline=true,
  rightline=true,
  leftline=true,
  linewidth=0.5pt,
  linecolor=black,
  backgroundcolor=magenta,
  innerleftmargin=10pt,
  innerrightmargin=10pt,
  innertopmargin=5pt,
  innerbottommargin=5pt,
  skipabove=\topsep,
  skipbelow=\topsep,
]{boite_proposition}

\newmdenv[
  nobreak=true,
  topline=true,
  bottomline=true,
  rightline=true,
  leftline=true,
  linewidth=0.5pt,
  linecolor=black,
  backgroundcolor=white,
  innerleftmargin=10pt,
  innerrightmargin=10pt,
  innertopmargin=5pt,
  innerbottommargin=5pt,
  skipabove=\topsep,
  skipbelow=\topsep,
]{boite_demonstration}

\newmdenv[
  nobreak=true,
  topline=true,
  bottomline=true,
  rightline=true,
  leftline=true,
  linewidth=0.5pt,
  linecolor=white,
  backgroundcolor=white,
  innerleftmargin=10pt,
  innerrightmargin=10pt,
  innertopmargin=5pt,
  innerbottommargin=5pt,
  skipabove=\topsep,
  skipbelow=\topsep,
]{boite_remarque}


\newenvironment{definition}[2]
{
    \vspace{15pt}
    \begin{boite_definition}
    \textbf{\textcolor{rouge}{Définition #1}}
    \if\relax\detokenize{#2}\relax
    \else
        \textit{ - #2}
    \fi \\ \\
}
{
    \end{boite_definition}
    
}

\newenvironment{exemple}[2]
{
    \vspace{15pt}
    \begin{boite_exemple}
    \textbf{\textcolor{bleu_fonce}{Exemple #1}}
    \if\relax\detokenize{#2}\relax
    \else
        \textit{ - #2}
    \fi \\ \\ 
}
{   
    \end{boite_exemple}
    
}

\newenvironment{proposition}[2]
{
    \vspace{15pt}
    \begin{boite_proposition}
    \textbf{\textcolor{rouge}{Proposition #1}}
    \if\relax\detokenize{#2}\relax
    \else
        \textit{ - #2}
    \fi \\ \\
}
{
    \end{boite_proposition}
    
}

\newenvironment{theoreme}[2]
{
    \vspace{15pt}
    \begin{boite_proposition}
    \textbf{\textcolor{rouge}{Théorème #1}} 
    \if\relax\detokenize{#2}\relax
    \else
        \textit{ - #2}
    \fi \\ \\
}
{
    \end{boite_proposition}
    
}

\newenvironment{demonstration}
{
    \vspace{15pt}
    \begin{boite_demonstration}
    \textbf{\textcolor{rouge}{Démonstration}}\\ \\
}
{
    \end{boite_demonstration}
    
}

\newenvironment{remarque}[2]
{
    \vspace{15pt}
    \begin{boite_remarque}
    \textbf{\textcolor{bleu_fonce}{Remarque #1}}
    \if\relax\detokenize{#2}\relax
    \else
        \textit{ - #2}
    \fi \\ \\   
}
{  
    \end{boite_remarque}
    
}



%Corrections
\newmdenv[
  nobreak=true,
  topline=true,
  bottomline=true,
  rightline=true,
  leftline=true,
  linewidth=0.5pt,
  linecolor=black,
  backgroundcolor=mayonnaise,
  innerleftmargin=10pt,
  innerrightmargin=10pt,
  innertopmargin=5pt,
  innerbottommargin=5pt,
  skipabove=\topsep,
  skipbelow=\topsep,
]{boite_question}


\newenvironment{question}[2]
{
    \vspace{15pt}
    \begin{boite_question}
    \textbf{\textcolor{rouge}{Question #1}}
    \if\relax\detokenize{#2}\relax
    \else
        \textit{ - #2}
    \fi \\ \\
}
{
    \end{boite_question}
    
}

\newenvironment{enumeratebf}{
    \begin{enumerate}[label=\textbf{\arabic*.}]
}
{
    \end{enumerate}
}

\begin{document}
\begin{adjustwidth}{-3cm}{-3cm}
\begin{document}
\begin{adjustwidth}{-3cm}{-3cm}
% commandes
\newcommand{\notion}[1]{\textcolor{vert_fonce}{\textit{#1}}}
\newcommand{\mb}[1]{\mathbb{#1}}
\newcommand{\mc}[1]{\mathcal{#1}}
\newcommand{\mr}[1]{\mathrm{#1}}
\newcommand{\code}[1]{\texttt{#1}}
\newcommand{\ccode}[1]{\texttt{|#1|}}
\newcommand{\ov}[1]{\overline{#1}}
\newcommand{\abs}[1]{|#1|}
\newcommand{\rev}[1]{\texttt{reverse(#1)}}
\newcommand{\crev}[1]{\texttt{|reverse(#1)|}}

\newcommand{\ie}{\textit{i.e.} }

\newcommand{\N}{\mathbb{N}}
\newcommand{\R}{\mathbb{R}}
\newcommand{\C}{\mathbb{C}}
\newcommand{\K}{\mathbb{K}}
\newcommand{\Z}{\mathbb{Z}}

\newcommand{\A}{\mathcal{A}}
\newcommand{\bigO}{\mathcal{O}}
\renewcommand{\L}{\mathcal{L}}

\newcommand{\rg}[0]{\mathrm{rg}}
\newcommand{\re}[0]{\mathrm{Re}}
\newcommand{\im}[0]{\mathrm{Im}}
\newcommand{\cl}[0]{\mathrm{cl}}
\newcommand{\grad}[0]{\vec{\mathrm{grad}}}
\renewcommand{\div}[0]{\mathrm{div}\,}
\newcommand{\rot}[0]{\vec{\mathrm{rot}}\,}
\newcommand{\vnabla}[0]{\vec{\nabla}}
\renewcommand{\vec}[1]{\overrightarrow{#1}}
\newcommand{\mat}[1]{\mathrm{Mat}_{#1}}
\newcommand{\matrice}[1]{\mathcal{M}_{#1}}
\newcommand{\sgEngendre}[1]{\left\langle #1 \right\rangle}
\newcommand{\gpquotient}[1]{\mathbb{Z} / #1\mathbb{Z}}
\newcommand{\norme}[1]{||#1||}
\renewcommand{\d}[1]{\,\mathrm{d}#1}
\newcommand{\adh}[1]{\overline{#1}}
\newcommand{\intint}[2]{\llbracket #1 ,\, #2 \rrbracket}
\newcommand{\seg}[2]{[#1\, ; \, #2]}
\newcommand{\scal}[2]{( #1 | #2 )}
\newcommand{\distance}[2]{\mathrm{d}(#1,\,#2)}
\newcommand{\inte}[2]{\int_{#1}^{#2}}
\newcommand{\somme}[2]{\sum_{#1}^{#2}}
\newcommand{\deriveref}[4]{\biggl( \frac{\text{d}^{#1}#2}{\text{d}#3^{#1}} \biggr)_{#4}}






\begin{definition}{6.16}{ensemble $\ell^1(E)$}
    Si $E$ est un $\K$-espace vectoriel normé,\, on note $\ell^1(E)$ l'ensemble des séries à termes dans $E^\N$ absolument convergentes. L'application suivante est alors une norme de $\ell^1(E)$~:
    $$\fonction{N_1}{\ell^1(E)}{\R_+}{u}{\somme{n=0}{+\infty}\norme{u_n}_E}$$
\end{definition}

\begin{definition}{6.19}{série géométrique en algèbre normée unitaire}
    Soit $\mc{A}$ une algèbre normée unitaire de dimension finie. Pour tout élément $u$ de $\mc{A}$ tel que $\norme{u}_{\mc{A}}<1$,  la série $\sum_n u^n$ est appelée \notion{série géométrique}. Cette série est absolument convergente et sa somme vaut~:
    $$\sum_{n=0}^{+\infty}u^n = (1_\mc{A} - u)^{-1}$$
\end{definition}

\begin{definition}{6.22}{série exponentielle}
   Soit $\mc{A}$ une algèbre normée unitaire de dimension finie. Pour tout élément $u$ de $\mc{A}$, la série $\displaystyle \sum_{n} \frac{u^n}{n!}$ est absolument convergente et appelée \notion{série exponentielle}. On note sa somme~:
    $$\exp(u) = \sum_{n=0}^{+\infty}\frac{u^n}{n!}$$
\end{definition}

\begin{definition}{6.26}{série alternée}
    On appelle \notion{série alternée} tout série $\displaystyle \sum_{n} u_n$ dont le terme général s'écrit $u_n = (-1)^na_n$, où $(a_n)_{n \in \N}$ est une suite de signe constant.
\end{definition}

\begin{theoreme}{6.27}{critère spécial des séries alternées}
    Soit $\displaystyle \sum_{n} u_n$ une série alternée. Si $(|u_n|)_{n \in \N}$ est décroissante et converge vers $0$, alors la série $\displaystyle \sum_{n} u_n$ converge.
\end{theoreme}

\begin{definition}{6.30}{produit de Cauchy de suites}
    Soit $\mc{A}$ une $\K$-algèbre normée unitaire de dimension finie. On appelle \notion{produit de Cauchy $u \star v$ de deux suites $(u_n)_{n\in\N}$ et $(v_n)_{n\in \N}$ de $\mc{A}^\N$} la suite $(w_n)_{n \in \N}$ définie par~:
    $$\forall n \in \N,\, w_n = \sum_{k=0}^{n}u_kv_{n-k} = \sum_{i+j = n}u_i v_j$$
   bien dans $\mc{A}^\N$, qui est alors muni d'une structure de $\K$-algèbre.
\end{definition}

\begin{proposition}{6.32}{convergence du produit de Cauchy de STP}
    Soit $\sum_n u_n$ et $\sum_n v_n$ deux séries réelles à termes positifs. Si $\sum_n u_n$ et $\sum_n v_n$ convergent, alors il en est de même pour la série de terme général $(u \star v)_n$. De plus~:
    $$\sum_{n=0}^{+\infty}\sum_{i+j=n}u_i v_j = \Bigg( \sum_{n=0}^{+\infty}u_n \Bigg)\Bigg( \sum_{n=0}^{+\infty}v_n \Bigg)$$
\end{proposition}

\begin{proposition}{6.32}{convergence du produit de Cauchy de séries d'élements d'une algèbre normée}
    Soit $\mc{A}$ une $\K$-algèbre normée unitaire de dimension finie. Soit $\sum_n u_n$ et $\sum_n v_n$ deux séries à termes dans $\mc{A}$. Si $\sum_n u_n$ et $\sum_n v_n$ convergent absolument, alors il en est de même pour la série de terme général $(u \star v)_n$. De plus~:
    $$\sum_{n=0}^{+\infty}\sum_{i+j=n}u_i v_j = \Bigg( \sum_{n=0}^{+\infty}u_n \Bigg)\Bigg( \sum_{n=0}^{+\infty}v_n \Bigg)$$
\end{proposition}

\begin{theoreme}{6.34}{propriété fondamentale de l'exponentielle}
   Soit $\mc{A}$ une $\K$-algèbre normée unitaire de dimension finie. Si $u$ et $v$ sont deux éléments de $\mc{A}$ qui commutent, alors :
    $$\exp(u+v) = \exp(u)\star \exp(v)$$
\end{theoreme}

\end{adjustwidth}
\end{document}