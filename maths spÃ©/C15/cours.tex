\documentclass{article}
\usepackage{amsmath,amssymb,mathtools}
\usepackage{xcolor}
\usepackage{minted}
\usepackage{enumitem}
\usepackage{multicol}
\usepackage{changepage}
\usepackage{stmaryrd}
\usepackage{graphicx}
\graphicspath{ {./images/} }
\usepackage[framemethod=tikz]{mdframed}
\usepackage{tikz,pgfplots}
\pgfplotsset{compat=1.18}

% physique
\definecolor{oranges}{RGB}{255, 242, 230}
\definecolor{rouges}{RGB}{255, 230, 230}
\definecolor{rose}{RGB}{255, 204, 204}

% maths - info
\definecolor{rouge_fonce}{RGB}{204, 0, 0}
\definecolor{rouge}{RGB}{255, 0, 0}
\definecolor{bleu_fonce}{RGB}{0, 0, 255}
\definecolor{vert_fonce}{RGB}{0, 69, 33}
\definecolor{vert}{RGB}{0,255,0}

\definecolor{orange_foncee}{RGB}{255, 153, 0}
\definecolor{myrtille}{RGB}{225, 225, 255}
\definecolor{mayonnaise}{RGB}{255, 253, 233}
\definecolor{magenta}{RGB}{224, 209, 240}
\definecolor{pomme}{RGB}{204, 255, 204}
\definecolor{mauve}{RGB}{255, 230, 255}


% Cours

\newmdenv[
  nobreak=true,
  topline=true,
  bottomline=true,
  rightline=true,
  leftline=true,
  linewidth=0.5pt,
  linecolor=black,
  backgroundcolor=mayonnaise,
  innerleftmargin=10pt,
  innerrightmargin=10pt,
  innertopmargin=5pt,
  innerbottommargin=5pt,
  skipabove=\topsep,
  skipbelow=\topsep,
]{boite_definition}

\newmdenv[
  nobreak=false,
  topline=true,
  bottomline=true,
  rightline=true,
  leftline=true,
  linewidth=0.5pt,
  linecolor=white,
  backgroundcolor=white,
  innerleftmargin=10pt,
  innerrightmargin=10pt,
  innertopmargin=5pt,
  innerbottommargin=5pt,
  skipabove=\topsep,
  skipbelow=\topsep,
]{boite_exemple}

\newmdenv[
  nobreak=true,
  topline=true,
  bottomline=true,
  rightline=true,
  leftline=true,
  linewidth=0.5pt,
  linecolor=black,
  backgroundcolor=magenta,
  innerleftmargin=10pt,
  innerrightmargin=10pt,
  innertopmargin=5pt,
  innerbottommargin=5pt,
  skipabove=\topsep,
  skipbelow=\topsep,
]{boite_proposition}

\newmdenv[
  nobreak=true,
  topline=true,
  bottomline=true,
  rightline=true,
  leftline=true,
  linewidth=0.5pt,
  linecolor=black,
  backgroundcolor=white,
  innerleftmargin=10pt,
  innerrightmargin=10pt,
  innertopmargin=5pt,
  innerbottommargin=5pt,
  skipabove=\topsep,
  skipbelow=\topsep,
]{boite_demonstration}

\newmdenv[
  nobreak=true,
  topline=true,
  bottomline=true,
  rightline=true,
  leftline=true,
  linewidth=0.5pt,
  linecolor=white,
  backgroundcolor=white,
  innerleftmargin=10pt,
  innerrightmargin=10pt,
  innertopmargin=5pt,
  innerbottommargin=5pt,
  skipabove=\topsep,
  skipbelow=\topsep,
]{boite_remarque}


\newenvironment{definition}[2]
{
    \vspace{15pt}
    \begin{boite_definition}
    \textbf{\textcolor{rouge}{Définition #1}}
    \if\relax\detokenize{#2}\relax
    \else
        \textit{ - #2}
    \fi \\ \\
}
{
    \end{boite_definition}
    
}

\newenvironment{exemple}[2]
{
    \vspace{15pt}
    \begin{boite_exemple}
    \textbf{\textcolor{bleu_fonce}{Exemple #1}}
    \if\relax\detokenize{#2}\relax
    \else
        \textit{ - #2}
    \fi \\ \\ 
}
{   
    \end{boite_exemple}
    
}

\newenvironment{proposition}[2]
{
    \vspace{15pt}
    \begin{boite_proposition}
    \textbf{\textcolor{rouge}{Proposition #1}}
    \if\relax\detokenize{#2}\relax
    \else
        \textit{ - #2}
    \fi \\ \\
}
{
    \end{boite_proposition}
    
}

\newenvironment{theoreme}[2]
{
    \vspace{15pt}
    \begin{boite_proposition}
    \textbf{\textcolor{rouge}{Théorème #1}} 
    \if\relax\detokenize{#2}\relax
    \else
        \textit{ - #2}
    \fi \\ \\
}
{
    \end{boite_proposition}
    
}

\newenvironment{demonstration}
{
    \vspace{15pt}
    \begin{boite_demonstration}
    \textbf{\textcolor{rouge}{Démonstration}}\\ \\
}
{
    \end{boite_demonstration}
    
}

\newenvironment{remarque}[2]
{
    \vspace{15pt}
    \begin{boite_remarque}
    \textbf{\textcolor{bleu_fonce}{Remarque #1}}
    \if\relax\detokenize{#2}\relax
    \else
        \textit{ - #2}
    \fi \\ \\   
}
{  
    \end{boite_remarque}
    
}



%Corrections
\newmdenv[
  nobreak=true,
  topline=true,
  bottomline=true,
  rightline=true,
  leftline=true,
  linewidth=0.5pt,
  linecolor=black,
  backgroundcolor=mayonnaise,
  innerleftmargin=10pt,
  innerrightmargin=10pt,
  innertopmargin=5pt,
  innerbottommargin=5pt,
  skipabove=\topsep,
  skipbelow=\topsep,
]{boite_question}


\newenvironment{question}[2]
{
    \vspace{15pt}
    \begin{boite_question}
    \textbf{\textcolor{rouge}{Question #1}}
    \if\relax\detokenize{#2}\relax
    \else
        \textit{ - #2}
    \fi \\ \\
}
{
    \end{boite_question}
    
}

\newenvironment{enumeratebf}{
    \begin{enumerate}[label=\textbf{\arabic*.}]
}
{
    \end{enumerate}
}

\begin{document}
\begin{adjustwidth}{-3cm}{-3cm}
\begin{document}
\begin{adjustwidth}{-3cm}{-3cm}
% commandes
\newcommand{\notion}[1]{\textcolor{vert_fonce}{\textit{#1}}}
\newcommand{\mb}[1]{\mathbb{#1}}
\newcommand{\mc}[1]{\mathcal{#1}}
\newcommand{\mr}[1]{\mathrm{#1}}
\newcommand{\code}[1]{\texttt{#1}}
\newcommand{\ccode}[1]{\texttt{|#1|}}
\newcommand{\ov}[1]{\overline{#1}}
\newcommand{\abs}[1]{|#1|}
\newcommand{\rev}[1]{\texttt{reverse(#1)}}
\newcommand{\crev}[1]{\texttt{|reverse(#1)|}}

\newcommand{\ie}{\textit{i.e.} }

\newcommand{\N}{\mathbb{N}}
\newcommand{\R}{\mathbb{R}}
\newcommand{\C}{\mathbb{C}}
\newcommand{\K}{\mathbb{K}}
\newcommand{\Z}{\mathbb{Z}}

\newcommand{\A}{\mathcal{A}}
\newcommand{\bigO}{\mathcal{O}}
\renewcommand{\L}{\mathcal{L}}

\newcommand{\rg}[0]{\mathrm{rg}}
\newcommand{\re}[0]{\mathrm{Re}}
\newcommand{\im}[0]{\mathrm{Im}}
\newcommand{\cl}[0]{\mathrm{cl}}
\newcommand{\grad}[0]{\vec{\mathrm{grad}}}
\renewcommand{\div}[0]{\mathrm{div}\,}
\newcommand{\rot}[0]{\vec{\mathrm{rot}}\,}
\newcommand{\vnabla}[0]{\vec{\nabla}}
\renewcommand{\vec}[1]{\overrightarrow{#1}}
\newcommand{\mat}[1]{\mathrm{Mat}_{#1}}
\newcommand{\matrice}[1]{\mathcal{M}_{#1}}
\newcommand{\sgEngendre}[1]{\left\langle #1 \right\rangle}
\newcommand{\gpquotient}[1]{\mathbb{Z} / #1\mathbb{Z}}
\newcommand{\norme}[1]{||#1||}
\renewcommand{\d}[1]{\,\mathrm{d}#1}
\newcommand{\adh}[1]{\overline{#1}}
\newcommand{\intint}[2]{\llbracket #1 ,\, #2 \rrbracket}
\newcommand{\seg}[2]{[#1\, ; \, #2]}
\newcommand{\scal}[2]{( #1 | #2 )}
\newcommand{\distance}[2]{\mathrm{d}(#1,\,#2)}
\newcommand{\inte}[2]{\int_{#1}^{#2}}
\newcommand{\somme}[2]{\sum_{#1}^{#2}}
\newcommand{\deriveref}[4]{\biggl( \frac{\text{d}^{#1}#2}{\text{d}#3^{#1}} \biggr)_{#4}}





\newcounter{chapitre}
\setcounter{chapitre}{15}

\newcommand{\tribu}[0]{\mathcal{T}}
\newcommand{\univ}[0]{\Omega}



\begin{definition}{15.1}{tribu}
    Pour un univers $\univ$ au plus dénombrable, on appelle \notion{tribu sur $\univ$} une partie $\tribu \subset \mc{P}(\univ)$ tel que~:
    \begin{enumeratebf}
        \item $\univ \in \tribu$
        \item Pour tout $A \in \tribu, \overline{A} \in \tribu$
        \item Pour toute suite $(A_n)_{n \in \N}$ d'éléments de $\tribu$, $\displaystyle \bigcup_{n \in \N}A_n \in \tribu$
    \end{enumeratebf}
    Les éléments de $\tribu$ sont appelés \notion{évènements}.
\end{definition}

\begin{definition}{15.4}{espace probabilisable}
    Soit $\univ$ un univers au plus dénombrable et $\tribu$ une tribu sur $\univ$. Le couple $(\Omega, \tribu)$ est appelé \notion{espace probabilisable}.
\end{definition}

\begin{definition}{15.5}{système complet d'évènements}
    Soit $(\univ, \tribu)$ un espace probabilisable associé à un univers $\univ$ au plus dénombrable. On dit qu'une famille au plus dénombrable $(A_i)_{i \in I} \in \tribu^I$ d'évènements constitue un \notion{système complet d'évènements} si~:
    $$\univ = \bigsqcup_{i \in I}A_i$$
\end{definition}


\newcommand{\proba}[0]{\mathbb{P}}

\begin{definition}{15.7}{probabilité sur un univers}
    Soit $(\univ, \tribu)$ un espace probabilisable associé à un univers $\univ$ au plus dénombrable. On appelle \notion{probabilité sur $(\univ, \tribu)$} une application $\proba : \tribu \to \seg{0}{1}$ telle que~:
    \begin{enumeratebf}
        \item $\proba(\univ) = 1$
        \item \notion{$\sigma$-additivité}~: Pour toute suite $(A_n)_{n \in \N}$ d'évènements deux à deux incompatibles, la série de terme général $\proba(A_n)$ converge et~:
        $$\proba\Bigg(\bigsqcup_{n \in \N}A_n\Bigg) = \sum_{n=0}^{+ \infty} \proba(A_n)$$
    \end{enumeratebf}
    On dit alors que $(\univ, \tribu, \proba)$ constitue un \notion{espace probabilisé}.
\end{definition}

\begin{theoreme}{15.17}{de la limite monotone}
    Soit $(\univ, \tribu, \proba)$ un espace probabilisé. Soit $(A_n)_{n\in \N} \in \tribu^\N$ une suite croissante d'évènements ($\forall n \in \N,\, A_n \subset A_{n+1}$). Alors~:
    $$\proba\Bigg(\bigcup_{n=0}^{+\infty}A_n\Bigg) = \lim_{n \to +\infty} \proba(A_n)$$
\end{theoreme}

\begin{proposition}{15.20}{inégalité de Boole}
    Soit $(\univ, \tribu, \proba)$ un espace probabilisé. Soit $(A_n)_{n\in \N} \in \tribu^\N$ une suite d'événements telle que la série de terme général $\proba(A_n)$ converge. Alors~:
    $$\proba\Bigg(\bigcup_{n=0}^{+\infty}A_n\Bigg) \leq \sum_{n=0}^{+\infty} \proba(A_n)$$
\end{proposition}

\begin{definition}{15.21}{événements négligeable, presque sûr}
    Soit $(\univ, \tribu, \proba)$ un espace probabilisé.
    \begin{itemize}
        \item Un événement $A$ est dit \notion{négligeable} si $\proba(A) = 0$.
        \item Un événement $A$ est dit \notion{presque sûr} si $\proba(A) = 1$.
    \end{itemize}
\end{definition}

\begin{proposition}{15.22 (3)}{intersection, union avec un événement négligeable}
    Soit $(\univ, \tribu, \proba)$ un espace probabilisé. Soit $A \in \tribu$ un événement négligeable. Alors~:
    $$\forall B \in \tribu,\, \begin{cases*}
        \proba(A\cap B) = 0 \\ \proba(A\cup B) = \proba(B)
    \end{cases*} $$
\end{proposition}

\begin{definition}{15.24}{probabilité conditionnelle}
    Soit $(\univ, \tribu, \proba)$ un espace probabilisé. Soit $A \in \tribu$ un événement non négligeable. Alors l'application~:
    \fonction{\proba_A}{\tribu}{\seg{0}{1}}{B}{\frac{\proba(A \cap B)}{\proba(A)}}
    est une probabilité, appelée \notion{probabilité conditionnelle sachant $A$}.
\end{definition}

\begin{theoreme}{15.26}{formule des probabilités composées}
    Soit $(\univ, \tribu, \proba)$ un espace probabilisé. Soit $(A_k)_{k \in \intint{1}{n}} \in \tribu^n$ une famille d'évènements telle que $\displaystyle \bigcap_{k=1}^{n}A_k$ ne soit pas négligeable. Alors~:
    $$\proba\Bigg(\bigcap_{k=1}^nA_k\Bigg) = \prod_{k=1}^n\proba\bigg( A_k\, \bigg\vert\,\bigcap_{i=1}^{k-1} A_j \bigg)$$
\end{theoreme}

\begin{theoreme}{15.28}{formule de probabilités totales}
    Soit $(\univ, \tribu, \proba)$ un espace probabilisé. Soit $(A_i)_{i \in I}$ un système complet d'évènement de probabilités non nulles, où $I$ est au plus dénombrable. Pour tout $B \in \tribu$~:
    $$\proba(B) = \sum_{i \in I} \proba(B \cap A_i)$$
\end{theoreme}

\begin{theoreme}{15.32}{formule de Bayes}
    Soit $(\univ, \tribu, \proba)$ un espace probabilisé. Soit $(A_i)_{i \in I}$ un système complet d'évènement de probabilités non nulles, où $I$ est au plus dénombrable. Pour tout $B \in \tribu$ de probabilité non nulle~: 
    $$\forall j \in I,\, \proba_B(A_j) = \frac{\proba_{A_j}(B)}{\displaystyle \sum_{i \in I} \proba_{A_i}(B)\proba(A_i)}$$
\end{theoreme}

\begin{definition}{15.37}{évènements mutuellement indépendants}
    Soit $(\univ, \tribu, \proba)$ un espace probabilisé. Soit $(A_i)_{i \in I}$ un système complet d'évènement, où $I$ est au plus dénombrable. $(A_i)_{i \in I}$ est une \notion{famille d'évènements mutuellement indépendants} si~:
    $$\forall J \in \mc{P}_f(I),\, \proba\Bigg(\bigcap_{j \in J} A_j\Bigg) = \prod_{j \in J}\proba(A_j)$$
\end{definition}

\begin{definition}{15.38}{prédicat vrai sur un ensemble au plus dénombrable}
    Soit $P$ un prédicat sur $I$ un ensemble au plus dénombrable. On dit que \notion{$P$ est vrai sur $I$} si~:
    $$\forall J \in \mc{P}_f(I),\, \forall x \in J,\, P(x)$$
\end{definition}

\end{adjustwidth}
\end{document}