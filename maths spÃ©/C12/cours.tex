\documentclass{article}
\usepackage{amsmath,amssymb,mathtools}
\usepackage{xcolor}
\usepackage{minted}
\usepackage{enumitem}
\usepackage{multicol}
\usepackage{changepage}
\usepackage{stmaryrd}
\usepackage{graphicx}
\graphicspath{ {./images/} }
\usepackage[framemethod=tikz]{mdframed}
\usepackage{tikz,pgfplots}
\pgfplotsset{compat=1.18}

% physique
\definecolor{oranges}{RGB}{255, 242, 230}
\definecolor{rouges}{RGB}{255, 230, 230}
\definecolor{rose}{RGB}{255, 204, 204}

% maths - info
\definecolor{rouge_fonce}{RGB}{204, 0, 0}
\definecolor{rouge}{RGB}{255, 0, 0}
\definecolor{bleu_fonce}{RGB}{0, 0, 255}
\definecolor{vert_fonce}{RGB}{0, 69, 33}
\definecolor{vert}{RGB}{0,255,0}

\definecolor{orange_foncee}{RGB}{255, 153, 0}
\definecolor{myrtille}{RGB}{225, 225, 255}
\definecolor{mayonnaise}{RGB}{255, 253, 233}
\definecolor{magenta}{RGB}{224, 209, 240}
\definecolor{pomme}{RGB}{204, 255, 204}
\definecolor{mauve}{RGB}{255, 230, 255}


% Cours

\newmdenv[
  nobreak=true,
  topline=true,
  bottomline=true,
  rightline=true,
  leftline=true,
  linewidth=0.5pt,
  linecolor=black,
  backgroundcolor=mayonnaise,
  innerleftmargin=10pt,
  innerrightmargin=10pt,
  innertopmargin=5pt,
  innerbottommargin=5pt,
  skipabove=\topsep,
  skipbelow=\topsep,
]{boite_definition}

\newmdenv[
  nobreak=false,
  topline=true,
  bottomline=true,
  rightline=true,
  leftline=true,
  linewidth=0.5pt,
  linecolor=white,
  backgroundcolor=white,
  innerleftmargin=10pt,
  innerrightmargin=10pt,
  innertopmargin=5pt,
  innerbottommargin=5pt,
  skipabove=\topsep,
  skipbelow=\topsep,
]{boite_exemple}

\newmdenv[
  nobreak=true,
  topline=true,
  bottomline=true,
  rightline=true,
  leftline=true,
  linewidth=0.5pt,
  linecolor=black,
  backgroundcolor=magenta,
  innerleftmargin=10pt,
  innerrightmargin=10pt,
  innertopmargin=5pt,
  innerbottommargin=5pt,
  skipabove=\topsep,
  skipbelow=\topsep,
]{boite_proposition}

\newmdenv[
  nobreak=true,
  topline=true,
  bottomline=true,
  rightline=true,
  leftline=true,
  linewidth=0.5pt,
  linecolor=black,
  backgroundcolor=white,
  innerleftmargin=10pt,
  innerrightmargin=10pt,
  innertopmargin=5pt,
  innerbottommargin=5pt,
  skipabove=\topsep,
  skipbelow=\topsep,
]{boite_demonstration}

\newmdenv[
  nobreak=true,
  topline=true,
  bottomline=true,
  rightline=true,
  leftline=true,
  linewidth=0.5pt,
  linecolor=white,
  backgroundcolor=white,
  innerleftmargin=10pt,
  innerrightmargin=10pt,
  innertopmargin=5pt,
  innerbottommargin=5pt,
  skipabove=\topsep,
  skipbelow=\topsep,
]{boite_remarque}


\newenvironment{definition}[2]
{
    \vspace{15pt}
    \begin{boite_definition}
    \textbf{\textcolor{rouge}{Définition #1}}
    \if\relax\detokenize{#2}\relax
    \else
        \textit{ - #2}
    \fi \\ \\
}
{
    \end{boite_definition}
    
}

\newenvironment{exemple}[2]
{
    \vspace{15pt}
    \begin{boite_exemple}
    \textbf{\textcolor{bleu_fonce}{Exemple #1}}
    \if\relax\detokenize{#2}\relax
    \else
        \textit{ - #2}
    \fi \\ \\ 
}
{   
    \end{boite_exemple}
    
}

\newenvironment{proposition}[2]
{
    \vspace{15pt}
    \begin{boite_proposition}
    \textbf{\textcolor{rouge}{Proposition #1}}
    \if\relax\detokenize{#2}\relax
    \else
        \textit{ - #2}
    \fi \\ \\
}
{
    \end{boite_proposition}
    
}

\newenvironment{theoreme}[2]
{
    \vspace{15pt}
    \begin{boite_proposition}
    \textbf{\textcolor{rouge}{Théorème #1}} 
    \if\relax\detokenize{#2}\relax
    \else
        \textit{ - #2}
    \fi \\ \\
}
{
    \end{boite_proposition}
    
}

\newenvironment{demonstration}
{
    \vspace{15pt}
    \begin{boite_demonstration}
    \textbf{\textcolor{rouge}{Démonstration}}\\ \\
}
{
    \end{boite_demonstration}
    
}

\newenvironment{remarque}[2]
{
    \vspace{15pt}
    \begin{boite_remarque}
    \textbf{\textcolor{bleu_fonce}{Remarque #1}}
    \if\relax\detokenize{#2}\relax
    \else
        \textit{ - #2}
    \fi \\ \\   
}
{  
    \end{boite_remarque}
    
}



%Corrections
\newmdenv[
  nobreak=true,
  topline=true,
  bottomline=true,
  rightline=true,
  leftline=true,
  linewidth=0.5pt,
  linecolor=black,
  backgroundcolor=mayonnaise,
  innerleftmargin=10pt,
  innerrightmargin=10pt,
  innertopmargin=5pt,
  innerbottommargin=5pt,
  skipabove=\topsep,
  skipbelow=\topsep,
]{boite_question}


\newenvironment{question}[2]
{
    \vspace{15pt}
    \begin{boite_question}
    \textbf{\textcolor{rouge}{Question #1}}
    \if\relax\detokenize{#2}\relax
    \else
        \textit{ - #2}
    \fi \\ \\
}
{
    \end{boite_question}
    
}

\newenvironment{enumeratebf}{
    \begin{enumerate}[label=\textbf{\arabic*.}]
}
{
    \end{enumerate}
}

\begin{document}
\begin{adjustwidth}{-3cm}{-3cm}
\begin{document}
\begin{adjustwidth}{-3cm}{-3cm}
% commandes
\newcommand{\notion}[1]{\textcolor{vert_fonce}{\textit{#1}}}
\newcommand{\mb}[1]{\mathbb{#1}}
\newcommand{\mc}[1]{\mathcal{#1}}
\newcommand{\mr}[1]{\mathrm{#1}}
\newcommand{\code}[1]{\texttt{#1}}
\newcommand{\ccode}[1]{\texttt{|#1|}}
\newcommand{\ov}[1]{\overline{#1}}
\newcommand{\abs}[1]{|#1|}
\newcommand{\rev}[1]{\texttt{reverse(#1)}}
\newcommand{\crev}[1]{\texttt{|reverse(#1)|}}

\newcommand{\ie}{\textit{i.e.} }

\newcommand{\N}{\mathbb{N}}
\newcommand{\R}{\mathbb{R}}
\newcommand{\C}{\mathbb{C}}
\newcommand{\K}{\mathbb{K}}
\newcommand{\Z}{\mathbb{Z}}

\newcommand{\A}{\mathcal{A}}
\newcommand{\bigO}{\mathcal{O}}
\renewcommand{\L}{\mathcal{L}}

\newcommand{\rg}[0]{\mathrm{rg}}
\newcommand{\re}[0]{\mathrm{Re}}
\newcommand{\im}[0]{\mathrm{Im}}
\newcommand{\cl}[0]{\mathrm{cl}}
\newcommand{\grad}[0]{\vec{\mathrm{grad}}}
\renewcommand{\div}[0]{\mathrm{div}\,}
\newcommand{\rot}[0]{\vec{\mathrm{rot}}\,}
\newcommand{\vnabla}[0]{\vec{\nabla}}
\renewcommand{\vec}[1]{\overrightarrow{#1}}
\newcommand{\mat}[1]{\mathrm{Mat}_{#1}}
\newcommand{\matrice}[1]{\mathcal{M}_{#1}}
\newcommand{\sgEngendre}[1]{\left\langle #1 \right\rangle}
\newcommand{\gpquotient}[1]{\mathbb{Z} / #1\mathbb{Z}}
\newcommand{\norme}[1]{||#1||}
\renewcommand{\d}[1]{\,\mathrm{d}#1}
\newcommand{\adh}[1]{\overline{#1}}
\newcommand{\intint}[2]{\llbracket #1 ,\, #2 \rrbracket}
\newcommand{\seg}[2]{[#1\, ; \, #2]}
\newcommand{\scal}[2]{( #1 | #2 )}
\newcommand{\distance}[2]{\mathrm{d}(#1,\,#2)}
\newcommand{\inte}[2]{\int_{#1}^{#2}}
\newcommand{\somme}[2]{\sum_{#1}^{#2}}
\newcommand{\deriveref}[4]{\biggl( \frac{\text{d}^{#1}#2}{\text{d}#3^{#1}} \biggr)_{#4}}






\newcommand{\intsemi}[0]{[a;\,b[}
\newcommand{\cm}{\mc{CM}\Big([a;\,b[,\K\Big)}
\newcommand{\cmplus}{\mc{CM}\Big([a;\,b[,\R_+\Big)}

\begin{definition}{12.1}{intégrale impropre}
    Soit $f \in \cm$. On dit que \notion{l'intégrale impropre $\displaystyle \int_a^b f(t)\d{t}$ converge} lorsque la fonction $\displaystyle x \mapsto \int_a^x f(t) \d{t}$ possède une limite finie en $b$.\\
    Autrement, on dit que \notion{l'intégrale impropre $\displaystyle \int_a^b f(t)\d{t}$ diverge}.
\end{definition}

\begin{definition}{12.5}{reste d'une intégrale impropre convergente}
    Soit $f \in \cm$ telle que l'intégrale $\displaystyle \int_{a}^{b}f(t)\d{t}$ converge. L'application~:
    \fonction{R}{\intsemi}{\K}{x}{\int_{x}^{b}f(t)\d{t}}
    est appelée \notion{reste de l'intégrale impropre convergente}.
\end{definition}

\begin{theoreme}{12.6}{de la limite monotone}
    Soit $f:I\to \R$ une fonction croissante. Quand $x$ tend vers la borne supérieure de $I$ (au sens large), $f(x)$ tend vers la borne supérieure (au sens large) de $f$ sur $I$~:
    $$\lim_{x \to \sup I} f(x) = \sup\{f(x),\, x \in I\}$$
\end{theoreme}

\begin{proposition}{12.8}{caractérisation de la convergence d'une intégrale impropre}
    Soit $f \in \cmplus$. L'intégrale $ \displaystyle \int_a^bf(t)\d{t}$ converge si et seulement si la fonction $\displaystyle x \mapsto \int_a^x f(t)\d{t}$ est majorée sur $\intsemi$.
\end{proposition}

\begin{definition}{12.14}{fonction intégrable}
    Soit $f:\intsemi \to \C$ un fonction continue par morceaux. On dit que \notion{$f$ est intégrable} lorsque l'intégrale impropre $\int_a^b \abs{f}(t)\d{t}$ est convergente. Le cas échéant on dit aussi que l'intégrale impropre $\int_a^b f(t) \d{t}$ est absolument convergente.
\end{definition}

\begin{theoreme}{12.16}{convergence absolue implique convergence}
    Soit $f:\intsemi \to \C$ un fonction continue par morceaux. Si $f$ est intégrable sur $\intsemi$, alors l'intégrale impropre $\int_a^b f(t)\d{t}$ est convergente.
\end{theoreme}

\begin{theoreme}{12.19}{fonction de Riemann}
    Soit $\alpha \in \R$. La fonction de Riemann $\displaystyle x \mapsto \frac{1}{x^\alpha}$ définie sur $]0 ; +\infty[$ est~:
    \begin{enumeratebf}
        \item intégrable en $0$ si et seulement si $\alpha < 1$
        \item intégrable en $+ \infty$ si et seulement si $\alpha > 1$
    \end{enumeratebf}
\end{theoreme}

\begin{theoreme}{12.20}{fonction de référence}
    Soit $\alpha \in \R$. La fonction $x \mapsto e^{-\alpha x}$ est intégrable sur $[0;+\infty[$ si et seulement si $\alpha > 0$. Le cas échéant~:
    $$\int_0^{+\infty}e^{-\alpha t} \d{t} = \frac{1}{\alpha}$$
\end{theoreme}

\begin{theoreme}{12.55}{de convergence dominée pour une suite de fonctions définies sur un intervalle}
    Soit $(f_n)_{n \in \N} \in \mc{F}(I,\K)^\N$. Sous réserve des hypothèses suivantes,
    \begin{enumeratebf}
        \item $\forall n \in \N,\, f_n \in \mc{CM}(I,\K)$.
        \item $(f_n)_{n \in \N}$ converge simplement sur $I$ et sa limite y est continue par morceaux.
        \item \notion{hypothèse de domination} : il existe $\varphi \in \mc{CM}(I, \R_+)$ une fonction intégrable telle que pour tout $n \in \N$, $\abs{f_n} \leq \varphi$
    \end{enumeratebf}
    les fonctions $f_n$ sont intégrables, ainsi que leur limite simple et on peut échanger les symboles "$\lim$" et "$\int$"~:
    $$\lim_{n \to +\infty} \int_{I} f_n = \int_{I} \lim_{n \to +\infty} f_n $$
\end{theoreme}

\begin{theoreme}{12.59}{de convergence dominée pour une série de fonctions définies sur un intervalle}
    Soit $(f_n)_{n \in \N} \in \mc{F}(I,\K)^\N$. Sous réserve des hypothèses suivantes,
    \begin{enumeratebf}
        \item $\forall n \in \N,\, f_n \in \mc{CM}(I,\K)$ et $f_n$ est intégrable.
        \item $\sum_n f_n$ converge simplement sur $I$ et sa somme y est continue par morceaux.
        \item la série $\sum_n \int_I \abs{f_n(t)}$ converge.
    \end{enumeratebf}
    la fonction $\displaystyle\sum_{n=0}^{+\infty}f_n$ est intégrable sur $I$ et on peut échanger les symboles "$\sum$" et "$\int$"~:
    $$\sum_{n=0}^{+\infty} \int_I f_n = \int_I \sum_{n=0}^{+\infty} f_n$$
\end{theoreme}

\begin{theoreme}{12.61}{théorème de convergence dominée à paramètre continu}
    Soit $A \subset \R^m$, $I$ un intervalle de $\R$ et $f \in \mc{F}(A\times I,\, \K)$. Sous réserve des hypothèses suivantes~:
    \begin{enumeratebf}
        \item $\forall x \in A,\,  f(x,\cdot) \in \mc{CM}(I,\K)$ où $f(x,\cdot): t \mapsto f(x,t)$.
        \item Pour un certain $x_0 \in \overline{A}$ la fonction $\displaystyle \lim_{x \to x_0}f(x,\cdot)$ est continue par morceaux sur $I$.
        \item \notion{hypothèse de domination} : il existe $\varphi \in \mc{CM}(I, \R_+)$ intégrable, majorant $f(x,\cdot)$ pour tout $x \in A$.
    \end{enumeratebf}
    Pour tout $x \in A$ la fonction $f(x, \cdot)$ est intégrable sur $I$, de même pour $\displaystyle \lim_{x \to x_0}f(x,\cdot)$, et on peut échanger les symboles "$\lim$" et "$\int$"~:
    $$\int_I \lim_{x \to x_0} f(x,\cdot) = \lim_{x \to x_0} \int_I f(x,\cdot)$$
\end{theoreme}

\begin{theoreme}{12.63}{théorème de continuité sous le signe $\int$}
    Soit $A \subset \R^m$, $I$ un intervalle de $\R$ et $f \in \mc{F}(A\times I,\, \K)$. Sous réserve des hypothèses suivantes~:
    \begin{enumeratebf}
        \item $\forall x \in A,\,  f(x,\cdot) \in \mc{CM}(I,\K)$ où $f(x,\cdot): t \mapsto f(x,t)$
        \item $\forall t \in I,\, f(\cdot, t) \in \mc{C}^0(A, \K)$ où $f(\cdot,t):x \mapsto f(x,t)$
        \item \notion{hypothèse de domination} : il existe $\varphi \in \mc{CM}(I, \R_+)$ intégrable, majorant $f(x,\cdot)$ pour tout $x \in A$.
    \end{enumeratebf}
    Pour tout $x \in A$ la fonction $f(x, \cdot)$ est intégrable sur $I$, et la fonction $\displaystyle x \mapsto \int_I f(x,t)\d{t}$ est continue sur $A$.
\end{theoreme}

\begin{theoreme}{12.69}{de classe $\mc{C}^1$ sous le signe $\int$}
    Soit $A$ et $I$ deux intervalle de $\R$ et $f \in \mc{F}(A\times I,\, \K)$. Sous réserve des hypothèses suivantes~:
    \begin{enumeratebf}
        \item $\forall t \in I,\, f(\cdot, t) \in \mc{C}^1(A, \K)$.
        \item $\displaystyle \forall x \in A,\, f(x,\cdot) \in \mc{CM}(I,K)$ et est intégrable.
        \item $\displaystyle \forall x \in A,\, \derivepar{}{f}{x}(x,\cdot) \in \mc{CM}(I,\K)$
        \item \notion{hypothèse de domination} : il existe $\varphi \in \mc{CM}(I, \R_+)$ intégrable, majorant $\displaystyle\Bigg\vert\derivepar{}{f}{x}(x,\cdot)\Bigg\vert$ pour tout $x \in A$.
    \end{enumeratebf}
    pour tout $x \in A$, la fonction $\displaystyle\derivepar{}{f}{x}(x,\cdot)$ est intégrable sur $I$ et la fonction $\displaystyle F : x \mapsto \int_I f(x,t)\d{t}$ est de classe $\mc{C}^1$ sur $A$ avec~:
    $$F^{(k)} : x \mapsto \int_I \derivepar{}{f}{x}(x,t)\d{t}$$
\end{theoreme}

\begin{theoreme}{12.74}{de classe $\mc{C}^k$ sous le signe $\int$, avec hypothèses minimales}
    Soit $A$ et $I$ deux intervalle de $\R$ et $f \in \mc{F}(A\times I,\, \K)$. Sous réserve des hypothèses suivantes~:
    \begin{enumeratebf}
        \item $\forall t \in I,\, f(\cdot, t) \in \mc{C}^k(A, \K)$.
        \item $\displaystyle \forall i \in \intint{0}{k-1},\, \forall x \in A,\, \derivepar{i}{f}{x}(x,\cdot) \in \mc{CM}(I,K)$ et est intégrable.
        \item $\displaystyle \forall x \in A,\, \derivepar{k}{f}{x}(x,\cdot) \in \mc{CM}(I,\K)$
        \item \notion{hypothèse de domination} : il existe $\varphi \in \mc{CM}(I, \R_+)$ intégrable, majorant $\displaystyle\Bigg\vert\derivepar{k}{f}{x}(x,\cdot)\Bigg\vert$ pour tout $x \in A$.
    \end{enumeratebf}
    pour tout $x \in A$, la fonction $\displaystyle\derivepar{k}{f}{x}(x,\cdot)$ est intégrable sur $I$ et la fonction $\displaystyle F : x \mapsto \int_I f(x,t)\d{t}$ est de classe $\mc{C}^k$ sur $A$ avec~:
    $$F^{(k)} : x \mapsto \int_I \derivepar{k}{f}{x}(x,t)\d{t}$$
\end{theoreme}




\end{adjustwidth}
\end{document}