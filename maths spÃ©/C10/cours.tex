\documentclass{article}
\usepackage{amsmath,amssymb,mathtools}
\usepackage{xcolor}
\usepackage{minted}
\usepackage{enumitem}
\usepackage{multicol}
\usepackage{changepage}
\usepackage{stmaryrd}
\usepackage{graphicx}
\graphicspath{ {./images/} }
\usepackage[framemethod=tikz]{mdframed}
\usepackage{tikz,pgfplots}
\pgfplotsset{compat=1.18}

% physique
\definecolor{oranges}{RGB}{255, 242, 230}
\definecolor{rouges}{RGB}{255, 230, 230}
\definecolor{rose}{RGB}{255, 204, 204}

% maths - info
\definecolor{rouge_fonce}{RGB}{204, 0, 0}
\definecolor{rouge}{RGB}{255, 0, 0}
\definecolor{bleu_fonce}{RGB}{0, 0, 255}
\definecolor{vert_fonce}{RGB}{0, 69, 33}
\definecolor{vert}{RGB}{0,255,0}

\definecolor{orange_foncee}{RGB}{255, 153, 0}
\definecolor{myrtille}{RGB}{225, 225, 255}
\definecolor{mayonnaise}{RGB}{255, 253, 233}
\definecolor{magenta}{RGB}{224, 209, 240}
\definecolor{pomme}{RGB}{204, 255, 204}
\definecolor{mauve}{RGB}{255, 230, 255}


% Cours

\newmdenv[
  nobreak=true,
  topline=true,
  bottomline=true,
  rightline=true,
  leftline=true,
  linewidth=0.5pt,
  linecolor=black,
  backgroundcolor=mayonnaise,
  innerleftmargin=10pt,
  innerrightmargin=10pt,
  innertopmargin=5pt,
  innerbottommargin=5pt,
  skipabove=\topsep,
  skipbelow=\topsep,
]{boite_definition}

\newmdenv[
  nobreak=false,
  topline=true,
  bottomline=true,
  rightline=true,
  leftline=true,
  linewidth=0.5pt,
  linecolor=white,
  backgroundcolor=white,
  innerleftmargin=10pt,
  innerrightmargin=10pt,
  innertopmargin=5pt,
  innerbottommargin=5pt,
  skipabove=\topsep,
  skipbelow=\topsep,
]{boite_exemple}

\newmdenv[
  nobreak=true,
  topline=true,
  bottomline=true,
  rightline=true,
  leftline=true,
  linewidth=0.5pt,
  linecolor=black,
  backgroundcolor=magenta,
  innerleftmargin=10pt,
  innerrightmargin=10pt,
  innertopmargin=5pt,
  innerbottommargin=5pt,
  skipabove=\topsep,
  skipbelow=\topsep,
]{boite_proposition}

\newmdenv[
  nobreak=true,
  topline=true,
  bottomline=true,
  rightline=true,
  leftline=true,
  linewidth=0.5pt,
  linecolor=black,
  backgroundcolor=white,
  innerleftmargin=10pt,
  innerrightmargin=10pt,
  innertopmargin=5pt,
  innerbottommargin=5pt,
  skipabove=\topsep,
  skipbelow=\topsep,
]{boite_demonstration}

\newmdenv[
  nobreak=true,
  topline=true,
  bottomline=true,
  rightline=true,
  leftline=true,
  linewidth=0.5pt,
  linecolor=white,
  backgroundcolor=white,
  innerleftmargin=10pt,
  innerrightmargin=10pt,
  innertopmargin=5pt,
  innerbottommargin=5pt,
  skipabove=\topsep,
  skipbelow=\topsep,
]{boite_remarque}


\newenvironment{definition}[2]
{
    \vspace{15pt}
    \begin{boite_definition}
    \textbf{\textcolor{rouge}{Définition #1}}
    \if\relax\detokenize{#2}\relax
    \else
        \textit{ - #2}
    \fi \\ \\
}
{
    \end{boite_definition}
    
}

\newenvironment{exemple}[2]
{
    \vspace{15pt}
    \begin{boite_exemple}
    \textbf{\textcolor{bleu_fonce}{Exemple #1}}
    \if\relax\detokenize{#2}\relax
    \else
        \textit{ - #2}
    \fi \\ \\ 
}
{   
    \end{boite_exemple}
    
}

\newenvironment{proposition}[2]
{
    \vspace{15pt}
    \begin{boite_proposition}
    \textbf{\textcolor{rouge}{Proposition #1}}
    \if\relax\detokenize{#2}\relax
    \else
        \textit{ - #2}
    \fi \\ \\
}
{
    \end{boite_proposition}
    
}

\newenvironment{theoreme}[2]
{
    \vspace{15pt}
    \begin{boite_proposition}
    \textbf{\textcolor{rouge}{Théorème #1}} 
    \if\relax\detokenize{#2}\relax
    \else
        \textit{ - #2}
    \fi \\ \\
}
{
    \end{boite_proposition}
    
}

\newenvironment{demonstration}
{
    \vspace{15pt}
    \begin{boite_demonstration}
    \textbf{\textcolor{rouge}{Démonstration}}\\ \\
}
{
    \end{boite_demonstration}
    
}

\newenvironment{remarque}[2]
{
    \vspace{15pt}
    \begin{boite_remarque}
    \textbf{\textcolor{bleu_fonce}{Remarque #1}}
    \if\relax\detokenize{#2}\relax
    \else
        \textit{ - #2}
    \fi \\ \\   
}
{  
    \end{boite_remarque}
    
}



%Corrections
\newmdenv[
  nobreak=true,
  topline=true,
  bottomline=true,
  rightline=true,
  leftline=true,
  linewidth=0.5pt,
  linecolor=black,
  backgroundcolor=mayonnaise,
  innerleftmargin=10pt,
  innerrightmargin=10pt,
  innertopmargin=5pt,
  innerbottommargin=5pt,
  skipabove=\topsep,
  skipbelow=\topsep,
]{boite_question}


\newenvironment{question}[2]
{
    \vspace{15pt}
    \begin{boite_question}
    \textbf{\textcolor{rouge}{Question #1}}
    \if\relax\detokenize{#2}\relax
    \else
        \textit{ - #2}
    \fi \\ \\
}
{
    \end{boite_question}
    
}

\newenvironment{enumeratebf}{
    \begin{enumerate}[label=\textbf{\arabic*.}]
}
{
    \end{enumerate}
}

\begin{document}
\begin{adjustwidth}{-3cm}{-3cm}
\begin{document}
\begin{adjustwidth}{-3cm}{-3cm}
% commandes
\newcommand{\notion}[1]{\textcolor{vert_fonce}{\textit{#1}}}
\newcommand{\mb}[1]{\mathbb{#1}}
\newcommand{\mc}[1]{\mathcal{#1}}
\newcommand{\mr}[1]{\mathrm{#1}}
\newcommand{\code}[1]{\texttt{#1}}
\newcommand{\ccode}[1]{\texttt{|#1|}}
\newcommand{\ov}[1]{\overline{#1}}
\newcommand{\abs}[1]{|#1|}
\newcommand{\rev}[1]{\texttt{reverse(#1)}}
\newcommand{\crev}[1]{\texttt{|reverse(#1)|}}

\newcommand{\ie}{\textit{i.e.} }

\newcommand{\N}{\mathbb{N}}
\newcommand{\R}{\mathbb{R}}
\newcommand{\C}{\mathbb{C}}
\newcommand{\K}{\mathbb{K}}
\newcommand{\Z}{\mathbb{Z}}

\newcommand{\A}{\mathcal{A}}
\newcommand{\bigO}{\mathcal{O}}
\renewcommand{\L}{\mathcal{L}}

\newcommand{\rg}[0]{\mathrm{rg}}
\newcommand{\re}[0]{\mathrm{Re}}
\newcommand{\im}[0]{\mathrm{Im}}
\newcommand{\cl}[0]{\mathrm{cl}}
\newcommand{\grad}[0]{\vec{\mathrm{grad}}}
\renewcommand{\div}[0]{\mathrm{div}\,}
\newcommand{\rot}[0]{\vec{\mathrm{rot}}\,}
\newcommand{\vnabla}[0]{\vec{\nabla}}
\renewcommand{\vec}[1]{\overrightarrow{#1}}
\newcommand{\mat}[1]{\mathrm{Mat}_{#1}}
\newcommand{\matrice}[1]{\mathcal{M}_{#1}}
\newcommand{\sgEngendre}[1]{\left\langle #1 \right\rangle}
\newcommand{\gpquotient}[1]{\mathbb{Z} / #1\mathbb{Z}}
\newcommand{\norme}[1]{||#1||}
\renewcommand{\d}[1]{\,\mathrm{d}#1}
\newcommand{\adh}[1]{\overline{#1}}
\newcommand{\intint}[2]{\llbracket #1 ,\, #2 \rrbracket}
\newcommand{\seg}[2]{[#1\, ; \, #2]}
\newcommand{\scal}[2]{( #1 | #2 )}
\newcommand{\distance}[2]{\mathrm{d}(#1,\,#2)}
\newcommand{\inte}[2]{\int_{#1}^{#2}}
\newcommand{\somme}[2]{\sum_{#1}^{#2}}
\newcommand{\deriveref}[4]{\biggl( \frac{\text{d}^{#1}#2}{\text{d}#3^{#1}} \biggr)_{#4}}






\begin{theoreme}{10.11}{dérivée d'une composée par une application linéaire}
    Soit $E$, $F$ deux $\R$-espaces vectoriels normés de dimension finie, $u$ une application linéaire de $E$ dans $F$, et $f$ une fonction de classe $\mc{C}^1$ définie sur un intervalle $I$ de $\R$ à valeurs dans $E$. Alors $u \circ f$ est une fonction de classe $\mc{C}^1(I,F)$ et~:
    $$(u\circ f)' = u \circ f'$$
\end{theoreme}

\begin{theoreme}{10.12}{dérivée d'une composée par une application bilinéaire}
    Soit $E$, $F$ et $G$ trois espaces vectoriels normés de dimension finie, $B$ une application bilinéaire de $E\times F$ vers $G$, et $f$ et $g$ deux fonctions de classe $\mc{C}^1$  sur un intervalle $I$ de $\R$ à valeurs respectives dans $E$ et $F$. Alors $B(f,g)$ est une fonction de classe $\mc{C}^1(I,G)$, et~:
    $$\Big(B(f,g)\Big)' = B(f',g) + B(f,g')$$
\end{theoreme}


\begin{theoreme}{10.29}{construction de l'intégrale d'une fonction continue par morceaux}
    Soit $E$ un espace vectoriel normé de dimension finie, $f \in \mc{C}\mc{M}(\seg{a}{b}, E)$. Si $(\varphi_n)_{n \in \N}$ est une suite de fonctions en escalier convergeant uniformément vers $f$, alors la suite $\displaystyle \Big( \int_{\seg{a}{b}} \varphi_n \Big)_{n \in \N}$ est convergente.
\end{theoreme}

\begin{definition}{10.30}{intégrale d'une fonction continue par morceaux}
    Soit $E$ un espace vectoriel normé de dimension finie, $f \in \mc{C}\mc{M}(\seg{a}{b}, E)$. Il existe par densité de $\mc{E}(\seg{a}{b}, E)$ dans $\mc{C}\mc{M}(\seg{a}{b}, E)$ une suite $(\varphi_n)_{n \in \N}$ de fonctions de $\mc{E}(\seg{a}{b})$ convergeant uniformément vers $f$. On appelle \notion{intégrale de $f$ sur $\seg{a}{b}$} le vecteur~:
    $$\int_{\seg{a}{b}} f = \lim_{n \to + \infty} \int_{\seg{a}{b}}\varphi_n$$
    Cette intégrale ne dépend pas de la suite de $\mc{E}(\seg{a}{b})^\N$ choisie.
\end{definition}

\begin{theoreme}{10.35}{fondamental du calcul intégral}
    Soit $f$ une fonction continue sur un intervalle $I$, à valeurs dans un espace vectoriel $E$ de dimension finie. Pour tout $a\in I$, l'application~:
    \fonction{F}{I}{\R}{x}{\inte{a}{x}f(t)\d{t}}
    est l'unique primitive de $f$ (sa dérivée est $f$) s'annulant en $a$. $F$ est donc de classe $\mc{C}^1$.
\end{theoreme}

\begin{theoreme}{10.39}{changement de variable}
    Soit $f:\seg{a}{b}\to E$ continue et $\varphi$ un $\mc{C}^1$-difféomorphisme de $\seg{a}{b}$ sur $\seg{\varphi(a)}{\varphi(b)}$.
    $$\inte{\varphi(a)}{\varphi(b)}f(x)\d{x} = \inte{a}{b}f\Bigl(\varphi(t)\Bigr)\varphi'(t)\d{t}$$
\end{theoreme}

\begin{theoreme}{10.42}{intégration d'un $o$}
    Soit $I$ un intervalle et $E$ de dimension finie. Soit $f:I \to E$ de classe $\mc{C}^1$ et $g:I\to \R$ de classe $\mc{C}^1$ avec $g' \geq 0$. Supposons pour $a \in I$ que $f' \underset{a}{=}o(g')$. Alors~:
    $$\norme{f(x) - f(a)}_E \underset{x \to a}{=} o\Big(\big\vert g(x) - g(a) \big\vert\Big)$$
\end{theoreme}

\begin{theoreme}{10.47}{formule de Taylor avec reste intégral}
    Soit $f\in \mc{D}^{n+1}(I,E)$ telle que $f^{(n+1)}$ est continue par morceaux sur l'intervalle $I$. Pour $a \in I$~:
    $$\forall x \in I,\, f(x) = \sum_{k=0}^n \frac{f^{(k)}(a)}{k!}(x-a)^k + \inte{a}{x}\frac{f^{(n+1)}(t)}{n!}(x-t)^n \d{t}$$
\end{theoreme}


\begin{theoreme}{10.48}{inégalité de Taylor-Lagrange}
    Soit $f\in \mc{D}^{n+1}(I,E)$ telle que $f^{(n+1)}$ est continue par morceaux sur l'intervalle $I$. Soit $\seg{a}{b} \subset I$ et $M$ un majorant de $\norme{f^{(n+1)}}_E$ sur $\seg{a}{b}$. Alors~:
    $$\Bigg\Vert f(b) - \sum_{k=0}^n \frac{f^{(k)}(a)}{k!}(b-a)^k\Bigg\Vert_E \leq M \bigg\vert \frac{(b-a)^{n+1}}{(n+1)!} \bigg\vert$$
\end{theoreme}

\begin{theoreme}{10.57}{échange limite-intégrale}
    Soit $(f_n)_{n \in \N} \in \mc{C}(\seg{a}{b},E)^\N$ convergeant uniformément vers $f$ sur $\seg{a}{b}$. Alors~:
    $$\lim_{n \to +\infty} \int_{\seg{a}{b}} f_n = \int_{\seg{a}{b}} \lim_{n \to +\infty} f_n $$
\end{theoreme}

\begin{theoreme}{10.62}{de convergence dominée pour une suite de fonctions}
    Soit $(f_n)_{n \in \N} \in \mc{F}(\seg{a}{b},\K)^\N$. Sous réserve des hypothèses suivantes,
    \begin{enumeratebf}
        \item $\forall n \in \N,\, f_n \in \mc{CM}(\seg{a}{b},\K)$.
        \item $(f_n)_{n \in \N}$ converge simplement sur $\seg{a}{b}$ et sa limite y est continue par morceaux.
        \item \notion{hypothèse de domination} : il existe $\varphi \in \mc{CM}(\seg{a}{b}, \R_+)$  telle que pour tout $n \in \N$,\, $\abs{f_n} \leq \varphi$
    \end{enumeratebf}
    on peut échanger les symboles "$\lim$" et "$\int$"~:
    $$\lim_{n \to +\infty} \int_{\seg{a}{b}} f_n = \int_{\seg{a}{b}} \lim_{n \to +\infty} f_n $$
\end{theoreme}

\begin{theoreme}{10.62}{de convergence dominée pour une série de fonctions}
    Soit $(f_n)_{n \in \N} \in \mc{F}(\seg{a}{b},\K)^\N$. Sous réserve des hypothèses suivantes,
    \begin{enumeratebf}
        \item $\forall n \in \N,\, f_n \in \mc{CM}(\seg{a}{b},\K)$.
        \item $(f_n)_{n \in \N}$ converge simplement sur $\seg{a}{b}$ et sa somme y est continue par morceaux.
        \item la série $\sum_n \inte{a}{b}\abs{f_n(t)}\d{t}$ converge
    \end{enumeratebf}
    on peut échanger les symboles "$\sum$" et "$\int$"~:
    $$\sum_{n=0}^{+\infty} \int_{\seg{a}{b}} f_n = \int_{\seg{a}{b}} \sum_{n=0}^{+\infty} f_n $$
\end{theoreme}

\begin{theoreme}{10.66}{primitivation d'une limite uniforme de suite de fonctions}
    Soit $(f_n)_{n \in \N} \in \mc{F}(I,E)^\N$. Sous réserve des hypothèses suivantes,
    \begin{enumeratebf}
        \item $\forall n \in \N,\, f_n \in \mc{C}^0(I,E)$.
        \item $(f_n)_{n \in \N}$ converge uniformément vers une même fonction $f$ sur tout segment inclus dans $I$.
    \end{enumeratebf}
    Pour tout $a \in I$, sur tout segment de $I$, la suite $(F_n)_{n \in \N}$ des primitives des $f_n$ respectives s'annulant en $a$ converge uniformément vers la primitive de $f$ s'annulant en $a$.
\end{theoreme}

\begin{theoreme}{10.66}{dérivation d'une limite uniforme de suite de fonctions}
    Soit $(f_n)_{n \in \N} \in \mc{F}(I,E)^\N$. Sous réserve des hypothèses suivantes,
    \begin{enumeratebf}
        \item $\forall n \in \N,\, f_n \in \mc{C}^1(I,E)$.
        \item $(f_n)_{n \in \N}$ converge simplement vers une fonction $f \in \mc{CM}(I,E)$
        \item $(f'_n)_{n \in \N}$ converge uniformément vers une même fonction $g$ sur tout segment inclus dans $I$.
    \end{enumeratebf}
    la fonction $f$ est de classe $\mc{C}^1(I,E)$ et $f' = g$.
\end{theoreme}

\begin{theoreme}{10.73}{dérivation $k$ fois d'une limite uniforme de suite de fonctions}
    Soit $(f_n)_{n \in \N} \in \mc{F}(I,E)^\N$. Sous réserve des hypothèses suivantes,
    \begin{enumeratebf}
        \item $\forall n \in \N,\, f_n \in \mc{C}^k(I,E)$.
        \item pour tout $i \in \intint{0}{k-1}$, la suite $(f_n^{(i)})_{n \in \N}$ converge simplement vers une fonction $f_i \in \mc{CM}(I,E)$
        \item $(f_n^{(k)})_{n \in \N}$ converge uniformément vers une même fonction $g$ sur tout segment inclus dans $I$.
    \end{enumeratebf}
    la fonction $f$ est de classe $\mc{C}^k(I,E)$ et $f^{(k)} = g$.
\end{theoreme}
\end{adjustwidth}
\end{document}