\documentclass{article}
\usepackage{amsmath,amssymb,mathtools}
\usepackage{xcolor}
\usepackage{minted}
\usepackage{enumitem}
\usepackage{multicol}
\usepackage{changepage}
\usepackage{stmaryrd}
\usepackage{graphicx}
\graphicspath{ {./images/} }
\usepackage[framemethod=tikz]{mdframed}
\usepackage{tikz,pgfplots}
\pgfplotsset{compat=1.18}

% physique
\definecolor{oranges}{RGB}{255, 242, 230}
\definecolor{rouges}{RGB}{255, 230, 230}
\definecolor{rose}{RGB}{255, 204, 204}

% maths - info
\definecolor{rouge_fonce}{RGB}{204, 0, 0}
\definecolor{rouge}{RGB}{255, 0, 0}
\definecolor{bleu_fonce}{RGB}{0, 0, 255}
\definecolor{vert_fonce}{RGB}{0, 69, 33}
\definecolor{vert}{RGB}{0,255,0}

\definecolor{orange_foncee}{RGB}{255, 153, 0}
\definecolor{myrtille}{RGB}{225, 225, 255}
\definecolor{mayonnaise}{RGB}{255, 253, 233}
\definecolor{magenta}{RGB}{224, 209, 240}
\definecolor{pomme}{RGB}{204, 255, 204}
\definecolor{mauve}{RGB}{255, 230, 255}


% Cours

\newmdenv[
  nobreak=true,
  topline=true,
  bottomline=true,
  rightline=true,
  leftline=true,
  linewidth=0.5pt,
  linecolor=black,
  backgroundcolor=mayonnaise,
  innerleftmargin=10pt,
  innerrightmargin=10pt,
  innertopmargin=5pt,
  innerbottommargin=5pt,
  skipabove=\topsep,
  skipbelow=\topsep,
]{boite_definition}

\newmdenv[
  nobreak=false,
  topline=true,
  bottomline=true,
  rightline=true,
  leftline=true,
  linewidth=0.5pt,
  linecolor=white,
  backgroundcolor=white,
  innerleftmargin=10pt,
  innerrightmargin=10pt,
  innertopmargin=5pt,
  innerbottommargin=5pt,
  skipabove=\topsep,
  skipbelow=\topsep,
]{boite_exemple}

\newmdenv[
  nobreak=true,
  topline=true,
  bottomline=true,
  rightline=true,
  leftline=true,
  linewidth=0.5pt,
  linecolor=black,
  backgroundcolor=magenta,
  innerleftmargin=10pt,
  innerrightmargin=10pt,
  innertopmargin=5pt,
  innerbottommargin=5pt,
  skipabove=\topsep,
  skipbelow=\topsep,
]{boite_proposition}

\newmdenv[
  nobreak=true,
  topline=true,
  bottomline=true,
  rightline=true,
  leftline=true,
  linewidth=0.5pt,
  linecolor=black,
  backgroundcolor=white,
  innerleftmargin=10pt,
  innerrightmargin=10pt,
  innertopmargin=5pt,
  innerbottommargin=5pt,
  skipabove=\topsep,
  skipbelow=\topsep,
]{boite_demonstration}

\newmdenv[
  nobreak=true,
  topline=true,
  bottomline=true,
  rightline=true,
  leftline=true,
  linewidth=0.5pt,
  linecolor=white,
  backgroundcolor=white,
  innerleftmargin=10pt,
  innerrightmargin=10pt,
  innertopmargin=5pt,
  innerbottommargin=5pt,
  skipabove=\topsep,
  skipbelow=\topsep,
]{boite_remarque}


\newenvironment{definition}[2]
{
    \vspace{15pt}
    \begin{boite_definition}
    \textbf{\textcolor{rouge}{Définition #1}}
    \if\relax\detokenize{#2}\relax
    \else
        \textit{ - #2}
    \fi \\ \\
}
{
    \end{boite_definition}
    
}

\newenvironment{exemple}[2]
{
    \vspace{15pt}
    \begin{boite_exemple}
    \textbf{\textcolor{bleu_fonce}{Exemple #1}}
    \if\relax\detokenize{#2}\relax
    \else
        \textit{ - #2}
    \fi \\ \\ 
}
{   
    \end{boite_exemple}
    
}

\newenvironment{proposition}[2]
{
    \vspace{15pt}
    \begin{boite_proposition}
    \textbf{\textcolor{rouge}{Proposition #1}}
    \if\relax\detokenize{#2}\relax
    \else
        \textit{ - #2}
    \fi \\ \\
}
{
    \end{boite_proposition}
    
}

\newenvironment{theoreme}[2]
{
    \vspace{15pt}
    \begin{boite_proposition}
    \textbf{\textcolor{rouge}{Théorème #1}} 
    \if\relax\detokenize{#2}\relax
    \else
        \textit{ - #2}
    \fi \\ \\
}
{
    \end{boite_proposition}
    
}

\newenvironment{demonstration}
{
    \vspace{15pt}
    \begin{boite_demonstration}
    \textbf{\textcolor{rouge}{Démonstration}}\\ \\
}
{
    \end{boite_demonstration}
    
}

\newenvironment{remarque}[2]
{
    \vspace{15pt}
    \begin{boite_remarque}
    \textbf{\textcolor{bleu_fonce}{Remarque #1}}
    \if\relax\detokenize{#2}\relax
    \else
        \textit{ - #2}
    \fi \\ \\   
}
{  
    \end{boite_remarque}
    
}



%Corrections
\newmdenv[
  nobreak=true,
  topline=true,
  bottomline=true,
  rightline=true,
  leftline=true,
  linewidth=0.5pt,
  linecolor=black,
  backgroundcolor=mayonnaise,
  innerleftmargin=10pt,
  innerrightmargin=10pt,
  innertopmargin=5pt,
  innerbottommargin=5pt,
  skipabove=\topsep,
  skipbelow=\topsep,
]{boite_question}


\newenvironment{question}[2]
{
    \vspace{15pt}
    \begin{boite_question}
    \textbf{\textcolor{rouge}{Question #1}}
    \if\relax\detokenize{#2}\relax
    \else
        \textit{ - #2}
    \fi \\ \\
}
{
    \end{boite_question}
    
}

\newenvironment{enumeratebf}{
    \begin{enumerate}[label=\textbf{\arabic*.}]
}
{
    \end{enumerate}
}

\begin{document}
\begin{adjustwidth}{-3cm}{-3cm}
\begin{document}
\begin{adjustwidth}{-3cm}{-3cm}
% commandes
\newcommand{\notion}[1]{\textcolor{vert_fonce}{\textit{#1}}}
\newcommand{\mb}[1]{\mathbb{#1}}
\newcommand{\mc}[1]{\mathcal{#1}}
\newcommand{\mr}[1]{\mathrm{#1}}
\newcommand{\code}[1]{\texttt{#1}}
\newcommand{\ccode}[1]{\texttt{|#1|}}
\newcommand{\ov}[1]{\overline{#1}}
\newcommand{\abs}[1]{|#1|}
\newcommand{\rev}[1]{\texttt{reverse(#1)}}
\newcommand{\crev}[1]{\texttt{|reverse(#1)|}}

\newcommand{\ie}{\textit{i.e.} }

\newcommand{\N}{\mathbb{N}}
\newcommand{\R}{\mathbb{R}}
\newcommand{\C}{\mathbb{C}}
\newcommand{\K}{\mathbb{K}}
\newcommand{\Z}{\mathbb{Z}}

\newcommand{\A}{\mathcal{A}}
\newcommand{\bigO}{\mathcal{O}}
\renewcommand{\L}{\mathcal{L}}

\newcommand{\rg}[0]{\mathrm{rg}}
\newcommand{\re}[0]{\mathrm{Re}}
\newcommand{\im}[0]{\mathrm{Im}}
\newcommand{\cl}[0]{\mathrm{cl}}
\newcommand{\grad}[0]{\vec{\mathrm{grad}}}
\renewcommand{\div}[0]{\mathrm{div}\,}
\newcommand{\rot}[0]{\vec{\mathrm{rot}}\,}
\newcommand{\vnabla}[0]{\vec{\nabla}}
\renewcommand{\vec}[1]{\overrightarrow{#1}}
\newcommand{\mat}[1]{\mathrm{Mat}_{#1}}
\newcommand{\matrice}[1]{\mathcal{M}_{#1}}
\newcommand{\sgEngendre}[1]{\left\langle #1 \right\rangle}
\newcommand{\gpquotient}[1]{\mathbb{Z} / #1\mathbb{Z}}
\newcommand{\norme}[1]{||#1||}
\renewcommand{\d}[1]{\,\mathrm{d}#1}
\newcommand{\adh}[1]{\overline{#1}}
\newcommand{\intint}[2]{\llbracket #1 ,\, #2 \rrbracket}
\newcommand{\seg}[2]{[#1\, ; \, #2]}
\newcommand{\scal}[2]{( #1 | #2 )}
\newcommand{\distance}[2]{\mathrm{d}(#1,\,#2)}
\newcommand{\inte}[2]{\int_{#1}^{#2}}
\newcommand{\somme}[2]{\sum_{#1}^{#2}}
\newcommand{\deriveref}[4]{\biggl( \frac{\text{d}^{#1}#2}{\text{d}#3^{#1}} \biggr)_{#4}}






\begin{definition}{7.1}{endomorphisme diagonalisable}
    Soit $E$ un $\K$-espace vectoriel de dimension finie $n$ et $u$ un endomorphisme de $E$. On dit que \notion{$u$ est diagonalisable} s'il existe une base $\mc{B}$ de $E$ dans laquelle la matrice de $u$ est diagonale.

\end{definition}

\begin{proposition}{7.2}{CNS de diagonalisabilité}
    Soit $E$ un $\K$-espace vectoriel de dimension finie $n$ et $u$ un endomorphisme de $E$. Les assertions suivantes sont équivalentes.
    \begin{enumeratebf}
        \item Il existe une base de $E$ constituée de vecteurs propres de $u$.
        \item L'endomorphisme $u$ est diagonalisable.
        \item Les sous-espaces propres de $u$ sont supplémentaires dans $E$ : $\displaystyle E = \bigoplus_{\lambda \in \spectre(u)}E_\lambda (u)$
    \end{enumeratebf}
\end{proposition}

\begin{theoreme}{7.6}{caractérisation de la diagonalisabilité par la dimension des sous-espaces propres}
    Soit $E$ un $\K$-espace vectoriel de dimension finie $n$ et $u$ un endomorphisme de $E$. $u$ est diagonalisable si et seulement si la somme des dimensions des sous-espaces propres de $u$ vaut la dimension de $E$~:
    $$\somme{\lambda \in \spectre(u) }{}\dim\big(E_\lambda(u)\big) = \dim(E)$$ 
\end{theoreme}

\begin{theoreme}{7.7}{caractérisation de la diagonalisabilité par la multiplicité de ses valeurs propres}
    Soit $E$ un $\K$-espace vectoriel de dimension finie $n$ et $u$ un endomorphisme de $E$. $u$ est diagonalisable si et seulement si $\chi_u$ est scindé et si la dimension de chaque sous-espace propre associé à la valeur propre $\lambda$ vaut la multiplicité de $\lambda$~:
    $$\forall \lambda \in \spectre(u),\, \dim\big(E_\lambda(u)\big) = m_\lambda$$
\end{theoreme}

\begin{theoreme}{7.9}{caractérisation de la diagonalisabilité par le polynôme minimal}
    Soit $E$ un $\K$-espace vectoriel de dimension finie $n$ et $u$ un endomorphisme de $E$. $u$ est diagonalisable si et seulement si $\mu_u$ est scindé à racines simples.
\end{theoreme}

\begin{definition}{7.14}{endomorphisme trigonalisable}
    Soit $E$ un $\K$-espace vectoriel de dimension finie $n$ et $u$ un endomorphisme de $E$. On dit que \notion{$u$ est trigonalisable} s'il existe une base $\mc{B}$ de $E$ dans laquelle la matrice de $u$ est triangulaire supérieure.
\end{definition}

\begin{proposition}{7.17}{point commun entre polynômes caractéristique et minimal}
    Soit $E$ un $\K$-espace vectoriel de dimension finie $n$ et $u$ un endomorphisme de $E$. $\chi_u$ est scindé si et seulement si $\mu_u$ est scindé.
\end{proposition}

\begin{theoreme}{7.18}{caractérisation de la trigonalisabilité}
    Soit $E$ un $\K$-espace vectoriel de dimension finie $n$ et $u$ un endomorphisme de $E$. $u$ est trigonalisable si et seulement si ses polynômes caractéristique et minimal sont scindés (c'est équivalent).
\end{theoreme}

$$\fonction{f}{\Z/np\Z}{\Z/n\Z \times \Z /p\Z}{\mr{cl}_{np}(x)}{\big(\mr{cl}_n(x),\, \mr{cl}_p(x)\big)}$$
\end{adjustwidth}
\end{document}