\documentclass{article}
\usepackage{amsmath,amssymb,mathtools}
\usepackage{xcolor}
\usepackage{minted}
\usepackage{enumitem}
\usepackage{multicol}
\usepackage{changepage}
\usepackage{stmaryrd}
\usepackage{graphicx}
\graphicspath{ {./images/} }
\usepackage[framemethod=tikz]{mdframed}
\usepackage{tikz,pgfplots}
\pgfplotsset{compat=1.18}

% physique
\definecolor{oranges}{RGB}{255, 242, 230}
\definecolor{rouges}{RGB}{255, 230, 230}
\definecolor{rose}{RGB}{255, 204, 204}

% maths - info
\definecolor{rouge_fonce}{RGB}{204, 0, 0}
\definecolor{rouge}{RGB}{255, 0, 0}
\definecolor{bleu_fonce}{RGB}{0, 0, 255}
\definecolor{vert_fonce}{RGB}{0, 69, 33}
\definecolor{vert}{RGB}{0,255,0}

\definecolor{orange_foncee}{RGB}{255, 153, 0}
\definecolor{myrtille}{RGB}{225, 225, 255}
\definecolor{mayonnaise}{RGB}{255, 253, 233}
\definecolor{magenta}{RGB}{224, 209, 240}
\definecolor{pomme}{RGB}{204, 255, 204}
\definecolor{mauve}{RGB}{255, 230, 255}


% Cours

\newmdenv[
  nobreak=true,
  topline=true,
  bottomline=true,
  rightline=true,
  leftline=true,
  linewidth=0.5pt,
  linecolor=black,
  backgroundcolor=mayonnaise,
  innerleftmargin=10pt,
  innerrightmargin=10pt,
  innertopmargin=5pt,
  innerbottommargin=5pt,
  skipabove=\topsep,
  skipbelow=\topsep,
]{boite_definition}

\newmdenv[
  nobreak=false,
  topline=true,
  bottomline=true,
  rightline=true,
  leftline=true,
  linewidth=0.5pt,
  linecolor=white,
  backgroundcolor=white,
  innerleftmargin=10pt,
  innerrightmargin=10pt,
  innertopmargin=5pt,
  innerbottommargin=5pt,
  skipabove=\topsep,
  skipbelow=\topsep,
]{boite_exemple}

\newmdenv[
  nobreak=true,
  topline=true,
  bottomline=true,
  rightline=true,
  leftline=true,
  linewidth=0.5pt,
  linecolor=black,
  backgroundcolor=magenta,
  innerleftmargin=10pt,
  innerrightmargin=10pt,
  innertopmargin=5pt,
  innerbottommargin=5pt,
  skipabove=\topsep,
  skipbelow=\topsep,
]{boite_proposition}

\newmdenv[
  nobreak=true,
  topline=true,
  bottomline=true,
  rightline=true,
  leftline=true,
  linewidth=0.5pt,
  linecolor=black,
  backgroundcolor=white,
  innerleftmargin=10pt,
  innerrightmargin=10pt,
  innertopmargin=5pt,
  innerbottommargin=5pt,
  skipabove=\topsep,
  skipbelow=\topsep,
]{boite_demonstration}

\newmdenv[
  nobreak=true,
  topline=true,
  bottomline=true,
  rightline=true,
  leftline=true,
  linewidth=0.5pt,
  linecolor=white,
  backgroundcolor=white,
  innerleftmargin=10pt,
  innerrightmargin=10pt,
  innertopmargin=5pt,
  innerbottommargin=5pt,
  skipabove=\topsep,
  skipbelow=\topsep,
]{boite_remarque}


\newenvironment{definition}[2]
{
    \vspace{15pt}
    \begin{boite_definition}
    \textbf{\textcolor{rouge}{Définition #1}}
    \if\relax\detokenize{#2}\relax
    \else
        \textit{ - #2}
    \fi \\ \\
}
{
    \end{boite_definition}
    
}

\newenvironment{exemple}[2]
{
    \vspace{15pt}
    \begin{boite_exemple}
    \textbf{\textcolor{bleu_fonce}{Exemple #1}}
    \if\relax\detokenize{#2}\relax
    \else
        \textit{ - #2}
    \fi \\ \\ 
}
{   
    \end{boite_exemple}
    
}

\newenvironment{proposition}[2]
{
    \vspace{15pt}
    \begin{boite_proposition}
    \textbf{\textcolor{rouge}{Proposition #1}}
    \if\relax\detokenize{#2}\relax
    \else
        \textit{ - #2}
    \fi \\ \\
}
{
    \end{boite_proposition}
    
}

\newenvironment{theoreme}[2]
{
    \vspace{15pt}
    \begin{boite_proposition}
    \textbf{\textcolor{rouge}{Théorème #1}} 
    \if\relax\detokenize{#2}\relax
    \else
        \textit{ - #2}
    \fi \\ \\
}
{
    \end{boite_proposition}
    
}

\newenvironment{demonstration}
{
    \vspace{15pt}
    \begin{boite_demonstration}
    \textbf{\textcolor{rouge}{Démonstration}}\\ \\
}
{
    \end{boite_demonstration}
    
}

\newenvironment{remarque}[2]
{
    \vspace{15pt}
    \begin{boite_remarque}
    \textbf{\textcolor{bleu_fonce}{Remarque #1}}
    \if\relax\detokenize{#2}\relax
    \else
        \textit{ - #2}
    \fi \\ \\   
}
{  
    \end{boite_remarque}
    
}



%Corrections
\newmdenv[
  nobreak=true,
  topline=true,
  bottomline=true,
  rightline=true,
  leftline=true,
  linewidth=0.5pt,
  linecolor=black,
  backgroundcolor=mayonnaise,
  innerleftmargin=10pt,
  innerrightmargin=10pt,
  innertopmargin=5pt,
  innerbottommargin=5pt,
  skipabove=\topsep,
  skipbelow=\topsep,
]{boite_question}


\newenvironment{question}[2]
{
    \vspace{15pt}
    \begin{boite_question}
    \textbf{\textcolor{rouge}{Question #1}}
    \if\relax\detokenize{#2}\relax
    \else
        \textit{ - #2}
    \fi \\ \\
}
{
    \end{boite_question}
    
}

\newenvironment{enumeratebf}{
    \begin{enumerate}[label=\textbf{\arabic*.}]
}
{
    \end{enumerate}
}

\begin{document}
\begin{adjustwidth}{-3cm}{-3cm}
\begin{document}
\begin{adjustwidth}{-3cm}{-3cm}
% commandes
\newcommand{\notion}[1]{\textcolor{vert_fonce}{\textit{#1}}}
\newcommand{\mb}[1]{\mathbb{#1}}
\newcommand{\mc}[1]{\mathcal{#1}}
\newcommand{\mr}[1]{\mathrm{#1}}
\newcommand{\code}[1]{\texttt{#1}}
\newcommand{\ccode}[1]{\texttt{|#1|}}
\newcommand{\ov}[1]{\overline{#1}}
\newcommand{\abs}[1]{|#1|}
\newcommand{\rev}[1]{\texttt{reverse(#1)}}
\newcommand{\crev}[1]{\texttt{|reverse(#1)|}}

\newcommand{\ie}{\textit{i.e.} }

\newcommand{\N}{\mathbb{N}}
\newcommand{\R}{\mathbb{R}}
\newcommand{\C}{\mathbb{C}}
\newcommand{\K}{\mathbb{K}}
\newcommand{\Z}{\mathbb{Z}}

\newcommand{\A}{\mathcal{A}}
\newcommand{\bigO}{\mathcal{O}}
\renewcommand{\L}{\mathcal{L}}

\newcommand{\rg}[0]{\mathrm{rg}}
\newcommand{\re}[0]{\mathrm{Re}}
\newcommand{\im}[0]{\mathrm{Im}}
\newcommand{\cl}[0]{\mathrm{cl}}
\newcommand{\grad}[0]{\vec{\mathrm{grad}}}
\renewcommand{\div}[0]{\mathrm{div}\,}
\newcommand{\rot}[0]{\vec{\mathrm{rot}}\,}
\newcommand{\vnabla}[0]{\vec{\nabla}}
\renewcommand{\vec}[1]{\overrightarrow{#1}}
\newcommand{\mat}[1]{\mathrm{Mat}_{#1}}
\newcommand{\matrice}[1]{\mathcal{M}_{#1}}
\newcommand{\sgEngendre}[1]{\left\langle #1 \right\rangle}
\newcommand{\gpquotient}[1]{\mathbb{Z} / #1\mathbb{Z}}
\newcommand{\norme}[1]{||#1||}
\renewcommand{\d}[1]{\,\mathrm{d}#1}
\newcommand{\adh}[1]{\overline{#1}}
\newcommand{\intint}[2]{\llbracket #1 ,\, #2 \rrbracket}
\newcommand{\seg}[2]{[#1\, ; \, #2]}
\newcommand{\scal}[2]{( #1 | #2 )}
\newcommand{\distance}[2]{\mathrm{d}(#1,\,#2)}
\newcommand{\inte}[2]{\int_{#1}^{#2}}
\newcommand{\somme}[2]{\sum_{#1}^{#2}}
\newcommand{\deriveref}[4]{\biggl( \frac{\text{d}^{#1}#2}{\text{d}#3^{#1}} \biggr)_{#4}}






\begin{theoreme}{25.5}{caractérisations métriques de domination et négligeabilité}
    Soit $f$ et $g$ définies sur $X$ et $x_0 \in \overline{X}$. Ainsi,
    \begin{enumeratebf}
        \item $\displaystyle f(x) \underset{x \to x_0}{=} o\bigr(g(x)\bigl) \quad \Leftrightarrow \quad \forall \epsilon > 0,\, \exists \eta > 0,\, \forall x \in X,\, \abs{x-x_0}<\eta \implies \abs{f(x)}\leq \epsilon \abs{g(x)}$ 
        \item $\displaystyle f(x) \underset{x \to x_0}{=} \bigO \bigr(g(x)\bigl) \quad \Leftrightarrow \quad \exists M \in \R,\, \exists \eta > 0,\, \forall x \in X,\, \abs{x-x_0}<\eta \implies \abs{f(x)}\leq M \abs{g(x)}$ 
    \end{enumeratebf}
\end{theoreme}
\begin{definition}{25.13}{développement de Taylor de $f$}
Soit $f \in \mc{D}^{n}(\{x_0\})$. Un \notion{développement de Taylor à l'ordre $n$ de $f$ en $x_0$} est un polynôme $P \in \R_n[X]$ vérifiant :
$$ \forall i \in \intint{0}{n}, \quad P^{(n)}(x_0) = f^{(n)}(x_0) $$
donc un polynôme dont la courbe est, en $x_0$, \notion{tangente à l'ordre $n$} à celle de $f$. Ce polynôme existe, est unique et donné par : 
$$P = \sum_{k=0}^{n}\frac{f^{(k)}(x_0)}{k!}(X-x_0)^k$$
\end{definition}

\begin{definition}{25.16}{reste de Taylor à l'ordre $n$}
    Soit $f:I \to \R$ de classe $\mc{D}^{n}$ sur $\{x_0\}$. On appelle \notion{reste de Taylor à l'ordre $n$ de $f$ en $x_0$} la fonction : 
    $$R_{n,x_0} : I \to \R \quad ; \quad x \mapsto f(x) - \sum_{k=0}^{n}\frac{f^{(k)}(x_0)}{k!}(x-x_0)^k$$
\end{definition}

\begin{proposition}{25.18}{fonction développable en série de Taylor}
    Soit $f:I \to \R$ de classe $\mc{C}^{\infty}$ sur $I$, $x_0 \in I$. Alors : 
    $$\Bigr(\forall x \in I, \, \lim_{n \to +\infty} R_{n,x_0}(x) = 0\Bigl) \quad \implies \quad \Bigr(\forall x \in I, \, f(x) = \sum_{k=0}^{+\infty}\frac{f^{(k)}(x_0)}{k!}(X-x_0)^k\Bigl)$$
    Dans ce cas, on dit que $f$ est \notion{développable en série de Taylor en $x_0$}.
\end{proposition}

\begin{theoreme}{25.20}{formule de Taylor avec reste intégral à l'ordre $n$ au point $a$}
    Soit $a<b$ et $f:\seg{a}{b} \to \R$ de classe $\mc{C}^{n+1}$ sur $\seg{a}{b}$. Alors
    $$\forall x \in \seg{a}{b},\, f(x) = \sum_{k=0}^{n}\frac{f^{(k)}(a)}{k!}(x-a)^k + \int_{a}^{x} \frac{f^{(n+1)}(t)}{n!}(x-t)^n\, \text{d}t $$

\end{theoreme}

\begin{theoreme}{25.27}{formule de Taylor-Young à l'ordre $n$ en $x_0$}
    Soit $I$ ouvert, $x_0 \in I$ et $f:I \to \R$ de classe $\mc{C}^{n}$ au voisinage de $x_0$. Alors : 
    $$f(x) \underset{x \to x_0}{=} \sum_{k=0}^{n}\frac{f^{(k)}(x_0)}{k!}(x-x_0)^k \, + o\bigl((x-x_0)^n\bigr)$$
\end{theoreme}

\begin{proposition}{25.34}{formule de Taylor pour les polynômes}
    Pour tous $n \in \N$ et $x_0 \in \R$, on a :
    $$\forall P \in \R_n[X],\, P = \sum_{k=0}^{n}\frac{P^{(k)}(x_0)}{k!}(X-x_0)^k$$
\end{proposition}

\begin{definition}{25.36}{développement limité}
    Soit $I$ ouvert, $x_0 \in I$ et $f:I \to \R$. On dit que le polynôme $ \displaystyle P = \sum_{k=0}^{n}a_k(X - x_0)^k \in \R_n[X]$ est un \notion{développement limité à l'ordre $n$ de $f$ en $x_0$} si on a :\\
    $$f(x) \underset{x \to x_0}{=}  P(x) + o\bigl((x-x_0)^n\bigr)$$ \\
    Si $f$ est de classe $\mc{C}^{n}$ au voisinage de $x_0$, un tel polynôme existe, est unique et donné par la formule de Taylor-Young:
    $$P = \sum_{k=0}^{n}\frac{f^{(k)}(x_0)}{k!}(X-x_0)^k$$
\end{definition}

\begin{proposition}{25.41}{DL de fonctions paires ou impaires}
    Soit $I$ ouvert tel que $0 \in I$ et $f:I \to \R$. \begin{itemize}
        \item Si $f$ est paire, alors si existence, ses développements limités en 0 ne présentent que des monômes de degré pair.
        \item Si $f$ est impaire, alors si existence, ses développements limités en 0 ne présentent que des monômes de degré impair.
    \end{itemize}
\end{proposition}

\begin{proposition}{25.42}{classe d'une fonction admettant un DL}
    Soit $I$ ouvert, $x_0 \in I$ et $f:I \to \R$.
    \begin{enumeratebf}
        \item Si $f$ admet un DL à l’ordre $0$ en $x_0$, alors $f$ est continue en $x_0$.
        \item Si $f$ admet un DL à l’ordre $1$ en $x_0$, alors $f$ est dérivable en $x_0$.
    \end{enumeratebf}
\end{proposition}

\begin{definition}{25.44}{DL au sens fort}
    Soit $I$ ouvert, $x_0 \in I$ et $f:I \to \R$. On dit que le polynôme $ \displaystyle P = \sum_{k=0}^{n}a_k(X - x_0)^k \in \R_n[X]$ est un \notion{développement limité au sens fort à l'ordre $n$ de $f$ en $x_0$} si on a :\\
    $$f(x) \underset{x \to x_0}{=}  P(x) + \bigO\bigl((x-x_0)^{n+1}\bigr)$$
\end{definition}

\begin{definition}{25.46}{troncature}
    Soit $m\leq n$ dans $\N$, et $\displaystyle P = \sum_{k=0}^{n}a_k(X - x_0)^k \in \R_n[X]$. La troncature de $P$ à l'ordre $m$ au voisinage de $x_0$ est le polynôme : 
    $$T_{m,x_0}(P) = \sum_{k=0}^{m}a_k(X - x_0)^k$$\\
    Ainsi, si : 
    $$f(x) \underset{x \to x_0}{=} P(x) \, + o\bigl((x-x_0)^n\bigr)$$
    alors,
    $$f(x) \underset{x \to x_0}{=} T_{m,x_0}(P)(x) \, + o\bigl((x-x_0)^m\bigr)$$
\end{definition}

\begin{proposition}{25.55}{somme de DL}
    Soit $I$ et $J$ ouverts tel que $(0,0) \in I\times J$, $f:I \to \R$,  $g:J \to \R$ et $(P,Q) \in \R_n[X]^2$. Si on a :
    $$\begin{cases}
        f(x) \underset{x \to 0}{=} P(x)  + o\bigl(x^n\bigr)\\
        g(x) \underset{x \to 0}{=} Q(x)  + o\bigl(x^n\bigr)
    \end{cases}$$
    Alors,
    $$(f+g)(x) \underset{x \to 0}{=} P(x) + Q(x) + o\bigl(x^n\bigr)$$
\end{proposition}

\begin{proposition}{25.56}{produit de DL}
    Soit $I$ et $J$ ouverts tel que $(0,0) \in I\times J$, $f:I \to \R$,  $g:J \to \R$ et $(P,Q) \in \R_n[X]^2$. Si on a :
    $$\begin{cases}
        f(x) \underset{x \to 0}{=} P(x)  + o\bigl(x^n\bigr)\\
        g(x) \underset{x \to 0}{=} Q(x)  + o\bigl(x^n\bigr)
    \end{cases}$$
    Alors,
    $$(fg)(x) \underset{x \to 0}{=} T_{n,0}(PQ)(x) + o\bigl(x^n\bigr)$$
\end{proposition}

\begin{proposition}{25.59}{composition de DL}
    Soit $I$ et $J$ ouverts tel que $(0,0) \in I\times J$, $f:I \to \R$ telle que $f(0) = 0$,  $g:J \to \R$ et $(P,Q) \in \R_n[X]^2$. Si on a :
    $$\begin{cases}
        f(x) \underset{x \to 0}{=} P(x)  + o\bigl(x^n\bigr)\\
        g(x) \underset{x \to 0}{=} Q(x)  + o\bigl(x^n\bigr)
    \end{cases}$$
    Alors,
    $$g \circ f(x) \underset{x \to 0}{=} T_{n,0}(Q\circ P)(x) + o\bigl(x^n\bigr)$$
\end{proposition}

\begin{proposition}{25.62}{DL d'une réciproque}
    Soit $I$ ouvert tel que $0 \in I$, $f:I \to \R$, $P \in \R_n[X]$ tels que :
    $$f(0) = 0 \quad \text{et} \quad   f(x) \underset{x \to 0}{=} P(x)  + o\bigl(x^n\bigr)$$ \\
    Si $f$ est bijective (ou au moins injective) au voisinage de $0$, alors il existe $Q \in \R_n[X]$ tel que :
    $$f^{-1}(x) \underset{x \to 0}{=} Q(x)  + o\bigl(x^n\bigr)$$
    Ce polynôme $Q$ s'identifie en résolvant :
    $$x \underset{x \to 0}{=} f^{-1} \circ f(x) + o\bigl(x^n\bigr)$$
\end{proposition}

\begin{proposition}{25.65}{DL d'un inverse}
    Soit $I$ ouvert tel que $0 \in I$, $f:I \to \R$, $ \displaystyle P = \sum_{k=0}^{n}a_kX^k \in \R_n[X]$ tels que : 
    $$f(0) \neq 0 \quad \text{et} \quad   f(x) \underset{x \to 0}{=} P(x)  + o\bigl(x^n\bigr)$$
    Comme $f(0) \neq 0$, on a également $P(0) \neq 0$. On peut alors écrire :
    $$P = a_0(1 +  Q ) \qquad \text{avec} \quad Q = a_0^{-1} \sum_{k=1}^{n}a_kX_k$$
    où ici $Q(0) = 0$. En remarquant que pour tout $x \in \R$, 
    $$\frac{1}{P(x)} = \frac{1}{a_0(1+Q(x))}$$
    on peut appliquer la formule de composition des développements limités sur $h \circ Q$ avec $h:x \mapsto \frac{1}{1+x} $ ($Q(0) = 0$), on a :\\
    $$\frac{1}{f(x)} \underset{x \to 0}{=} \frac{1}{a_0} \times T_{n,0}\Biggl(\sum_{k=0}^{n}(-1)^k Q^k\Biggr) \, + o\bigl(x^n\bigr)$$
\end{proposition}

\begin{proposition}{25.65}{primitive de DL}
    Soit $I$ ouvert tel que $0 \in I$, $f \in \mc{D}^1(I,\R) $, $ \displaystyle P = \sum_{k=0}^{n-1}a_kX^k \in \R_{n-1}[X]$ tels que :
    $$f'(x) \underset{x \to 0}{=} P(x)  + o\bigl(x^{n-1}\bigr)$$\\
    Alors $f$ admet un développement limité à l'ordre $n$, en $0$, donné par :
    \begin{align*}
        f(x) &\underset{x \to 0}{=} f(0) + \sum_{k=1}^{n} \frac{a_{k-1}}{k} x^k \, + o(x^n) \\
        &\underset{x \to 0}{=} f(0) + a_0x + \frac{a_1}{2}x^2 + \frac{a_2}{3}x^3 + \dots + \frac{a_{n-1}}{n}x^n + o(x^n)
     \end{align*}
\end{proposition}





\begin{proposition}{25.87 (1)}{DL de $x \mapsto e^x$}
    Le développement limité au rang $n \in \N$ de $x \mapsto e^x$ en $0$ est :
    \begin{align*}
        e^x &\underset{x \to 0}{=} \sum_{k=0}^{n} \frac{x^k}{k!} \, + o(x^n) \\
        &\underset{x \to 0}{=} 1 + x + \frac{x^2}{2!} + \frac{x^3}{3!} + \dots + \frac{x^n}{n!} + o(x^n)
    \end{align*}
\end{proposition}

\begin{proposition}{25.87 (2)}{DL de $x \mapsto \ln(1+x)$}
    Le développement limité au rang $n \in \N$ de $x \mapsto \ln(1+x)$  en $0$ est :
    \begin{align*}
        \ln(1+x) &\underset{x \to 0}{=} \sum_{k=1}^{n} (-1)^{k+1}\frac{x^k}{k} \, + o(x^n) \\
        &\underset{x \to 0}{=} x - \frac{x^2}{2} + \frac{x^3}{3} - \frac{x^4}{4} + \dots + \frac{(-1)^{n+1}x^n}{n!} + o(x^n)
    \end{align*}
\end{proposition}

\begin{proposition}{25.87 (3)}{DL de $x \mapsto \cos(x)$}
    Le développement limité au rang $n \in \N$ de $x \mapsto \cos(x)$ en $0$ est :
    \begin{align*}
        \cos(x) &\underset{x \to 0}{=} \sum_{k=0}^{\lfloor \frac{n}{2} \rfloor} \frac{(-1)^{k}x^{2k}}{(2k)!} \, + o(x^n) \\
        &\underset{x \to 0}{=} 1 - \frac{x^2}{2!} + \frac{x^4}{4!} - \frac{x^6}{6!} + \dots + \frac{(-1)^{\lfloor \frac{n}{2}\rfloor} x^{(2\lfloor\frac{n}{2}\rfloor)}}{(2\lfloor\frac{n}{2}\rfloor)!} + o(x^n)
    \end{align*}
\end{proposition}

\begin{proposition}{25.87 (4)}{DL de $x \mapsto \sin(x)$}
    Le développement limité au rang $n \in \N$ de $x \mapsto \sin(x)$ en $0$ est :
    \begin{align*}
        \sin(x) &\underset{x \to 0}{=} \sum_{k=0}^{\lfloor \frac{n-1}{2} \rfloor} \frac{(-1)^{k}x^{2k+1}}{(2k+1)!} \, + o(x^n) \\
        &\underset{x \to 0}{=} x - \frac{x^3}{3!} + \frac{x^5}{5!} - \frac{x^7}{7!} + \dots + \frac{(-1)^{\lfloor \frac{n-1}{2}\rfloor} x^{(2\lfloor\frac{n-1}{2}\rfloor+1)}}{(2\lfloor\frac{n-1}{2}\rfloor+1)!} + o(x^n)
    \end{align*}
\end{proposition}

\begin{proposition}{25.87 (5)}{DL de $x \mapsto \tan(x)$}
    Le développement limité au rang $10$ de $x \mapsto \tan(x)$ en $0$ est :
    \begin{align*}
        \tan(x) \underset{x \to 0}{=} x + \frac{1}{3}x^3 + \frac{2}{15}x^5 + \frac{17}{315}x^7 + \frac{62}{2835} x^9+ o(x^{10})
    \end{align*}
\end{proposition}

\begin{proposition}{25.87 (6)}{DL de $x \mapsto \arctan(x)$}
    Le développement limité au rang $n \in \N$ de $x \mapsto \arctan(x)$ en $0$ est :
    \begin{align*}
        \arctan(x) &\underset{x \to 0}{=} \sum_{k=0}^{\lfloor \frac{n-1}{2} \rfloor} \frac{(-1)^{k}x^{2k+1}}{2k+1} \, + o(x^n) \\
        &\underset{x \to 0}{=} x - \frac{x^3}{3} + \frac{x^5}{5} - \frac{x^7}{7} + \dots + \frac{(-1)^{\lfloor \frac{n-1}{2}\rfloor} x^{(2\lfloor\frac{n-1}{2}\rfloor)}}{(2\lfloor\frac{n-1}{2}\rfloor)} + o(x^n)
    \end{align*}
\end{proposition}

\begin{proposition}{25.87 (7)}{DL de $x \mapsto \cosh(x)$}
    Le développement limité au rang $n \in \N$ de $x \mapsto \cosh(x)$ en $0$ est :
    \begin{align*}
        \cosh(x) &\underset{x \to 0}{=} \sum_{k=0}^{\lfloor \frac{n}{2} \rfloor} \frac{x^{2k}}{(2k)!} \, + o(x^n) \\
        &\underset{x \to 0}{=} 1 + \frac{x^2}{2!} + \frac{x^4}{4!} + \frac{x^6}{6!} + \dots + \frac{x^{(2\lfloor\frac{n}{2}\rfloor)}}{(2\lfloor\frac{n}{2}\rfloor)!} + o(x^n)
    \end{align*}
\end{proposition}

\begin{proposition}{25.87 (8)}{DL de $x \mapsto \sinh(x)$}
    Le développement limité au rang $n \in \N$ de $x \mapsto \sinh(x)$ en $0$ est :
    \begin{align*}
        \sinh(x) &\underset{x \to 0}{=} \sum_{k=0}^{\lfloor \frac{n-1}{2} \rfloor} \frac{x^{2k+1}}{(2k+1)!} \, + o(x^n) \\
        &\underset{x \to 0}{=} x + \frac{x^3}{3!} + \frac{x^5}{5!} + \frac{x^7}{7!} + \dots + \frac{x^{(2\lfloor\frac{n-1}{2}\rfloor)}}{(2\lfloor\frac{n-1}{2}\rfloor)!} + o(x^n)
    \end{align*}
\end{proposition}

\begin{proposition}{25.87 (9)}{DL de $x \mapsto \tanh(x)$}
    Le développement limité au rang $10$ de $x \mapsto \tanh(x)$ en $0$ est :
    \begin{align*}
        \tanh(x) \underset{x \to 0}{=} x - \frac{1}{3}x^3 + \frac{2}{15}x^5 - \frac{17}{315}x^7 + \frac{62}{2835} x^9+ o(x^{10})
    \end{align*}
\end{proposition}


\begin{proposition}{25.87 (10)}{DL de $x \mapsto (1+x)^{\alpha}$}
    Soit $\alpha \in \R$. Le développement limité au rang $n \in \N$ de $x \mapsto (1+x)^{\alpha}$ en $0$ est :
    \begin{align*}
        (1+x)^{\alpha} &\underset{x \to 0}{=} \sum_{k=0}^{n} \frac{x^k}{k!} \prod_{i=0}^{k-1}(\alpha-i) \, + o(x^n) \\
        &\underset{x \to 0}{=} 1 + \alpha x + \alpha(\alpha-1) \frac{x^2}{2} + \alpha(\alpha-1)(\alpha-2)\frac{x^3}{6} + \dots + \frac{x^n}{n!}\prod_{i=0}^{n-1}(\alpha-i) \,+  o(x^n)
    \end{align*}
\end{proposition}

\begin{proposition}{25.87 (11)}{DL de $x \mapsto \frac{1}{1+x}$}
    Le développement limité au rang $n \in \N$ de $x \mapsto \frac{1}{1+x}$ en $0$ est :
    \begin{align*}
        \frac{1}{1+x} &\underset{x \to 0}{=} \sum_{k=0}^{n} (-1)^k x^k \, + o(x^n) \\
        &\underset{x \to 0}{=} 1 - x + x^2 - x^3 + \dots + (-1)^n x^n +  o(x^n)
    \end{align*}
\end{proposition}

\begin{proposition}{25.87 (12)}{DL de $x \mapsto \frac{1}{1-x}$}
    Le développement limité au rang $n \in \N$ de $x \mapsto \frac{1}{1-x}$ en $0$ est :
    \begin{align*}
        \frac{1}{1-x} &\underset{x \to 0}{=} \sum_{k=0}^{n} x^k \, + o(x^n) \\
        &\underset{x \to 0}{=} 1 + x + x^2 + x^3 + \dots + x^n +  o(x^n)
    \end{align*}
\end{proposition}

\end{adjustwidth}
\end{document}