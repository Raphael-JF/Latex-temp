% commandes
\newcommand{\notion}[1]{\textcolor{vert_fonce}{\textit{#1}}}
\newcommand{\mb}[1]{\mathbb{#1}}
\newcommand{\mc}[1]{\mathcal{#1}}
\newcommand{\mr}[1]{\mathrm{#1}}
\newcommand{\code}[1]{\texttt{#1}}
\newcommand{\ccode}[1]{\texttt{|#1|}}
\newcommand{\ov}[1]{\overline{#1}}
\newcommand{\abs}[1]{|#1|}
\newcommand{\rev}[1]{\texttt{reverse(#1)}}
\newcommand{\crev}[1]{\texttt{|reverse(#1)|}}

\newcommand{\ie}{\textit{i.e.} }

\newcommand{\N}{\mathbb{N}}
\newcommand{\R}{\mathbb{R}}
\newcommand{\C}{\mathbb{C}}
\newcommand{\K}{\mathbb{K}}
\newcommand{\Z}{\mathbb{Z}}

\newcommand{\A}{\mathcal{A}}
\newcommand{\bigO}{\mathcal{O}}
\renewcommand{\L}{\mathcal{L}}

\newcommand{\rg}[0]{\mathrm{rg}}
\newcommand{\re}[0]{\mathrm{Re}}
\newcommand{\im}[0]{\mathrm{Im}}
\newcommand{\cl}[0]{\mathrm{cl}}
\newcommand{\grad}[0]{\vec{\mathrm{grad}}}
\renewcommand{\div}[0]{\mathrm{div}\,}
\newcommand{\rot}[0]{\vec{\mathrm{rot}}\,}
\newcommand{\vnabla}[0]{\vec{\nabla}}
\renewcommand{\vec}[1]{\overrightarrow{#1}}
\newcommand{\mat}[1]{\mathrm{Mat}_{#1}}
\newcommand{\matrice}[1]{\mathcal{M}_{#1}}
\newcommand{\sgEngendre}[1]{\left\langle #1 \right\rangle}
\newcommand{\gpquotient}[1]{\mathbb{Z} / #1\mathbb{Z}}
\newcommand{\norme}[1]{||#1||}
\renewcommand{\d}[1]{\,\mathrm{d}#1}
\newcommand{\adh}[1]{\overline{#1}}
\newcommand{\intint}[2]{\llbracket #1 ,\, #2 \rrbracket}
\newcommand{\seg}[2]{[#1\, ; \, #2]}
\newcommand{\scal}[2]{( #1 | #2 )}
\newcommand{\distance}[2]{\mathrm{d}(#1,\,#2)}
\newcommand{\inte}[2]{\int_{#1}^{#2}}
\newcommand{\somme}[2]{\sum_{#1}^{#2}}
\newcommand{\deriveref}[4]{\biggl( \frac{\text{d}^{#1}#2}{\text{d}#3^{#1}} \biggr)_{#4}}





\documentclass{article}
\usepackage{amsmath,amssymb,mathtools}
\usepackage{xcolor}
\usepackage{enumitem}
\usepackage{multicol}
\usepackage{changepage}
\usepackage{stmaryrd}
\usepackage{graphicx}
\usepackage[framemethod=tikz]{mdframed}
\usepackage{tikz,pgfplots}
\pgfplotsset{compat=1.18}

% physique
\definecolor{oranges}{RGB}{255, 242, 230}
\definecolor{rouges}{RGB}{255, 230, 230}
\definecolor{rose}{RGB}{255, 204, 204}

% maths - info
\definecolor{rouge_fonce}{RGB}{204, 0, 0}
\definecolor{bleu_fonce}{RGB}{0, 0, 255}
\definecolor{vert_fonce}{RGB}{0, 69, 33}

\definecolor{orange_foncee}{RGB}{255, 153, 0}
\definecolor{myrtille}{RGB}{225, 225, 255}
\definecolor{mayonnaise}{RGB}{255, 253, 233}
\definecolor{magenta}{RGB}{224, 209, 240}
\definecolor{pomme}{RGB}{204, 255, 204}
\definecolor{mauve}{RGB}{255, 230, 255}


% Cours

\newmdenv[
  nobreak=true,
  topline=true,
  bottomline=true,
  rightline=true,
  leftline=true,
  linewidth=0.5pt,
  linecolor=black,
  backgroundcolor=mayonnaise,
  innerleftmargin=10pt,
  innerrightmargin=10pt,
  innertopmargin=10pt,
  innerbottommargin=10pt,
  skipabove=\topsep,
  skipbelow=\topsep,
]{boite_definition}

\newmdenv[
  nobreak=true,
  topline=true,
  bottomline=true,
  rightline=true,
  leftline=true,
  linewidth=0.5pt,
  linecolor=white,
  backgroundcolor=white,
  innerleftmargin=10pt,
  innerrightmargin=10pt,
  innertopmargin=10pt,
  innerbottommargin=10pt,
  skipabove=\topsep,
  skipbelow=\topsep,
]{boite_exemple}

\newmdenv[
  nobreak=true,
  topline=true,
  bottomline=true,
  rightline=true,
  leftline=true,
  linewidth=0.5pt,
  linecolor=black,
  backgroundcolor=magenta,
  innerleftmargin=10pt,
  innerrightmargin=10pt,
  innertopmargin=10pt,
  innerbottommargin=10pt,
  skipabove=\topsep,
  skipbelow=\topsep,
]{boite_proposition}

\newmdenv[
  nobreak=true,
  topline=true,
  bottomline=true,
  rightline=true,
  leftline=true,
  linewidth=0.5pt,
  linecolor=black,
  backgroundcolor=white,
  innerleftmargin=10pt,
  innerrightmargin=10pt,
  innertopmargin=10pt,
  innerbottommargin=10pt,
  skipabove=\topsep,
  skipbelow=\topsep,
]{boite_demonstration}

\newmdenv[
  nobreak=true,
  topline=true,
  bottomline=true,
  rightline=true,
  leftline=true,
  linewidth=0.5pt,
  linecolor=white,
  backgroundcolor=white,
  innerleftmargin=10pt,
  innerrightmargin=10pt,
  innertopmargin=10pt,
  innerbottommargin=10pt,
  skipabove=\topsep,
  skipbelow=\topsep,
]{boite_remarque}


\newenvironment{definition}[2]
{
    \begin{boite_definition}
    \textbf{\textcolor{rouge_fonce}{Définition #1}} \textit{#2} \\ \\
}
{
    \end{boite_definition}
    \vspace{15pt}
}

\newenvironment{exemple}[2]
{
    \begin{boite_exemple}
    \textbf{\textcolor{bleu_fonce}{Exemple #1}} \textit{#2} \\
    \begin{itemize}[label=$\blacktriangleright \quad $]                    
}
{   
    \end{itemize}
    \end{boite_exemple}
    \vspace{15pt}
}

\newenvironment{proposition}[2]
{
    \begin{boite_proposition}
    \textbf{\textcolor{rouge_fonce}{Proposition #1}} \textit{#2} \\ \\
}
{
    \end{boite_proposition}
    \vspace{15pt}
}

\newenvironment{theoreme}[2]
{
    \begin{boite_proposition}
    \textbf{\textcolor{rouge_fonce}{Théorème #1}} \textit{#2} \\ \\
}
{
    \end{boite_proposition}
    \vspace{15pt}
}

\newenvironment{demonstration}
{
    \begin{boite_demonstration}
    \textbf{\textcolor{rouge_fonce}{Démonstration}}\\ \\
}
{
    \end{boite_demonstration}
    \vspace{15pt}
}

\newenvironment{remarque}[2]
{
    \begin{boite_remarque}
    \textbf{\textcolor{bleu_fonce}{Remarque #1}}\textit{#2} \\ \\
    \begin{itemize}[label=$\blacktriangleright \quad $ ]                    
}
{   
    \end{itemize}
    \end{boite_remarque}
    \vspace{15pt}
}



%Corrections
\newmdenv[
  nobreak=true,
  topline=true,
  bottomline=true,
  rightline=true,
  leftline=true,
  linewidth=0.5pt,
  linecolor=black,
  backgroundcolor=mayonnaise,
  innerleftmargin=10pt,
  innerrightmargin=10pt,
  innertopmargin=10pt,
  innerbottommargin=10pt,
  skipabove=\topsep,
  skipbelow=\topsep,
]{boite_question}



\newenvironment{question}[2]
{
    \begin{boite_question}
    \textbf{\textcolor{rouge_fonce}{Question #1}} \textit{#2} \\ \\
}
{
    \end{boite_question}
    \vspace{15pt}
}

\newenvironment{enumeratebf}{
    \begin{enumerate}[label=\textbf{\arabic*.}]
}
{
    \end{enumerate}
}

\begin{document}
\begin{adjustwidth}{-3cm}{-3cm}
\everymath{\displaystyle}

    \begin{definition}{17.1}{ - fraction rationnelle}
        Dans $\mb{K}[X] \times (\mb{K}[X] \backslash \{0\})$ On définit la relation d'équivalence $\mc{R}$ en posant : \begin{align*}
            &(P,Q)\mc{R}(R,S) &&\\
            \Leftrightarrow \quad& P/Q = R/S && \text{(Cette étape n'est qu'à titre explicatif dans la mesure où l'opération / n'est pas définie)} \\
            \Leftrightarrow \quad& PS = RQ &&\\
        \end{align*}
    On appelle \notion{fraction rationnelle} à coefficients dans $\mb{K}$ toute classe d'équivalence pour la relation $\mc{R}$. La classe de $(P,Q)$ est alors notée $\displaystyle \frac{P}{Q}$. On a donc : 
    $$\frac{P}{Q} = \{(R,S) \in \mb{K}[X] \times (\mb{K}[X] \backslash \{0\}), PS = RQ\}$$
    On dit que $(P,Q)$ est un \notion{représentant} de la fraction $\displaystyle\frac{P}{Q}$. L'ensemble des fractions rationnelles est noté $\mb{K}(X)$ et la relation $\mc{R}$ est appelée \notion{égalité des fractions rationnelles}.

    \end{definition} 

    \begin{proposition}{16.4}{ - structure de $\mb{K}(X)$}
        $(\mb{K}(X),+,\times)$ est un corps commutatif et $(\mb{K}(X),+,\times,\cdot)$ (où $\cdot$ est la loi externe) est une $\mb{K}$-algèbre commutative. \\ \\
        L'application $\varphi : \mb{K}[X] \rightarrow \mb{K}(X)$ définie par $ \varphi(P) = \frac{P}{1}$ est un morphisme d'algèbres injectif.
    \end{proposition}

    \begin{definition}{17.7}{ - représentant irréductible}
        Soit $F = \frac{P}{Q}$ une fraction. On dit que $\frac{P}{Q}$ est un \notion{représentant irréductible} lorsque $P \wedge Q = 1$ et que $Q$ est unitaire. Toute fration rationnelle de $\mb{K}(X)$ admet un \underline{unique} (dénominateur unitaire) représentant irréductible.
    \end{definition}

    \begin{theoreme}{17.34}{ - décomposition en éléments simples}
        Soit $F = \frac{A}{B}$ une fraction sous forme irréductible, et $B = \prod_{i=1}^{k}P_i^{\alpha_i}$ sa décomposition en produit de polynômes irréductibles. Il existe des polynômes $(U_{i})_{i \in \llbracket 1,k \rrbracket}$ tels que $$F = E + \sum_{i=1}^{k}\frac{U_{i}}{P_i^{\alpha_i}} \quad \text{avec $\deg(\frac{U_{i}}{P_i})<0$} $$ \\
        De plus, pour $n \in \mb{N}^*$, Si $T \in \mc{I}_{\mb{K}[X]} $ et $\deg(\frac{A}{T^n})<0$, alors il existe des polynômes $V_1,\dots,V_n$ tels que $$\frac{A}{T^n} = \sum_{k=1}^{n}\frac{V_k}{T^k} \quad \text{avec $\deg(\frac{V_k}{T^k})<0$}$$ \\
        Finalement, Il existe des polynômes $(U_{i,j})_{i \in \llbracket 1,k \rrbracket, j \in \llbracket 1,\alpha_i \rrbracket}$ tels que $$F = E + \sum_{i=1}^{k}\sum_{j=1}^{\alpha_i}\frac{U_{i,j}}{P_i^j} \quad \text{avec $\deg(\frac{U_{i,j}}{P_i})<0$} $$ \\ \\
        Cette décomposition est unique.
    \end{theoreme}

    \begin{proposition}{17.40}{ - cas d'un pôle d'ordre $n \in \mb{N}^*$ pour une fraction de $\mb{C}(X)$}
        Si $a \in \mb{C}$ est un pôle d'ordre de multiplicité $n \in \mb{N}^*$ de $F \in \mathbb{C}(X)$, alors la partie polaire de $F$ relative à $a$ est, en posant $H = (X-a)^n F$ : $$P_F(a) = \sum_{k=1}^{n}\frac{H^{(k-1)}(a)}{(X-a)^k} = \frac{H(a)}{X-a} + \frac{H'(a)}{(X-a)^2}+\dots + \frac{H^{(n-1)}(a)}{(X-a)^n}$$
        
    \end{proposition}

    \begin{remarque}{17.51}{ - primitives d'éléments simples de première espèce}
        Un élément simple de première espèce est de la forme $\frac{1}{(X-a)^n}$, avec $n \in \mathbb{N}^*$.
        $$\int^{x} \frac{1}{(t-a)^n}dt = \begin{cases}\ln|x-a| & \text{si $n=1$} \\ \\\frac{-1}{(n-1)(x-a)^{n-1}} & \text{si $n > 1$}\end{cases}$$
    \end{remarque}

    \begin{remarque}{17.51}{ - primitives d'éléments simples de deuxième espèce}
        Un élément simple de seconde espèce est de la forme $\frac{aX+b}{(X^2+pX+c)^n}$, avec $n \in \mb{N}^*$. On ne traite que le cas $n=1$ : \\
        \begin{align*}
            \frac{aX+b}{X^2+pX+q} &=\frac{\frac{a}{2}2X}{X^2+pX+q} + \frac{b}{X^2+pX+q} \qquad\text{On fait apparaître la dérivée du trinôme au numérateur} \\ \\
            &=\frac{\frac{a}{2}(2X+p-p)}{X^2+pX+q} + \frac{b}{X^2+pX+q}\qquad \\ \\
            &=\frac{\frac{a}{2}(2X+p)}{X^2+pX+q} + \frac{b-\frac{a}{2}p}{X^2+pX+q}\qquad\\ \\
            &=\frac{\frac{a}{2}(2X+p)}{X^2+pX+q} + \frac{b-\frac{a}{2}p}{(X+\frac{p}{2})^2+\frac{4q-p^2}{4}} \qquad \text{On passe le dénominateur sous forme canonique}\\ \\
            &=\frac{\frac{a}{2}(2X+p)}{X^2+pX+q} + \frac{(b-\frac{ap}{2})(\frac{4}{4q-p^2})}{\frac{4}{4q-p^2}(X+\frac{p}{2})^2+1}\qquad \text{On divise par $4q-p^2>0$ pour avoir une forme $\alpha\frac{u'}{u^2+1}$}\\\\
            &=\frac{\frac{a}{2}(2X+b)}{X^2+bX+q} + \frac{(b-\frac{ap}{2})(\frac{4}{4q-p^2})}{(\frac{2}{\sqrt{4q-p^2}}X+\frac{p}{\sqrt{4q-p^2}})^2+1}\qquad \text{On fait rentrer $\frac{4}{4q-p^2}>0$ dans $u$ avec $x \mapsto \sqrt{x}$} \\\\
            \Aboxed{\frac{aX+b}{X^2+pX+q}&=\frac{\frac{a}{2}(2X+b)}{X^2+pX+q} + \frac{\frac{2}{\sqrt{4q-p^2}}}{(\frac{2}{\sqrt{4q-p^2}}X+\frac{p}{\sqrt{4q-p^2}})^2+1}\frac{2b-ap}{\sqrt{4q-p^2}}}
        \\\end{align*}
        Passage au calcul intégral :
            $$\Aboxed{\int^x \frac{at+b}{t^2+pt+q}dt = \frac{a}{2}{\ln|x^2 + px + q|} + \frac{2b-ap}{\sqrt{4q-p^2}}\arctan(\frac{2x+p}{\sqrt{4q-p^2}})}$$
    \end{remarque}
\end{adjustwidth}
\end{document}