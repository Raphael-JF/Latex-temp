% commandes
\newcommand{\notion}[1]{\textcolor{vert_fonce}{\textit{#1}}}
\newcommand{\mb}[1]{\mathbb{#1}}
\newcommand{\mc}[1]{\mathcal{#1}}
\newcommand{\code}[1]{\texttt{#1}}
\newcommand{\ccode}[1]{\texttt{|#1|}}
\newcommand{\ov}[1]{\overline{#1}}
\newcommand{\abs}[1]{|#1|}
\newcommand{\rev}[1]{\texttt{reverse(#1)}}
\newcommand{\crev}[1]{\texttt{|reverse(#1)|}}

\newcommand{\ie}{\textit{i.e.} }

\newcommand{\N}{\mathbb{N}}
\newcommand{\R}{\mathbb{R}}
\newcommand{\C}{\mathbb{C}}
\newcommand{\K}{\mathbb{K}}

\newcommand{\A}{\mathcal{A}}
\newcommand{\bigO}{\mathcal{O}}
\renewcommand{\L}{\mathcal{L}}

\newcommand{\rg}[0]{\text{rg}}
\newcommand{\re}[0]{\text{Re}}
\newcommand{\im}[0]{\text{Im}}
\newcommand{\cl}[0]{\text{cl}}
\newcommand{\mat}[1]{\text{Mat}_{#1}}
\newcommand{\matrice}[1]{\mathcal{M}_{#1}}
\newcommand{\sgEngendre}[1]{\left\langle #1 \right\rangle}
\newcommand{\norme}[1]{||#1||}
\renewcommand{\d}[1]{\,\text{d}#1}
\newcommand{\intint}[2]{\llbracket #1 ,\, #2 \rrbracket}
\newcommand{\seg}[2]{[#1\, ; \, #2]}
\newcommand{\scal}[2]{\left\langle #1 ,\, #2 \right\rangle}
\newcommand{\inte}[2]{\int_{#1}^{#2}}
\newcommand{\somme}[2]{\sum_{#1}^{#2}}





\input{../stock/en-tete.tex}

\begin{question}{I.1) }{Définition inductive de la concaténation en Ocaml}
    Pour toutes listes \code{l} et \code{q}, et tout élément \code{e} :
    $\begin{cases}
        \code{[] @ l = l} \\
        \code{(e::q @ l = e::(q @ l))}
    \end{cases}$

\end{question}

\begin{question}{I.2) }{Rédaction de démonstration par induction structurelle}
    Soit \code{l2} une liste.
    Montrons par récurrence structurelle sur \code{l1} que $\ccode{l1 @ l2} = \ccode{l1 + l2}$ pour toutes listes \code{l1} et \code{l2}.
   \begin{itemize}
       \item Si \code{l1 = []}, $\ccode{l1 @ l2} = \ccode{l2} = \ccode{l1 + l2}$
       \item Si $\code{l1} = \code{e::q}$, \hbox{supposons que} $\ccode{q @ l2} = \ccode{q} + \ccode{l2}$, pour toute suite l2 (\notion{hypothèse d'induction}) ()
    \end{itemize}
    \begin{align*}
        \ccode{l1 @ l2} & = \ccode{(e::q) @ l2} && \\
        & = \ccode{e::(q @ l2)} \quad &&\text{par définition de \code{@}} \\
        & = 1 + \ccode{q @ l2} &&\text{par définition de $\code{l} \mapsto \code{l}$} \\
        & = 1 + \ccode{q} + \ccode{l2}  &&\text{par hypothèse d'induction} \\
        & = \ccode{e::q} + \ccode{l2} && \\
        \ccode{l1 @ l2} & = \ccode{l1} + \ccode{l2} &&
    \end{align*}
    Ainsi, par induction structurelle, $\ccode{l1 @ l2} = \ccode{l1} + \ccode{l2}$, pour toutes listes \code{l1} et \code{l2}, et ce, indépendamment du choix de \code{l2}
\end{question}


\begin{question}{I.2) }{}
    Montrons par induction structurelle sur \code{l1} que pour toute liste \code{l1}, $\crev{l1} = \ccode{l1}$. \begin{itemize}
        \item Si $\code{l1} = []$, $\rev{l1} = \rev{[]} = \code{[]}  = \code{l1}$
        \item Si $\code{l1} = \code{e::q}$, on suppose que $\crev{q} = \ccode{q}$ \begin{align*}
            \crev{l1} &= \crev{e::q} &&\\
            &= \code{|reverse(q) @ [e]|} && \text{définition de \code{reverse}}\\
            &= \crev{q} + \ccode{[e]} && \\
            &= \ccode{q} + \ccode{[e]} && \text{hypothèse d'induction}\\
            &= \ccode{e::q} && \\
            \crev{l1} &= \ccode{l1} &&
        \end{align*}
    \end{itemize}
    Ainsi, par induction structurelle, pour toute liste \code{l1}, $\crev{l1} = \ccode{l1}$.
\end{question}

\begin{question}{I.3) }{}
    Montrons, par induction structurelle sur \code{l1}, que pour toute liste \code{l1}, $$\rev{l1 @ l2} = \code{\rev{l2} @ \rev{l1}}$$. \begin{itemize}
        \item Si $\code{l1} = \code{[]}$, alors $\rev{[] @ l2} = \rev{l2} = \code{\rev{l2} @ \rev{[]}}$        
        \item Si $\code{l1} = \code{e::q}$, supposons que $\rev{q @ l2} = \code{\rev{l2} @ \rev{q}}$. \begin{align*}
            \text{Alors, } \rev{l1 @ l2} &= \rev{e::q @ l2} && \\
            &= \code{\rev{q @ l2} @ [e]} && \text{par définition} \\
            &= \code{\rev{l2} @ \rev{q} @ [e]} && \text{hypothèse d'induction} \\
            &= \code{\rev{l2} @ \rev{e::q}} && \text{par définition} \\
            &= \code{\rev{l2} @ \rev{l1}}
        \end{align*}
    \end{itemize}
    Donc, par induction structurelle sur \code{l1}, $\forall \code{l1}, \rev{l1 @ l2} = \code{\rev{l2} @ \rev{l1}}$
\end{question}
\input{../stock/pied.tex}