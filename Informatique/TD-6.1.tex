\documentclass{article}
\usepackage{amsmath,amssymb}
\usepackage{xcolor}
\usepackage{enumitem}
\usepackage{multicol}
\usepackage{changepage}
\usepackage{stmaryrd}
\usepackage[framemethod=tikz]{mdframed}

 \usepackage{verbatim}

% physique
\definecolor{oranges}{RGB}{255, 242, 230}
\definecolor{rouges}{RGB}{255, 230, 230}
\definecolor{rose}{RGB}{255, 204, 204}

% maths - info
\definecolor{rouge_fonce}{RGB}{204, 0, 0}
\definecolor{bleu_fonce}{RGB}{0, 0, 255}
\definecolor{vert_fonce}{RGB}{0, 69, 33}

\definecolor{orange_foncee}{RGB}{255, 153, 0}
\definecolor{myrtille}{RGB}{225, 225, 255}
\definecolor{mayonnaise}{RGB}{255, 253, 233}
\definecolor{magenta}{RGB}{224, 209, 240}
\definecolor{pomme}{RGB}{204, 255, 204}
\definecolor{mauve}{RGB}{255, 230, 255}

% commandes
\newcommand{\notion}[1]{\textcolor{vert_fonce}{\textit{#1}}}
\newcommand{\mb}[1]{\mathbb{#1}}
\newcommand{\mc}[1]{\mathcal{#1}}
\newcommand{\cons}[1]{\cons{#1}}
\newcommand{\ov}[1]{\overline{#1}}
\newcommand{\abso}[1]{|#1|}

\newmdenv[
  nobreak=true,
  topline=true,
  bottomline=true,
  rightline=true,
  leftline=true,
  linewidth=0.5pt,
  linecolor=black,
  backgroundcolor=mayonnaise,
  innerleftmargin=10pt,
  innerrightmargin=10pt,
  innertopmargin=10pt,
  innerbottommargin=10pt,
  skipabove=\topsep,
  skipbelow=\topsep,
]{boite_question}


\newenvironment{question}[2]
{
    \begin{boite_question}
    \textbf{\textcolor{rouge_fonce}{Question #1}} \textit{#2} \\ \\
}
{
    \end{boite_question}
    \vspace{15pt}
}


\renewcommand{\labelitemi}{--}

\begin{document}
\begin{adjustwidth}{-3cm}{-3cm}

\begin{question}{I.1) }{Définition inductive de la concaténation en Ocaml}
    Pour toutes listes \cons{l} et \cons{q}, et tout élément \cons{e} :
    $\begin{cases}
        \cons{[]@l = l} \\
        \cons{(e::q@l = e::(q@l))}
    \end{cases}$

\end{question}

\begin{question}{I.2) }{Rédaction de démonstration par induction structurelle}
    Soit \cons{l2} une liste.
    Montrons par récurrence structurelle sur \cons{l1} que $|\cons{l1 @ l2}| = |\cons{l1} + \cons{l2}|$ pour toutes listes \cons{l1} et \cons{l2}.
   \begin{itemize}
       \item Si \cons{l1 = []}, $|\cons{l1 @ l2}| = |\cons{l2}| = |\cons{l1}| + |\cons{l2}|$
       \item Si \cons{l1 = e::q}, \boxed{\text{supposons que}} $|\cons{q @ l2}| = |\cons{q}| + |\cons{l2}|$, pour toute suite l2 (\notion{hypothèse d'induction}) 
    \end{itemize}
    % \begin{equation*}
    \begin{align*}
        |\cons{l1 @ l2}| & = |\cons{(e::q) @ l2}| && \\
        & = |\cons{e::(q @ l2)}| \quad &&\text{par définition de \cons{@}} \\
        & = 1 + |\cons{q @ l2}| &&\text{par définition de $\cons{l} \mapsto \cons{l}$} \\
        & = 1 + |\cons{q}| + |\cons{l2}|  &&\text{par hypothèse d'induction} \\
        & = |\cons{e::q}| + |\cons{l2}| && \\
        |\cons{l1 @ l2}| & = |\cons{l1}| + |\cons{l2}| &&
    \end{align*}
    % \end{equation*}
    Ainsi, par induction structurelle, $|\cons{l1 @ l2}| = |\cons{l1}| + |\cons{l2}|$, pour toutes listes \cons{l1} et \cons{l2}, et ce, indépendamment du choix de |\cons{l2}|
\end{question}



\end{adjustwidth}
\end{document}