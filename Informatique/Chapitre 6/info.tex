\documentclass{article}
\usepackage{amsmath,amssymb}
\usepackage{xcolor}
\usepackage{enumitem}
\usepackage{multicol}
\usepackage{changepage}

% physique
\definecolor{oranges}{RGB}{255, 242, 230}
\definecolor{rouges}{RGB}{255, 230, 230}
\definecolor{rose}{RGB}{255, 204, 204}

% maths
\definecolor{saumon}{RGB}{224, 209, 240}


\renewcommand{\labelitemi}{--}
\begin{document}
\begin{adjustwidth}{-3cm}{-3cm}

Définition 6.16
Soit $E$ un ensemble ordonné non vide et $e \in E$. On dit que :
    \begin{itemize}
        \item $p \in E$ est un prédécesseur immédiat de $e$ si $p<e$ et il n'existe pas d'élement $a\in e$ tel que $p<a<e$
        \item $s \in E$ est successeur  immédiat de $e$ si $e<s$ et il n'existe pas d'élement $a\in e$ tel que $e<a<s$
    \end{itemize}
Exemple 6.17
Dans $\mathbb{N}$ muni de l'ordre usuel :
\begin{itemize}
    \item $\forall n \in \mathbb{N}$, $n+1$ est le successeur immédiat de $n$.
    \item $\forall n \in \mathbb{N}^*$, $n+1$ est le prédécesseur immédiat de $n$.
    \item $0$ n'a pas de prédécesseur (en particulier immédiat).
\end{itemize}

Exemple 6.18
Dans $\mathbb{R}$ muni de l'ordre usuel, aucun élément $e$ n'a de prédécesseur (resp.successeur) immédiat puisque si $a<e$ alors en particulier $a<\frac{a+e}{2}<e$.

Définition 6.19
Soit $E$ un ensemble ordonné non vide et $e \in E$. On dit que \begin{itemize}
    \item $e$ est un élément minimal de $E$ s'il n'admet pas de prédécesseur.
    \item $e$ est un élément maximal de $E$ s'il n'admet pas de successeur.
\end{itemize}

Exemple 6.20
Soit $E$ un ensemble. L'ensemble $A = \mathcal{P}(E)\\ \{\varnothing\}$ des parties non vides de $E$ muni de l'inclusion et ordonné. Si $E \neq \varnothing$, $E$ est l'élément maximal de $A$, et $\forall e \in E$, $\{e\}$ est un élément maximal de $A$.

Remarque 6.21
L'ensemble précédent montre en particulier qu'un ensemble peut tout à fait avoir plusieurs éléments minimaux ou maximaux.

Définition 6.22
Soit $E$ un ensemble ordonné non vide et $e\in E$. On dit que :
\begin{itemize}
    \item $e$ est le \textbf{plus grand élément} de E si $\forall x \in E, x \leq e$.
    \item $e$ est le \textbf{plus petit élément} de E si $\forall x \in E, x \geq e$.
\end{itemize}

Démonstration : (preuve de l'unicité)
Supposons par l'absurde, qu'il n'y a pas unicité du plus petit élément. 
Soit $e$ et $e'$ deux plus petits éléments distincts de $E$. 
Alors, par définition, ($e$ est un plus petit élément $E$, $e \leq e')$. de même, $e' \leq e$. 
Par antisymétrie de $\leq$, $e=e'$. Absurde.
On montre de même l'unicité du plus grand élément, s'il existe.

Définition 6.23 : \textit{Ordre bien fondé}
Soit $(E,\leq)$ un ensemble ordonné.
On dit que $\leq$ est un ordre bien fondé si toute partie non vide de $E$ admet au moins un élément minimal.

Exemple 6.24 : \textit{Ordres bien fondés}
    \begin{itemize}
        \item l'ordre usuel sur l'ensemble $\mathbb{N}$ des entiers naturels est bien fondé.
        \item l'inclusion sur les parties d'un ensemble fini est bien fondée.
        \item la relation de divisibilité sur l'ensemble $\mathbb{N}^*$ est un ordre bien fondé.
    \end{itemize}

    Exemple 6.25 (Ordres non bien fondés)
    \begin{itemize}
        \item l'ordre usuel sur $\mathbb{Z}$ ou sur $\mathbb{R_+}$
        \item L'inclusion sur les parties d'un ensemble infini n'est pas bien fondée.
    \end{itemize}

    


\end{adjustwidth}
\end{document}