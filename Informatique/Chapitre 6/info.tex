\documentclass{article}
\usepackage{amsmath,amssymb}
\usepackage{xcolor}
\usepackage{enumitem}
\usepackage{multicol}
\usepackage{changepage}
\usepackage{stmaryrd}

% physique
\definecolor{oranges}{RGB}{255, 242, 230}
\definecolor{rouges}{RGB}{255, 230, 230}
\definecolor{rose}{RGB}{255, 204, 204}

% maths
\definecolor{saumon}{RGB}{224, 209, 240}


\renewcommand{\labelitemi}{--}
\begin{document}
\begin{adjustwidth}{-3cm}{-3cm}

Définition 6.16
Soit $E$ un ensemble ordonné non vide et $e \in E$. On dit que :
    \begin{itemize}
        \item $p \in E$ est un prédécesseur immédiat de $e$ si $p<e$ et il n'existe pas d'élement $a\in e$ tel que $p<a<e$
        \item $s \in E$ est successeur  immédiat de $e$ si $e<s$ et il n'existe pas d'élement $a\in e$ tel que $e<a<s$
    \end{itemize}
Exemple 6.17
Dans $\mathbb{N}$ muni de l'ordre usuel :
\begin{itemize}
    \item $\forall n \in \mathbb{N}$, $n+1$ est le successeur immédiat de $n$.
    \item $\forall n \in \mathbb{N}^*$, $n+1$ est le prédécesseur immédiat de $n$.
    \item $0$ n'a pas de prédécesseur (en particulier immédiat).
\end{itemize}

Exemple 6.18
Dans $\mathbb{R}$ muni de l'ordre usuel, aucun élément $e$ n'a de prédécesseur (resp.successeur) immédiat puisque si $a<e$ alors en particulier $a<\frac{a+e}{2}<e$.

Définition 6.19
Soit $E$ un ensemble ordonné non vide et $e \in E$. On dit que \begin{itemize}
    \item $e$ est un élément minimal de $E$ s'il n'admet pas de prédécesseur.
    \item $e$ est un élément maximal de $E$ s'il n'admet pas de successeur.
\end{itemize}

Exemple 6.20
Soit $E$ un ensemble. L'ensemble $A = \mathcal{P}(E)\\ \{\varnothing\}$ des parties non vides de $E$ muni de l'inclusion et ordonné. Si $E \neq \varnothing$, $E$ est l'élément maximal de $A$, et $\forall e \in E$, $\{e\}$ est un élément maximal de $A$.

Remarque 6.21
L'ensemble précédent montre en particulier qu'un ensemble peut tout à fait avoir plusieurs éléments minimaux ou maximaux.

Définition 6.22
Soit $E$ un ensemble ordonné non vide et $e\in E$. On dit que :
\begin{itemize}
    \item $e$ est le \textbf{plus grand élément} de E si $\forall x \in E, x \leq e$.
    \item $e$ est le \textbf{plus petit élément} de E si $\forall x \in E, x \geq e$.
\end{itemize}

Démonstration : (preuve de l'unicité)
Supposons par l'absurde, qu'il n'y a pas unicité du plus petit élément. 
Soit $e$ et $e'$ deux plus petits éléments distincts de $E$. 
Alors, par définition, ($e$ est un plus petit élément $E$, $e \leq e')$. de même, $e' \leq e$. 
Par antisymétrie de $\leq$, $e=e'$. Absurde.
On montre de même l'unicité du plus grand élément, s'il existe.

Définition 6.23 : \textit{Ordre bien fondé}
Soit $(E,\leq)$ un ensemble ordonné.
On dit que $\leq$ est un ordre bien fondé si toute partie non vide de $E$ admet au moins un élément minimal.

Exemple 6.24 : \textit{Ordres bien fondés}
    \begin{itemize}
        \item l'ordre usuel sur l'ensemble $\mathbb{N}$ des entiers naturels est bien fondé.
        \item l'inclusion sur les parties d'un ensemble fini est bien fondée.
        \item la relation de divisibilité sur l'ensemble $\mathbb{N}^*$ est un ordre bien fondé.
    \end{itemize}

    Exemple 6.25 \textit{Ordres non bien fondés}
    \begin{itemize}
        \item l'ordre usuel sur $\mathbb{Z}$ ou sur $\mathbb{R_+}$
        \item L'inclusion sur les parties d'un ensemble infini n'est pas bien fondée.
    \end{itemize}

    Propriété 6.26
    Soit $(A,\leq_A) et (B,\leq_B)$ deux ensembles ordonnés.
    Si $\leq_A$ et $\leq_B$ sont bien fondées, l'ordre lexicographique défini sur $A \times B$ est bien fondé.

    Démonstration : 
    Soit $X$ une partie non vide de $A \times B$. Montrons qu'elle admet un élément minimal.
    On note $A_X = \{a \in A, \exists b \in B, (a,b) \in X\}$. $X$ est non vide, donc $A_X$ l'est également.
    De plus, comme $\leq_A$ est bien fondé, $A_X$ admet un élément minimal.
    Soit donc $a_0 \in A$ un élément minimal de $A_X$. On considère alors l'ensemble $B_0 = \{b \in B, (a_0,b) \in X\}$.
    Par définition de $a_0$, $B_0$ est non vide, alors, $\leq_B$ étant aussi bien fondé, $B_0$ admet un élément minimal $b_0$. l'élémet $x_0 = (a_0, b_0)$ est alors un élément minimal de $X$.
    En effet, Soit $(a,b) \in X$ tel que $(a,b) \leq (a_0,b_0)$ i.e. tel que $a < a_0 ou (a = a_0 et b \leq_B b_0)$.
    $a \in A_X$ donc, par minimalité de $a_0$, $a \nless a_0$, on a donc $a = a_0$ et $b \leq_B b_0$.
    De même, $b \in B_0$ puisque $(a,b) = (a_0,b)$. Par minimalité de $b_0$ dans $B_0$, on a donc $b = b_0$ d'où $(a,b) = (a_0,b_0)$.

    Propriété 6.27
    Soit $((E_i,\leq_i))_\{i \in \llbracket 1,n \rrbracket \}$ une famille finie d'ensembles munis d'ordres bien fondés. ($n \geq 2$). L'ordre produit défini sur $\displaystyle \prod_{i=1}^{n}E_i = E_1 \times \ldots \times E_n$ est bien fondé.

    Démonstration Soit $A$ une partie non vide de $E_1 \times \ldots \times E_n$.
    On pose $A_1 = \{a_1 \in E_1, \exists (x_1,\ldots,x_n) \in A^n, x_1 = a_1\}$. Comme $A$ est non vide, $A_1$ est une partie non vide de $E_1$ qui admet donc un élément minimal $m_1$.
    On pose $A_2 = \{a_2 \in E_2, \exists (x_1,\ldots,x_n) \in A^n, x_2 = a_2 \text{ et } x_1 = m_1\}$. Comme $A$ est non vide, $A_2$ est une partie non vide de $E_2$ qui admet donc un élément minimal $m_2$. 

    On construit ainsi $n$ ensembles non vides définis pour tout $i \in \mathbb{N}$ par : 
    $$\begin{cases}
        A_{i+1} = \{a_{i+1} \in E_{i+1}, \exists (x_1, \ldots, x_n) \in A^n, \forall j \in \llbracket 1, i \rrbracket, x_j = m_j \text{ et } x_{i+1} = a_{i+1}\} \\
        m_i \text{ est un élément minimal de } A_i, \forall i \in \llbracket 1,n \rrbracket 
    \end{cases}$$
    L'élément $m = (m_1, \ldots, m_n)$ est alors, par construction, un élément minimal de $A$.

    Remarque 6.28
    Si $E$ est muni d'un ordre total et bien fondé, alors toute partie non vide de $E$ admet un plus petit élément. On parle alors de bon ordre et d'ensemble bien ordonné.

    Définition 6.29
    Soit $E$ un ensemble. On appelle prédicat sur $E$ toute propriété $P$ dépendant d'éléments de $E$.
    Lorsque $P$ dépend de $n$ paramètres, on dit que $P$ est d'arité $n$. On note alors $\forall (x_1, \ldots, x_n) \in E^n$. \begin{itemize}
        \item $P(x_1,...,x_n)$ lorsque la propriété est vraie.
        \item $\lnot P(x_1,...,x_n)$ lorsque la propriété est fausse.
    \end{itemize}

    Remarque 6.30
    Une relation bianire est en fait un prédicat d'arité 2.

    Théorème 6.31
    Soit $(E,\leq)$ un ensemble ordonné. Les propositions suivantes sont équivalentes : \begin{enumerate}
        \item $\leq$ est un ordre bien fondé.
        \item Il n'existe pas de suite infinie strictement décroissante d'élements de $E$.
        \item Pour tout prédicat $P$ sur $E$, si: $$\forall (x,y) \in E^2, x > y \implies P(x)$$
    \end{enumerate}


\end{adjustwidth}
\end{document}