% commandes
\newcommand{\notion}[1]{\textcolor{vert_fonce}{\textit{#1}}}
\newcommand{\mb}[1]{\mathbb{#1}}
\newcommand{\mc}[1]{\mathcal{#1}}
\newcommand{\mr}[1]{\mathrm{#1}}
\newcommand{\code}[1]{\texttt{#1}}
\newcommand{\ccode}[1]{\texttt{|#1|}}
\newcommand{\ov}[1]{\overline{#1}}
\newcommand{\abs}[1]{|#1|}
\newcommand{\rev}[1]{\texttt{reverse(#1)}}
\newcommand{\crev}[1]{\texttt{|reverse(#1)|}}

\newcommand{\ie}{\textit{i.e.} }

\newcommand{\N}{\mathbb{N}}
\newcommand{\R}{\mathbb{R}}
\newcommand{\C}{\mathbb{C}}
\newcommand{\K}{\mathbb{K}}
\newcommand{\Z}{\mathbb{Z}}

\newcommand{\A}{\mathcal{A}}
\newcommand{\bigO}{\mathcal{O}}
\renewcommand{\L}{\mathcal{L}}

\newcommand{\rg}[0]{\mathrm{rg}}
\newcommand{\re}[0]{\mathrm{Re}}
\newcommand{\im}[0]{\mathrm{Im}}
\newcommand{\cl}[0]{\mathrm{cl}}
\newcommand{\grad}[0]{\vec{\mathrm{grad}}}
\renewcommand{\div}[0]{\mathrm{div}\,}
\newcommand{\rot}[0]{\vec{\mathrm{rot}}\,}
\newcommand{\vnabla}[0]{\vec{\nabla}}
\renewcommand{\vec}[1]{\overrightarrow{#1}}
\newcommand{\mat}[1]{\mathrm{Mat}_{#1}}
\newcommand{\matrice}[1]{\mathcal{M}_{#1}}
\newcommand{\sgEngendre}[1]{\left\langle #1 \right\rangle}
\newcommand{\gpquotient}[1]{\mathbb{Z} / #1\mathbb{Z}}
\newcommand{\norme}[1]{||#1||}
\renewcommand{\d}[1]{\,\mathrm{d}#1}
\newcommand{\adh}[1]{\overline{#1}}
\newcommand{\intint}[2]{\llbracket #1 ,\, #2 \rrbracket}
\newcommand{\seg}[2]{[#1\, ; \, #2]}
\newcommand{\scal}[2]{( #1 | #2 )}
\newcommand{\distance}[2]{\mathrm{d}(#1,\,#2)}
\newcommand{\inte}[2]{\int_{#1}^{#2}}
\newcommand{\somme}[2]{\sum_{#1}^{#2}}
\newcommand{\deriveref}[4]{\biggl( \frac{\text{d}^{#1}#2}{\text{d}#3^{#1}} \biggr)_{#4}}





\documentclass{article}
\usepackage{amsmath,amssymb,mathtools}
\usepackage{xcolor}
\usepackage{enumitem}
\usepackage{multicol}
\usepackage{changepage}
\usepackage{stmaryrd}
\usepackage{graphicx}
\usepackage[framemethod=tikz]{mdframed}
\usepackage{tikz,pgfplots}
\pgfplotsset{compat=1.18}

% physique
\renewcommand*{\overrightarrow}[1]{\vbox{\halign{##\cr 
  \tiny\rightarrowfill\cr\noalign{\nointerlineskip\vskip1pt} 
  $#1\mskip2mu$\cr}}}
  
\definecolor{oranges}{RGB}{255, 242, 230}
\definecolor{rouges}{RGB}{255, 230, 230}
\definecolor{rose}{RGB}{255, 204, 204}

% maths - info
\definecolor{rouge_fonce}{RGB}{204, 0, 0}
\definecolor{rouge}{RGB}{255, 0, 0}
\definecolor{bleu_fonce}{RGB}{0, 0, 255}
\definecolor{vert_fonce}{RGB}{0, 69, 33}
\definecolor{vert}{RGB}{0,255,0}

\definecolor{orange_foncee}{RGB}{255, 153, 0}
\definecolor{myrtille}{RGB}{225, 225, 255}
\definecolor{mayonnaise}{RGB}{255, 253, 233}
\definecolor{magenta}{RGB}{224, 209, 240}
\definecolor{pomme}{RGB}{204, 255, 204}
\definecolor{mauve}{RGB}{255, 230, 255}


% Cours

\newmdenv[
  nobreak=true,
  topline=true,
  bottomline=true,
  rightline=true,
  leftline=true,
  linewidth=0.5pt,
  linecolor=black,
  backgroundcolor=mayonnaise,
  innerleftmargin=10pt,
  innerrightmargin=10pt,
  innertopmargin=5pt,
  innerbottommargin=5pt,
  skipabove=\topsep,
  skipbelow=\topsep,
]{boite_definition}

\newmdenv[
  nobreak=true,
  topline=true,
  bottomline=true,
  rightline=true,
  leftline=true,
  linewidth=0.5pt,
  linecolor=white,
  backgroundcolor=white,
  innerleftmargin=10pt,
  innerrightmargin=10pt,
  innertopmargin=5pt,
  innerbottommargin=5pt,
  skipabove=\topsep,
  skipbelow=\topsep,
]{boite_exemple}

\newmdenv[
  nobreak=true,
  topline=true,
  bottomline=true,
  rightline=true,
  leftline=true,
  linewidth=0.5pt,
  linecolor=black,
  backgroundcolor=magenta,
  innerleftmargin=10pt,
  innerrightmargin=10pt,
  innertopmargin=5pt,
  innerbottommargin=5pt,
  skipabove=\topsep,
  skipbelow=\topsep,
]{boite_proposition}

\newmdenv[
  nobreak=true,
  topline=true,
  bottomline=true,
  rightline=true,
  leftline=true,
  linewidth=0.5pt,
  linecolor=black,
  backgroundcolor=white,
  innerleftmargin=10pt,
  innerrightmargin=10pt,
  innertopmargin=5pt,
  innerbottommargin=5pt,
  skipabove=\topsep,
  skipbelow=\topsep,
]{boite_demonstration}

\newmdenv[
  nobreak=true,
  topline=true,
  bottomline=true,
  rightline=true,
  leftline=true,
  linewidth=0.5pt,
  linecolor=white,
  backgroundcolor=white,
  innerleftmargin=10pt,
  innerrightmargin=10pt,
  innertopmargin=5pt,
  innerbottommargin=5pt,
  skipabove=\topsep,
  skipbelow=\topsep,
]{boite_remarque}


\newenvironment{definition}[2]
{
    \vspace{15pt}
    \begin{boite_definition}
    \textbf{\textcolor{rouge}{Définition #1}}
    \if\relax\detokenize{#2}\relax
    \else
        \textit{ - #2}
    \fi \\ \\
}
{
    \end{boite_definition}
    
}

\newenvironment{exemple}[2]
{
    \vspace{15pt}
    \begin{boite_exemple}
    \textbf{\textcolor{bleu_fonce}{Exemple #1}}
    \if\relax\detokenize{#2}\relax
    \else
        \textit{ - #2}
    \fi \\ \\ 
}
{   
    \end{boite_exemple}
    
}

\newenvironment{proposition}[2]
{
    \vspace{15pt}
    \begin{boite_proposition}
    \textbf{\textcolor{rouge}{Proposition #1}}
    \if\relax\detokenize{#2}\relax
    \else
        \textit{ - #2}
    \fi \\ \\
}
{
    \end{boite_proposition}
    
}

\newenvironment{theoreme}[2]
{
    \vspace{15pt}
    \begin{boite_proposition}
    \textbf{\textcolor{rouge}{Théorème #1}} 
    \if\relax\detokenize{#2}\relax
    \else
        \textit{ - #2}
    \fi \\ \\
}
{
    \end{boite_proposition}
    
}

\newenvironment{demonstration}
{
    \vspace{15pt}
    \begin{boite_demonstration}
    \textbf{\textcolor{rouge}{Démonstration}}\\ \\
}
{
    \end{boite_demonstration}
    
}

\newenvironment{remarque}[2]
{
    \vspace{15pt}
    \begin{boite_remarque}
    \textbf{\textcolor{bleu_fonce}{Remarque #1}}
    \if\relax\detokenize{#2}\relax
    \else
        \textit{ - #2}
    \fi \\ \\   
}
{  
    \end{boite_remarque}
    
}



%Corrections
\newmdenv[
  nobreak=true,
  topline=true,
  bottomline=true,
  rightline=true,
  leftline=true,
  linewidth=0.5pt,
  linecolor=black,
  backgroundcolor=mayonnaise,
  innerleftmargin=10pt,
  innerrightmargin=10pt,
  innertopmargin=5pt,
  innerbottommargin=5pt,
  skipabove=\topsep,
  skipbelow=\topsep,
]{boite_question}


\newenvironment{question}[2]
{
    \vspace{15pt}
    \begin{boite_question}
    \textbf{\textcolor{rouge}{Question #1}}
    \if\relax\detokenize{#2}\relax
    \else
        \textit{ - #2}
    \fi \\ \\
}
{
    \end{boite_question}
    
}

% Cours

\newmdenv[
  nobreak=false,
  topline=true,
  bottomline=true,
  rightline=true,
  leftline=true,
  linewidth=0.5pt,
  linecolor=black,
  backgroundcolor=mayonnaise,
  innerleftmargin=10pt,
  innerrightmargin=10pt,
  innertopmargin=5pt,
  innerbottommargin=5pt,
  skipabove=\topsep,
  skipbelow=\topsep,
]{boite_definition_break}

\newmdenv[
  nobreak=false,
  topline=true,
  bottomline=true,
  rightline=true,
  leftline=true,
  linewidth=0.5pt,
  linecolor=white,
  backgroundcolor=white,
  innerleftmargin=10pt,
  innerrightmargin=10pt,
  innertopmargin=5pt,
  innerbottommargin=5pt,
  skipabove=\topsep,
  skipbelow=\topsep,
]{boite_exemple_break}

\newmdenv[
  nobreak=false,
  topline=true,
  bottomline=true,
  rightline=true,
  leftline=true,
  linewidth=0.5pt,
  linecolor=black,
  backgroundcolor=magenta,
  innerleftmargin=10pt,
  innerrightmargin=10pt,
  innertopmargin=5pt,
  innerbottommargin=5pt,
  skipabove=\topsep,
  skipbelow=\topsep,
]{boite_proposition_break}

\newmdenv[
  nobreak=false,
  topline=true,
  bottomline=true,
  rightline=true,
  leftline=true,
  linewidth=0.5pt,
  linecolor=black,
  backgroundcolor=white,
  innerleftmargin=10pt,
  innerrightmargin=10pt,
  innertopmargin=5pt,
  innerbottommargin=5pt,
  skipabove=\topsep,
  skipbelow=\topsep,
]{boite_demonstration_break}

\newmdenv[
  nobreak=false,
  topline=true,
  bottomline=true,
  rightline=true,
  leftline=true,
  linewidth=0.5pt,
  linecolor=white,
  backgroundcolor=white,
  innerleftmargin=10pt,
  innerrightmargin=10pt,
  innertopmargin=5pt,
  innerbottommargin=5pt,
  skipabove=\topsep,
  skipbelow=\topsep,
]{boite_remarque_break}


\newenvironment{definition_break}[2]
{
    \vspace{15pt}
    \begin{boite_definition_break}
    \textbf{\textcolor{rouge}{Définition #1}}
    \if\relax\detokenize{#2}\relax
    \else
        \textit{ - #2}
    \fi \\ \\
}
{
    \end{boite_definition_break}
    
}

\newenvironment{exemple_break}[2]
{
    \vspace{15pt}
    \begin{boite_exemple_break}
    \textbf{\textcolor{bleu_fonce}{Exemple #1}}
    \if\relax\detokenize{#2}\relax
    \else
        \textit{ - #2}
    \fi \\ \\ 
}
{   
    \end{boite_exemple_break}
    
}

\newenvironment{proposition_break}[2]
{
    \vspace{15pt}
    \begin{boite_proposition_break}
    \textbf{\textcolor{rouge}{Proposition #1}}
    \if\relax\detokenize{#2}\relax
    \else
        \textit{ - #2}
    \fi \\ \\
}
{
    \end{boite_proposition_break}
    
}

\newenvironment{theoreme_break}[2]
{
    \vspace{15pt}
    \begin{boite_proposition_break}
    \textbf{\textcolor{rouge}{Théorème #1}} 
    \if\relax\detokenize{#2}\relax
    \else
        \textit{ - #2}
    \fi \\ \\
}
{
    \end{boite_proposition_break}
    
}

\newenvironment{demonstration_break}
{
    \vspace{15pt}
    \begin{boite_demonstration_break}
    \textbf{\textcolor{rouge}{Démonstration}}\\ \\
}
{
    \end{boite_demonstration_break}
    
}

\newenvironment{remarque_break}[2]
{
    \vspace{15pt}
    \begin{boite_remarque}
    \textbf{\textcolor{bleu_fonce}{Remarque #1}}
    \if\relax\detokenize{#2}\relax
    \else
        \textit{ - #2}
    \fi \\ \\   
}
{  
    \end{boite_remarque}
    
}



%Corrections
\newmdenv[
  nobreak=false,
  topline=true,
  bottomline=true,
  rightline=true,
  leftline=true,
  linewidth=0.5pt,
  linecolor=black,
  backgroundcolor=mayonnaise,
  innerleftmargin=10pt,
  innerrightmargin=10pt,
  innertopmargin=5pt,
  innerbottommargin=5pt,
  skipabove=\topsep,
  skipbelow=\topsep,
]{boite_question_break}


\newenvironment{question_break}[2]
{
    \vspace{15pt}
    \begin{boite_question_break}
    \textbf{\textcolor{rouge}{Question #1}}
    \if\relax\detokenize{#2}\relax
    \else
        \textit{ - #2}
    \fi \\ \\
}
{
    \end{boite_question_break}
    
}

\newenvironment{enumeratebf}{
    \begin{enumerate}[label=\textbf{\arabic*.}]
}
{
    \end{enumerate}
}

\begin{document}
\begin{adjustwidth}{-3cm}{-3cm}

\begin{theoreme}{6.31}{caractérisation de la bonne fondation d'un ordre}
Soit $(E,\leq)$ un ensemble ordonné. Les propositions suivantes sont équivalentes : \begin{enumeratebf}
    \item $\leq$ est un ordre bien fondé.
    \item Il n'existe pas de suite infinie strictement décroissante d'élements de $E$.
    \item Pour tout prédicat $P$ sur $E$, $$\biggl(\forall x \in E, \Bigl(\forall y \in E, x>y \implies P(y)\Bigr) \implies P(x)\biggr) \implies \biggl(\forall x \in E, P(x)\biggr) $$
\end{enumeratebf}
\end{theoreme}


\begin{demonstration}
    (1) $\implies$ (2) : $\quad$ Supposons que $\leq$ est un ordre bien fondé, et par l'absurde, que $(x_n)_{n \in \mathbb{N}}$ est une suite infinie strictement décroissante d'éléments de $E$. Alors l'ensemble non vide $\{x_n, n\in \mathbb{N}\} \subset E$ admet un élément minimal $x_k$; Ainsi, par décroissance stricte de $(x_n)$, $x_{k+1} < x_k$, ce qui contredit la minimalité de $x_k$.
\end{demonstration}

\begin{demonstration}
    (2) $\implies$ (3) : $\quad$ Soit $P$ un prédicat sur $E$. On suppose par l'absurde : \begin{align*}
        &\text{(2) et } \lnot \text{(3)} \\
        \text{donc} \quad& \text{(2) et }  \biggl(\forall x \in E, \Bigl(\forall y \in E, x>y \implies P(y)\Bigr) \implies P(x)\biggr) \text{ et } \biggl(\exists x \in E, \lnot P(x)\biggr) \\
        \text{donc} \quad& \text{(2) et }  \biggl(\forall x \in E,   \lnot P(x) \implies \Bigl(\exists y \in E, x>y \text{ et } \lnot P(y)\Bigr) \biggr) \text{ et } \biggl(\exists x \in E, \lnot P(x)\biggr)  \\
    \end{align*}
    Soit $x \in E$ tel que $\lnot P(x)$. On choisit $y \in E$ tel que $x>y$ et $\lnot P(y)$. De même, on peut encore choisir $z \in E$ tel que $y>z$ et $\lnot P(z)$. D'après le principe de récurrence, on construit ainsi une suite strictement décroissante $(x_i) \in E^\N$, telle que pour tout $i \in \N, \lnot P(x_i)$. Ce qui contredit la propriété (2).
\end{demonstration}

\begin{demonstration}
    \item (3) $\implies$ (1) : Remarquons d'abord que : 
    \begin{align*}
        &\text{(3)} \implies \text{(1)} \\
        \Leftrightarrow  \quad &\Biggl(\Biggl[\biggl(\forall x \in E, \Bigl(\forall y \in E, x>y \implies P(y)\Bigr) \implies P(x)\biggr) \implies \biggl(\forall x \in E, P(x)\biggr) \Biggr] \implies \text{(1)} \Biggr)\\
        \Leftrightarrow \quad &\Biggl( \Biggl[\biggl(\exists x \in E, \lnot P(x)\biggr) \implies \biggl(\exists x \in E, \Bigl(\forall y \in E, x>y \implies P(y)\Bigr) \text{ et } \lnot P(x)\biggr)\Biggr] \implies \text{(1)}\Biggr)
    \end{align*}
    Soit $A \subset E$, tel que $A \neq \varnothing$. On note $P$ le prédicat de $E$ tel que $A \neq \varnothing$. Pour tout $x \in E$ on note $P(x) : \,  "x \notin A"$. Nécessairement, il existe $x \in E$ tel que \begin{align*}
        &x \in A \\
        \text{donc} \quad &\lnot P(x)
    \end{align*}
    On choisit alors $m \in A$ tel que : \begin{align*}
        &\lnot P(m) \text{ et } \bigl( \forall y \in E, y<m \implies P(y) \bigr)\\
        \text{alors} \quad &m \in A \text{ et } \bigl( \forall y \in E, y<m \implies y \notin A \bigr)
    \end{align*}
    $m$ est alors l'élément minimal de $A$. donc $\leq$ est un ordre bien fondé.
\end{demonstration}

\begin{theoreme}{6.48}{preuve par induction structurelle}
    Soit $X \subset E$ un ensemble défini de manière inductive par un ensemble de base $B$ et un ensemble de constructeurs $\mathcal{K}$. Soit $\mathcal{P}$ un prédicat sur $E$. \\ si on a : \begin{enumeratebf}
        \item $\forall b \in B, \mathcal{P}(b)$
        \item $\forall f : E^p \rightarrow E \in \mathcal{K}, \, \forall x_1,\ldots, x_p \in X, \quad \Bigl(\forall i \in \llbracket 1,p \rrbracket, \mathcal{P}(x_i)\Bigr) \implies \Bigl(\mathcal{P}\bigl(f(x_1,\ldots,x_p)\bigr)\Bigr)$
    \end{enumeratebf}
    Alors, $$\forall x \in X, \mathcal{P}(X)$$
\end{theoreme}

\begin{demonstration}
    On définit la suite $(Y_n)$ par $\begin{cases}
        Y_0 =  B \\
        \forall n \in \mathbb{N}, Y_{n+1} = Y_n \cup \mathcal{K}(Y_n)
    \end{cases}$ \\ \\
    alors $X = \bigcup_{n \in \mathbb{N}}Y_n$. 
    \begin{definition}{temporaire}{hauteur d'un élément d'ensemble inductif défini récursivement}
        Pour tout $x \in X$ on définit alors la \notion{hauteur} de $x$ par $h(x) = \min \{n \in \mathbb{N}, x \in Y_n\}$.
    \end{definition}
    On suppose que : 
    \begin{enumeratebf}
        \item $\forall b \in B, \mathcal{P}(b)$
        \item $\forall f : E^p \rightarrow E \in \mathcal{K}, \, \forall x_1,\ldots, x_p \in X, \quad \Bigl(\forall i \in \llbracket 1,p \rrbracket, \mathcal{P}(x_i)\Bigr) \implies \Bigl(\mathcal{P}\bigl(f(x_1,\ldots,x_p)\bigr)\Bigr)$
    \end{enumeratebf}
    Montrons par récurrence forte sur la hauteur $n$ des éléments de $X$ que $\forall x \in X, \mathcal{P}(X)$.
    \begin{itemize}
        \item Si $n = 0$ : \\Soit $x \in X$ tel que $h(x) = 0$, i.e. tel que $x \in Y_0 = B$. Alors d'après la propriété (1), $\mathcal{P}(x)$
        \item Soit $n \in \mathbb{N}$ fixé. On suppose que $$\forall y \in X, h(y) \leq n \implies \mc{P}(y)$$ Soit $x \in X$ tel que $h(x) = n+1$. Montrons que $\mathcal{P}(x)$ est encore vraie.\\Par définition de la hauteur, $x \in Y_{n+1} = Y_{n} \cup \mathcal{K}(Y_n)$. De plus, $x \notin Y_n$ ($h(x) = n+1$).\\Alors, $x \in \mathcal{K}(Y_n)$.\\Soit donc $f : E^p \rightarrow E \in \mathcal{K}$ et $y_1, \ldots, y_p \in Y_n$ tels que $x = f(y_1,\ldots y_p)$.\\Donc $P(y_i)$ est vraie par hypothèse de récurrence ($\forall i \in \llbracket 1,p \rrbracket, h(y_i) \leq n$).\\D'après la proposition (2), on a donc $\mathcal{P}(x)$. 
        \item Ainsi, par récurrence forte sur $n$, $$\forall n \in \mathbb{N}, \forall x \in X \text{ de hauteur } n, \mathcal{P}(x)$$
    \end{itemize}
\end{demonstration}

\begin{demonstration}
    Les fonctions \code{echange} et \code{pere} ont une complexité constante.\\
    De plus, la fonction \code{percolation\_haut} parcourt chaque niveau de profondeur du tas dans le pire des cas, chaque itération se faisant en temps constant. Sa complexité est donc $\bigO(\log(n))$ (hauteur = $\lfloor \log_2(n) \rfloor$ car arbre quasi-complet).\\
    La fonction \code{insere}, exécutant un nombre borné d'opérations élémentaires et faisant un appel à \code{percolation\_haut}, a donc une complexité en $\bigO(\log(n))$.
\end{demonstration}

\begin{demonstration}
    Les fonctions \code{echange}, et \code{fils\_g} et \code{fils\_d} ont une complexité constante.\\
    De plus, la fonction \code{insere} exécute un nombre borné d'opérations élémentaires en plus de parcourir chaque niveau de profondeur du tas dans le pire des cas, chaque itération se faisant en temps constant.\\
    Sa complexité est donc en $\bigO(\log(n))$ (hauteur = $\lfloor \log_2(n) \rfloor$ car arbre quasi-complet).
\end{demonstration}
\end{adjustwidth}
\end{document}