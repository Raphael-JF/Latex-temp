% commandes
\newcommand{\notion}[1]{\textcolor{vert_fonce}{\textit{#1}}}
\newcommand{\mb}[1]{\mathbb{#1}}
\newcommand{\mc}[1]{\mathcal{#1}}
\newcommand{\mr}[1]{\mathrm{#1}}
\newcommand{\code}[1]{\texttt{#1}}
\newcommand{\ccode}[1]{\texttt{|#1|}}
\newcommand{\ov}[1]{\overline{#1}}
\newcommand{\abs}[1]{|#1|}
\newcommand{\rev}[1]{\texttt{reverse(#1)}}
\newcommand{\crev}[1]{\texttt{|reverse(#1)|}}

\newcommand{\ie}{\textit{i.e.} }

\newcommand{\N}{\mathbb{N}}
\newcommand{\R}{\mathbb{R}}
\newcommand{\C}{\mathbb{C}}
\newcommand{\K}{\mathbb{K}}
\newcommand{\Z}{\mathbb{Z}}

\newcommand{\A}{\mathcal{A}}
\newcommand{\bigO}{\mathcal{O}}
\renewcommand{\L}{\mathcal{L}}

\newcommand{\rg}[0]{\mathrm{rg}}
\newcommand{\re}[0]{\mathrm{Re}}
\newcommand{\im}[0]{\mathrm{Im}}
\newcommand{\cl}[0]{\mathrm{cl}}
\newcommand{\grad}[0]{\vec{\mathrm{grad}}}
\renewcommand{\div}[0]{\mathrm{div}\,}
\newcommand{\rot}[0]{\vec{\mathrm{rot}}\,}
\newcommand{\vnabla}[0]{\vec{\nabla}}
\renewcommand{\vec}[1]{\overrightarrow{#1}}
\newcommand{\mat}[1]{\mathrm{Mat}_{#1}}
\newcommand{\matrice}[1]{\mathcal{M}_{#1}}
\newcommand{\sgEngendre}[1]{\left\langle #1 \right\rangle}
\newcommand{\gpquotient}[1]{\mathbb{Z} / #1\mathbb{Z}}
\newcommand{\norme}[1]{||#1||}
\renewcommand{\d}[1]{\,\mathrm{d}#1}
\newcommand{\adh}[1]{\overline{#1}}
\newcommand{\intint}[2]{\llbracket #1 ,\, #2 \rrbracket}
\newcommand{\seg}[2]{[#1\, ; \, #2]}
\newcommand{\scal}[2]{( #1 | #2 )}
\newcommand{\distance}[2]{\mathrm{d}(#1,\,#2)}
\newcommand{\inte}[2]{\int_{#1}^{#2}}
\newcommand{\somme}[2]{\sum_{#1}^{#2}}
\newcommand{\deriveref}[4]{\biggl( \frac{\text{d}^{#1}#2}{\text{d}#3^{#1}} \biggr)_{#4}}





\documentclass{article}
\usepackage{amsmath,amssymb,mathtools}
\usepackage{xcolor}
\usepackage{enumitem}
\usepackage{multicol}
\usepackage{changepage}
\usepackage{stmaryrd}
\usepackage{graphicx}
\usepackage[framemethod=tikz]{mdframed}
\usepackage{tikz,pgfplots}
\pgfplotsset{compat=1.18}

% physique
\definecolor{oranges}{RGB}{255, 242, 230}
\definecolor{rouges}{RGB}{255, 230, 230}
\definecolor{rose}{RGB}{255, 204, 204}

% maths - info
\definecolor{rouge_fonce}{RGB}{204, 0, 0}
\definecolor{bleu_fonce}{RGB}{0, 0, 255}
\definecolor{vert_fonce}{RGB}{0, 69, 33}

\definecolor{orange_foncee}{RGB}{255, 153, 0}
\definecolor{myrtille}{RGB}{225, 225, 255}
\definecolor{mayonnaise}{RGB}{255, 253, 233}
\definecolor{magenta}{RGB}{224, 209, 240}
\definecolor{pomme}{RGB}{204, 255, 204}
\definecolor{mauve}{RGB}{255, 230, 255}


% Cours

\newmdenv[
  nobreak=true,
  topline=true,
  bottomline=true,
  rightline=true,
  leftline=true,
  linewidth=0.5pt,
  linecolor=black,
  backgroundcolor=mayonnaise,
  innerleftmargin=10pt,
  innerrightmargin=10pt,
  innertopmargin=10pt,
  innerbottommargin=10pt,
  skipabove=\topsep,
  skipbelow=\topsep,
]{boite_definition}

\newmdenv[
  nobreak=true,
  topline=true,
  bottomline=true,
  rightline=true,
  leftline=true,
  linewidth=0.5pt,
  linecolor=white,
  backgroundcolor=white,
  innerleftmargin=10pt,
  innerrightmargin=10pt,
  innertopmargin=10pt,
  innerbottommargin=10pt,
  skipabove=\topsep,
  skipbelow=\topsep,
]{boite_exemple}

\newmdenv[
  nobreak=true,
  topline=true,
  bottomline=true,
  rightline=true,
  leftline=true,
  linewidth=0.5pt,
  linecolor=black,
  backgroundcolor=magenta,
  innerleftmargin=10pt,
  innerrightmargin=10pt,
  innertopmargin=10pt,
  innerbottommargin=10pt,
  skipabove=\topsep,
  skipbelow=\topsep,
]{boite_proposition}

\newmdenv[
  nobreak=true,
  topline=true,
  bottomline=true,
  rightline=true,
  leftline=true,
  linewidth=0.5pt,
  linecolor=black,
  backgroundcolor=white,
  innerleftmargin=10pt,
  innerrightmargin=10pt,
  innertopmargin=10pt,
  innerbottommargin=10pt,
  skipabove=\topsep,
  skipbelow=\topsep,
]{boite_demonstration}

\newmdenv[
  nobreak=true,
  topline=true,
  bottomline=true,
  rightline=true,
  leftline=true,
  linewidth=0.5pt,
  linecolor=white,
  backgroundcolor=white,
  innerleftmargin=10pt,
  innerrightmargin=10pt,
  innertopmargin=10pt,
  innerbottommargin=10pt,
  skipabove=\topsep,
  skipbelow=\topsep,
]{boite_remarque}


\newenvironment{definition}[2]
{
    \begin{boite_definition}
    \textbf{\textcolor{rouge_fonce}{Définition #1}} \textit{#2} \\ \\
}
{
    \end{boite_definition}
    \vspace{15pt}
}

\newenvironment{exemple}[2]
{
    \begin{boite_exemple}
    \textbf{\textcolor{bleu_fonce}{Exemple #1}} \textit{#2} \\
    \begin{itemize}[label=$\blacktriangleright \quad $]                    
}
{   
    \end{itemize}
    \end{boite_exemple}
    \vspace{15pt}
}

\newenvironment{proposition}[2]
{
    \begin{boite_proposition}
    \textbf{\textcolor{rouge_fonce}{Proposition #1}} \textit{#2} \\ \\
}
{
    \end{boite_proposition}
    \vspace{15pt}
}

\newenvironment{theoreme}[2]
{
    \begin{boite_proposition}
    \textbf{\textcolor{rouge_fonce}{Théorème #1}} \textit{#2} \\ \\
}
{
    \end{boite_proposition}
    \vspace{15pt}
}

\newenvironment{demonstration}
{
    \begin{boite_demonstration}
    \textbf{\textcolor{rouge_fonce}{Démonstration}}\\ \\
}
{
    \end{boite_demonstration}
    \vspace{15pt}
}

\newenvironment{remarque}[2]
{
    \begin{boite_remarque}
    \textbf{\textcolor{bleu_fonce}{Remarque #1}}\textit{#2} \\ \\
    \begin{itemize}[label=$\blacktriangleright \quad $ ]                    
}
{   
    \end{itemize}
    \end{boite_remarque}
    
}



%Corrections
\newmdenv[
  nobreak=true,
  topline=true,
  bottomline=true,
  rightline=true,
  leftline=true,
  linewidth=0.5pt,
  linecolor=black,
  backgroundcolor=mayonnaise,
  innerleftmargin=10pt,
  innerrightmargin=10pt,
  innertopmargin=10pt,
  innerbottommargin=10pt,
  skipabove=\topsep,
  skipbelow=\topsep,
]{boite_question}


\newenvironment{question}[2]
{
    \begin{boite_question}
    \textbf{\textcolor{rouge_fonce}{Question #1}} \textit{#2} \\ \\
}
{
    \end{boite_question}
    \vspace{15pt}
}

\begin{document}
\begin{adjustwidth}{-3cm}{-3cm}
Nous avons jusqu'à présent étudié des structures de données séquentielles (piles, files, tables de hachage, etc.). Dans ce chapitre, on étudie des structures de donénes hiérarchiques. \\
On peut définir la structure d'\notion{arbre} de différentes façons, avec quelques subtilités, ce qui offre une certaine souplesse et permet de s'adapter au problème étudié. \textbf{Il convient d'être particulièrement vigilant à la lecture des sujets proposés pour identifier correctement la structure attendue}.
\section{Arbres binaires}

\begin{definition}{7.1}{ - Arbre binaire}
    Un \notion{arbre binaire} est un ensemble (éventuellement vide) de \notion{noeuds} et est défini de manière inductive. Un arbre binaire est : \begin{itemize}
        \item Ou bien \notion{l'arbre vide}
        \item Ou bien constitué d'un noeud $R$, appelé \notion{racine}, et de deux \notion{sous-arbres} binaires. Si existence, on appelle \notion{fils gauche} (resp. \notion{fils droit}) du noeud $R$ la racine du sous-arbre gauche (resp. droit). \\ \textit{Voir Figure 1}
    \end{itemize}
\end{definition}

\begin{remarque}{7.2}{ - à propos la précédente définition}
    \item Dans cette définition, si on permute les sous-arbres gauche et droit, on obtient un arbre différent. Ainsi, les deux arbres représentés ci-dessous sont deux arbres distincts : \textit{Voir Figure 2}
\end{remarque}

\begin{definition}{7.3}{ - feuille, père, arête, taille}
    \begin{itemize}
        \item Une \notion{feuille} est un noeud dont les sous-arbres gauche et droit sont l'arbre vide.
        \item Un noeud est le \notion{père} de ses éventuels fils. La liaison d'un père vers un fils est appelée une \notion{une arête}.
        \item Tous les noeuds d'un arbre, hormis la racine et les feuilles, sont appelés des \notion{noeuds internes}.
        \item La \notion{taille} d'un arbre est son nombre de noeuds. On note $\abs{\mc{A}}$ la taille de l'arbre $\mc{A}$
    \end{itemize}
\end{definition}

\begin{remarque}{7.4}{ - définition inductive de la taille d'un arbre binaire}
    \item La taille de l'arbre vide est nulle
    \item Un arbre binaire comportant un sous-arbre gauche de taille $n_g$ et un sous-arbre droit de taille $n_d$ est de taille $1+n_g+n_d$
\end{remarque}

\begin{definition}{7.5}{ - hauteur, profondeur}
    La \notion{hauteur} d'un arbre binaire est définie inductivement : \begin{itemize}
        \item L'arbre vide est de hauteur : $-1$
        \item Un arbre binaire comportant un sous-arbre gauche de hauteur $h_g$ et un sous-arbre droit de hauteur $h_d$ est de hauteur $1+\max{h_g,h_d}$.
    \end{itemize}
    La \notion{profondeur} d'un noeud est la distance de celui-ci à la racine (si existence). La hauteur d'un arbre est alors la profondeur maximale de ses neuds, et donc la profondeur maximale de ses feuilles.
\end{definition}

\begin{remarque}{7.6}{ - sur les étiquettes}
    Les noeuds et les arêtes d'un arbre peuvent porter des \notion{étiquettes} : on leur associe une valeur.
\end{remarque}

\begin{exemple}{7.7}{}
    \item Considérons l'arbre binaire suivant.
\end{exemple}
\end{adjustwidth}
\end{document}