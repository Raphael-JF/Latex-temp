\documentclass{article}
\usepackage{amsmath,amssymb,mathtools}
\usepackage{esint} % intégrale avec un round
\usepackage{xcolor}
\usepackage{listings}
% \usepackage{minted}
\usepackage{enumitem}
\usepackage{fourier-orns}
\usepackage{multicol}
\usepackage{changepage}
\usepackage{stmaryrd}
\usepackage{graphicx}
\graphicspath{ {./images/} }
\usepackage[framemethod=tikz]{mdframed}
\usepackage{tikz,pgfplots}
\pgfplotsset{compat=1.18}
\usetikzlibrary{arrows}
\usepackage{forest}
\usepackage{hyperref}

% physique
% \renewcommand*{\overrightarrow}[1]{\vbox{\halign{##\cr 
%   \tiny\rightarrowfill\cr\noalign{\nointerlineskip\vskip1pt} 
%   $#1\mskip2mu$\cr}}}

  \newenvironment{enumeratebf}{
    \begin{enumerate}[label=\textbf{\arabic*.}]
}
{
    \end{enumerate}
}
  
\definecolor{oranges}{RGB}{255, 242, 230}
\definecolor{rouges}{RGB}{255, 230, 230}
\definecolor{rose}{RGB}{255, 204, 204}

% maths - info
\definecolor{rouge_fonce}{RGB}{204, 0, 0}
\definecolor{rouge}{RGB}{255, 0, 0}
\definecolor{bleufonce}{RGB}{0, 0, 255}
\definecolor{vert_fonce}{RGB}{0, 69, 33}
\definecolor{vert}{RGB}{0,255,0}

\definecolor{orange_foncee}{RGB}{255, 153, 0}
\definecolor{myrtille}{RGB}{225, 225, 255}
\definecolor{mayonnaise}{RGB}{255, 253, 233}
\definecolor{magenta}{RGB}{224, 209, 240}
\definecolor{pomme}{RGB}{204, 255, 204}
\definecolor{mauve}{RGB}{255, 230, 255}


% Cours

\newmdenv[
  nobreak=true,
  topline=true,
  bottomline=true,
  rightline=true,
  leftline=true,
  linewidth=0.5pt,
  linecolor=black,
  backgroundcolor=mayonnaise,
  innerleftmargin=10pt,
  innerrightmargin=2.5em,
  innertopmargin=5pt,
  innerbottommargin=5pt,
  skipabove=\topsep,
  skipbelow=\topsep,
]{boite_definition}


\newenvironment{definition}[2]
{
    \vspace{15pt}
    \begin{boite_definition}
    \textbf{\textcolor{rouge}{Définition #1}}
    \if\relax\detokenize{#2}\relax
    \else
        \textit{ - #2}
    \fi \\ \\
}
{
    \end{boite_definition}
    
}

\newmdenv[
  nobreak=true,
  topline=true,
  bottomline=true,
  rightline=true,
  leftline=true,
  linewidth=0.5pt,
  linecolor=white,
  backgroundcolor=white,
  innerleftmargin=10pt,
  innerrightmargin=2.5em,
  innertopmargin=5pt,
  innerbottommargin=5pt,
  skipabove=\topsep,
  skipbelow=\topsep,
]{boite_exemple}


\newenvironment{exemple}[2]
{
    \vspace{15pt}
    \begin{boite_exemple}
    \textbf{\textcolor{bleufonce}{Exemple #1}}
    \if\relax\detokenize{#2}\relax
    \else
        \textit{ - #2}
    \fi \\ \\ 
}
{   
    \end{boite_exemple}
    
}


\newmdenv[
  nobreak=true,
  topline=true,
  bottomline=true,
  rightline=true,
  leftline=true,
  linewidth=0.5pt,
  linecolor=black,
  backgroundcolor=magenta,
  innerleftmargin=10pt,
  innerrightmargin=2.5em,
  innertopmargin=5pt,
  innerbottommargin=5pt,
  skipabove=\topsep,
  skipbelow=\topsep,
]{boite_proposition}

\newenvironment{proposition}[2]
{
    \vspace{15pt}
    \begin{boite_proposition}
    \textbf{\textcolor{rouge}{Proposition #1}}
    \if\relax\detokenize{#2}\relax
    \else
        \textit{ - #2}
    \fi \\ \\
}
{
    \end{boite_proposition}
}


\newmdenv[
  nobreak=true,
  topline=true,
  bottomline=true,
  rightline=true,
  leftline=true,
  linewidth=0.5pt,
  linecolor=black,
  backgroundcolor=magenta,
  innerleftmargin=10pt,
  innerrightmargin=2.5em,
  innertopmargin=5pt,
  innerbottommargin=5pt,
  skipabove=\topsep,
  skipbelow=\topsep,
]{boite_theoreme}


\newenvironment{theoreme}[2]
{
    \vspace{15pt}
    \begin{boite_theoreme}
    \textbf{\textcolor{rouge}{Théorème #1}} 
    \if\relax\detokenize{#2}\relax
    \else
        \textit{ - #2}
    \fi \\ \\
}
{
    \end{boite_theoreme}
    
}


\newmdenv[
  nobreak=true,
  topline=true,
  bottomline=true,
  rightline=true,
  leftline=true,
  linewidth=0.5pt,
  linecolor=black,
  backgroundcolor=white,
  innerleftmargin=10pt,
  innerrightmargin=2.5em,
  innertopmargin=5pt,
  innerbottommargin=5pt,
  skipabove=\topsep,
  skipbelow=\topsep,
]{boite_demonstration}


\newenvironment{demonstration}
{
    \vspace{15pt}
    \begin{boite_demonstration}
    \textbf{\textcolor{rouge}{Démonstration}}\\ \\
}
{
    \end{boite_demonstration}
    
}


\newmdenv[
  nobreak=true,
  topline=true,
  bottomline=true,
  rightline=true,
  leftline=true,
  linewidth=0.5pt,
  linecolor=white,
  backgroundcolor=white,
  innerleftmargin=10pt,
  innerrightmargin=2.5em,
  innertopmargin=5pt,
  innerbottommargin=5pt,
  skipabove=\topsep,
  skipbelow=\topsep,
]{boite_remarque}


\newenvironment{remarque}[2]
{
    \vspace{15pt}
    \begin{boite_remarque}
    \textbf{\textcolor{bleufonce}{Remarque #1}}
    \if\relax\detokenize{#2}\relax
    \else
        \textit{ - #2}
    \fi \\ \\   
}
{  
    \end{boite_remarque}
    
}

\newmdenv[
  nobreak=true,
  topline=true,
  bottomline=true,
  rightline=true,
  leftline=true,
  linewidth=0.5pt,
  linecolor=bleufonce,
  backgroundcolor=white,
  innerleftmargin=10pt,
  innerrightmargin=2.5em,
  innertopmargin=5pt,
  innerbottommargin=5pt,
  skipabove=\topsep,
  skipbelow=\topsep,
]{boite_OCaml}


\definecolor{keywordcolor}{RGB}{133, 153, 0}  % les mots-clés
\definecolor{commentcolor}{RGB}{147, 161, 161} % les commentaires
\definecolor{stringcolor}{RGB}{42, 161, 152}  % les chaînes de caractères
\newenvironment{OCaml}[2]
{   
    \vspace{15pt}
    \begin{boite_OCaml}
    \textbf{\textcolor{bleufonce}{Implémentation #1}}
    \if\relax\detokenize{#2}\relax
    \else
        \textit{ - #2}
    \fi \\ \\   

    \lstset{
    language=[Objective]Caml,
    basicstyle=\ttfamily,          % Police par défaut pour le code
    keywordstyle=\color{keywordcolor}, % Mots-clés en bleu doux
    commentstyle=\color{commentcolor}, % Commentaires en vert pâle
    stringstyle=\color{stringcolor},   % Chaînes en orange léger
    backgroundcolor=\color{white},   % Fond très clair
    numbers=left,                  % Numérotation à gauche
    numberstyle=\ttfamily,             % Taille des numéros de ligne
    stepnumber=1,                  % Numérotation de chaque ligne
    frame=single,                  % Cadre autour du code
    breaklines=true,               % Retour à la ligne automatique
    tabsize=2,                        % Taille des tabulations          
    }

    \begin{lstlisting}
}
{      
    \end{lstlisting}
    \end{boite_OCaml}
}

\newmdenv[
  nobreak=true,
  topline=true,
  bottomline=true,
  rightline=true,
  leftline=true,
  linewidth=0.5pt,
  linecolor=black,
  backgroundcolor=mayonnaise,
  innerleftmargin=10pt,
  innerrightmargin=2.5em,
  innertopmargin=5pt,
  innerbottommargin=5pt,
  skipabove=\topsep,
  skipbelow=\topsep,
]{boite_question}


\newenvironment{question}[2]
{
    \vspace{15pt}
    \begin{boite_question}
    \textbf{\textcolor{rouge}{Question #1}}
    \if\relax\detokenize{#2}\relax
    \else
        \textit{ - #2}
    \fi \\ \\
}
{
    \end{boite_question}
    
}

\newmdenv[
  nobreak=true,
  topline=true,
  bottomline=true,
  rightline=true,
  leftline=true,
  linewidth=0.5pt,
  linecolor=black,
  backgroundcolor=white,
  innerleftmargin=10pt,
  innerrightmargin=2.5em,
  innertopmargin=5pt,
  innerbottommargin=5pt,
  skipabove=\topsep,
  skipbelow=\topsep,
]{boite_corollaire}



\newenvironment{corollaire}[2]
{
    \vspace{15pt}
    \begin{boite_corollaire}
    \textbf{\textcolor{rouge}{Corollaire #1}}
    \if\relax\detokenize{#2}\relax
    \else
        \textit{ - #2}
    \fi \\ \\   
}
{
    \end{boite_corollaire}
    
}

\begin{document}
\begin{adjustwidth}{-3cm}{-3cm}
% commandes
\newcommand{\notion}[1]{\textcolor{vert_fonce}{\textit{#1}}}
\newcommand{\mb}[1]{\mathbb{#1}}
\newcommand{\mc}[1]{\mathcal{#1}}
\newcommand{\code}[1]{\texttt{#1}}
\newcommand{\ccode}[1]{\texttt{|#1|}}
\newcommand{\ov}[1]{\overline{#1}}
\newcommand{\abs}[1]{|#1|}
\newcommand{\rev}[1]{\texttt{reverse(#1)}}
\newcommand{\crev}[1]{\texttt{|reverse(#1)|}}

\newcommand{\ie}{\textit{i.e.} }

\newcommand{\N}{\mathbb{N}}
\newcommand{\R}{\mathbb{R}}
\newcommand{\C}{\mathbb{C}}
\newcommand{\K}{\mathbb{K}}

\newcommand{\A}{\mathcal{A}}
\newcommand{\bigO}{\mathcal{O}}
\renewcommand{\L}{\mathcal{L}}

\newcommand{\rg}[0]{\text{rg}}
\newcommand{\re}[0]{\text{Re}}
\newcommand{\im}[0]{\text{Im}}
\newcommand{\cl}[0]{\text{cl}}
\newcommand{\mat}[1]{\text{Mat}_{#1}}
\newcommand{\matrice}[1]{\mathcal{M}_{#1}}
\newcommand{\sgEngendre}[1]{\left\langle #1 \right\rangle}
\newcommand{\norme}[1]{||#1||}
\renewcommand{\d}[1]{\,\text{d}#1}
\newcommand{\intint}[2]{\llbracket #1 ,\, #2 \rrbracket}
\newcommand{\seg}[2]{[#1\, ; \, #2]}
\newcommand{\scal}[2]{\left\langle #1 ,\, #2 \right\rangle}
\newcommand{\inte}[2]{\int_{#1}^{#2}}
\newcommand{\somme}[2]{\sum_{#1}^{#2}}






\begin{definition}{2.8}{éléments associés}
    Deux un éléments d'un anneaux sont dits \notion{associés} lorsqu'ils sont égaux à multiplication près par un élément inversible.
\end{definition}

\begin{definition}{2.20 (1)}{diviseurs de zéro}
    Soit $(A,\, +,\, \times)$ un anneau. on appelle \notion{diviseurs de zéro} deux éléments $a$ et $b$ de $A$,tels que $ab = 0_A$
\end{definition}

\begin{definition}{2.20 (2)}{anneau intègre}
    Soit $(A,\, +,\, \times)$ un anneau. on dit que \notion{$A$ est intègre} lorsque~:
    \begin{enumeratebf}
        \item $A \neq \{0_A\}$
        \item $\times$ est commutative
        \item $A$ n'admet pas de diviseur zéro~: $\forall (a,b) \in A^2,\, a \neq 0 \mr{ et } b\neq \implies ab \neq 0$
    \end{enumeratebf}
\end{definition}

\begin{theoreme}{2.23}{caractérisation de la structure de corps}
    Soit $(\K,\, +,\, \times)$ un ensemble muni de deux lois de composition internes. $\K$ est un corps si et seulement si~:
    \begin{enumeratebf}
        \item $(\K,\, +)$ est un groupe abélien
        \item $(\K \setminus \{0_\K\},\, \times)$ est un groupe abélien
        \item $\times$ est distributive sur $+$
    \end{enumeratebf}
\end{theoreme}

\begin{theoreme}{2.26}{caractérisation de sous-corps}
     Soit $(\K,\, +,\, \times)$ un corps et $L \subset \K$. $L$ est un sous-corps de $\K$ si et seulement si~:
     \begin{enumeratebf}
        \item $L$ est un sous-anneau de $\K$
        \item tout élément non nul de $L$ est inversible dans $L$
     \end{enumeratebf}
\end{theoreme}

\begin{proposition}{2.29}{condition suffisante de caractère de corps}
    Tout anneau intègre fini est un corps.
\end{proposition}

\begin{definition}{2.30}{structure d'idéal}
    Soit $(A,\, +,\, \times)$ un anneau commutatif. On appelle \notion{idéal de $A$} une partie $I$ de $A$ telle que~:
    \begin{enumeratebf}
        \item $(I,+)$ est un sous-groupe de $(A,+)$
        \item $I$ est attracteur pour $\times$~: $\forall i \in I,\, \forall a \in A,\, ai = ia \in I$
    \end{enumeratebf}
\end{definition}

\begin{proposition}{2.32}{images directe et réciproque d'un idéal}
    Soit $A$ et $B$ deux anneaux commutatifs, $f:A\to B$ un morphisme d'anneaux.
    \begin{enumeratebf}
        \item L'image directe d'un idéal de $A$ par $f$ est un idéal de $B$
        \item L'image réciproque d'un idéal de $B$ par $f$ est un idéal de $A$
    \end{enumeratebf}
\end{proposition}

\begin{definition}{2.34}{idéal engendré par un élément}
    Soit $(A,\, +,\, \times)$ un anneau commutatif. Soit $x \in A$
    L'ensemble $xA$ des multiples de $x$ dans $A$ est un idéal de $A$, appelé \notion{idéal engendré par $x$}. On le note $(x)$.
\end{definition}

\begin{theoreme}{2.36}{idéaux de $(\Z,\, +,\, \times)$}
    L'ensemble des idéaux de $(\Z,\, +,\, \times)$ est $\{n\Z,\, n \in \Z\}$~: $\Z$ est principal (car intègre aussi).
\end{theoreme}

\begin{definition}{2.37}{plus grand commun diviseur de deux entiers}
    Étant donnés deux entiers $a$ et $b$, l'ensemble $a\Z + b\Z$ est un idéal de $\Z$, son unique générateur positif est appelé \notion{le plus grand diviseur commun de $a$ et $b$}. Ainsi,\, 
    $$a\Z + b\Z = (a\wedge b) \Z$$
\end{definition}

\begin{definition}{2.39}{plus petit commun multiple de deux entiers}
    Étant donnés deux entiers $a$ et $b$, l'ensemble $a\Z \cap b\Z$ est un idéal de $\Z$, son unique générateur positif est appelé \notion{plus petit commun multiple de $a$ et $b$}. Ainsi,\, 
    $$a\Z \cap b\Z = (a \vee b) \Z$$
\end{definition}

\begin{theoreme}{2.44}{produit d'anneaux quotients}
    Soit $(n,p)\in \N^2$. Si $n$ et $p$ sont premiers entre eux, alors les anneaux $\Z/np\Z$ et $\Z/n\Z \times \Z/p\Z$ sont isomorphes.
\end{theoreme}

\begin{theoreme}{2.45}{inversibles de $(\Z /n\Z,\, +,\,  \times)$}
    Soit $k \in \Z$. la classe $\cl(k)$ est inversible dans $(\Z /n\Z,\, +,\,  \times)$ si et seulement si $k \wedge n = 1$.
\end{theoreme}

\begin{definition}{2.48}{fonction indicatrice d'Euler}
    On appelle \notion{fonction indicatrice d'Euler} la fonction $\varphi:\N^*\to\N$ qui à $n$ associe le nombre $\varphi(n)$ d'entiers de l'intervalle $\intint{1}{n}$ premiers avec $n$. En fait,
    $$\forall n \in \N^*,\,  \varphi(n) = \mc{U}\Big((\Z/n\Z,\, +,\, \times)\Big)$$
\end{definition}

\begin{proposition}{2.48 bis}{image de la fonction indicatrice d'Euler par un entier premier}
    Soit $p \in \mb{P}$. On a~:
    $$\varphi(p) = p-1$$
\end{proposition}

\begin{proposition}{2.49}{théorème d'Euler}
    Soit $k$ et $n$ deux entiers premiers entre eux. Alors on a~:
    \begin{align*}
            &k^{\varphi(n)} \equiv 1 \, [n]\\
        \text{et donc} \quad &k^{\varphi(n)+1} \equiv k \, [n]
    \end{align*}
\end{proposition}

\begin{proposition}{2.50}{caractérisation du caractère de corps de $\gpquotient{n}$}
    $\gpquotient{n}$ est un corps si et seulement si $n$ est premier.
\end{proposition}

\begin{theoreme}{2.51}{petit théorème de Fermat}
    Soit $p \in \mb{P}$. Pour tout entier $x$, $$x^p \equiv x \,[p]$$
\end{theoreme}

\begin{theoreme}{2.52}{groupe des inversibles de $\gpquotient{p}$, $p \in \mb{P}$}
    Soit $p \in \mb{P}$. Alors le groupe multiplicatif $\mc{U}(\gpquotient{p})$ est isomorphe au groupe additif $\gpquotient{(p-1)}$
\end{theoreme}

\begin{proposition}{2.54}{indicatrice d'Euler du produit de premiers entre eux}
    Soit $m$ et $n$ premiers entre eux. Alors~:
    $$\varphi(mn) = \varphi(m)\varphi(n)$$
\end{proposition}

\begin{proposition}{2.55}{indicatrice d'Euler d'une puissance d'un premier}
    Soit $p$ un nombre premier et $\alpha \in \N^*$. Alors~:
    $$\varphi(p^\alpha) = p^\alpha - p^{\alpha -1}$$
\end{proposition}

\begin{definition}{2.72}{stucture d'algèbre}
    Un ensemble $(\mc{A},\, +,\, \times,\, \cdot)$ :  muni de deux lois de composition internes $+$ et $\times$, et d’une loi $\cdot$ externe sur $\K$, est une \notion{$\K$-algèbre} si~:
    \begin{enumeratebf}
        \item $(\mc{A},\, +,\, \times)$ est un anneau.
        \item $(\mc{A},\, +,\, \cdot)$ est un $\K$-espace vectoriel.
        \item $\times$ et $\cdot$ sont compatibles~:
        $$\forall (a,b) \in \mc{A}^2,\, \forall (\lambda, \mu) \in \K^2,\, (a \cdot x) \times (b \cdot y) = (ab) \cdot (x \times y)$$
    \end{enumeratebf}
\end{definition}

\begin{definition}{2.74}{sous-algèbre}
    Un ensemble $\mc{B}$ est une  \notion{sous-algèbre} d'une algèbre $\mc{A}$ si~:
    \begin{enumeratebf}
        \item $\mc{B}$ est un sous-anneau de $\mc{A}$.
        \item $\mc{B}$ est un sous-espace vectoriel de $\mc{A}$.
    \end{enumeratebf}
    mais il suffit d'avoir~:
    \begin{enumeratebf}
        \item $1_\mc{A} \in \mc{B}$
        \item $\mc{B}$ stable par $\times$
        \item $\mc{B}$ stable par combinaison linéaire
    \end{enumeratebf}
\end{definition}

\begin{definition}{2.76}{morphisme d'algèbre}
    Soit $(\mc{A},\, +_\mc{A},\, \times_\mc{A},\, \cdot_\mc{A})$ et $(\mc{B},\, +_\mc{B},\, \times_\mc{B},\, \cdot_\mc{B})$ deux $\K$-algèbres. $f:A\to B$ est un \notion{morphisme de $\K$-algèbres} si~:
    \begin{enumeratebf}
        \item $f$ est linéaire.
        \item $f$ respecte le produit.
    \end{enumeratebf}
\end{definition}

\input{../../stock/pied.tex}