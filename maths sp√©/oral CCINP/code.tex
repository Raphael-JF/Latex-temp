\documentclass[11pt]{article}
\usepackage[utf8]{inputenc}
\usepackage[T1]{fontenc}
\usepackage{lmodern}
\usepackage[french]{babel}
\usepackage{amsmath, amssymb, amsthm}
\usepackage{enumitem}
\usepackage{geometry}
\geometry{margin=2.5cm}
\usepackage{xcolor}
\usepackage{hyperref}
\hypersetup{
    colorlinks=true,
    linkcolor=blue,
    urlcolor=cyan
}

% Commandes perso utiles
\newcommand{\R}{\mathbb{R}}
\newcommand{\N}{\mathbb{N}}
\newcommand{\Z}{\mathbb{Z}}
\newcommand{\C}{\mathbb{C}}


\title{Planche d'oral de Mathématiques du CCINP}
\author{MPI Session 2025}
\date{}

\begin{document}

\maketitle

\section*{Premier exercice sur huit points}

    Exercice 102 concernant le minimum de variables aléatoires.


\section*{Second exercice sur douze points}

Pour $a \in [-1;1[$, on définit pour $x>0$, et $n \in \N^*$~:
$$u_n(x)=\frac{a^n}{x+n}$$
\begin{enumerate}[label=\textbf{\arabic*.}]
    \item Étudier les convergences de $\sum_{n > 0}u_n$. Si existence, on note $S(x) = \sum_{n=1}^{+\infty}u_n(x)$.
    \item Montrer que $S$ est continue.
    \item Montrer que $S$ est de classe $\mathcal{C}^1$ sur $\R_+^*$
\end{enumerate}

\section*{Second exercice - Corrigé}
\begin{enumerate}[label=\textbf{\arabic*.}]
    \item Pour $x > 0$, $n \in \N^*$, et $a \in ]-1;1[$~:
        $$\vert u_n(x) \vert \leq \frac{\vert a \vert^n}{n}$$
    La majoration est indépendante de la variable de $u_n$. Ainsi~:
    \begin{align*}
        \lVert u_n \rVert_\infty^{\R_+^*} &\leq \frac{\vert a \vert^n}{n}\\
        &\leq \vert a \vert^n \qquad \text{car $n \in \N^*$}
    \end{align*}
    Comme $a \in ]-1;1[$, $\sum_{n>0} \vert a \vert^n $ converge. Par théorème de comparaison de séries à termes positifs par majoration, $\sum_{n > 0}u_n$ converge normalement sur $\R_+^*$ si $a \in ]-1;1[$.\\
    Pour $a = -1$, Le critère spécial des séries alternées s'applique pour tout $x$~: il y a convergence simple de $\sum_{n > 0}u_n$ sur $\R_+^*$. En outre~:
    \begin{align*}
        \vert R_n(x)\vert &\leq \vert u_{n+1}\vert\\
        &=\frac{1}{n+1+x} \\
        &\leq \frac{1}{n+1}
    \end{align*}
    La majoration est indépendante de la variable de $R_n$. Ainsi~:
    $$\lVert R_n \rVert_\infty^{\R_+^*} \leq \frac{1}{n+1}\xrightarrow{n \to +\infty} 0$$
    On a donc la convergence uniforme sur $\R_+^*$ dans le cas où $a = -1$. Cependant, la convergence normale est perdue~:
    $$2 \geq \lVert u_n \rVert_\infty^{\R_+^*} \geq  \vert u_n(x) \vert = \frac{1}{n+x} \underset{n \to +\infty}{\sim} \frac{1}{n}$$
    Par théorème de comparaison de séries à termes positifs par minoration avec un terme semblable au terme général de la série harmonique, $\sum_{n > 0}u_n$ ne converge pas normalement sur $\R_+^*$.
    \item D'après la question précédente on a convergence uniforme sur $\R_+^*$ de $\sum_{n > 0}u_n$. Or $(\sum_{i=1}^{n}u_i)_{n \in \N^*}$ est une suite de fonctions continues sur $\R_+^*$, d'où la continuité de sa limite uniforme.
    \item On applique le théorème de la classe $\mathcal{C}^1$ d'une limite uniforme~:
    \begin{itemize}
        \item $\sum_{n>0}u_n$ converge simplement sur $\R_+^*$. On le doit à la première question.
        \item pour $n>0$ et $a$ fixé, $u_n$ est de classe $\mathcal{C}^1$~:
            $$u_n'(x) = \frac{-a^n}{(n+x)^2}$$
        \item $\sum_{n>0}u_n'$ converge uniformément sur $\R_+^*$. En effet~:
        $$\vert u_n'(x) \vert \leq \frac{1}{n^2}$$
        La majoration est indépendante de la variable de $u_n'$. Ainsi~:
        $$\lVert u_n' \rVert_\infty^{\R_+^*}\leq \frac{1}{n^2}$$
        Par comparaison de séries à termes positifs par majoration avec le terme général d'une série de Riemann convergente, on a la convergence normale, et donc la converge uniforme.
    \end{itemize}
    D'après le théorème, $S$ est de classe $\mathcal{C}^1$ sur $\R_+^*$.
\end{enumerate}

\vspace{1cm}
\end{document}
