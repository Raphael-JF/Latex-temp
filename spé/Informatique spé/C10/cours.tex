\documentclass{article}
\usepackage{titling} % Personnalisation du titre
\usepackage[left=20mm, right=20mm]{geometry}
\usepackage{amsmath,amssymb,mathtools}
\usepackage{esint} % intégrale avec un round
\usepackage{xcolor}
\usepackage[utf8]{inputenc}
\usepackage{listings}
\usepackage{enumitem}
\usepackage{multicol}
\usepackage{stmaryrd}
\usepackage{graphicx}
\graphicspath{ {./images/} }
\usepackage[framemethod=tikz]{mdframed}
\usepackage{tikz,pgfplots}
\pgfplotsset{compat=1.18}
\usetikzlibrary{arrows}
\usepackage{forest}
\usepackage{titlesec}
\setlength{\parindent}{0pt}

\pretitle{\begin{center}\Huge\bfseries}
\posttitle{\end{center}}
\date{}
\renewcommand{\thesection}{\Roman{section}} 
\titleformat{\section}
  {\Large\bfseries} % Style du titre
  {\thesection} % Numéro de section
  {0.5cm} % Espacement entre numéro et titre
  {} % Pas de préfixe supplémentaire
\titleformat{\subsection}
  {\large\bfseries} % Style du titre
  {\thesubsection} % Numéro de section
  {0.4cm} % Espacement entre numéro et titre
  {} % Pas de préfixe supplémentair


\newenvironment{enumeratebf}{
    \begin{enumerate}[label=\textbf{\arabic*.}]
}
{
    \end{enumerate}
}
  
\definecolor{oranges}{RGB}{255, 242, 230}
\definecolor{rouges}{RGB}{255, 230, 230}
\definecolor{rose}{RGB}{255, 204, 204}

% maths - info
\definecolor{rouge_fonce}{RGB}{204, 0, 0}
\definecolor{rouge}{RGB}{255, 0, 0}
\definecolor{bleufonce}{RGB}{0, 0, 255}
\definecolor{vert_fonce}{RGB}{0, 69, 33}
\definecolor{vert}{RGB}{0,255,0}

\definecolor{orange_foncee}{RGB}{255, 153, 0}
\definecolor{myrtille}{RGB}{225, 225, 255}
\definecolor{mayonnaise}{RGB}{255, 253, 233}
\definecolor{magenta}{RGB}{224, 209, 240}
\definecolor{pomme}{RGB}{204, 255, 204}
\definecolor{mauve}{RGB}{255, 230, 255}


% Cours

\newmdenv[
    nobreak=true,
    topline=true,
    bottomline=true,
    rightline=true,
    leftline=true,
    linewidth=0.5pt,
    linecolor=black,
    backgroundcolor=mayonnaise,
    innerleftmargin=10pt,
    innerrightmargin=2.5em,
    innertopmargin=5pt,
    innerbottommargin=5pt,
    skipabove=-0.5cm,
    skipbelow=-0.25cm,
]{boite_definition}

\newcounter{boite}
\setcounter{boite}{1}
\newenvironment{definition}[2]
{
    \vspace{15pt}
    \begin{boite_definition}
    \if\relax\detokenize{#1}\relax
        \textbf{\textcolor{rouge}{Définition \arabic{chapitre}.\arabic{boite}}}%
        \if\relax\detokenize{#2}\relax
        \else
            \textit{ - #2}
        \fi
        \stepcounter{boite}
    \else
        \textbf{\textcolor{rouge}{Définition #1}}%
        \if\relax\detokenize{#2}\relax
        \else
            \textit{ - #2}
        \fi
    \fi \\
    
    
}
{
    \end{boite_definition}
}

\newmdenv[
  nobreak=true,
  topline=true,
  bottomline=true,
  rightline=true,
  leftline=true,
  linewidth=0.5pt,
  linecolor=white,
  backgroundcolor=white,
  innerleftmargin=10pt,
  innerrightmargin=2.5em,
  innertopmargin=5pt,
  innerbottommargin=5pt,
  skipabove=-0.5cm,
  skipbelow=-0.25cm,
]{boite_exemple}


\newenvironment{exemple}[2]
{
    \vspace{15pt}
    \begin{boite_exemple}
    \if\relax\detokenize{#1}\relax
        \textbf{\textcolor{bleufonce}{Exemple \arabic{chapitre}.\arabic{boite}}}%
        \if\relax\detokenize{#2}\relax
        \else
            \textit{ - #2}
        \fi
        \stepcounter{boite}
    \else
        \textbf{\textcolor{bleufonce}{Exemple #1}}%
        \if\relax\detokenize{#2}\relax
        \else
            \textit{ - #2}
        \fi
    \fi \\
    
    
}
{
    \end{boite_exemple}
}

\newmdenv[
  nobreak=true,
  topline=true,
  bottomline=true,
  rightline=true,
  leftline=true,
  linewidth=0.5pt,
  linecolor=black,
  backgroundcolor=magenta,
  innerleftmargin=10pt,
  innerrightmargin=2.5em,
  innertopmargin=5pt,
  innerbottommargin=5pt,
  skipabove=-0.5cm,
  skipbelow=-0.25cm,
]{boite_proposition}

\newenvironment{proposition}[2]
{
    \vspace{15pt}
    \begin{boite_proposition}
    \if\relax\detokenize{#1}\relax
        \textbf{\textcolor{rouge}{Proposition \arabic{chapitre}.\arabic{boite}}}%
        \if\relax\detokenize{#2}\relax
        \else
            \textit{ - #2}
        \fi
        \stepcounter{boite}
    \else
        \textbf{\textcolor{rouge}{Proposition #1}}%
        \if\relax\detokenize{#2}\relax
        \else
            \textit{ - #2}
        \fi
    \fi \\
    
    
}
{
    \end{boite_proposition}
}

\newmdenv[
  nobreak=true,
  topline=true,
  bottomline=true,
  rightline=true,
  leftline=true,
  linewidth=0.5pt,
  linecolor=black,
  backgroundcolor=magenta,
  innerleftmargin=10pt,
  innerrightmargin=2.5em,
  innertopmargin=5pt,
  innerbottommargin=5pt,
  skipabove=-0.5cm,
  skipbelow=-0.25cm,
]{boite_theoreme}


\newenvironment{theoreme}[2]
{
    \vspace{15pt}
    \begin{boite_theoreme}
    \if\relax\detokenize{#1}\relax
        \textbf{\textcolor{rouge}{Théorème \arabic{chapitre}.\arabic{boite}}}%
        \if\relax\detokenize{#2}\relax
        \else
            \textit{ - #2}
        \fi
        \stepcounter{boite}
    \else
        \textbf{\textcolor{rouge}{Théorème #1}}%
        \if\relax\detokenize{#2}\relax
        \else
            \textit{ - #2}
        \fi
    \fi \\
    
    
}
{
    \end{boite_theoreme}
}


\newmdenv[
  nobreak=true,
  topline=true,
  bottomline=true,
  rightline=true,
  leftline=true,
  linewidth=0.5pt,
  linecolor=black,
  backgroundcolor=white,
  innerleftmargin=10pt,
  innerrightmargin=2.5em,
  innertopmargin=5pt,
  innerbottommargin=5pt,
  skipabove=-0.5cm,
  skipbelow=-0.25cm,
]{boite_demonstration}


\newenvironment{demonstration}
{
    \vspace{15pt}
    \begin{boite_demonstration}
    \textbf{\textcolor{rouge}{Démonstration}}\\ \\
}
{
    \end{boite_demonstration}
    
}


\newmdenv[
  nobreak=true,
  topline=true,
  bottomline=true,
  rightline=true,
  leftline=true,
  linewidth=0.5pt,
  linecolor=white,
  backgroundcolor=white,
  innerleftmargin=10pt,
  innerrightmargin=2.5em,
  innertopmargin=5pt,
  innerbottommargin=5pt,
  skipabove=-0.5cm,
  skipbelow=-0.25cm,
]{boite_remarque}


\newenvironment{remarque}[2]
{
    \vspace{15pt}
    \begin{boite_remarque}
    \if\relax\detokenize{#1}\relax
        \textbf{\textcolor{bleufonce}{Remarque \arabic{chapitre}.\arabic{boite}}}%
        \if\relax\detokenize{#2}\relax
        \else
            \textit{ - #2}
        \fi
        \stepcounter{boite}
    \else
        \textbf{\textcolor{bleufonce}{Remarque #1}}%
        \if\relax\detokenize{#2}\relax
        \else
            \textit{ - #2}
        \fi
    \fi \\
    
    
}
{
    \end{boite_remarque}
}

\newmdenv[
  nobreak=true,
  topline=true,
  bottomline=true,
  rightline=true,
  leftline=true,
  linewidth=0.5pt,
  linecolor=bleufonce,
  backgroundcolor=white,
  innerleftmargin=10pt,
  innerrightmargin=2.5em,
  innertopmargin=5pt,
  innerbottommargin=5pt,
  skipabove=-0.5cm,
  skipbelow=-0.25cm,
]{boite_implementation}


\definecolor{keywordcolor}{RGB}{133, 153, 0}  % les mots-clés
\definecolor{commentcolor}{RGB}{147, 161, 161} % les commentaires
\definecolor{stringcolor}{RGB}{42, 161, 152}  % les chaînes de caractères

\lstnewenvironment{lstOCaml}
{\lstset{
    language=[Objective]Caml,
    basicstyle=\ttfamily,
    keywordstyle=\color{keywordcolor},
    commentstyle=\color{commentcolor},
    stringstyle=\color{stringcolor},
    backgroundcolor=\color{white},
    numbers=left,
    numberstyle=\ttfamily,
    numbersep=-1.5em,
    stepnumber=1,
    frame=l,
    framexleftmargin=-2.25em,
    tabsize=2,
    literate=%
    {é}{{\'e}}{1}%
    {è}{{\`e}}{1}%
    {à}{{\`a}}{1}%
    {ç}{{\c{c}}}{1}%
    {œ}{{\oe}}{1}%
    {ù}{{\`u}}{1}%
    {É}{{\'E}}{1}%
    {È}{{\`E}}{1}%
    {À}{{\`A}}{1}%
    {Ç}{{\c{C}}}{1}%
    {Œ}{{\OE}}{1}%
    {Ê}{{\^E}}{1}%
    {ê}{{\^e}}{1}%
    {î}{{\^i}}{1}%
    {ô}{{\^o}}{1}%
    {û}{{\^u}}{1}%
    {ä}{{\"{a}}}1
    {ë}{{\"{e}}}1
    {ï}{{\"{i}}}1
    {ö}{{\"{o}}}1
    {ü}{{\"{u}}}1
    {û}{{\^{u}}}1
    {â}{{\^{a}}}1
    {Â}{{\^{A}}}1
    {Î}{{\^{I}}}1
}}{}
 
\lstnewenvironment{lstC}
{\lstset{
    language=C,
    basicstyle=\ttfamily,
    keywordstyle=\color{keywordcolor},
    commentstyle=\color{commentcolor},
    stringstyle=\color{stringcolor},
    backgroundcolor=\color{white},
    numbers=left,
    numberstyle=\ttfamily,
    numbersep=-1.5em,
    stepnumber=1,
    frame=l,
    framexleftmargin=-2.25em,
    tabsize=2,
    literate=%
    {é}{{\'e}}{1}%
    {è}{{\`e}}{1}%
    {à}{{\`a}}{1}%
    {ç}{{\c{c}}}{1}%
    {œ}{{\oe}}{1}%
    {ù}{{\`u}}{1}%
    {É}{{\'E}}{1}%
    {È}{{\`E}}{1}%
    {À}{{\`A}}{1}%
    {Ç}{{\c{C}}}{1}%
    {Œ}{{\OE}}{1}%
    {Ê}{{\^E}}{1}%
    {ê}{{\^e}}{1}%
    {î}{{\^i}}{1}%
    {ô}{{\^o}}{1}%
    {û}{{\^u}}{1}%
    {ä}{{\"{a}}}1
    {ë}{{\"{e}}}1
    {ï}{{\"{i}}}1
    {ö}{{\"{o}}}1
    {ü}{{\"{u}}}1
    {û}{{\^{u}}}1
    {â}{{\^{a}}}1
    {Â}{{\^{A}}}1
    {Î}{{\^{I}}}1
}}{}


\lstdefinelanguage{LNat}{
    morekeywords={tant,que,pour,tout,si,sinon,initialiser,renvoyer,attendre la fin, afficher},
    sensitive=false,
    morecomment=[l]{//},
}

\lstnewenvironment{lstLNat}
{\lstset{
    language=LNat,
    basicstyle=\ttfamily,
    keywordstyle=\color{keywordcolor},
    commentstyle=\color{commentcolor},
    stringstyle=\color{stringcolor},
    backgroundcolor=\color{white},
    numbers=left,
    numberstyle=\ttfamily,
    numbersep=-1.5em,
    stepnumber=1,
    frame=l,
    mathescape=true,
    framexleftmargin=-2.25em,
    tabsize=2,
    literate=%
    {é}{{\'e}}{1}%
    {è}{{\`e}}{1}%
    {à}{{\`a}}{1}%
    {ç}{{\c{c}}}{1}%
    {œ}{{\oe}}{1}%
    {ù}{{\`u}}{1}%
    {É}{{\'E}}{1}%
    {È}{{\`E}}{1}%
    {À}{{\`A}}{1}%
    {Ç}{{\c{C}}}{1}%
    {Œ}{{\OE}}{1}%
    {Ê}{{\^E}}{1}%
    {ê}{{\^e}}{1}%
    {î}{{\^i}}{1}%
    {ô}{{\^o}}{1}%
    {û}{{\^u}}{1}%
    {ä}{{\"{a}}}1
    {ë}{{\"{e}}}1
    {ï}{{\"{i}}}1
    {ö}{{\"{o}}}1
    {ü}{{\"{u}}}1
    {û}{{\^{u}}}1
    {â}{{\^{a}}}1
    {Â}{{\^{A}}}1
    {Î}{{\^{I}}}1}
}{}

\newenvironment{implementation}[1]
{   
    \vspace{15pt}
    \begin{boite_implementation}
    \textbf{\textcolor{bleufonce}{Implémentation}}\textit{ - #1}
     \\ \\
}
{    
    \end{boite_implementation}
}

\newmdenv[
  nobreak=true,
  topline=true,
  bottomline=true,
  rightline=true,
  leftline=true,
  linewidth=0.5pt,
  linecolor=black,
  backgroundcolor=mayonnaise,
  innerleftmargin=10pt,
  innerrightmargin=2.5em,
  innertopmargin=5pt,
  innerbottommargin=5pt,
  skipabove=-0.5cm,
  skipbelow=-0.25cm,
]{boite_question}


\newenvironment{question}[2]
{
    \vspace{15pt}
    \begin{boite_question}
    \if\relax\detokenize{#1}\relax
        \textbf{\textcolor{rouge}{Question \arabic{chapitre}.\arabic{boite}}}%
        \if\relax\detokenize{#2}\relax
        \else
            \textit{ - #2}
        \fi
        \stepcounter{boite}
    \else
        \textbf{\textcolor{rouge}{Question #1}}%
        \if\relax\detokenize{#2}\relax
        \else
            \textit{ - #2}
        \fi
    \fi \\
    
    
}
{
    \end{boite_question}
}

\newmdenv[
  nobreak=true,
  topline=true,
  bottomline=true,
  rightline=true,
  leftline=true,
  linewidth=0.5pt,
  linecolor=black,
  backgroundcolor=white,
  innerleftmargin=10pt,
  innerrightmargin=2.5em,
  innertopmargin=5pt,
  innerbottommargin=5pt,
  skipabove=-0.5cm,
  skipbelow=-0.25cm,
]{boite_corollaire}



\newenvironment{corollaire}[2]
{
    \vspace{15pt}
    \begin{boite_corollaire}
    \if\relax\detokenize{#1}\relax
        \textbf{\textcolor{rouge}{Corollaire \arabic{chapitre}.\arabic{boite}}}%
        \if\relax\detokenize{#2}\relax
        \else
            \textit{ - #2}
        \fi
        \stepcounter{boite}
    \else
        \textbf{\textcolor{rouge}{Corollaire #1}}%
        \if\relax\detokenize{#2}\relax
        \else
            \textit{ - #2}
        \fi
    \fi \\
    
    
}
{
    \end{boite_corollaire}
}

\newcounter{chapitre}
\setcounter{chapitre}{10}

\renewcommand{\newline}{\\}
\title{\Large Chapitre 10 \newline \Huge Décidabilité}

\begin{document}
% commandes
\newcommand{\notion}[1]{\textcolor{vert_fonce}{\textit{#1}}}
\newcommand{\mb}[1]{\mathbb{#1}}
\newcommand{\mc}[1]{\mathcal{#1}}
\newcommand{\mr}[1]{\mathrm{#1}}
\newcommand{\code}[1]{\texttt{#1}}
\newcommand{\ccode}[1]{\texttt{|#1|}}
\newcommand{\ov}[1]{\overline{#1}}
\newcommand{\abs}[1]{|#1|}
\newcommand{\rev}[1]{\texttt{reverse(#1)}}
\newcommand{\crev}[1]{\texttt{|reverse(#1)|}}

\newcommand{\ie}{\textit{i.e.} }

\newcommand{\N}{\mathbb{N}}
\newcommand{\R}{\mathbb{R}}
\newcommand{\C}{\mathbb{C}}
\newcommand{\K}{\mathbb{K}}
\newcommand{\Z}{\mathbb{Z}}

\newcommand{\A}{\mathcal{A}}
\newcommand{\bigO}{\mathcal{O}}
\renewcommand{\L}{\mathcal{L}}

\newcommand{\rg}[0]{\mathrm{rg}}
\newcommand{\re}[0]{\mathrm{Re}}
\newcommand{\im}[0]{\mathrm{Im}}
\newcommand{\cl}[0]{\mathrm{cl}}
\newcommand{\grad}[0]{\vec{\mathrm{grad}}}
\renewcommand{\div}[0]{\mathrm{div}\,}
\newcommand{\rot}[0]{\vec{\mathrm{rot}}\,}
\newcommand{\vnabla}[0]{\vec{\nabla}}
\renewcommand{\vec}[1]{\overrightarrow{#1}}
\newcommand{\mat}[1]{\mathrm{Mat}_{#1}}
\newcommand{\matrice}[1]{\mathcal{M}_{#1}}
\newcommand{\sgEngendre}[1]{\left\langle #1 \right\rangle}
\newcommand{\gpquotient}[1]{\mathbb{Z} / #1\mathbb{Z}}
\newcommand{\norme}[1]{||#1||}
\renewcommand{\d}[1]{\,\mathrm{d}#1}
\newcommand{\adh}[1]{\overline{#1}}
\newcommand{\intint}[2]{\llbracket #1 ,\, #2 \rrbracket}
\newcommand{\seg}[2]{[#1\, ; \, #2]}
\newcommand{\scal}[2]{( #1 | #2 )}
\newcommand{\distance}[2]{\mathrm{d}(#1,\,#2)}
\newcommand{\inte}[2]{\int_{#1}^{#2}}
\newcommand{\somme}[2]{\sum_{#1}^{#2}}
\newcommand{\deriveref}[4]{\biggl( \frac{\text{d}^{#1}#2}{\text{d}#3^{#1}} \biggr)_{#4}}





\maketitle

Au programme : concepts à comprendre, démonstration à connaître !

\section{Problème de décision et décidabilité}

\begin{definition}{}{problème de décision}
    Un \notion{problème de décision} est un problème dont la réponse attendue est binaire : Vrai ou Faux. Plus précisément, un problème $\mc{P}$ est la donnée de~:
    \begin{itemize}
        \item $I$ un ensemble d'instances
        \item $S$ un ensemble de solutions : l'union des solutions pour chaque instance.
        \item $f : I \to S$ ou pour $i$ une instance, $f(i)$ est la réponse attendue pour l'instance $i$.
    \end{itemize}
    Pour un problème de décision, $S = \{\mr{Vrai}, \mr{Faux}\}$. On appelle la fonction $f$ \notion{fonction de prédicat} du problème $\mc{P}$ de décision.\\\\
    $\mc{P}$ peut aussi être défini à l'aide d'un sous-ensemble $P$ de $I \times S$ tel que ~:
    $$(i,s) \in P \Leftrightarrow s \text{ solution de $\mc{P}$ pour l'instance $i$}$$
\end{definition}



Pour $\mc{P}$ un problème de décision, défini par $f : I \to \{\mr{Vrai},\, \mr{Faux}\}$ sa fonction de prédicat, on définit~:
$$I_{\mc{P}}^+ = \{i \in I,\, f(i) = \mr{Vrai}\}$$
l'ensemble des \notion{instances passives} du problème $\mc{P}$ de décision.


\begin{definition}{}{décidabilité d'un problème de décision}
    Un problème de décision $\mc{P}$ est dit \notion{décidable} lorsqu'il existe $\mc{A}$ un algorithme qui pour toute instance du problème $\mc{P}$ renvoie la solution attendue.\\
    Autrement dit, pour $f : I \to S$ la fonction de prédicat de $\mc{P}$, il existe un algorithme $\mc{A}$ tel que~:
    $$\forall i \in I, f(i) = \mc{A}(i)$$
    $f(i)$ est ici la solution attendue tandis que $\mc{A}(i)$ est la solution attendue pour $i$.\\\\
    Le cas échéant, $\mc{A}$ termine pour toute instance.
    \notion{$\mc{A}$ résout $\mc{P}$} ou que \notion{$\mc{P}$ est décidé par $\mc{A}$}.
\end{definition}

\begin{definition}{}{indécidabilité d'un problème}
    Un problème de décision $\mc{P}$ est dit \notion{indécidable} lorsqu'il n'existe pas d'algorithme resolvant $\mc{P}$.
\end{definition}

\begin{remarque}{}{sur l'indécidabilité}
    Un problème indécidable est un problème intrinsèquement infaisable~: inutile d'essayer de le résoudre car c'est impossible.
\end{remarque}

\begin{exemple}{}{de problème décidable}
    $f : I \to S = \{\mr{Vrai},\, \mr{Faux}\}$\\
    $$f(\code{l}) = \mr{Vrai} \Leftrightarrow \text{\code{l} a exactement 5 éléments}$$
    $f$ est la fonction de prédicat d'un problème de décision~:
    \begin{itemize}
        \item \textbf{Instance}~: \code{l} une liste d'entiers
        \item \textbf{Question}~: Est-ce que \code{l} contient 5 éléments.
    \end{itemize}
    Ce problème est décidable car on peut écrire en OCaml~:
    \begin{lstOCaml}
    let longueur 5 l =
        match l with
        | e1::e2::e3::e4::e5::[] -> true
        | _ -> false
    \end{lstOCaml}
\end{exemple}

\subsection{semi-décidabilité (HP)}

\begin{definition}{}{semi-décidabilité d'un problème}
    Un problème de décision $\mc{P}$ défini par $f:I \to \{\mr{Vrai},\, \mr{Faux}\}$ est dit \notion{semi-décidable} lorsqu'il existe un algorithme $\mc{A}$ tel que pour $i \in I$~:
    \begin{itemize}
        \item si $f(i) = \mr{Vrai}$ alors $\mc{A}(i) = f(i)$ et $\mc{A}$ termine sur $i$.
        \item si $f(i) = \mr{Faux}$, alors $\mc{A}(i) = f(i)$ et $\mc{A}$ termine ou bien $\mc{A}$ ne termine pas sur $i$. Ecrire systeme.
    \end{itemize}
\end{definition}

\begin{exemple}{}{important}
    Il existe des problèmes non semi-décidables. On considère par exemple le suivant~:
    \begin{itemize}
        \item \textbf{Instance}~: une fonction \code{f : string -> bool} et \code{f} son code source.
        \item \textbf{Question}~: est ce que l'appel \code{f code\_f} ne renvoie pas true ? \ie est ce que l'appel renvoie false ou bien ne termine pas ? N'a-t-on que ces deux possibilités ?
    \end{itemize}
    Montrons que ce problème n'est pas semi-décidable.\\\\
    Supposons par l'absurde qu'il existe un algorithme $\mc{A}$ implémenté par une fonction \code{diag : string -> bool} qui semi-décide le problème. Par définition~:
    \begin{enumerate}
        \item \code{diag code\_f} renvoie \code{true} lorsque \code{f code\_f} ne renvoie pas \code{true}
        \item \code{diag code\_f} renvoie \code{false} ou ne termine pas lorsque \code{f code\_f} revoie true.
    \end{enumerate}
    On applique la fonction \code{diag} à son propre code, noté \code{code\_diag}.
    \begin{itemize}
        \item Si \code{diag code\_diag} renvoie \code{false}~: par définition de \code{diag} (2.), on a \code{diag code\_diag} renvoie \code{true}. Il y a alors contradiction.
        \item Si \code{diag code\_diag} renvoie \code{true}. Par définition de \code{diag} (1.), on a \code{diag code\_diag} ne renvoie pas \code{true}. C'est également absurde.
        \item Si \code{diag code\_diag} ne termine pas ou échoue, par définition de \code{diag} (2.), \code{diag code\_diag} renvoie \code{true}. On a une contradiction.
    \end{itemize}
    Aucune possibilité n'est viable. D'où l'absurdité de l'hypothèse.
\end{exemple}

\subsection{Problème de l'arrêt (au programme)}

\begin{definition}{}{problème de l'arrêt}
    Le \notion{problème de l'arrêt} est le problème qui consiste à décider si un programme ou un algorithme termine sur une entrée.
    \begin{itemize}
        \item \textbf{Instance}~: un programme \code{p} donné par son code \code{code\_p} et une entrée \code{x}
        \item \textbf{Question}~: est ce que l'appel \code{p x} termine ?
    \end{itemize}
\end{definition}

\begin{theoreme}{}{indécidabilité du problème de l'arrêt}
    Le problème de l'arrêt est indécidable.
\end{theoreme}
La démonstration est la suivante. Elle est à connaître impérativement.\\\\
\begin{demonstration}
    Supposons par l'absurde qu'il existe $\mc{A}$ un algorithme implément par une fonction \code{arret} qui résout le problème de l'arrêt. Par définition de décidabilité~:
    \begin{enumerate}
        \item \code{arret code\_p x} renvoie \code{true} lorsque \code{p x} termine.
        \item \code{arret code\_p code\_x} renvoie \code{false} lorsque \code{p x} ne termine pas.
    \end{enumerate}
    On écrit~:
    \begin{lstOCaml}
        let rec boucle (b:bool) :int =
            match b with
            | true -> boucle true
            | false -> 0
    
        let absurde code_p = boucle (arret code_p code_p)
    \end{lstOCaml}
    
    On s'intéresse à l'appel \code{arret code\_absurde code\_absurde}.
    \begin{itemize}
        \item Si \code{arret code\_absurde code\_absurde} renvoie \code{true}, par définition de \code{arret} (1.), \code{absurde code\_absurde} termine.
        Or cet appel \code{absurde code\_absurde} correspond à \code{boucle (arret code\_absurde code\_absurde)} qui ne termine pas par construction, alors que \code{arret code\_absurde code\_absurde} renvoie \code{true} : ceci constitue une contradiction.
        \item Si \code{arret code\_absurde code\_absurde} renvoie \code{false}. Par définition de \code{arret} (2.), \code{absurde code\_absurde} ne termine pas.
        Or \code{absurde code\_absurde} correspond à \code{boucle (arret code\_absurde code\_absurde)} qui termine par construction, alors que \notion{arret code\_absurde code\_absurde} renvoie \code{false}. Contradiction.
    \end{itemize}
\end{demonstration}



\begin{remarque}{}{usage de chaînes de caractères}
    Dans les démonstrations, on préfèrera la manipulation de chaînes de caractères associées à ce qui n'en est pas. En effet, en machine, tout est représenté par des chaînes et en particulier les machines de Turing ne manipulent que ça. C'est plus formel.
\end{remarque}

Fin cours 12/02/2025

\section{Réduction (entre problèmes de décision)}

\newcommand{\algo}{\mc{A}}


\begin{definition}{}{fonction calculable}
    Une fonction $f : E \to F$ est dite \notion{calculable} lorsqu'il existe un algorithme $\algo$ tel que pour toute entrée $e \in E$, $\algo$ appliqué à $e$ termine et renvoie $f(e)$ en un temps fini.
\end{definition}

\begin{remarque}{}{importante sur les fonctions calculables}
    Il existe des fonctions qui ne sont pas calculables~:
    \begin{itemize}
        \item $\{0;1\}^\N$ est indénombrable
        \item l'ensemble des algorithmes est dénombrable, car on peut numéroter chaque possibilité du $i$-ème caractère, puis le produit cartésien d'ensembles dénombrables est dénombrables~:
        $$\abs{\algo} \leq \abs{\Sigma^*} = \abs{\bigcup_{n \in \N}\Sigma^n}$$
        On peut les énumérer par taille.
    \end{itemize}
\end{remarque}

\begin{remarque}{}{}
    Il existe moins d'algorithmes que de fonctions.
\end{remarque}
\newcommand{\pbm}{\mc{P}}


\begin{definition}{}{réduction calculatoire}
    Soit $\pbm_1$ et $\pbm_2$ deux problèmes de décision définis par leurs fonction de prédicat $f_1 : I_1 \to \{\mr{Vrai},\, \mr{Faux}\}$ et $f_2 : I_2 \to \{\mr{Vrai},\, \mr{Faux}\}$. On dit que \notion{$\pbm_1$ se réduit calculatoirement à $\pbm_2$} ($\pbm_1 \leq_m \pbm_2$), lorsqu'il existe $g : I_1 \to I_2$ une fonction calculable telle que~:
    $$\forall e \in I_1,\, f_1(e) = f_2\Big(g(e)\Big)$$
\end{definition}

\begin{remarque}{}{réduction calculatoire}
    $\pbm_1 \leq_m \pbm_2$ revient à dire que "$\pbm_1$ est plus facile que $\pbm_2$".\\
    On note $\algo_2$ un algorithme qui résout $\mc{P}_2$ et $\mc{A}_g$ un algorithme qui calcule la fonction $g$. On pose alors $\mc{A}_1$ l'algorithme~:
    \begin{lstLNat}
    $\algo_1$(e):
        e' = $\algo_g$(e)
        renvoyer $\algo_2$(e')
    \end{lstLNat}
    $\algo_1$ résout alors $\pbm_1$.
\end{remarque}

\begin{proposition}{}{}
    Soit $\pbm_1$ et $\pbm_2$ deux problèmes de décision définis par leurs fonction de prédicat $f_1 : I_1 \to \{\mr{Vrai},\, \mr{Faux}\}$ et $f_2 : I_2 \to \{\mr{Vrai},\, \mr{Faux}\}$. Si $\pbm_1 \leq_m \pbm_2$ et $\mc{P}_1$ est indécidable, alors $\pbm_2$ est indécidable.
\end{proposition}

\begin{demonstration}
    Supposons par l'absurde $\pbm_2$ décidable, alors il existe $\algo_2$ un algorithme qui résout $\pbm_2$. $\pbm_1 \leq_m \pbm_2$ donne l'existence de $g : I_1 \to I_2$ calculable telle que~:
    $$\forall e \in I_1,\, f_1(e) = f_2\Big(g(e)\Big)$$
    On construit $\mc{A_1}$ comme dans la remarque précédente, il résout $\mc{P_1}$. D'où la contradiction.
\end{demonstration}

\begin{exemple}{}{problème ZERO (V1)}
    Le problème ZERO est le suivant~:
    \begin{itemize}
        \item \textbf{Instance} : Une fonction (en programmation) \code{f} de code \code{code\_f} et une entrée \code{x}.
        \item \textbf{Question} : Est-ce que \code{f} appliquée à \code{x} renvoie \code{0} ?
    \end{itemize}
    On cherche à montrer que ZERO est indécidable. Pour cela on peut montrer~:
    $$\mr{ARRET} \leq_m \mr{ZERO}$$
    On pose~:
    $$g : (\code{f},\, \code{x}) \mapsto (\code{f'},\, \code{x})$$
    où \code{f'} est définie par~:
    \begin{lstLNat}
    let f' x = let _ = f x in 0
    \end{lstLNat}
    qui calcule \code{f x}, et renvoie \code{0}.\\\\

    $g$ est calculable, car un algorithme transformant $(\code{f},\, \code{x})$ en $(\code{f'},\, \code{x})$ est le suivant~:
    \begin{lstLNat}
    let g f =
        let f' x = let _ 0 = f x in 0
        in f'
    \end{lstLNat}
    \begin{itemize}
        \item Si $(\code{f},\, \code{x}) \in \mr{ARRET}^+$ (instance positive de ARRET), alors \code{f x} termine, puis \code{f' x} termine et renvoie \code{0}. On a bien $(\code{f'},\, \code{x}) \in \mr{ZERO}^+$
        \item Si $(\code{f},\, \code{x}) \notin \mr{ARRET}^+$, alors $\code{f x}$ ne termine pas, donc $\code{f' x}$ ne termine pas et ne renvoie pas \code{0}, donc $(\code{f'},\, \code{x}) \notin \mr{ZERO}^+$
    \end{itemize}
    On a $(\code{f},\, \code{x}) \in \mr{ARRET}^+ \Leftrightarrow (\code{f'},\, \code{x}) \in \mr{ZERO}^+$
\end{exemple}

\begin{exemple}{numero}{problème ZERO (V2)}
    à recopier
\end{exemple}

\begin{remarque}{}{many to one}
    Cette façon de faire des réductions entre problèmes s'appelle réduction "\notion{many to one}" car la fonction $g$ n'est pas toujours injective : on peut avoir plusieurs instances pour une seule et même sortie (en parlant de $g$).
\end{remarque}

\section{Complément HP : coproblème}

\newcommand{\copbm}{\mr{co}\mc{P}}


\begin{definition}{}{coproblème}
    Soit $\pbm$ un problème de décision défini par $f:I \to \{\mr{Vrai},\mr{Faux}\}$. On définit le \notion{coproblème} de $\pbm$, noté $\copbm$ par la fonction de prédicat~:
    \fonction{\mr{co}f}{I}{\{\mr{Vrai},\, \mr{Faux}\}}{e}{\begin{cases*}
    \mr{Vrai} &si $f(e) = \mr{Faux}$ \\
    \mr{Faux} &si $f(e) = \mr{Vrai}$
    \end{cases*}}
    Autrement dit, on fait la négation de la question correspondante.
\end{definition}

\begin{exemple}{}{coARRET}
    \begin{itemize}
        \item \textbf{Instance}~: un programme \code{p} donné par son code \code{code\_p} et une entrée \code{x}
        \item \textbf{Question}~: est ce que l'appel \code{p x} ne termine pas ?
    \end{itemize}
\end{exemple}

\begin{proposition}{}{}
    Soit $\pbm$ un problème de décision. Si $\pbm$ et $\copbm$ sont semi-décidable, alors ils sont décidables.
\end{proposition}

\begin{demonstration}
    On a $\algo$ et $co\algo$ deux algorithmes qui semi-décident $\mc{P}$ et $\copbm$.\\
    Pour une instance positive de $\pbm$, $\algo$ termine et pour une instance négative, c'est $co\algo$ termine. On les lance en parallèle~:
    \begin{lstLNat}
    semaphore s initialise à 0
    r variable globale
    RESOUT(e):
        r = $\mr{A}$(e)
        s.post()
    
    CORESOUT(e):
        r = NOT(co$\mc{A}$(e))
        s.post()

    ALGORITHME(e):
        T1 = fil réalisant RESOUT(e)
        T2 = fil réalisant CORESOUT(e)
        s.wait()
        renvoyer r
    \end{lstLNat}
\end{demonstration}

\begin{definition}{}{co-semi-décidabilité}
    Pour $\pbm$ un problème de décision, lorsque $\copbm$ est semi-décidable, on dit que \notion{$\mc{P}$ est co-semi-décidable}.
\end{definition}

\end{document}