\documentclass{article}
\usepackage{amsmath,amssymb,mathtools}
\usepackage{xcolor}
\usepackage{minted}
\usepackage{enumitem}
\usepackage{multicol}
\usepackage{changepage}
\usepackage{stmaryrd}
\usepackage{graphicx}
\graphicspath{ {./images/} }
\usepackage[framemethod=tikz]{mdframed}
\usepackage{tikz,pgfplots}
\pgfplotsset{compat=1.18}

% physique
\definecolor{oranges}{RGB}{255, 242, 230}
\definecolor{rouges}{RGB}{255, 230, 230}
\definecolor{rose}{RGB}{255, 204, 204}

% maths - info
\definecolor{rouge_fonce}{RGB}{204, 0, 0}
\definecolor{rouge}{RGB}{255, 0, 0}
\definecolor{bleu_fonce}{RGB}{0, 0, 255}
\definecolor{vert_fonce}{RGB}{0, 69, 33}
\definecolor{vert}{RGB}{0,255,0}

\definecolor{orange_foncee}{RGB}{255, 153, 0}
\definecolor{myrtille}{RGB}{225, 225, 255}
\definecolor{mayonnaise}{RGB}{255, 253, 233}
\definecolor{magenta}{RGB}{224, 209, 240}
\definecolor{pomme}{RGB}{204, 255, 204}
\definecolor{mauve}{RGB}{255, 230, 255}


% Cours

\newmdenv[
  nobreak=true,
  topline=true,
  bottomline=true,
  rightline=true,
  leftline=true,
  linewidth=0.5pt,
  linecolor=black,
  backgroundcolor=mayonnaise,
  innerleftmargin=10pt,
  innerrightmargin=10pt,
  innertopmargin=5pt,
  innerbottommargin=5pt,
  skipabove=\topsep,
  skipbelow=\topsep,
]{boite_definition}

\newmdenv[
  nobreak=false,
  topline=true,
  bottomline=true,
  rightline=true,
  leftline=true,
  linewidth=0.5pt,
  linecolor=white,
  backgroundcolor=white,
  innerleftmargin=10pt,
  innerrightmargin=10pt,
  innertopmargin=5pt,
  innerbottommargin=5pt,
  skipabove=\topsep,
  skipbelow=\topsep,
]{boite_exemple}

\newmdenv[
  nobreak=true,
  topline=true,
  bottomline=true,
  rightline=true,
  leftline=true,
  linewidth=0.5pt,
  linecolor=black,
  backgroundcolor=magenta,
  innerleftmargin=10pt,
  innerrightmargin=10pt,
  innertopmargin=5pt,
  innerbottommargin=5pt,
  skipabove=\topsep,
  skipbelow=\topsep,
]{boite_proposition}

\newmdenv[
  nobreak=true,
  topline=true,
  bottomline=true,
  rightline=true,
  leftline=true,
  linewidth=0.5pt,
  linecolor=black,
  backgroundcolor=white,
  innerleftmargin=10pt,
  innerrightmargin=10pt,
  innertopmargin=5pt,
  innerbottommargin=5pt,
  skipabove=\topsep,
  skipbelow=\topsep,
]{boite_demonstration}

\newmdenv[
  nobreak=true,
  topline=true,
  bottomline=true,
  rightline=true,
  leftline=true,
  linewidth=0.5pt,
  linecolor=white,
  backgroundcolor=white,
  innerleftmargin=10pt,
  innerrightmargin=10pt,
  innertopmargin=5pt,
  innerbottommargin=5pt,
  skipabove=\topsep,
  skipbelow=\topsep,
]{boite_remarque}


\newenvironment{definition}[2]
{
    \vspace{15pt}
    \begin{boite_definition}
    \textbf{\textcolor{rouge}{Définition #1}}
    \if\relax\detokenize{#2}\relax
    \else
        \textit{ - #2}
    \fi \\ \\
}
{
    \end{boite_definition}
    
}

\newenvironment{exemple}[2]
{
    \vspace{15pt}
    \begin{boite_exemple}
    \textbf{\textcolor{bleu_fonce}{Exemple #1}}
    \if\relax\detokenize{#2}\relax
    \else
        \textit{ - #2}
    \fi \\ \\ 
}
{   
    \end{boite_exemple}
    
}

\newenvironment{proposition}[2]
{
    \vspace{15pt}
    \begin{boite_proposition}
    \textbf{\textcolor{rouge}{Proposition #1}}
    \if\relax\detokenize{#2}\relax
    \else
        \textit{ - #2}
    \fi \\ \\
}
{
    \end{boite_proposition}
    
}

\newenvironment{theoreme}[2]
{
    \vspace{15pt}
    \begin{boite_proposition}
    \textbf{\textcolor{rouge}{Théorème #1}} 
    \if\relax\detokenize{#2}\relax
    \else
        \textit{ - #2}
    \fi \\ \\
}
{
    \end{boite_proposition}
    
}

\newenvironment{demonstration}
{
    \vspace{15pt}
    \begin{boite_demonstration}
    \textbf{\textcolor{rouge}{Démonstration}}\\ \\
}
{
    \end{boite_demonstration}
    
}

\newenvironment{remarque}[2]
{
    \vspace{15pt}
    \begin{boite_remarque}
    \textbf{\textcolor{bleu_fonce}{Remarque #1}}
    \if\relax\detokenize{#2}\relax
    \else
        \textit{ - #2}
    \fi \\ \\   
}
{  
    \end{boite_remarque}
    
}



%Corrections
\newmdenv[
  nobreak=true,
  topline=true,
  bottomline=true,
  rightline=true,
  leftline=true,
  linewidth=0.5pt,
  linecolor=black,
  backgroundcolor=mayonnaise,
  innerleftmargin=10pt,
  innerrightmargin=10pt,
  innertopmargin=5pt,
  innerbottommargin=5pt,
  skipabove=\topsep,
  skipbelow=\topsep,
]{boite_question}


\newenvironment{question}[2]
{
    \vspace{15pt}
    \begin{boite_question}
    \textbf{\textcolor{rouge}{Question #1}}
    \if\relax\detokenize{#2}\relax
    \else
        \textit{ - #2}
    \fi \\ \\
}
{
    \end{boite_question}
    
}

\newenvironment{enumeratebf}{
    \begin{enumerate}[label=\textbf{\arabic*.}]
}
{
    \end{enumerate}
}

\begin{document}
\begin{adjustwidth}{-3cm}{-3cm}
\begin{document}
\begin{adjustwidth}{-3cm}{-3cm}
% commandes
\newcommand{\notion}[1]{\textcolor{vert_fonce}{\textit{#1}}}
\newcommand{\mb}[1]{\mathbb{#1}}
\newcommand{\mc}[1]{\mathcal{#1}}
\newcommand{\mr}[1]{\mathrm{#1}}
\newcommand{\code}[1]{\texttt{#1}}
\newcommand{\ccode}[1]{\texttt{|#1|}}
\newcommand{\ov}[1]{\overline{#1}}
\newcommand{\abs}[1]{|#1|}
\newcommand{\rev}[1]{\texttt{reverse(#1)}}
\newcommand{\crev}[1]{\texttt{|reverse(#1)|}}

\newcommand{\ie}{\textit{i.e.} }

\newcommand{\N}{\mathbb{N}}
\newcommand{\R}{\mathbb{R}}
\newcommand{\C}{\mathbb{C}}
\newcommand{\K}{\mathbb{K}}
\newcommand{\Z}{\mathbb{Z}}

\newcommand{\A}{\mathcal{A}}
\newcommand{\bigO}{\mathcal{O}}
\renewcommand{\L}{\mathcal{L}}

\newcommand{\rg}[0]{\mathrm{rg}}
\newcommand{\re}[0]{\mathrm{Re}}
\newcommand{\im}[0]{\mathrm{Im}}
\newcommand{\cl}[0]{\mathrm{cl}}
\newcommand{\grad}[0]{\vec{\mathrm{grad}}}
\renewcommand{\div}[0]{\mathrm{div}\,}
\newcommand{\rot}[0]{\vec{\mathrm{rot}}\,}
\newcommand{\vnabla}[0]{\vec{\nabla}}
\renewcommand{\vec}[1]{\overrightarrow{#1}}
\newcommand{\mat}[1]{\mathrm{Mat}_{#1}}
\newcommand{\matrice}[1]{\mathcal{M}_{#1}}
\newcommand{\sgEngendre}[1]{\left\langle #1 \right\rangle}
\newcommand{\gpquotient}[1]{\mathbb{Z} / #1\mathbb{Z}}
\newcommand{\norme}[1]{||#1||}
\renewcommand{\d}[1]{\,\mathrm{d}#1}
\newcommand{\adh}[1]{\overline{#1}}
\newcommand{\intint}[2]{\llbracket #1 ,\, #2 \rrbracket}
\newcommand{\seg}[2]{[#1\, ; \, #2]}
\newcommand{\scal}[2]{( #1 | #2 )}
\newcommand{\distance}[2]{\mathrm{d}(#1,\,#2)}
\newcommand{\inte}[2]{\int_{#1}^{#2}}
\newcommand{\somme}[2]{\sum_{#1}^{#2}}
\newcommand{\deriveref}[4]{\biggl( \frac{\text{d}^{#1}#2}{\text{d}#3^{#1}} \biggr)_{#4}}






\begin{implementation}{tri par fusion}
    \begin{lstOCaml}
    let rec casser l =
        match l with
        | [] -> [], []
        | [e1] -> [e1], []
        | e1::e2::q -> 
            let l1, l2 = casser q in
            e1::l1, e2::l2

    let rec fusion l1 l2 = 
        match l1, l2 with
        | [], _ -> l2
        | _, [] -> l1
        | e1::q1, e2::q2 ->
            if e2 > e1 then
                e1::(fusion q1 l2)
            else
                e2::(fusion l1 q2)

    let rec tri_fusion l =
        match l with
        | [] -> []
        | [e1] -> [e1]
        | _ -> 
            let l1, l2 = casser l in
            fusion (tri_fusion l1) (tri_fusion l2)
    \end{lstOCaml}
\end{implementation}

\begin{implementation}{parcours en largeur d'un graphe (1/2)}
    \begin{lstOCaml}
    type graphe = int list array
    type file = {entrants:int list; sortants:int list}
        
    let file_vide () = {entrants = []; sortants = []}
                        
    let pop_opt f = 
        let rec retourne f1 =
            match f1.entrants with 
            | [] -> f1
            | e::q -> retourne {entrants=q; sortants = e::f1.sortants}
        in let f2 = if f.sortants = [] then retourne f else f
        in match f2.sortants with
        | [] -> file_vide (), None (* file vide *)
        | e::q -> {entrants=f2.entrants; sortants=q}, Some e
    \end{lstOCaml}
\end{implementation}

\begin{implementation}{parcours en largeur d'un graphe (2/2)}
    \begin{lstOCaml}
    let rec ajoute f liste = 
        match liste with 
        | [] -> f
        | e::q -> ajoute {entrants = e::f.entrants; sortants = f.sortants} q            
        
    let parcours_largeur g s = 
        let n = Array.length g in
        let non_vus = Array.make n true in
        let rec parcours f = 
            match pop_opt f with 
            | _, None -> []
            | f1, Some v when non_vus.(v) ->
                non_vus.(v) <- false;
                v::(parcours (ajoute f1 g.(v)))
            | f1, Some v ->
                parcours f1
        in parcours {entrants=[s]; sortants=[]}
    \end{lstOCaml}
\end{implementation}

\begin{implementation}{liste chainée en C (1/3)}
    \begin{lstC}
    typedef int elemtype;

    struct Maillon{
        elemtype val;
        struct Maillon* suivant;
    };
    typedef struct Maillon maillon;

    \end{lstC}
\end{implementation}

\begin{implementation}{liste chainée en C (2/3)}
    \begin{lstC}
    maillon* ajoute(elemtype x, maillon* c){
        maillon* res = malloc(sizeof(maillon));
        assert(res != NULL);
        res->val = x;
        res->suivant = c;
        return res;
    };
    \end{lstC}
\end{implementation}

\begin{implementation}{liste chainée en C (3/3)}
    \begin{lstC}
    int main(){
        maillon* a = ajoute(1,NULL);
        a = ajoute(2,a);
        a = ajoute(3,a);
        return 0;
    };
    \end{lstC}
\end{implementation}

\begin{implementation}{file d'entiers}
    \begin{lstC}
    struct Maillon{
        int val;
        struct Maillon* suivant;
    };
    typedef struct Maillon maillon;

    struct File{
        maillon* e; //maillon d'entrée
        maillon* s; //maillon de sortie
    };
    typedef struct File file;

    file* file_vide(){
        file* res = malloc(sizeof(file));
        assert(res != NULL);
        res->e = NULL;
        res->s = NULL;
        return res;
    }
    \end{lstC}
\end{implementation}

\begin{implementation}{file de priorité : type et fonction $\code{redim}$}
    \begin{lstOCaml}
    type tas_binaire_min = {
        mutable nb_elts:int; 
        mutable data: (char*int) array
    }

    let redim tbmin new_taille = 
        assert (new_taille >= tbmin.nb_elts);
        let new_data = Array.make new_taille ('\000',0) in
        for i=0 to tbmin.nb_elts - 1 do
            new_data.(i) <- tbmin.data.(i)
        done;
        tbmin.data <- new_data
    \end{lstOCaml}
\end{implementation}

\begin{implementation}{file de priorité : fonction $\code{percole\_haut}$}
    \begin{lstOCaml}
    let percole_haut tbmin i_depart =
        let rec percole i =
            let daron = if (i-1)/2 < 0 then 0 else (i-1)/2 in
            if prio tbmin.data.(daron) > prio tbmin.data.(i) then begin
                let temp = tbmin.data.(i) in
                tbmin.data.(i) <- tbmin.data.(daron);
                tbmin.data.(daron) <- temp;
                percole daron;
            end
        in if tbmin.nb_elts <> 0 then
            percole i_depart
    \end{lstOCaml}
\end{implementation}

\begin{implementation}{file de priorité : fonction $\code{percole\_bas}$}
    \begin{lstOCaml}
    let percole_bas tbmin i_depart =
        let rec percole i =
            let max = tbmin.nb_elts - 1 in
            let fils_g = if 2*i+1 <= max then 2*i+1 else max in
            let fils_d = if 2*i+2 <= max then 2*i+2 else max in
            if (prio tbmin.data.(fils_d) < prio tbmin.data.(i)|| 
                prio tbmin.data.(fils_g) < prio tbmin.data.(i)) then begin
                (*on va percoler le fils de plus basse priorité*)
                if prio tbmin.data.(fils_d) < prio tbmin.data.(fils_g) then
                    let temp = tbmin.data.(fils_d) in
                    tbmin.data.(fils_d) <- tbmin.data.(i);
                    tbmin.data.(i) <- temp;
                    percole fils_d
                else 
                    let temp = tbmin.data.(fils_g) in
                    tbmin.data.(fils_g) <- tbmin.data.(i);
                    tbmin.data.(i) <- temp;
                    percole fils_g
            end
        in if tbmin.nb_elts <> 0 then
            percole i_depart        
    \end{lstOCaml}
\end{implementation}

\begin{implementation}{file de priorité : type et fonction $\code{file\_vide}$}
    \begin{lstOCaml}
    type tas_binaire_min = {
        mutable nb_elts:int; 
        mutable data: (char*int) array
    }

    let tbmin_vide () = {nb_elts = 0; data = [||]}
    \end{lstOCaml}
\end{implementation}

\begin{implementation}{file de priorité : fonction $\code{ajoute}$ (qui remplace aussi)}
    \begin{lstOCaml}
    let tbmin_ajoute tbmin x p  =
        (*redimensionnement*)
        let n = Array.length tbmin.data in
        if tbmin.nb_elts >= n then 
            redim tbmin (2*n+1);
    
        (*vérification de la présence éventuelle de x*)
        let deja_present = ref false in (*indice de x si existence, sinon -1*)
        for i=0 to tbmin.nb_elts-1 do
            match tbmin.data.(i) with
            | (elt,prio) when elt = x -> 
                tbmin.data.(i) <- (x, p); 
                deja_present := true;
                if prio < p then 
                percole_bas tbmin i
                else if prio > p then
                percole_haut tbmin i
            | _ -> ()
        done;
    
        (*ajout et percolations vers le haut*)
        if not !deja_present then begin
            tbmin.data.(tbmin.nb_elts) <- (x,p);
            tbmin.nb_elts <- tbmin.nb_elts + 1;
            percole_haut tbmin (tbmin.nb_elts - 1)
        end
    \end{lstOCaml}
\end{implementation}

\begin{implementation}{file de priorité : fonction $\code{pop\_opt}$}
    \begin{lstOCaml}
    let tbmin_pop tbmin =
        if tbmin.nb_elts = 0 then None else
    
     (* remplacement de la première case par la dernière *)
        let res =  tbmin.data.(0) in 
        tbmin.data.(0) <- tbmin.data.(tbmin.nb_elts-1);
        tbmin.nb_elts <- tbmin.nb_elts - 1;
    
     (* redimensionnement *)
        let n = Array.length tbmin.data in
        if tbmin.nb_elts <= n/2 then 
            redim tbmin (n/2);
    
        (*percolations du nouveau premier élément*)
        percole_bas tbmin 0;
        Some res
    \end{lstOCaml}
\end{implementation}

\begin{implementation}{algorithme de Dijkstra (1/2)}
    On suppose implémentée la structure de file de priorité min.
    \begin{lstOCaml}
    let algo_dijkstra (g:graphe) (s:char) (t:char) =
        let non_vus = Array.make 256 true in
     (* le prédécesseur de chaque sommet *)
        let pred = Array.make 256 '\000' in 
     (* mémoïsation : distances de s à chaque sommet *)
        let dist = Array.make 256 max_int in 
        dist.(int_of_char s) <- 0;
        let file_p = tbmin_vide () in
        tbmin_ajoute file_p s 0;
    
        let rec reconstruire chemin (sommet:char) =
            match sommet with
            | '\000' -> chemin
            | _ -> reconstruire (sommet::chemin) pred.(int_of_char sommet)
    \end{lstOCaml}
\end{implementation}

\begin{implementation}{algorithme de Dijkstra (2/2)}
    \begin{lstOCaml}
        in let rec traitement u voisins =
         (* pour chaque voisin v de u, si favorable, 
            on remplace par la distance la plus courte puis on avance, 
            sinon on avance directement *)
            match voisins with
            | [] -> ()
            | (v,w)::q  when non_vus.(int_of_char v) && 
              dist.(int_of_char u) + w < dist.(int_of_char v) ->
                dist.(int_of_char v) <- dist.(int_of_char u) + w;
                tbmin_ajoute file_p v dist.(int_of_char v);
                pred.(int_of_char v) <- u;
                traitement u q
            | (v,w)::q -> traitement u q
        in let rec parcours () = 
            match tbmin_pop file_p with
            | None -> failwith "chemin inexistant"
            | Some (u,_) when u=t -> reconstruire [] u
            | Some (u,_) when non_vus.(int_of_char u) ->
                non_vus.(int_of_char u) <- false;
                traitement u g.(int_of_char u);
                parcours ()
            | Some _ -> parcours ()
        in parcours ()
    \end{lstOCaml}
\end{implementation}

\begin{implementation}{structure Unir et Trouver avec forêt, doublement optimisée (1/2)}
    \begin{lstOCaml}
    type pile_spaghetti = { 
        mutable parent: pile_spaghetti option; 
        mutable rang: int;
        valeur: int;
    }
    let creer x =
     (* création d'une classe d'équivalence *)
        {parent=None; rang=0; valeur=x}
    
    let rec trouver x =
     (* recherche du représentant
        on fait de la COMPRESSIONS DE CHEMIN *)
        match x.parent with
        | None -> x
        | Some e -> 
            let res = trouver e in
            x.parent <- Some res; (* applatissement de l'arbre *)
            res
    \end{lstOCaml}
\end{implementation}

\begin{implementation}{structure Unir et Trouver avec forêt, doublement optimisée (2/2)}
    \begin{lstOCaml}
    let unir x y =
     (* réunion de deux classes d'équivalence 
        on fait de l'UNION PAR RANG *)
        let parent_x = trouver x in
        let parent_y = trouver y in
        if parent_x <> parent_y then begin
            match parent_x, parent_y with
            | a,b when a.rang = b.rang ->
                a.parent <- Some b;
                b.rang <- b.rang + 1
            | a,b when a.rang < b.rang -> a.parent <- Some b
            | a,b when a.rang > b.rang -> b.parent <- Some a
            | _ -> () (* ne sert à rien *)
        end
    \end{lstOCaml}
\end{implementation}

\begin{implementation}{création du tableau \code{[1,2,3]} en C}
    \begin{lstC}
    int main(){
        int a[3] = {1,2,3};
        return 0;
    }
    \end{lstC}
\end{implementation}

\begin{implementation}{entraînement à la création d'une instance de type sur le segment de données}
    \begin{lstC}
    struct RegExp{
        char etiquette;
        struct RegExp *filsd;
        struct RegExp *filsg;
    };
    typedef struct RegExp RegExp;
    
    
    int main(){
        RegExp un = {
            .etiquette = 'a',
            .filsd = NULL,
            .filsg = NULL,
        };
        return 0;
    }
    \end{lstC}
\end{implementation}

\begin{implementation}{algorithme de Rabin-Karp}
    \begin{itemize}
        \item \textbf{Entrée} : \begin{itemize}
            \item un texte \code{t} sur un alphabet 
            \item un motif \code{m} sur un alphabet $\Sigma$, de longueur inférieure à  celle du texte.
        \end{itemize}
        \item \textbf{Sortie} : le nombre d'occurrences du motif dans le texte
    \end{itemize}
    On associe à un caractère $c \in \Sigma$ un entier naturel unique, qu'on notera également $c$. On note également $b = \abs{\Sigma}$. pour un mot $a_0 \dots a_{\abs{m}-1}$, on note~:
    $$h(t_0\dots t_{\abs{m}-1}) = \sum_{i=0}^{\abs{m}-1}t_i b^{\abs{m}-1-i}$$
    la fonction $h$ ainsi créée est injective  (par liberté de $(b^{\abs{m}-1},\dots, b^0)$). On va calculer l'image par $h$ du motif et la comparer avec chaque facteur de texte de même longueur que le motif (en effectuant un \notion{glissement} pour avancer de proche en proche). En cas d'égalité, on aura trouvé une occurrence par injectivité de $h$.\\\\
    \textbf{limitation} : le calcul de $h$ peut engendrer des dépassements de mémoire. On va donc calculer les hachés modulo $p$ un entier choisi~:\begin{itemize}
        \item \textbf{premier avec $b$} (ou juste premier) pour ne pas lourdement altérer l'écriture en base $b$.
        \item \textbf{grand} pour limiter les \notion{collisions} (cas de non unicité de l'image de $h$)
        \item \textbf{assez petit pour que le produit $bp$ soit représentable} (pour appliquer le modulo)
    \end{itemize}
    on perd alors l'injectivité de $h$. En cas d'égalité il faudra donc comparer les hachés.
    \begin{lstLNat}
    Rabin_Karp(m,t):
        $h_\code{m}$ = 0
        $h_\mr{courant}$ = 0
        pour $i = 0$ jusqu'à $\abs{m}-1$:
            $h_\code{m}$ = $b\times h_\code{m} \, +\, m_i$ mod $p$
            $h_\mr{courant}$ = $b\times h_\mr{courant} \, +\, t_i$ mod $p$
        compteur = 0
        b_puiss = $b^{m-1}$ mod $p$
        pour $i = 0$ jusqu'à $\abs{t}-\abs{m}$:
            si $h_\code{m}$ == $h_\mr{courant}$ et motif == $\code{t[i]}\dots \code{t[i+\abs{m}]}$: // évaluation paresseuse
                compteur = compteur + 1
            $h_\mr{courant}$ = $b \big( h_\mr{courant} - \code{b\_puiss} \times \code{t[i]} \big) + \code{t[i+\abs{m}]}$ mod $p$ // glissement
        renvoyer compteur


    \end{lstLNat}
\end{implementation}

\end{adjustwidth}
\end{document}
