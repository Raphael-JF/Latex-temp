\documentclass[a4paper,12pt]{article}

% Encodage et langue
\usepackage[utf8]{inputenc}
\usepackage[T1]{fontenc}
\usepackage[french]{babel}

\newcommand{\ttp}[0]{\underline{\textit{Traité Théologico-politique}} }

\newcommand{\lti}{\underline{\textit{Le Temps de l'innocence}} }


% Marges et mise en page
\usepackage[margin=2.5cm]{geometry}

% Titre du document
\title{Cours de Français\\ \Large \textit{\lti : Newland Archer, le conformiste malgré lui}}
\author{Raphaël JONTEF}
\date{\today}

\begin{document}
\maketitle

Dans ce contexte aristocrate, le regard singulier de Newland Archer va servir de point focal de la narration. Newland est à la fois le produit d'un mileu et conserve un regard critique sur celui-ci. Il en est le parangon mais incarne un regard critique.\\
Toute la tension du roman est contenu dans Newland Archer et son dilettantisme (amateurisme, opposé au professionnalisme) qui l'empêche d'actualiser (au sens de traduire en acte) son regarde critique.

\section{Newland : conformiste et moraliste}

\subsection{la reproduction sociale}

La reproduction sociale consiste dans un milieu social à secreter les mêmes conditionnements mentaux et physiques. (par exemple le corps ouvrier n'est pas du tout le même que le corps aristocratique).

Cette repoduction sociale est incarnée dans la scène initiale de l'Opéra. L'Opéra est un lieu de parade sociale : le spectacle n'a pas lieu que sur scène mais aussi dans le public. L'escalier d'entrée, tout pimpant qu'il est, est la scène construite pour cette ostentation sociale.\\\\

Archer (p.25) prétend vouloir une femme sensuelle mais exige (conformément aux codes sociaux) une femme vierge : c'est incompatible. Il "lorge" (mate) les femmes qui sont "les produits de ce système" (elles sont des choses, à cause du système). Il se prétend supérieur à ses pairs, mais pourtant il les suit car il sait qu'"il serait de mauvais goût de faire cavalier seul". Il n'est visiblement pas disposé à se passer de ses privilèges.\\
Le choix de May Welland comme épouse contribue à ce renoncement choisi. Pourquoi alors choisit-il May ?\\\\

Newland est le fleuron de la société, et May est la femme parfait au vu de la société. Il est favorable aux deux de se marrier, donc ils le font.

\subsection{le renversement anthropologique}

Il faut assurer la reproduction du milieu (donc ici, la reproduction tout court).\\
Newland Archer décrit son propre milieu comme un individu extérieur à celui-ci~:
\begin{itemize}
    \item à la fois avec un regard d'anthropologique
    \item à la fois avec un regard de moraliste (émet des jugements sur le comportement de ses contemporains). Le moraliste a souvent un comportement réactionnaire : "mes contemporains se comportent mal aujourd'hui : avant c'était mieux.". Ils dénoncent une société contemporaine corrompue par 
\end{itemize}

Archer juge ses contemporains avec du recul : il a de fait un regard de moraliste. Puis, le moraliste s'intéresse aux individus là où l'anthropologue étudie les groupe sociaux : Archer le fait aussi.\\
D'ailleurs l'anthropologie apparaît au début du $XX^{\mr{eme}}$ siècle.\\\\

Extrait du musée d'anthropologie (p.284) : Archer et Hélène se donnent rendez-vous au musée. Pourquoi ? Car c'est désert, la contemplation est permise (aujourd'hui ce n'est plus le cas). Olenska est vêtue de la tête au pieds de fourures, et Archer observe admire détail d'elle. Il ne regarde pas les oeuvres, c'est Olenska qu'il regarde.

\subsection{la distance satirique (!= de satyrique, propre au satyre) et la marginalisation d'Archer}

Archer se sent à la fois solidaire au vieux New-York et supérieurs pour des raisons culturelles à ses pairs. On sait qu'il se fait expédier à grands frais des ouvrages de Londres, qu'il connaît les peintres français. En cette fin du XIX eme, c'est l'excellence même, le summum de la culture.\\
Ce n'est qu'au milieu du XXème que les américains conçoivent un art propre à eux.

Pourtant Arche répugne à faire cavalier seul. Il est donc partagé.\\\\

Il est sur le point de sympathiser avec M. Rivière, un précepteur français et May l'en interdit car il n'est pas de leur milieu. D'ailleurs il ne rencontre que des américains expatriés en Europe, il ne parle pas aux autochtones.\\
Archer dénonce l'arbitraire des conventions sociales, dont notamment celle du mariage. Dans le cas des Beaufort, c'est parfaitement ça : il la trompe constamment, mais revient avec des bijoux donc c ok.\\
Archer dénonce aussi la condition féminine. Il voit très bien au moment de se fiancer que lui a une expérience antérieure et sa femme aucune. Il y a aussi hypocrisie dans le fait que le divorce ne soit jamais une option pour la comptesse (parce que ça dérogerait à l'ordre social et que le mariage est le pilier de cet ordre).\\\\
Newland prend conscience que c'est tout la société qui repose sur le mariage~:\\
"une morne association d'intérêts matériels"\\
Il ne s'agit que d'une affaire de patrimoine. Où est le progrès historique par rapport à Eschyles ? Nulle part, si ce n'est qu'on peut maintenant se marier avec quelqu'un d'autre que son cousin.

Ainsi Archer se marginalise. Il tient à sa mère et à sa soeur des propos féministes. Elle sont choquées, scandalisée qu'il défende la condition des femmes : il est typique des victimes d'un système que d'y adhérer pleinement.\\
Un jour Archer réagit face à Lawrence Lefert quand celui-ci tient des propos désobligeants vis-à-vis de la comptesse Olenska.\\\\
Le clan des Mingott-Welland finit par exclure Archer des tractations concernant le mariage d'Ellen. C'est le comble car il n'a jamais rien fait contre le clan : ce qui le trahit est que le clan veut qu'Ellen ne divorce pas. Le mari a l'air épouvantable (il veut la racheter, il la trompe) et Archer ne le comprend pas. Pourtant, Archer répercute les propos du clan, il va dire à Olenska ce qu'on lui dit de dire.\\\\
Au cours du roman on fait à Archer des propositions de travail.\\
D'abord Monsieur Rivière lui offre une nouvelle ébauche d'orientation intellectuelle. Il y renonce pourtant dès que sa femme l'en déconseille.\\
On lui propose aussi une vie politique, car dans son milieu ça ne se fait pas que de se mêler à la population.\\\\
Wharton dira d'Archer qu'il est passé à côté de "la fleur de la vie" (l'épanouissement amoureux).

\section{La prison du mariage}

\subsection{La lâcheté personnelle d'Archer a favorisé la conspiration de tout un milieu}

Cette conspiration apparâit lors du dîner d'adieu d'Ellen. Le camp Mingott-Welland organise se dîner pour maintenir les apparencese et se préserver. Archer y va en s'imaginant que tout n'est peut-être pas perdu, "qu'il n'a pas encore perdu toutes ses chances". Voir p.299 (bas)et 300. On assiste à la mutilation à la société d'Ellen Olenska.

Il y a une communauté masculine et une communauté féminine, bien soudée. On l'a vu avec cette société des hommes au début, et on le voit par le fait que May prévienne sa mère et sa grand-mère avant de prévenir son mari.







\end{document}