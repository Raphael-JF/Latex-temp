\documentclass{article}
\usepackage{amsmath,amssymb,mathtools}
\usepackage{xcolor}
\usepackage{minted}
\usepackage{enumitem}
\usepackage{multicol}
\usepackage{changepage}
\usepackage{stmaryrd}
\usepackage{graphicx}
\graphicspath{ {./images/} }
\usepackage[framemethod=tikz]{mdframed}
\usepackage{tikz,pgfplots}
\pgfplotsset{compat=1.18}

% physique
\definecolor{oranges}{RGB}{255, 242, 230}
\definecolor{rouges}{RGB}{255, 230, 230}
\definecolor{rose}{RGB}{255, 204, 204}

% maths - info
\definecolor{rouge_fonce}{RGB}{204, 0, 0}
\definecolor{rouge}{RGB}{255, 0, 0}
\definecolor{bleu_fonce}{RGB}{0, 0, 255}
\definecolor{vert_fonce}{RGB}{0, 69, 33}
\definecolor{vert}{RGB}{0,255,0}

\definecolor{orange_foncee}{RGB}{255, 153, 0}
\definecolor{myrtille}{RGB}{225, 225, 255}
\definecolor{mayonnaise}{RGB}{255, 253, 233}
\definecolor{magenta}{RGB}{224, 209, 240}
\definecolor{pomme}{RGB}{204, 255, 204}
\definecolor{mauve}{RGB}{255, 230, 255}


% Cours

\newmdenv[
  nobreak=true,
  topline=true,
  bottomline=true,
  rightline=true,
  leftline=true,
  linewidth=0.5pt,
  linecolor=black,
  backgroundcolor=mayonnaise,
  innerleftmargin=10pt,
  innerrightmargin=10pt,
  innertopmargin=5pt,
  innerbottommargin=5pt,
  skipabove=\topsep,
  skipbelow=\topsep,
]{boite_definition}

\newmdenv[
  nobreak=false,
  topline=true,
  bottomline=true,
  rightline=true,
  leftline=true,
  linewidth=0.5pt,
  linecolor=white,
  backgroundcolor=white,
  innerleftmargin=10pt,
  innerrightmargin=10pt,
  innertopmargin=5pt,
  innerbottommargin=5pt,
  skipabove=\topsep,
  skipbelow=\topsep,
]{boite_exemple}

\newmdenv[
  nobreak=true,
  topline=true,
  bottomline=true,
  rightline=true,
  leftline=true,
  linewidth=0.5pt,
  linecolor=black,
  backgroundcolor=magenta,
  innerleftmargin=10pt,
  innerrightmargin=10pt,
  innertopmargin=5pt,
  innerbottommargin=5pt,
  skipabove=\topsep,
  skipbelow=\topsep,
]{boite_proposition}

\newmdenv[
  nobreak=true,
  topline=true,
  bottomline=true,
  rightline=true,
  leftline=true,
  linewidth=0.5pt,
  linecolor=black,
  backgroundcolor=white,
  innerleftmargin=10pt,
  innerrightmargin=10pt,
  innertopmargin=5pt,
  innerbottommargin=5pt,
  skipabove=\topsep,
  skipbelow=\topsep,
]{boite_demonstration}

\newmdenv[
  nobreak=true,
  topline=true,
  bottomline=true,
  rightline=true,
  leftline=true,
  linewidth=0.5pt,
  linecolor=white,
  backgroundcolor=white,
  innerleftmargin=10pt,
  innerrightmargin=10pt,
  innertopmargin=5pt,
  innerbottommargin=5pt,
  skipabove=\topsep,
  skipbelow=\topsep,
]{boite_remarque}


\newenvironment{definition}[2]
{
    \vspace{15pt}
    \begin{boite_definition}
    \textbf{\textcolor{rouge}{Définition #1}}
    \if\relax\detokenize{#2}\relax
    \else
        \textit{ - #2}
    \fi \\ \\
}
{
    \end{boite_definition}
    
}

\newenvironment{exemple}[2]
{
    \vspace{15pt}
    \begin{boite_exemple}
    \textbf{\textcolor{bleu_fonce}{Exemple #1}}
    \if\relax\detokenize{#2}\relax
    \else
        \textit{ - #2}
    \fi \\ \\ 
}
{   
    \end{boite_exemple}
    
}

\newenvironment{proposition}[2]
{
    \vspace{15pt}
    \begin{boite_proposition}
    \textbf{\textcolor{rouge}{Proposition #1}}
    \if\relax\detokenize{#2}\relax
    \else
        \textit{ - #2}
    \fi \\ \\
}
{
    \end{boite_proposition}
    
}

\newenvironment{theoreme}[2]
{
    \vspace{15pt}
    \begin{boite_proposition}
    \textbf{\textcolor{rouge}{Théorème #1}} 
    \if\relax\detokenize{#2}\relax
    \else
        \textit{ - #2}
    \fi \\ \\
}
{
    \end{boite_proposition}
    
}

\newenvironment{demonstration}
{
    \vspace{15pt}
    \begin{boite_demonstration}
    \textbf{\textcolor{rouge}{Démonstration}}\\ \\
}
{
    \end{boite_demonstration}
    
}

\newenvironment{remarque}[2]
{
    \vspace{15pt}
    \begin{boite_remarque}
    \textbf{\textcolor{bleu_fonce}{Remarque #1}}
    \if\relax\detokenize{#2}\relax
    \else
        \textit{ - #2}
    \fi \\ \\   
}
{  
    \end{boite_remarque}
    
}



%Corrections
\newmdenv[
  nobreak=true,
  topline=true,
  bottomline=true,
  rightline=true,
  leftline=true,
  linewidth=0.5pt,
  linecolor=black,
  backgroundcolor=mayonnaise,
  innerleftmargin=10pt,
  innerrightmargin=10pt,
  innertopmargin=5pt,
  innerbottommargin=5pt,
  skipabove=\topsep,
  skipbelow=\topsep,
]{boite_question}


\newenvironment{question}[2]
{
    \vspace{15pt}
    \begin{boite_question}
    \textbf{\textcolor{rouge}{Question #1}}
    \if\relax\detokenize{#2}\relax
    \else
        \textit{ - #2}
    \fi \\ \\
}
{
    \end{boite_question}
    
}

\newenvironment{enumeratebf}{
    \begin{enumerate}[label=\textbf{\arabic*.}]
}
{
    \end{enumerate}
}

\begin{document}
\begin{adjustwidth}{-3cm}{-3cm}
\begin{document}
\begin{adjustwidth}{-3cm}{-3cm}
% commandes
\newcommand{\notion}[1]{\textcolor{vert_fonce}{\textit{#1}}}
\newcommand{\mb}[1]{\mathbb{#1}}
\newcommand{\mc}[1]{\mathcal{#1}}
\newcommand{\mr}[1]{\mathrm{#1}}
\newcommand{\code}[1]{\texttt{#1}}
\newcommand{\ccode}[1]{\texttt{|#1|}}
\newcommand{\ov}[1]{\overline{#1}}
\newcommand{\abs}[1]{|#1|}
\newcommand{\rev}[1]{\texttt{reverse(#1)}}
\newcommand{\crev}[1]{\texttt{|reverse(#1)|}}

\newcommand{\ie}{\textit{i.e.} }

\newcommand{\N}{\mathbb{N}}
\newcommand{\R}{\mathbb{R}}
\newcommand{\C}{\mathbb{C}}
\newcommand{\K}{\mathbb{K}}
\newcommand{\Z}{\mathbb{Z}}

\newcommand{\A}{\mathcal{A}}
\newcommand{\bigO}{\mathcal{O}}
\renewcommand{\L}{\mathcal{L}}

\newcommand{\rg}[0]{\mathrm{rg}}
\newcommand{\re}[0]{\mathrm{Re}}
\newcommand{\im}[0]{\mathrm{Im}}
\newcommand{\cl}[0]{\mathrm{cl}}
\newcommand{\grad}[0]{\vec{\mathrm{grad}}}
\renewcommand{\div}[0]{\mathrm{div}\,}
\newcommand{\rot}[0]{\vec{\mathrm{rot}}\,}
\newcommand{\vnabla}[0]{\vec{\nabla}}
\renewcommand{\vec}[1]{\overrightarrow{#1}}
\newcommand{\mat}[1]{\mathrm{Mat}_{#1}}
\newcommand{\matrice}[1]{\mathcal{M}_{#1}}
\newcommand{\sgEngendre}[1]{\left\langle #1 \right\rangle}
\newcommand{\gpquotient}[1]{\mathbb{Z} / #1\mathbb{Z}}
\newcommand{\norme}[1]{||#1||}
\renewcommand{\d}[1]{\,\mathrm{d}#1}
\newcommand{\adh}[1]{\overline{#1}}
\newcommand{\intint}[2]{\llbracket #1 ,\, #2 \rrbracket}
\newcommand{\seg}[2]{[#1\, ; \, #2]}
\newcommand{\scal}[2]{( #1 | #2 )}
\newcommand{\distance}[2]{\mathrm{d}(#1,\,#2)}
\newcommand{\inte}[2]{\int_{#1}^{#2}}
\newcommand{\somme}[2]{\sum_{#1}^{#2}}
\newcommand{\deriveref}[4]{\biggl( \frac{\text{d}^{#1}#2}{\text{d}#3^{#1}} \biggr)_{#4}}






\begin{definition}{5.1}{vecteur élémentaire}
    Soit $n \in \N$. On appelle vecteur élémentaire d'indice $i\in \intint{1}{n}$ le vecteur~:
    $$E_i =
    \begin{pmatrix}
    0 \\
    \vdots \\
    0 \\
    1 \\
    0 \\
    \vdots \\
    0
    \end{pmatrix}
    \begin{array}{c}
     \\
     \\
     \xleftarrow{\text{indice } i} \\
     \\
     \\
    \end{array}$$
    La famille $(E_i)_{i \in \intint{i}{n}}$ est alors la base canonique de $\R^n$. Puis, en remarquant que~:
    $$\forall (i,j) \in \intint{1}{n}^2,\, E_i \times E_j^\top = E_{i,j},$$
    On retrouve la base canonique de $\matrice{n}(\R)$.
\end{definition}

\begin{proposition}{5.2}{produit de matrices élémentaires}
    Soit $(i,j,k,l) \in \intint{1}{n}^4$. On a~:
    $$E_{i,j} \times E_{k,l} = \delta_{j,k} E_{i,l} $$
\end{proposition}

\begin{definition}{5.3}{somme directe d'espaces vectoriels}
    Soit $E$ un $\K$-espace vectoriel et $(F_1, \dots , F_p)$ une famille d'au moins deux sous-espaces vectoriels de $E$. La \notion{somme $\somme{i=1}{p}F_i$ est directe} lorsque la décomposition d'un élément de cette somme en somme d'élements de chaque sous-espace vectoriel existe et est unique~:
    $$\bigoplus_{i=1}^{p} F_i = \{x \in E,\, \exists!(x_1,\, \dots,\, x_p),\, x = x_1 + \dots + x_p\}$$
\end{definition}

\begin{proposition}{5.4}{caractérisation du caractère direct par la décomposition du neutre additif}
    Soit $E$ un $\K$-espace vectoriel et $(F_1, \dots , F_p)$ une famille d'au moins deux sous-espaces vectoriels de $E$. La somme $\somme{i=1}{p}F_i$ est directe si et seulement si $0_E$ se décompose de manière unique en la somme des neutres additifs des sous-espaces vectoriels (qui sont tous $0_E$)~:
    $$\forall (x_1,\, \dots,\, x_p) \in F_1 \times \dots \times F_p,\, \somme{i=1}{p}x_i = 0_E \implies x_1 = \dots = x_p = 0_E$$
\end{proposition}

\begin{proposition}{5.5}{caractérisation du caractère direct}
    Soit $E$ un $\K$-espace vectoriel et $(F_1, \dots , F_p)$ une famille d'au moins deux sous-espaces vectoriels de $E$. La somme $ \displaystyle  \somme{i=1}{p}F_i$ est directe si et seulement si~:
        $$ \bigg( \bigoplus_{i=1}^{p-1} F_i \bigg) \cap F_p = \{0_E\}.$$
\end{proposition}

\begin{proposition}{5.9}{dimension d'une somme d'espaces vectoriels de dimiension finie}
    Soit $E$ un $\K$-espace vectoriel et $(F_1, \dots , F_p)$ une famille d'au moins deux sous-espaces vectoriels de dimension finie de $E$. Alors~:
    $$\dim\bigg(\somme{i=1}{p}F_i\bigg) \leq \somme{i=1}{p}\dim(F_i)$$
De plus l’égalité est vérifiée si et seulement si la somme est directe.
\end{proposition}

\begin{proposition}{5.38}{stabilité des sous-algèbres de $\matrice{n}(\K)$ par prise d'inverse}
    Soit $\mc{A}$ une sous-algèbre de $\matrice{n}(\K)$. Toute matrice $M$ inversible de $\mc{A}$ a pour inverse un élément de $\mc{A}$.
\end{proposition}

\begin{definition}{5.41}{sous-algèbre engendrée par un endomorphisme}
    Soit $E$ un $\K$-espace vectoriel et $u \in \mc{L}(E)$. L'image du morphisme d'algèbres $P \mapsto P(u)$ est appelée \notion{sous-algèbre engendrée par $u$} et notée $\K[u]$.
\end{definition}

\begin{definition}{5.42}{polynôme caractéristique d'un endomorphisme}
    Soit $E$ un $\K$-espace vectoriel de dimension finie $n$ et $u \in \mc{L}(E)$. On appelle \notion{polynôme caractéristique de $u$} le polynôme~:
    $$\chi_u = \det(X \mr{id}_E - u)$$
    On dispose d'une définition tout à fait analogue pour une matrice $A \in \matrice{n}(\K)$~:
    $$\chi_u = \det(X I_n - A)$$
\end{definition}

\begin{proposition}{5.44}{expression du polynôme caractéristique d'un endomorphisme}
    Soit $E$ un $\K$-espace vectoriel de dimension finie $n$ et $u \in \mc{L}(E)$. On a~:
    $$\chi_u = X^n - \mr{tr}(u)X^{n-1} + \dots + (-1)^n\det(u)$$
    
\end{proposition}

\begin{theoreme}{5.47}{de décomposition des noyaux}
    Soit $E$ un $\K$-espace vectoriel, $u \in \mc{L}(E)$ et $(P,Q) \in \K[X]^2$ tel que $P \wedge Q = 1$. On a~:
    $$\ker\big(A(u)\big) \oplus \ker\big(B(u) \big) = \ker\big(AB(u)\big) $$
\end{theoreme}

\begin{definition}{5.49}{sous-espace propre associé à une valeur propre}
    Soit $E$ un $\K$-espace vectoriel, $u \in \mc{L}(E)$ et $\lambda$ une valeur propre de $u$. On appelle \notion{sous-espace propre associé à $\lambda$} l'espace $\ker(u - \lambda \mr{id}_E)$, supposément non réduit à $\{0_E\}$, sans quoi $\lambda$ ne serait valeur propre.
\end{definition}

\begin{theoreme}{5.52}{lien entre valeurs propres et racines du polynôme caractéristique}
    Soit $E$ un $\K$-espace vectoriel de dimension finie $n$ et $u \in \mc{L}(E)$. Les valeurs propres de $u$ sont exactement les racines de $\chi_u$~:
    $$\mr{Sp}(u) = Z(\chi_u)$$
\end{theoreme}

\begin{definition}{5.53}{multiplicité d'une valeur propre}
    Soit $E$ un $\K$-espace vectoriel de dimension finie $n$ et $u \in \mc{L}(E)$. La \notion{multiplicité $m_\lambda$ d'une valeur propre $\lambda$} de $u$ est sa multiplicité en tant que racine de $\chi_u$.
\end{definition}

\begin{theoreme}{5.63}{spectres de $u$ et $P(u)$}
    Soit $E$ un $\K$-espace vectoriel, $u \in \mc{L}(E)$ et $P \in \K[X]$. pour toute valeur propre $\lambda$ de $u$, $P(\lambda)$ est valeur propre de $P(u)$~:
    $$P\big(\mr{Sp}(u)\big) \subset \mr{Sp}\big(P(u)\big)$$
\end{theoreme}

\begin{proposition}{5.64}{lien entre valeurs propres et racines d'un polynôme annulateur}
    Soit $E$ un $\K$-espace vectoriel, $u \in \mc{L}(E)$ et $P \in \K[X]$ annulateur de $u$. Les valeurs propres de $u$ sont à rechecher parmi les racines de $P$~:
    $$\mr{Sp}(u) \subset Z(P)$$
    Ce résultat ne tient pas compte de la multiplicité.
\end{proposition}

\begin{proposition}{5.64}{lien entre valeurs propres et racines du polynôme minimal}
    Soit $E$ un $\K$-espace vectoriel et $u \in \mc{L}(E)$. Les valeurs propres de $u$ sont exactement les racines de $\mu_u$~:
    $$\mr{Sp}(u) = Z(\mu_u)$$
    Ce résultat ne tient pas compte de la multiplicité.
\end{proposition}

\begin{theoreme}{5.70}{de Cayley-Hamilton}
    Soit $E$ un $\K$-espace vectoriel de dimension finie et $u \in \mc{L}(E)$. $\chi_u$ est annulateur de $u$.
\end{theoreme}

\end{adjustwidth}
\end{document}