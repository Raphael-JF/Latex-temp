\documentclass{article}
\usepackage{amsmath,amssymb,mathtools}
\usepackage{xcolor}
\usepackage{minted}
\usepackage{enumitem}
\usepackage{multicol}
\usepackage{changepage}
\usepackage{stmaryrd}
\usepackage{graphicx}
\graphicspath{ {./images/} }
\usepackage[framemethod=tikz]{mdframed}
\usepackage{tikz,pgfplots}
\pgfplotsset{compat=1.18}

% physique
\definecolor{oranges}{RGB}{255, 242, 230}
\definecolor{rouges}{RGB}{255, 230, 230}
\definecolor{rose}{RGB}{255, 204, 204}

% maths - info
\definecolor{rouge_fonce}{RGB}{204, 0, 0}
\definecolor{rouge}{RGB}{255, 0, 0}
\definecolor{bleu_fonce}{RGB}{0, 0, 255}
\definecolor{vert_fonce}{RGB}{0, 69, 33}
\definecolor{vert}{RGB}{0,255,0}

\definecolor{orange_foncee}{RGB}{255, 153, 0}
\definecolor{myrtille}{RGB}{225, 225, 255}
\definecolor{mayonnaise}{RGB}{255, 253, 233}
\definecolor{magenta}{RGB}{224, 209, 240}
\definecolor{pomme}{RGB}{204, 255, 204}
\definecolor{mauve}{RGB}{255, 230, 255}


% Cours

\newmdenv[
  nobreak=true,
  topline=true,
  bottomline=true,
  rightline=true,
  leftline=true,
  linewidth=0.5pt,
  linecolor=black,
  backgroundcolor=mayonnaise,
  innerleftmargin=10pt,
  innerrightmargin=10pt,
  innertopmargin=5pt,
  innerbottommargin=5pt,
  skipabove=\topsep,
  skipbelow=\topsep,
]{boite_definition}

\newmdenv[
  nobreak=false,
  topline=true,
  bottomline=true,
  rightline=true,
  leftline=true,
  linewidth=0.5pt,
  linecolor=white,
  backgroundcolor=white,
  innerleftmargin=10pt,
  innerrightmargin=10pt,
  innertopmargin=5pt,
  innerbottommargin=5pt,
  skipabove=\topsep,
  skipbelow=\topsep,
]{boite_exemple}

\newmdenv[
  nobreak=true,
  topline=true,
  bottomline=true,
  rightline=true,
  leftline=true,
  linewidth=0.5pt,
  linecolor=black,
  backgroundcolor=magenta,
  innerleftmargin=10pt,
  innerrightmargin=10pt,
  innertopmargin=5pt,
  innerbottommargin=5pt,
  skipabove=\topsep,
  skipbelow=\topsep,
]{boite_proposition}

\newmdenv[
  nobreak=true,
  topline=true,
  bottomline=true,
  rightline=true,
  leftline=true,
  linewidth=0.5pt,
  linecolor=black,
  backgroundcolor=white,
  innerleftmargin=10pt,
  innerrightmargin=10pt,
  innertopmargin=5pt,
  innerbottommargin=5pt,
  skipabove=\topsep,
  skipbelow=\topsep,
]{boite_demonstration}

\newmdenv[
  nobreak=true,
  topline=true,
  bottomline=true,
  rightline=true,
  leftline=true,
  linewidth=0.5pt,
  linecolor=white,
  backgroundcolor=white,
  innerleftmargin=10pt,
  innerrightmargin=10pt,
  innertopmargin=5pt,
  innerbottommargin=5pt,
  skipabove=\topsep,
  skipbelow=\topsep,
]{boite_remarque}


\newenvironment{definition}[2]
{
    \vspace{15pt}
    \begin{boite_definition}
    \textbf{\textcolor{rouge}{Définition #1}}
    \if\relax\detokenize{#2}\relax
    \else
        \textit{ - #2}
    \fi \\ \\
}
{
    \end{boite_definition}
    
}

\newenvironment{exemple}[2]
{
    \vspace{15pt}
    \begin{boite_exemple}
    \textbf{\textcolor{bleu_fonce}{Exemple #1}}
    \if\relax\detokenize{#2}\relax
    \else
        \textit{ - #2}
    \fi \\ \\ 
}
{   
    \end{boite_exemple}
    
}

\newenvironment{proposition}[2]
{
    \vspace{15pt}
    \begin{boite_proposition}
    \textbf{\textcolor{rouge}{Proposition #1}}
    \if\relax\detokenize{#2}\relax
    \else
        \textit{ - #2}
    \fi \\ \\
}
{
    \end{boite_proposition}
    
}

\newenvironment{theoreme}[2]
{
    \vspace{15pt}
    \begin{boite_proposition}
    \textbf{\textcolor{rouge}{Théorème #1}} 
    \if\relax\detokenize{#2}\relax
    \else
        \textit{ - #2}
    \fi \\ \\
}
{
    \end{boite_proposition}
    
}

\newenvironment{demonstration}
{
    \vspace{15pt}
    \begin{boite_demonstration}
    \textbf{\textcolor{rouge}{Démonstration}}\\ \\
}
{
    \end{boite_demonstration}
    
}

\newenvironment{remarque}[2]
{
    \vspace{15pt}
    \begin{boite_remarque}
    \textbf{\textcolor{bleu_fonce}{Remarque #1}}
    \if\relax\detokenize{#2}\relax
    \else
        \textit{ - #2}
    \fi \\ \\   
}
{  
    \end{boite_remarque}
    
}



%Corrections
\newmdenv[
  nobreak=true,
  topline=true,
  bottomline=true,
  rightline=true,
  leftline=true,
  linewidth=0.5pt,
  linecolor=black,
  backgroundcolor=mayonnaise,
  innerleftmargin=10pt,
  innerrightmargin=10pt,
  innertopmargin=5pt,
  innerbottommargin=5pt,
  skipabove=\topsep,
  skipbelow=\topsep,
]{boite_question}


\newenvironment{question}[2]
{
    \vspace{15pt}
    \begin{boite_question}
    \textbf{\textcolor{rouge}{Question #1}}
    \if\relax\detokenize{#2}\relax
    \else
        \textit{ - #2}
    \fi \\ \\
}
{
    \end{boite_question}
    
}

\newenvironment{enumeratebf}{
    \begin{enumerate}[label=\textbf{\arabic*.}]
}
{
    \end{enumerate}
}

\begin{document}
\begin{adjustwidth}{-3cm}{-3cm}
\begin{document}
\begin{adjustwidth}{-3cm}{-3cm}
% commandes
\newcommand{\notion}[1]{\textcolor{vert_fonce}{\textit{#1}}}
\newcommand{\mb}[1]{\mathbb{#1}}
\newcommand{\mc}[1]{\mathcal{#1}}
\newcommand{\mr}[1]{\mathrm{#1}}
\newcommand{\code}[1]{\texttt{#1}}
\newcommand{\ccode}[1]{\texttt{|#1|}}
\newcommand{\ov}[1]{\overline{#1}}
\newcommand{\abs}[1]{|#1|}
\newcommand{\rev}[1]{\texttt{reverse(#1)}}
\newcommand{\crev}[1]{\texttt{|reverse(#1)|}}

\newcommand{\ie}{\textit{i.e.} }

\newcommand{\N}{\mathbb{N}}
\newcommand{\R}{\mathbb{R}}
\newcommand{\C}{\mathbb{C}}
\newcommand{\K}{\mathbb{K}}
\newcommand{\Z}{\mathbb{Z}}

\newcommand{\A}{\mathcal{A}}
\newcommand{\bigO}{\mathcal{O}}
\renewcommand{\L}{\mathcal{L}}

\newcommand{\rg}[0]{\mathrm{rg}}
\newcommand{\re}[0]{\mathrm{Re}}
\newcommand{\im}[0]{\mathrm{Im}}
\newcommand{\cl}[0]{\mathrm{cl}}
\newcommand{\grad}[0]{\vec{\mathrm{grad}}}
\renewcommand{\div}[0]{\mathrm{div}\,}
\newcommand{\rot}[0]{\vec{\mathrm{rot}}\,}
\newcommand{\vnabla}[0]{\vec{\nabla}}
\renewcommand{\vec}[1]{\overrightarrow{#1}}
\newcommand{\mat}[1]{\mathrm{Mat}_{#1}}
\newcommand{\matrice}[1]{\mathcal{M}_{#1}}
\newcommand{\sgEngendre}[1]{\left\langle #1 \right\rangle}
\newcommand{\gpquotient}[1]{\mathbb{Z} / #1\mathbb{Z}}
\newcommand{\norme}[1]{||#1||}
\renewcommand{\d}[1]{\,\mathrm{d}#1}
\newcommand{\adh}[1]{\overline{#1}}
\newcommand{\intint}[2]{\llbracket #1 ,\, #2 \rrbracket}
\newcommand{\seg}[2]{[#1\, ; \, #2]}
\newcommand{\scal}[2]{( #1 | #2 )}
\newcommand{\distance}[2]{\mathrm{d}(#1,\,#2)}
\newcommand{\inte}[2]{\int_{#1}^{#2}}
\newcommand{\somme}[2]{\sum_{#1}^{#2}}
\newcommand{\deriveref}[4]{\biggl( \frac{\text{d}^{#1}#2}{\text{d}#3^{#1}} \biggr)_{#4}}





\newcounter{chapitre}
\setcounter{chapitre}{numéro}

\begin{definition}{}{opérateur $\vnabla$}
    $\vnabla$ est un \notion{opérateur différentiel vectoriel}.
    \begin{enumeratebf}
        \item dans la base cartésienne : $$ \displaystyle \vnabla = \frac{\partial}{\partial x}\vec{u_x} + \frac{\partial}{\partial y}\vec{u_y} + \frac{\partial}{\partial z}\vec{u_z}$$
        \item dans la base polaire : $$ \vnabla =  \displaystyle {\partial \over \partial r}\overrightarrow u_r
        + {1 \over r}{\partial \over \partial \theta }\overrightarrow u_\theta
        + {\partial \over \partial z} \overrightarrow u_z$$
        \item dans la base sphérique : $$ \vnabla = \displaystyle {\partial \over \partial r}\overrightarrow u_r
        + {1 \over r}  {\partial \over \partial \theta}\overrightarrow u_\theta
        + {1 \over r\sin\theta} {\partial \over \partial \varphi}\overrightarrow u_\varphi$$
    \end{enumeratebf}
\end{definition}

\begin{definition}{}{gradient d'une fonction scalaire}
    le \notion{gradient d'une fonction scalaire} $f$ correspond au produit du vecteur $\vnabla$ par $f$~:
    $$\grad f = \vnabla f$$
\end{definition}

\begin{definition}{}{divergence d'une fonction vectorielle}
    la \notion{divergence d'une fonction vectorielle} $\vec{A}$ correspond au produit scalaire de $\vnabla$ par $\vec{A}$~:
    $$\div \vec{A} = \vnabla \cdot \vec{A}$$
\end{definition}

\begin{definition}{}{rotationnel d'une fonction vectorielle}
    Le \notion{rotationnel d'une fonction vectorielle} $\vec{A}$ correspond au produit vectoriel de $\vnabla$ par $\vec{A}$~:
    $$\rot \vec{A} = \vnabla \wedge \vec{A}$$
\end{definition}

\begin{definition}{}{laplacien d'une fonction scalaire en coordonnées cartésiennes}
    Le \notion{laplacien d'une fonction scalaire} $f$ est, en coordonnées cartésiennes~:
    $$\Delta f = \derivepar{2}{f}{x} + \derivepar{2}{f}{y} + \derivepar{2}{f}{z}$$
\end{definition}

\begin{theoreme}{}{de Green-Ostrogradski}
    Soit $\mc{S}$ une surface orientée vers l'extérieur délimitant un volume $\mc{V}$. Pour toute fonction vectorielle $\vec{A}$ de classe $\mc{C}^1$ de l'espace~:
    $$\iiint_\mc{V}(\div \vec{A})\d{\tau} = \oiint_\mc{S}\vec{A} \cdot \vec{\d{S}}$$
\end{theoreme}

\begin{theoreme}{}{équation locale de Maxwell-Gauss}
    En tout point $M$ de l'espace~:
    $$\div \vec{E} = \frac{\rho}{\epsilon_0}$$
    avec $\begin{cases*}
        $\epsilon_0$ \text{ la permittivité diélectrique du vide} \\
        $\rho$ \text{ la densité volumique de charge}
    \end{cases*}$
\end{theoreme}



\begin{definition}{}{surface de Gauss}
    On appelle \notion{surface de Gauss} un objet topologique~:
    \begin{enumeratebf}
        \item fermé ;
        \item comportant le point $M$ d'étude ;
        \item facilitant les calculs sachant les symétries et variance de la distribution des charges.
    \end{enumeratebf}
\end{definition}

\begin{theoreme}{}{théorème de Gauss}
    Le flux $\Phi$ du champ électrostatique sortant d’une surface de Gauss $\Sigma_G$ est relié à la charge $Q_\mr{int}$ contenue à l’intérieur de cette surface~:
    $$\Phi = \oiint_{\Sigma_G} \vec{E} \cdot \d{\vec{S}} = \frac{Q_\mr{int}}{\epsilon_0}$$
    avec $\begin{cases*}
        $\epsilon_0$ \text{ la permittivité diélectrique du vide} \\
        $Q_\mr{int} = \iiint_{V(\Sigma_G)} \rho \d{\tau}$ \text{ la charge contenue à l’intérieur de $\Sigma_G$}\\
        $\rho$ \text{ la densité volumique de charge}

    \end{cases*}$
\end{theoreme}


\end{adjustwidth}
\end{document}