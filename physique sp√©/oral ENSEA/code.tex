\documentclass[11pt]{article}
\usepackage[utf8]{inputenc}
\usepackage[T1]{fontenc}
\usepackage{lmodern}
\usepackage[french]{babel}
\usepackage{amsmath, amssymb, amsthm}
\usepackage{enumitem}
\usepackage{geometry}
\geometry{margin=2.5cm}
\usepackage{xcolor}
\usepackage{hyperref}
\hypersetup{
    colorlinks=true,
    linkcolor=blue,
    urlcolor=cyan
}

% Commandes perso utiles
\newcommand{\R}{\mathbb{R}}
\newcommand{\N}{\mathbb{N}}
\newcommand{\Z}{\mathbb{Z}}
\newcommand{\C}{\mathbb{C}}
\newcommand{\ds}{\displaystyle}
\newcommand{\vect}[1]{\vec{\mathbf{#1}}}
\newcommand{\e}{\mathrm{e}}
\newcommand{\ii}{\mathrm{i}}
\newcommand{\dd}{\,\mathrm{d}}

\title{Planche d'oral de Physique de l’ENSEA}
\author{MPI Session 2025}
\date{}

\begin{document}

\maketitle

\section*{Question de cours}

« Interféromètre de Michelson réglé en coin d'air, franges d'égale épaisseur. »

Difficile de comprendre ce qui est exactement attendu. En fournissant le calcul de la différence de marche $\delta = 2\alpha x $. En y ajoutant l'expression de l'éclairement résultant, et de la condition d'interférence constructive, le correcteur semble satisfait, \underline{même si la condition que je trouve est fausse} à un facteur près.

\section*{Énoncé de l'exercice}
On considère un cylindre d'axe $(Oz)$, de hauteur $H$ et de rayon $R\ll H $, chargé uniformément d'une densité surfacique de charge $\sigma_0$. \textit{On donne les équations de Maxwell dans le cas général}.
\begin{enumerate}[label=\textbf{\arabic*.}]
    \item Qu'advient-il des équations de Maxwell en électrostatique ?
    \item Calculer le champ électrostatique $\vec{E}(M)$ de deux manières différentes en tout point $M$ de l'espace.
    \item En déduire le potentiel électrostatique $V(M)$ en tout point de l'espace.
\end{enumerate}

\vspace{1cm}

\section*{Remarques personnelles}
\textit{Examinateur sur son portable, semblant inintéressé et acquiesce systématiquement sans rien ajouter ni rectifier. Me laisse patauger sur les conditions limites du potentiel\dots}

\end{document}
