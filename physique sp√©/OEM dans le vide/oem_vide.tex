\documentclass{article}
\usepackage{titling} % Personnalisation du titre
\usepackage[left=20mm, right=20mm]{geometry}
\usepackage{amsmath,amssymb,mathtools}
\usepackage{esint} % intégrale avec un round
\usepackage{xcolor}
\usepackage[utf8]{inputenc}
\usepackage{listings}
\usepackage{enumitem}
\usepackage{multicol}
\usepackage{stmaryrd}
\usepackage{graphicx}
\graphicspath{ {./images/} }
\usepackage[framemethod=tikz]{mdframed}
\usepackage{tikz,pgfplots}
\pgfplotsset{compat=1.18}
\usetikzlibrary{arrows}
\usepackage{forest}
\usepackage{titlesec}
\setlength{\parindent}{0pt}

\pretitle{\begin{center}\Huge\bfseries}
\posttitle{\end{center}}
\date{}
\renewcommand{\thesection}{\Roman{section}} 
\titleformat{\section}
  {\Large\bfseries} % Style du titre
  {\thesection} % Numéro de section
  {0.5cm} % Espacement entre numéro et titre
  {} % Pas de préfixe supplémentaire
\titleformat{\subsection}
  {\large\bfseries} % Style du titre
  {\thesubsection} % Numéro de section
  {0.4cm} % Espacement entre numéro et titre
  {} % Pas de préfixe supplémentair


\newenvironment{enumeratebf}{
    \begin{enumerate}[label=\textbf{\arabic*.}]
}
{
    \end{enumerate}
}
  
\definecolor{oranges}{RGB}{255, 242, 230}
\definecolor{rouges}{RGB}{255, 230, 230}
\definecolor{rose}{RGB}{255, 204, 204}

% maths - info
\definecolor{rouge_fonce}{RGB}{204, 0, 0}
\definecolor{rouge}{RGB}{255, 0, 0}
\definecolor{bleufonce}{RGB}{0, 0, 255}
\definecolor{vert_fonce}{RGB}{0, 69, 33}
\definecolor{vert}{RGB}{0,255,0}

\definecolor{orange_foncee}{RGB}{255, 153, 0}
\definecolor{myrtille}{RGB}{225, 225, 255}
\definecolor{mayonnaise}{RGB}{255, 253, 233}
\definecolor{magenta}{RGB}{224, 209, 240}
\definecolor{pomme}{RGB}{204, 255, 204}
\definecolor{mauve}{RGB}{255, 230, 255}


% Cours

\newmdenv[
    nobreak=true,
    topline=true,
    bottomline=true,
    rightline=true,
    leftline=true,
    linewidth=0.5pt,
    linecolor=black,
    backgroundcolor=mayonnaise,
    innerleftmargin=10pt,
    innerrightmargin=2.5em,
    innertopmargin=5pt,
    innerbottommargin=5pt,
    skipabove=-0.5cm,
    skipbelow=-0.25cm,
]{boite_definition}

\newcounter{boite}
\setcounter{boite}{1}
\newenvironment{definition}[2]
{
    \vspace{15pt}
    \begin{boite_definition}
    \if\relax\detokenize{#1}\relax
        \textbf{\textcolor{rouge}{Définition \arabic{chapitre}.\arabic{boite}}}%
        \if\relax\detokenize{#2}\relax
        \else
            \textit{ - #2}
        \fi
        \stepcounter{boite}
    \else
        \textbf{\textcolor{rouge}{Définition #1}}%
        \if\relax\detokenize{#2}\relax
        \else
            \textit{ - #2}
        \fi
    \fi \\
    
    
}
{
    \end{boite_definition}
}

\newmdenv[
  nobreak=true,
  topline=true,
  bottomline=true,
  rightline=true,
  leftline=true,
  linewidth=0.5pt,
  linecolor=white,
  backgroundcolor=white,
  innerleftmargin=10pt,
  innerrightmargin=2.5em,
  innertopmargin=5pt,
  innerbottommargin=5pt,
  skipabove=-0.5cm,
  skipbelow=-0.25cm,
]{boite_exemple}


\newenvironment{exemple}[2]
{
    \vspace{15pt}
    \begin{boite_exemple}
    \if\relax\detokenize{#1}\relax
        \textbf{\textcolor{bleufonce}{Exemple \arabic{chapitre}.\arabic{boite}}}%
        \if\relax\detokenize{#2}\relax
        \else
            \textit{ - #2}
        \fi
        \stepcounter{boite}
    \else
        \textbf{\textcolor{bleufonce}{Exemple #1}}%
        \if\relax\detokenize{#2}\relax
        \else
            \textit{ - #2}
        \fi
    \fi \\
    
    
}
{
    \end{boite_exemple}
}

\newmdenv[
  nobreak=true,
  topline=true,
  bottomline=true,
  rightline=true,
  leftline=true,
  linewidth=0.5pt,
  linecolor=black,
  backgroundcolor=magenta,
  innerleftmargin=10pt,
  innerrightmargin=2.5em,
  innertopmargin=5pt,
  innerbottommargin=5pt,
  skipabove=-0.5cm,
  skipbelow=-0.25cm,
]{boite_proposition}

\newenvironment{proposition}[2]
{
    \vspace{15pt}
    \begin{boite_proposition}
    \if\relax\detokenize{#1}\relax
        \textbf{\textcolor{rouge}{Proposition \arabic{chapitre}.\arabic{boite}}}%
        \if\relax\detokenize{#2}\relax
        \else
            \textit{ - #2}
        \fi
        \stepcounter{boite}
    \else
        \textbf{\textcolor{rouge}{Proposition #1}}%
        \if\relax\detokenize{#2}\relax
        \else
            \textit{ - #2}
        \fi
    \fi \\
    
    
}
{
    \end{boite_proposition}
}

\newmdenv[
  nobreak=true,
  topline=true,
  bottomline=true,
  rightline=true,
  leftline=true,
  linewidth=0.5pt,
  linecolor=black,
  backgroundcolor=magenta,
  innerleftmargin=10pt,
  innerrightmargin=2.5em,
  innertopmargin=5pt,
  innerbottommargin=5pt,
  skipabove=-0.5cm,
  skipbelow=-0.25cm,
]{boite_theoreme}


\newenvironment{theoreme}[2]
{
    \vspace{15pt}
    \begin{boite_theoreme}
    \if\relax\detokenize{#1}\relax
        \textbf{\textcolor{rouge}{Théorème \arabic{chapitre}.\arabic{boite}}}%
        \if\relax\detokenize{#2}\relax
        \else
            \textit{ - #2}
        \fi
        \stepcounter{boite}
    \else
        \textbf{\textcolor{rouge}{Théorème #1}}%
        \if\relax\detokenize{#2}\relax
        \else
            \textit{ - #2}
        \fi
    \fi \\
    
    
}
{
    \end{boite_theoreme}
}


\newmdenv[
  nobreak=true,
  topline=true,
  bottomline=true,
  rightline=true,
  leftline=true,
  linewidth=0.5pt,
  linecolor=black,
  backgroundcolor=white,
  innerleftmargin=10pt,
  innerrightmargin=2.5em,
  innertopmargin=5pt,
  innerbottommargin=5pt,
  skipabove=-0.5cm,
  skipbelow=-0.25cm,
]{boite_demonstration}


\newenvironment{demonstration}
{
    \vspace{15pt}
    \begin{boite_demonstration}
    \textbf{\textcolor{rouge}{Démonstration}}\\ \\
}
{
    \end{boite_demonstration}
    
}


\newmdenv[
  nobreak=true,
  topline=true,
  bottomline=true,
  rightline=true,
  leftline=true,
  linewidth=0.5pt,
  linecolor=white,
  backgroundcolor=white,
  innerleftmargin=10pt,
  innerrightmargin=2.5em,
  innertopmargin=5pt,
  innerbottommargin=5pt,
  skipabove=-0.5cm,
  skipbelow=-0.25cm,
]{boite_remarque}


\newenvironment{remarque}[2]
{
    \vspace{15pt}
    \begin{boite_remarque}
    \if\relax\detokenize{#1}\relax
        \textbf{\textcolor{bleufonce}{Remarque \arabic{chapitre}.\arabic{boite}}}%
        \if\relax\detokenize{#2}\relax
        \else
            \textit{ - #2}
        \fi
        \stepcounter{boite}
    \else
        \textbf{\textcolor{bleufonce}{Remarque #1}}%
        \if\relax\detokenize{#2}\relax
        \else
            \textit{ - #2}
        \fi
    \fi \\
    
    
}
{
    \end{boite_remarque}
}

\newmdenv[
  nobreak=true,
  topline=true,
  bottomline=true,
  rightline=true,
  leftline=true,
  linewidth=0.5pt,
  linecolor=bleufonce,
  backgroundcolor=white,
  innerleftmargin=10pt,
  innerrightmargin=2.5em,
  innertopmargin=5pt,
  innerbottommargin=5pt,
  skipabove=-0.5cm,
  skipbelow=-0.25cm,
]{boite_implementation}


\definecolor{keywordcolor}{RGB}{133, 153, 0}  % les mots-clés
\definecolor{commentcolor}{RGB}{147, 161, 161} % les commentaires
\definecolor{stringcolor}{RGB}{42, 161, 152}  % les chaînes de caractères

\lstnewenvironment{lstOCaml}
{\lstset{
    language=[Objective]Caml,
    basicstyle=\ttfamily,
    keywordstyle=\color{keywordcolor},
    commentstyle=\color{commentcolor},
    stringstyle=\color{stringcolor},
    backgroundcolor=\color{white},
    numbers=left,
    numberstyle=\ttfamily,
    numbersep=-1.5em,
    stepnumber=1,
    frame=l,
    framexleftmargin=-2.25em,
    tabsize=2,
    literate=%
    {é}{{\'e}}{1}%
    {è}{{\`e}}{1}%
    {à}{{\`a}}{1}%
    {ç}{{\c{c}}}{1}%
    {œ}{{\oe}}{1}%
    {ù}{{\`u}}{1}%
    {É}{{\'E}}{1}%
    {È}{{\`E}}{1}%
    {À}{{\`A}}{1}%
    {Ç}{{\c{C}}}{1}%
    {Œ}{{\OE}}{1}%
    {Ê}{{\^E}}{1}%
    {ê}{{\^e}}{1}%
    {î}{{\^i}}{1}%
    {ô}{{\^o}}{1}%
    {û}{{\^u}}{1}%
    {ä}{{\"{a}}}1
    {ë}{{\"{e}}}1
    {ï}{{\"{i}}}1
    {ö}{{\"{o}}}1
    {ü}{{\"{u}}}1
    {û}{{\^{u}}}1
    {â}{{\^{a}}}1
    {Â}{{\^{A}}}1
    {Î}{{\^{I}}}1
}}{}
 
\lstnewenvironment{lstC}
{\lstset{
    language=C,
    basicstyle=\ttfamily,
    keywordstyle=\color{keywordcolor},
    commentstyle=\color{commentcolor},
    stringstyle=\color{stringcolor},
    backgroundcolor=\color{white},
    numbers=left,
    numberstyle=\ttfamily,
    numbersep=-1.5em,
    stepnumber=1,
    frame=l,
    framexleftmargin=-2.25em,
    tabsize=2,
    literate=%
    {é}{{\'e}}{1}%
    {è}{{\`e}}{1}%
    {à}{{\`a}}{1}%
    {ç}{{\c{c}}}{1}%
    {œ}{{\oe}}{1}%
    {ù}{{\`u}}{1}%
    {É}{{\'E}}{1}%
    {È}{{\`E}}{1}%
    {À}{{\`A}}{1}%
    {Ç}{{\c{C}}}{1}%
    {Œ}{{\OE}}{1}%
    {Ê}{{\^E}}{1}%
    {ê}{{\^e}}{1}%
    {î}{{\^i}}{1}%
    {ô}{{\^o}}{1}%
    {û}{{\^u}}{1}%
    {ä}{{\"{a}}}1
    {ë}{{\"{e}}}1
    {ï}{{\"{i}}}1
    {ö}{{\"{o}}}1
    {ü}{{\"{u}}}1
    {û}{{\^{u}}}1
    {â}{{\^{a}}}1
    {Â}{{\^{A}}}1
    {Î}{{\^{I}}}1
}}{}


\lstdefinelanguage{LNat}{
    morekeywords={tant,que,pour,tout,si,sinon,initialiser,renvoyer,attendre la fin, afficher},
    sensitive=false,
    morecomment=[l]{//},
}

\lstnewenvironment{lstLNat}
{\lstset{
    language=LNat,
    basicstyle=\ttfamily,
    keywordstyle=\color{keywordcolor},
    commentstyle=\color{commentcolor},
    stringstyle=\color{stringcolor},
    backgroundcolor=\color{white},
    numbers=left,
    numberstyle=\ttfamily,
    numbersep=-1.5em,
    stepnumber=1,
    frame=l,
    mathescape=true,
    framexleftmargin=-2.25em,
    tabsize=2,
    literate=%
    {é}{{\'e}}{1}%
    {è}{{\`e}}{1}%
    {à}{{\`a}}{1}%
    {ç}{{\c{c}}}{1}%
    {œ}{{\oe}}{1}%
    {ù}{{\`u}}{1}%
    {É}{{\'E}}{1}%
    {È}{{\`E}}{1}%
    {À}{{\`A}}{1}%
    {Ç}{{\c{C}}}{1}%
    {Œ}{{\OE}}{1}%
    {Ê}{{\^E}}{1}%
    {ê}{{\^e}}{1}%
    {î}{{\^i}}{1}%
    {ô}{{\^o}}{1}%
    {û}{{\^u}}{1}%
    {ä}{{\"{a}}}1
    {ë}{{\"{e}}}1
    {ï}{{\"{i}}}1
    {ö}{{\"{o}}}1
    {ü}{{\"{u}}}1
    {û}{{\^{u}}}1
    {â}{{\^{a}}}1
    {Â}{{\^{A}}}1
    {Î}{{\^{I}}}1}
}{}

\newenvironment{implementation}[1]
{   
    \vspace{15pt}
    \begin{boite_implementation}
    \textbf{\textcolor{bleufonce}{Implémentation}}\textit{ - #1}
     \\ \\
}
{    
    \end{boite_implementation}
}

\newmdenv[
  nobreak=true,
  topline=true,
  bottomline=true,
  rightline=true,
  leftline=true,
  linewidth=0.5pt,
  linecolor=black,
  backgroundcolor=mayonnaise,
  innerleftmargin=10pt,
  innerrightmargin=2.5em,
  innertopmargin=5pt,
  innerbottommargin=5pt,
  skipabove=-0.5cm,
  skipbelow=-0.25cm,
]{boite_question}


\newenvironment{question}[2]
{
    \vspace{15pt}
    \begin{boite_question}
    \if\relax\detokenize{#1}\relax
        \textbf{\textcolor{rouge}{Question \arabic{chapitre}.\arabic{boite}}}%
        \if\relax\detokenize{#2}\relax
        \else
            \textit{ - #2}
        \fi
        \stepcounter{boite}
    \else
        \textbf{\textcolor{rouge}{Question #1}}%
        \if\relax\detokenize{#2}\relax
        \else
            \textit{ - #2}
        \fi
    \fi \\
    
    
}
{
    \end{boite_question}
}

\newmdenv[
  nobreak=true,
  topline=true,
  bottomline=true,
  rightline=true,
  leftline=true,
  linewidth=0.5pt,
  linecolor=black,
  backgroundcolor=white,
  innerleftmargin=10pt,
  innerrightmargin=2.5em,
  innertopmargin=5pt,
  innerbottommargin=5pt,
  skipabove=-0.5cm,
  skipbelow=-0.25cm,
]{boite_corollaire}



\newenvironment{corollaire}[2]
{
    \vspace{15pt}
    \begin{boite_corollaire}
    \if\relax\detokenize{#1}\relax
        \textbf{\textcolor{rouge}{Corollaire \arabic{chapitre}.\arabic{boite}}}%
        \if\relax\detokenize{#2}\relax
        \else
            \textit{ - #2}
        \fi
        \stepcounter{boite}
    \else
        \textbf{\textcolor{rouge}{Corollaire #1}}%
        \if\relax\detokenize{#2}\relax
        \else
            \textit{ - #2}
        \fi
    \fi \\
    
    
}
{
    \end{boite_corollaire}
}


\newcounter{chapitre}
\setcounter{chapitre}{23}

\title{\Large Chapitre 23 \\ \Huge Ondes électromagnétiques dans le vide}

\begin{document}
% commandes
\newcommand{\notion}[1]{\textcolor{vert_fonce}{\textit{#1}}}
\newcommand{\mb}[1]{\mathbb{#1}}
\newcommand{\mc}[1]{\mathcal{#1}}
\newcommand{\mr}[1]{\mathrm{#1}}
\newcommand{\code}[1]{\texttt{#1}}
\newcommand{\ccode}[1]{\texttt{|#1|}}
\newcommand{\ov}[1]{\overline{#1}}
\newcommand{\abs}[1]{|#1|}
\newcommand{\rev}[1]{\texttt{reverse(#1)}}
\newcommand{\crev}[1]{\texttt{|reverse(#1)|}}

\newcommand{\ie}{\textit{i.e.} }

\newcommand{\N}{\mathbb{N}}
\newcommand{\R}{\mathbb{R}}
\newcommand{\C}{\mathbb{C}}
\newcommand{\K}{\mathbb{K}}
\newcommand{\Z}{\mathbb{Z}}

\newcommand{\A}{\mathcal{A}}
\newcommand{\bigO}{\mathcal{O}}
\renewcommand{\L}{\mathcal{L}}

\newcommand{\rg}[0]{\mathrm{rg}}
\newcommand{\re}[0]{\mathrm{Re}}
\newcommand{\im}[0]{\mathrm{Im}}
\newcommand{\cl}[0]{\mathrm{cl}}
\newcommand{\grad}[0]{\vec{\mathrm{grad}}}
\renewcommand{\div}[0]{\mathrm{div}\,}
\newcommand{\rot}[0]{\vec{\mathrm{rot}}\,}
\newcommand{\vnabla}[0]{\vec{\nabla}}
\renewcommand{\vec}[1]{\overrightarrow{#1}}
\newcommand{\mat}[1]{\mathrm{Mat}_{#1}}
\newcommand{\matrice}[1]{\mathcal{M}_{#1}}
\newcommand{\sgEngendre}[1]{\left\langle #1 \right\rangle}
\newcommand{\gpquotient}[1]{\mathbb{Z} / #1\mathbb{Z}}
\newcommand{\norme}[1]{||#1||}
\renewcommand{\d}[1]{\,\mathrm{d}#1}
\newcommand{\adh}[1]{\overline{#1}}
\newcommand{\intint}[2]{\llbracket #1 ,\, #2 \rrbracket}
\newcommand{\seg}[2]{[#1\, ; \, #2]}
\newcommand{\scal}[2]{( #1 | #2 )}
\newcommand{\distance}[2]{\mathrm{d}(#1,\,#2)}
\newcommand{\inte}[2]{\int_{#1}^{#2}}
\newcommand{\somme}[2]{\sum_{#1}^{#2}}
\newcommand{\deriveref}[4]{\biggl( \frac{\text{d}^{#1}#2}{\text{d}#3^{#1}} \biggr)_{#4}}





\maketitle

\begin{theoreme}{}{formule du double rotationnel}
    Pour $A$ une fonction vectorielle de l'espace~:
    $$\rot \Big( \rot(\vec{A}) \Big) = \grad(\div \vec{A} ) - \vec{\Delta} (\vec{A})$$
    où intervient l'opérateur \notion{laplacien vectoriel}, s'écrivant en coordonnées cartésiennes~:
    $$\vec{\Delta}(\vec{A}) = \derivepar{2}{\vec{A}}{x} + \derivepar{2}{\vec{A}}{y} + \derivepar{2}{\vec{A}}{z}$$
\end{theoreme}

\begin{theoreme}{}{équation de d'Alembert}
    Dans le vide, les champs électrique et magnétique vérifient tous deux une même équation de propagation, dite \notion{équation de d'Alembert}~:
    $$\vec{\Delta}(\vec{E}) = \frac{1}{c^2} \derivepar{2}{\vec{E}}{t} \qquad \mr{et} \qquad \vec{\Delta}(\vec{B}) = \frac{1}{c^2} \derivepar{2}{\vec{B}}{t}$$
    où $\displaystyle c = \frac{1}{\sqrt{\varepsilon_0 \mu_0}}$ est la célérité de l'onde électromagnétique.
\end{theoreme}

\begin{definition}{}{surface d'onde}
    On appelle \notion{surface d'onde} une surface de l'espace sur laquelle le champ électromagnétique est uniforme à tout instant.\\
    Moralement, tous les points d'une surface d'onde sont dans le même état vibratoire à tout instant.
\end{definition}

\begin{definition}{}{onde plane}
    Une onde est dite \notion{plane} si ses surfaces d'onde sont des plans parallèles.
\end{definition}

\begin{proposition}{}{ondes planes}
    Une onde plane ne dépend que du temps et d'une dimension cartésienne, puisque tous les plans d'ondes sont parallèles et que la perturbation est la même sur ces plans.
\end{proposition}

\begin{definition}{}{onde plane progressive}
    Une onde plane est dite \notion{progressive} lorsqu'elle se propage~:
    \begin{enumeratebf}
        \item dans un sens bien déterminé
        \item sans étalement
        \item sans déformation
    \end{enumeratebf}
    En pratique, l'expression du signal doit présenter un couplage linéaire entre les variables temporelle et spatiale.
\end{definition}

\begin{definition}{}{onde plane progressive sinusoïdale}
    Une onde plane progressive est dite \notion{sinusoïdale}, ou bien \notion{harmonique} si son signal physique réel s'écrit~:
    $$\vec{A}(x,t) = A_0 \cos(\omega t \pm kx + \varphi)\vec{e_p}$$
    Son signal physique complexe s'écrit~:
    \begin{align*}\vecc{A}(x,t) &= A_0 e^{i(\omega t \pm kx + \varphi)}\vec{e_p}\\
        &= \vecc{A_0} e^{i(\omega t \pm kx)}
    \end{align*}
    où $\vecc{A_0} = A_0 e^{i\varphi}\vec{e_p}$.
\end{definition}

\begin{theoreme}{}{équation de dispersion dans le vide}
    L'équation de dispersion lie les pulsations spatiale et temporelle d'une onde plane progressive harmonique. Pour une OPPH électromagnétique dans le vide, la relation de dispersion est la suivante~:
    $$\omega = kc \qquad \text{ soit } \qquad \lambda = \frac{c}{f}$$
\end{theoreme}

\begin{definition}{}{vecteur d'onde d'une OPPH}
    On appelle \notion{vecteur d'onde} d'une OPPH le vecteur de norme $k$, de direction et sens ceux de la propagation de l'onde~:
    $$\vec{k} = \frac{2\pi}{\lambda}\vec{u}$$
\end{definition}

\begin{definition}{}{opérateur nabla appliqué à une représentation complexe d'OPPH}
    Pour $\vecc{E}(x,t) &= \underline{E_0} e^{i(\omega t - \vec{k} \cdot \vec{OM})}\vec{e_p}$, on a~:
    $$\begin{cases*}
        \displaystyle\derivepar{}{\vecc{E}}{x} = -ik_x\vecc{E}\\
        \displaystyle\derivepar{}{\vecc{E}}{y} = -ik_y\vecc{E}\\
        \displaystyle\derivepar{}{\vecc{E}}{z} = -ik_z\vecc{E}\\
    \end{cases*}$$
    On identifie alors $\vec{\nabla} = -i\vec{k}$.
\end{definition}

\begin{theoreme}{}{équations de Maxwell-Gauss et Maxwell-Thomson en représentation complexe}
    En représentation complexe d'une OPPH électromagnétique, les équations de Maxwell-Gauss et Maxwell-Thomson s'écrivent~:
    $$\vec{k} \cdot \vecc{E} = 0 \qquad \text{(M.G.)}$$
    $$\vec{k} \cdot \vecc{B} = 0 \qquad \text{(M.T.)}$$
\end{theoreme}

\begin{theoreme}{}{équations de Maxwell-Faraday et Maxwell-Ampère en représentation complexe}
    En représentation complexe d'une OPPH électromagnétique, les équations de Maxwell-Faraday et Maxwell-Ampère s'écrivent~:
    $$\vec{k} \wedge \vecc{E} = \omega \vecc{B} \qquad \text{(M.F.)}$$
    $$\vec{k} \wedge \vecc{B} = - \varepsilon_0 \mu_0 \omega \vecc{E} \qquad \text{(M.A.)}$$
\end{theoreme}

\begin{theoreme}{}{relation de structure d'une OPPH et d'une OPP}
    Pour une OPPH, d'après les équations de Maxwell en représentation complexe (avec passage en partie réelle)~:
    $$\vec{B} = \frac{\vec{u} \wedge \vec{E}}{c}$$
    Pour une OPP, en tant que somme d'OPPH de même sens de propagation, on a par linéarité la même relation.
\end{theoreme}

\begin{definition}{}{polarisation d'une OPP électromagnétique}
    On appelle \notion{polarisation} d'une OPP la direction du vecteur champ électrique de l'onde.
\end{definition}

\begin{definition}{}{OPP électromagnétique non polarisée}
    Une OPP électromagnétique est dite \notion{non polarisée} si sa direction de polarisation fluctue rapidement et aléatoirement.
\end{definition}

\begin{definition}{}{OPP électromagnétique polarisée rectilignement}
    Par opposition aux OPP électromagnétiques non polarisées, une OPP électromagnétique est dite \notion{polarisée rectilignement} si la direction du vecteur champ électrique est constante.
\end{definition}

\end{document}