\documentclass[a4paper,12pt]{article}

% Encodage et langue
\usepackage[utf8]{inputenc}
\usepackage[T1]{fontenc}
\usepackage[french]{babel}

\newcommand{\ttp}[0]{\underline{\textit{Traité Théologico-politique}} }


% Marges et mise en page
\usepackage[margin=2.5cm]{geometry}

% Titre du document
\title{Cours de Français\\ \Large \textit{"Des fondements de L'État"}}
\author{Raphaël JONTEF}
\date{\today}

\begin{document}
\maketitle
Comment préserver la communauté de la confrontation des pouvoirs (politique et religieux) et ce qui en résulte : les états dans l'État ?\\\\

Lorsque Spinoza évoque l'histoire des hébreux, il explique que c'est le seul à réunir une religion à un peuple, là où les autres religions ont vocation à rayonner dans le monde. Puisque la Terre est dite Sainte, les autres ne peuvent qu'être impures, d'où l'isolement de l'État hébreux, jugeant les autres états d'impurs. Spinoza tire de cette réflexion historique deux constats sur la communauté politique~:
\begin{enumerate}
    \item le souverain doit pouvoir gouverner de manière absolument souveraine. Il ne doit pas avoir de concurrence : pas d'état dans l'État.
    \item les hommes d'église ne doivent pas s'occuper des affaires politiques. Autrement, l'état est voué à l'échec. Selon Spinoza, la parole divine n'est pas accessible aux hommes et que quiconque le réclame ne le fait que pour asseoir sa domination. C'est là que la religion bascule dans le fanatisme. Elle n'est plus une forme d'amour et de tolérance.
\end{enumerate}
\\
Spinoza se demande Raphaël ï donc comment la communauté peut s'ouvrir à la pluralité, au lieu de la diaboliser (exactement comme le herem de Spinoza). Comment faire coexister les divisions nécessaires (car les êtres humains diffèrent) sans menacer la concorde, l'ordre public.

\section{Le contrat social Spinoziste}

Titre du chapitre XVI : "Des fondements de l'". Spinoza s'oppose à la théorie naturaliste de Hobbes et Rousseau. Chez ces deux philosophes, l'État est fondé sur la base d'un état de nature : à l'origine, les hommes y auraient vécu (hypothétiquement). Hobbes pense que l'insécurité y était omni-présente, le chaos partout. Rousseau pense que l'homme est bon à l'état de nature et pense que c'est la société qui corrompt. Chez les deux portant, il y a rupture de l'état de nature à l'état politique. \\
Spinoza diffère de cet approche. Selon lui, quelque chose relevant de l'état de nature doit subsister en nous : toutes les créatures vivantes sont animées par leur volonté de puissance, d'épanouissement (cf. \textit{connatus}). Ceci constitue notre volonté première commune. On ne peut alors imaginer que c'est derrière nous, car manifestement c'est en nous : on ne se rend compte de ce qu'on a que quand on l'a perdu. Comment harmoniser donc cette condition ? On débouche sur la notion de démocratie, régime qui permettrait de respecter ce droit naturel, tout en permettant 
\end{document}